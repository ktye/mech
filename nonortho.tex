\paragraph{Nichtorthogonalit\"at}
Mit einem einfachen Modell wollen wir den Effekt transienter Verst\"arkung infolge nichtorthogonaler Eigenfunktionen demonstrieren.\\
Dazu betrachten wir ein lineares dynamisches System zweiter Ordnung mit dem Zustandsvektor $x(t)=(u,v)$
\begin{equation}
	\frac{\mathrm{d}x}{\mathrm{d}t} = Ax.
\end{equation}
Die Systemmatrix $A$ habe ged\"ampfte Eigenwerte $\Re\lambda_{1,2}<0$ mit den zugeh\"origen nichtorthogonale Eigenvektoren
\begin{equation}
	x_1 = (1,0) \qquad x_2 = (\cos{\varphi},\sin\varphi).
\end{equation}
Um den Effekt der nichtorthogonalen Eigenvektoren hervorzuheben soll der eingeschlossene Winkel $\varphi$ klein sein $0<\varphi\ll\pi/2$.\\
Die Eigenwerte und Eigenvektoren korrespondieren mit der Systemmatrix
\begin{equation}
	A = \begin{pmatrix}\lambda_1&(\lambda_2-\lambda_1)\cot\varphi\\0&\lambda_2\end{pmatrix}.
\end{equation}
Die allgemeine L\"osung des homogenen Systems setzt sich linear aus den Basisl\"osungen zusammen
\begin{equation}
	x(t) = \alpha x_1 e^{\lambda_1 t} + \beta x_2 e^{\lambda_2 t}.
\end{equation}
W\"ahlen wir als Anfangsbedingung einen Vektor, welcher orthogonal zur Vorzugsrichtung des Systems steht
\begin{equation}
	x(0) = (0,1),
\end{equation}
dann lautet die L\"osung
\begin{eqnarray}
	u(t) &=& (e^{\lambda_2 t} - e^{\lambda_1 t}) \cot\varphi \\
	v(t) &=& e^{\lambda_2 t}.
\end{eqnarray}
Da die Realteile der Eigenwerte $\lambda_{1,2}$ negativ sind, n\"ahern sich beide Komponenten f\"ur lange Zeiten exponentiell der Null\"osung.
Trotzdem w\"achst die Komponente in Vorzugsrichtung des Systems $u$ zwischenzeitlich stark, da der Faktor $\cot\varphi$ f\"ur kleine Winkel wie $1/\varphi$ singul\"ar wird.
Das h\"angt damit zusammen, da\ss\ die Faktoren $\alpha$ und $\beta$ sehr gro\ss\ sein m\"ussen, um eine endliche Anfangskonfiguration $x(0)$ darzustellen, welche orthogonal auf die Vorzugsrichtung steht.
Zum Zeitpunkt $t=0$ kompensieren sich die Anteile nahezu gegenseitig.
Wenn die Eigenwerte ungleich sind $\lambda_1\neq\lambda_2$, dann klingen die Anteile im weiteren Verlauf unterschiedlich schnell ab, soda\ss\ vor\"ubergehend einer der beiden Anteile dominiert.\\
\textit{Wird ein ged\"ampftes System mit Vorzugsrichtung orthogonal zu dieser Richtung angeregt, so weicht es parallel zur Vorzugsrichtung aus, bevor es asymptotisch abklingt.
Die Intensit\"at der Verst\"arkung ist das Produkt aus dem Kotangens des eingeschlossenen Winkels der Eigenvektoren und der Differenz der Exponentiall\"osungen.}

