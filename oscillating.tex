% $Id: diss.tex,v 1.79 2006/05/31 10:11:22 elmar Exp $
% vim: foldmethod=indent
\newif\ifklein
%\kleinfalse		% a4paper
\kleintrue		% a5paper
\ifklein
\documentclass[10pt,a5paper,oneside,draft]{book}	
\else
\documentclass[12pt,a4paper,oneside,draft]{book}	
\fi
	\usepackage[centertags]{amsmath}
	\numberwithin{equation}{chapter}
\usepackage{scale}	% scale by sqrt{2}: a5->a4
	\usepackage{amsfonts}
	\usepackage{amssymb}
	\usepackage{mathrsfs}
	\usepackage{textcomp}
	\usepackage{german}
	\usepackage{epic}
	\usepackage{eepic}
	\usepackage{rotating}

	%%footnotes
	\usepackage[perpage,symbol*]{footmisc}

	%\renewcommand{\thechapter}{\S\,\arabic{chapter}}
	\renewcommand{\thesection}{\Alph{section}}
	
	%\renewcommand{\thefigure}{\arabic{figure}}

	%%% package titlesec ( see $TEXMF/doc/latex/styles/titlesec.dvi )
	\usepackage[small,compact,center,clearempty,pagestyles]{titlesec}

	%% titleformat
	%\titleformat{\chapter}[display]{\normalfont\small}{\filcenter{{\chaptertitlename}} \thechapter}{0.5ex}{\Large\bfseries\filcenter\pagestyle{main}}[\pagestyle{main}]
	%\titleformat{\section}[hang]{\normalfont\bfseries\center}{\large\thesection.}{.5em}{}
	%\titleformat{\subsection}[hang]{\normalfont\normalfont\center}{\large\thesubsection.}{.5em}{}
	\titleformat{\chapter}[hang]{\Large\bfseries\center}{\S\,\thechapter.}{.5em}{}
	\titleformat{\section}[hang]{\large\bfseries\center}{\thesection.}{.5em}{}

	%\titlespacing*{\part}		{0pt}{50pt}{40pt}
	\titlespacing*{\chapter}	{0pt}{7ex}{5ex}
	\titlespacing*{\section}	{0pt}{5ex}{3ex}

	%% pagestyles
	\renewpagestyle{plain}[]{
	}
	%\newpagestyle{main}[\small\normalfont]{
	\newpagestyle{main}[\normalfont]{
		\sethead[\S\,\thechapter.][\textit{\chaptertitle}][\thepage]{\S\,\thechapter.}{\textit{\chaptertitle}}{\normalfont{\thepage}}
	}
	\newpagestyle{intro}[\normalfont]{
		%\sethead[\normalfont{\thepage}][\textit{Einleitung}][]{}{}{\normalfont{\thepage}}
		\sethead[][\textit{Einleitung}][\thepage]{}{\textit{Einleitung}}{\normalfont{\thepage}}
	}
	\newpagestyle{epilogue}[\normalfont]{
		\sethead[][\textit{Zusammenfassung}][\thepage]{}{\textit{Zusammenfassung}}{\normalfont{\thepage}}
	}
	\newpagestyle{literatur}[\normalfont]{
		\sethead[][\textit{Literatur}][\thepage]{}{\textit{Literatur}}{\normalfont{\thepage}}
	}

	% default: 100
	\tolerance=400	

	\ifklein
	\def\bibspace{\hspace*{18pt}}
	\else
	\def\bibspace{\hspace*{21pt}}
	\fi
	\newcommand{\todo}[1]{[\textbf{#1}]}
\ifklein
\unitlength=1pt		% for a5paper
\else
\unitlength=1.4pt	% for a4paper
\fi
\begin{document}

\setcounter{page}{-1}
\renewcommand\baselinestretch{1.2}\normalfont
\chapter*{Stabilit\"atstheorie f\"ur die oszillierende Rohrstr\"omung}

\renewcommand\baselinestretch{1}\normalfont
%% Inhalt
\chapter*{Inhalt}
	\newcommand{\cleantoc}[2]{#1\,\dotfill\,#2\\}
	\newcommand{\chaptoc}[3]{\makebox[2em][l]{\S\,#1.}#2\,\dotfill\,#3\\}
	\newcommand{\sectoc}[3]{\hspace*{2em}\makebox[1.5em][l]{#1.}#2\,\dotfill\,#3\\}

	\renewcommand\baselinestretch{1.2}\normalfont
	\cleantoc{Einleitung}{\pageref{sec:einleitung}}
	\chaptoc{1}{Oszillierende Rohrstr\"omung}{\pageref{sec:rohrstroemung}}
	\chaptoc{2}{Einf\"uhrung in die Stabilit\"atstheorie}{\pageref{sec:stabilitaetstheorie}}
	\chaptoc{3}{Quasistatische Methode}{\pageref{sec:quasistatik}}
		\sectoc{A}{Schie{\ss}verfahren}{\pageref{sec:schiessverfahren}}
		\sectoc{B}{Galerkin-Entwicklung}{\pageref{sec:galerkin}}
	\chaptoc{4}{Aussage der Floquet-\!Theorie}{\pageref{sec:floquet}}
	\chaptoc{5}{Berechnungen}{\pageref{sec:berechnungen}}
	\chaptoc{6}{Vergleich mit Experimenten}{\pageref{sec:experimente}}
	\cleantoc{Zusammenfassung}{\pageref{sec:zusammenfassung}}
	\cleantoc{Literatur}{\pageref{sec:literatur}}

	\renewcommand\baselinestretch{1}\normalfont

%% Einleitung
\chapter*{Einleitung}\label{sec:einleitung}
\pagestyle{intro}
%\setcounter{page}{1}
Mit der \textit{Stabilit\"at der oszillierenden Rohrstr\"o"-mung} untersuchen wir die Bewegung einer viskosen Fl\"ussigkeit innerhalb eines unendlich langen Rohres, angetrieben durch einen zeitlich oszillierenden axialen Druckgradienten.
Im Zentrum der Diskussion steht die Frage, ob kleine St\"or"-ungen die Grundstr\"o"-mung, welche aus der L\"osung der Bewegungsgleichungen resultiert, grunds\"atzlich \"andern, oder ob sie nach gewisser Zeit abklingen.
Dies bezeichnen wir als das \textit{Stabilit\"atsproblem} und nennen die Grundstr\"o"-mung entweder \textit{stabil} oder \textit{instabil} gegen\"uber den St\"or"-ungen.\\

Der Bewegungszustand einer Fl\"ussigkeit l\"a\ss t sich in zwei grunds\"atzlich verschiedene Zust\"ande einteilen.
Bei der \textit{laminaren} Bewegung str\"omt die Fl\"ussigkeit in Schichten, die sich nicht vermischen.
Im Gegensatz dazu wird eine ungeordnete teils chaotische Str\"omungsform als \textit{turbulente} Fl\"ussigkeitsbewegung bezeichnet.\\
Unter technischen Gesichtspunkten ist es oft unwesentlich zu wissen, welche Gestalt eine Str\"omung im Detail annimmt.
Es kann jedoch entscheidend sein, ob es sich im ganzen um eine eher laminare oder turbulente Str\"omung handelt.\\
Welche der Formen bevorzugt wird, h\"angt von den Anwendungsf\"allen ab.
In verfahrenstechnischen Anlagen wird beispielsweise zur Herstellung eines gleichm\"a\ss igen Produktes eine laminare Str\"omung der Zustr\"ome erforderlich sein.
Sollen Fl\"ussigkeiten vermischt werden, kann hingegen eine turbulente Str\"omung von Vorteil sein.
Ein weiteres wichtiges Kriterium ist der Str\"omungswiderstand.
Er kann sich mit dem \"Ubergang von laminarer zu turbulenter Str\"omung sprunghaft \"andern.\\
Mathematisch betrachtet sind beide Formen L\"osungen der selben hydrodynamischen Bewegungsgleichungen.
Hierbei ist die laminare Str\"omung in einfachen Geometrien oft analytisch l\"osbar.
Die zentrale Fragestellung ist, welche der Str\"omungsformen sich unter bestimmten gegebenen Bedingungen einstellt.
Die Stabilit\"atstheorie ist ein Versuch, diese Frage zu beantworten.\\
Wir gehen daf\"ur von der analytischen laminaren L\"osung aus und ver\"andern diese ein wenig.
Anschlie\ss end studieren wir die weitere zeitliche Entwicklung der Str\"omung.
Falls sich nach gewisser Zeit wieder die laminare L\"osung einstellt, gilt diese als stabil, und wir gehen davon aus, da\ss\ sie in der Natur in dieser Weise vorkommt.\\
Falls sich jedoch in der weiteren Entwicklung eine v\"ollig andere Str\"omung ergibt, dann bezeichnen wir die laminare L\"osung als instabil.
Sie kann in dieser Form nicht existieren, da sie von jeder noch so kleinen St\"orung ver\"andert wird, obwohl sie mathematisch aus den Bewegungsgleichungen resultiert.
Eine instabile Str\"omung ist im Umkehrschlu\ss\ nicht unbedingt turbulent.
Es kann sich zwar eine turbulente Str\"omung einstellen, falls die untersuchte Str\"omung instabil ist, es kann sich jedoch auch eine weitere Str\"omung ausbilden, die wir von ihrem Ph\"anotyp als laminar bezeichnen m\"u\ss ten.
Auch Zwischentypen oder eine nicht \textit{voll ausgebildete} turbulente Str\"omung sind m\"oglich.\\

In dieser Arbeit stellen wir die Stabilit\"atsfrage bez\"uglich einer \textit{instation\"aren} Grundstr\"omung.
Instation\"ar bedeutet, da\ss\ sich die Str\"omung mit der Zeit ver\"andert.
Als Beispiel behandeln wir die \textit{oszillierende Rohrstr\"omung}.
Diese ist einfach genug, da\ss\ wir eine analytische Darstellung der laminaren L\"osung angeben k\"onnen.
Wir wollen eine \"Ubersicht der Stabilit\"atsverh\"altnisse dieser Str\"omung in Abh\"anigkeit aller ihrer Parameter erstellen, denn es l\"a\ss t sich vermuten, da\ss\ das Stabilit\"atskriterium von den Parametern der Str\"omung beeinflu\ss t wird.\\

Die oszillierende Rohrstr\"omung ist zwar eine idealisierte Str\"omung, doch sie kann als Modell f\"ur viele in Natur und Technik vorkommende Str\"omungen verwendet werden.
In der Biostr\"omungsmechanik hat sie Modellcharakter f\"ur die Str\"o"-mung in Blutgef\"a\ss en, oder den Gasaustausch in den Atemwegen.
Im Bereich der Mikrofluidik wird sie benutzt, um die Str\"omung peristaltischer Pumpen nachzubilden.
Im Maschinenbau spiegelt die Str\"omung die Verh\"altnisse in Kolbenpumpen oder allgemein in hydraulischen oder pneumatischen Rohrsystemen wider.\\

Die Rohrstr\"o"-mung geh\"ort zur Klasse der \textit{Parallelstr\"o"-mungen}, deren Stabilit\"atsanalyse zu den klassischen Problemen der Hydrodynamik z\"ahlt.
Parallelstr\"omungen zeichnen sich dadurch aus, da\ss\ alle Teilchen in die gleiche, oder die entgegengesetzte Richtung flie\ss en.
Dabei nehmen ihre Geschwindigkeiten im allgemeinen verschiedene Werte an.
Neben der Rohrstr\"omung fallen auch die ebenen Couette- und Poiseuillestr\"omungen in diese Kategorie.\\
Die meisten Untersuchungen \"uber das Stabilit\"atsverhalten von Parallelstr\"omungen behandeln das station\"are Problem, bei welchem sich die Geschwindigkeiten der laminaren Str\"omung im Gegensatz zu jenen unserer oszillierenden Str\"omung, zeitlich nicht ver\"andern.\\
In den fr\"uhen Arbeiten wurde versucht, das Statilit\"atsverhalten von Fl\"ussigkeiten mit Hilfe von Theorien \"uber reibungsfreie Fluide zu kl\"aren.
\textsl{Helmholtz}\footnote{\label{bib:helmholtz_fluessigkeit}\textsl{Helmholtz}: \textit{Preu{\ss}.\ Akad.\ Wiss.} \textbf{23} (1868)}
widmete sich der Frage, wie es dazu kommt, da\ss\ in Fl\"ussigkeitsstr\"omen um spitze Kanten Unregelm\"a\ss igkeiten entstehen, welche in elektrischen Feldern mit scheinbar analogen Gleichungen nicht vorhanden sind.\\
\textsl{Rayleigh}\footnote{\label{bib:rayleigh}\textsl{Rayleigh}: \textit{Phil.\ Mag.} \textbf{34} (1892)}
gelang es, eine Stabilit\"atstheorie f\"ur Parallelstr\"omungen aufzustellen, welche auf der Annahme reibungsfreier Fl\"ussigkeiten beruht.
Diese kann in manchen F\"allen als Grenzfall bei niedrig viskosen Fl\"ussigkeiten herangezogen werden, allerdings stimmt das Ergebnis selten mit jenem einer Theorie \"uber schwach reibungsbehaftete Fl\"ussigkeiten \"uberein.
Die mathematischen Probleme, die sich ergeben, wenn man die Viskosit\"at gegen null gehen l\"a\ss t, werden in der Arbeit von \textsl{Case}\footnote{\label{bib:case2}\textsl{Case}: \textit{J.\ Fluid.\ Mech.} \textbf{10} (1961)}
behandelt.\\
In zwei grundlegenden Arbeiten stellte \textsl{Reynolds} die Rolle der Viskosit\"at bei der Ermittlung der Stabilit\"atsgrenze hervor.\footnote{\label{bib:reynolds_experiment}\textsl{Reynolds}: \textit{Phil.\ Trans.} \textbf{174} (1883), \label{bib:reynolds_criterion}\textbf{186} (1895)}
Durch seinen bekannten Farbfadenversuch ermittlete er den la"-mi"-nar-tur"-bu"-lent"-en \"Ubergang einer station\"aren Rohrstr\"omung und f\"uhrte als Kriterium f\"ur den Umschlag einen Parameter ein, welcher heute als Reynoldssche Zahl bekannt ist.
In der zweiten Arbeit widmete sich \textsl{Reynolds} einer theoretischen Erkl\"arung der gefundenen Zusammenh\"ange und versuchte, die turbulente Str\"omung mittles einer Analogie zur kinetischen Gastheorie zu ergr\"unden.\\
Au\ss er \textsl{Reynolds} nahmen es sich auch andere Wissenschaftler zur Aufgabe, die experimentell gefundene Stabilit\"atsgrenze zu berechnen.
Die hierbei entwickleten Theorien basieren auf zwei verschiedenen Ans\"atzen:
der \textit{Energiemethode} und der \textit{Methode der kleinen Schwingungen}.\\
Erstere bilanziert den Energietransfer von der Grundstr\"omung in die St\"orungen.
\textsl{Serrin}\footnote{\label{bib:serrin}\textsl{Serrin}: \textit{Arch.\ Rat.\ Mech.\ Anal.} \textbf{3} (1959)}
weist mittles Variationsrechnung nach, da\ss\ es f\"ur station\"are Str\"omungen mit beliebiger Umrandung eine minimale Reynoldssche Zahl gibt, unterhalb derer die laminare Str\"omung stets stabil ist.\footnote{F\"ur nichtnewtonsche Fluide gilt diese Aussage im Allgemeinen nicht.}
Die Methode liefert im Umkehrschlu\ss\ leider keine Aussage, ab welcher Reynoldsschen Zahl eine Instabilit\"at zu erwarten ist.
Da die gefundenen Zahlenwerte sehr niedrig sind, ist die Aussagekraft dieser Theorien f\"ur das Auffinden des laminar-turbulenten \"Ubergangs begrenzt.\\
Die Methode der kleinen Schwingungen wurde erstmals von \textsl{Kelvin}\footnote{\label{bib:kelvin}\textsl{Kelvin}: \textit{Phil.\ Mag.} \textbf{24} (1887)}
auf das hydrodynamische Problem angewandt und sp\"ater von \textsl{Sommerfeld}\footnote{\label{bib:sommerfeld_congresso}\textsl{Sommerfeld}: \textit{4 Cong.\ Mat.} \textbf{III} (1908)}
und seinen Sch\"ulern \textsl{Hopf}\footnote{\label{bib:hopf}\textsl{Hopf}: \textit{Ann.\ Phys.} \textbf{44} (1914)}
und \textsl{Heisenberg}\footnote{\label{bib:heisenberg}\textsl{Heisenberg}: \textit{Annalen der Physik} \textbf{74} (1924)}
weiterentwickelt.
Sie beschr\"ankten sich der mathematischen Einfachheit wegen auf die ebenen Couette- und Poiseuillestr\"o"-mungen.
Die Theorie ist heute bekannt unter dem Namen \textit{Orr-Sommerfeld-\!Theorie}.\footnote{\label{bib:orr}\textsl{Orr}: \textit{Proc.\ Roy.\ Irish Acad.} \textbf{XXVII} (1907)}
\textsl{Sexl}\footnote{\label{bib:sexl}\textsl{Sexl}: \textit{Ann.\ Phys.} \textbf{83} (1927)}
verallgemeinerte den Ansatz und formulierte die St\"or"-ungsgleichungen f\"ur die Rohrstr\"o"-mung in zylindrischen Koordinaten.\\
Keine der Arbeiten konnte die Stabilit\"atsgrenze der Rey"-noldsschen Experimente theoretisch untermauern.
Es wurde vielmehr vermutet, da\ss\ die station\"are Rohrstr\"o"-mung f\"ur beliebige Rey"-noldssche Zahlen stabil gegen\"uber kleinen St\"or"-ungen bleibe.
Da man zu jener Zeit auf asymptotische Methoden angewiesen war, konnte dies erst durch eine numerische L\"osung des Ei"-gen"-wert"-pro"-blems best\"atigt werden, welche \textsl{Salwenn und Grosch}\footnote{\label{bib:salwen_grosch}\textsl{Salwen, Grosch}: \textit{J.\ Fluid.\ Mech.} \textbf{54} (1972)} f\"ur eine gro\ss e Anzahl an Parametern durchf\"uhrten.\\
%{Heisenbergsche Arbeit: ``Das Turbulenzproblem der Hydrodynamik ist ein Problem der energetischen nicht der dynamischen Stabilit\"at''}
%{\textsl{Heisenberg} (1948): ``Da\ss\  die turbulente Bewegung die Regel, die laminare die Ausnahme ist, folgt einfach aus der Tatsache, da\ss\  eine Fl\"ussigkeit sehr viele Freiheitsgrade besitzt.''}
%{``Die Verteilung der Energie auf die anderen Wellenl\"angenbereiche geschieht stehts durch die nichtlinearen, die Tr\"agheit darstellenden Glieder [$\ldots$] die Geschwindigkeiten m\"ussen also einen gewissen Mindestbetrag erreichen, damit die Tr\"agheitsglieder \"uberhaupt eine erhebliche Rolle spielen k\"onnen.''}
%{Statistische Methoden, isotrope Turbulenz: \textsl{T.~von Karmann, G.I. Taylor, C.F. von Weizs\"acker (1948)}}

Das hydrodynamische System hat einige Eigenschaften, die eine Stabilit\"atsanalyse erschweren.
\begin{itemize}
	\item Es handelt sich um ein nichtlineares mehrdimensionales System mit nichtkonstanten Koeffizienten.
	\item Die Dispersionsrelation, d.h.\ der Zusammenhang zwischen Wellenl\"ange und Frequenz der St\"or"-ungen, liegt nicht explizit vor und ist mehrdeutig.
	\item Es gibt einen kleinen Parameter, der sich schlecht auf die Kondition auswirkt, was zu numerischen Problemen f\"uhren kann.
	\item Der Parameterraum ist hochdimensional, was einen hohen Rechenaufwand zur Folge hat.
\end{itemize}

Das Problem der Nichtlinearit\"at kann umgangen werden, da die St\"or"-gr\"o\ss en per Definition zun\"achst klein gegen\"uber der Hauptl\"os"-ung sind.
Terme h\"oherer Ordnung in den St\"or"-gr\"o\ss en k\"onnen vernachl\"assigt werden, was auf ein lineares System f\"uhrt.\\
Das lineare System hat den Vorteil, da\ss\ das Superpositionsprinzip angewendet werden kann, das es \"uberhaupt erst erm\"oglicht, eine gro\ss e Klasse m\"oglicher St\"or"-ungen zu erfassen.
Die Aussage einer linearen Stabilit\"atsanalyse gilt f\"ur Zeiten, in denen die St\"or"-ungen klein sind.
Daher gilt sie f\"ur instabile Systeme nur begrenzt.
Wenn die St\"or"-ungen eine gewisse Gr\"o\ss e \"uberschreiten, k\"onnen wir sicher sein, da\ss\ die Vorhersagen der linearen Theorie ihre G\"ultigkeit verloren haben.
In der weiteren Entwicklung spielen die nichtlinearen Terme eine entscheidende Rolle.
Es ist jedoch nicht notwendigerweise der Fall, da\ss\ sich aus einer instabilen laminaren Str\"omung stets eine turbulente Str\"omung entwickelt.
Die nichtlinearen Terme k\"onnen auch dazu f\"uhren, da\ss\ die St\"omung in eine neue Ordnung---m\"oglicherweise ein periodisches Muster---gef\"uhrt wird, die ihrerseits stabil erscheint.
Man nennt diese Ph\"anomene \textit{Sekund\"arstr\"o"-mungen}.
Der Stabilit\"atsbegriff einer linearen Theorie kann diese nicht mit einschlie\ss en.\\

Von einem energetischen Standpunkt aus, ist die Voraussetzung einer Instabilit\"at ein Transfer kinetischer Energie von der Grundstr\"o"-mung in die St\"or"-ungen.
Bei diesem Vorgang spielen nichtlineare Effekte keine Rolle.
Dies kann als weitere Rechtfertigung einer linearen Theorie gelten.\\
In den letzten Jahren gibt es allerdings eine Theorie, die einen Instabilit\"atsmechanismus vorschl\"agt, bei dem es trotz linearer Stabilit\"at und kleiner Anfangsst\"or"-ungen zu In"-sta"-bi"-li"-t\"at"-en kommen kann.
Aufgrund der Nichtorthogonalit\"at der Eigenfunktionen kann es m\"oglich sein, da\ss\ kleine St\"or"-ungen kurzfristig wachsen bis sie eine Gr\"o\ss e erreicht haben, bei der nichtlineare Effekte zu einer Umverteilung der Energie in den St\"or"-ungen f\"uhren k\"onnen.\footnote{\label{bib:trefethen}\textsl{Trefethen, Trefethen, Reddy, Driscoll}: \textit{Science} \textbf{261} (1993)\\\bibspace\label{bib:reddy}\textsl{Reddy, Schmidt, Henningson}: \textit{SIAM J.\ Appl.\ Math.} \textbf{53} (1993)\\\bibspace\label{bib:grossmann}\textsl{Grossmann}: \textit{Rev.\ Mod.\ Phys.} \textbf{72} (2000)}\\
Bereits \textsl{Hamel}\footnote{\label{bib:hamel}\textsl{Hamel}: \textit{Ges.\ Wiss.\ G{\"o}ttingen} \textbf{} (1911)}
kritisierte die Sommerfeldsche Arbeit und wies auf die M\"og"-lichkeit von Instabilit\"at trotz ged\"ampfter Eigenmoden hin.
Die versprochene n\"ahere Erl\"auterung lie\ss\ er jedoch aus.
Ob er den selben Effekt im Sinne hatte ist ungewi\ss.\\

In der Theorie der hydrodynamischen Stabilit\"at unterscheidet man das \textit{\"ortliche} und das \textit{zeitliche} Problem, sowie die mathematisch verwandten Themen der \textit{konvektiven} und \textit{absoluten} Instabilit\"at.
Diese Begriffe wurden im Bereich der Plasmaphysik gepr\"agt und sp\"ater in der Hydrodynamik aufgegriffen.\footnote{\textsl{Landau, Lifschitz}: (1983)\\\bibspace\label{bib:watson_j}\textsl{Watson}: \textit{J.\ Fluid.\ Mech.} \textbf{14} (1962)\\\bibspace\label{bib:huerre_monkewitz}\textsl{Huerre, Monkewitz}: \textit{Ann.\ Rev.\ Fluid Mech.} \textbf{22} (1990)}\\
Im Mittelpunkt dieser Thematik steht die \textit{Dispersionsrelation}, die Beziehung zwischen Frequenz und Wellenl\"ange der St\"or"-ungen.
Beim zeitlichen Problem fragt man nach der Entwicklung eines Anfangszustandes, bei dem St\"or"-ungen durch ihre spektralen Komponenten dargestellt werden.
Die Dispersionsrelation wird unter Vorgabe reeller Wellenl\"angen, nach komplexen Frequenzen gel\"ost.
Falls komplexe Frequenzen mit anfachenden Eigenschaften existieren, gilt das zeitliche Problem als instabil.\\
Das \"ortliche Problem behandelt den umgekehrten Fall.
Anstatt eines Anfangswertproblems mit rellen Wellenzahlen, erfolgt die Vorgabe reeller Frequenzen.
Diese Art der Problemstellung kommt experimentellen Anordnungen nach, bei denen eine Str\"o"-mung an einer festen Stelle harmonisch gest\"ort wird.
Aus der Dispersionsrelation erh\"alt man im Falle einer Instabilit\"at eine komplexe Wellenzahl, d.h.\ es ergibt sich ein Ansteigen der St\"or"-ung im Ort.
Im Gegensatz zum zeitlichen Problem ist das Kriterium nicht hinreichend.
Ein Gegenbeispiel ist das Modell transversaler Schwingungen einer elastisch gebetteten Saite---eine Wellengleichung mit linearer R\"uckstellung.
F\"ur einige Frequenzen hat ihre Dispersionsrelation komplex konjugierte L\"osungen und ist trotzdem stabil.
Eine Antwort auf die Stabilit\"atsfrage gibt erst eine genauere Untersuchung der Topologie der Dispersionsrelation.\footnote{\textsl{Clemmow, Dougherty} (1969)}
Die entsprechende Unterscheidung f\"uhrt beim zeitlichen Problem auf die Begriffe der konvektiven und absoluten Instabilit\"at.\\

Eine Anfangsst\"or"-ung, welche zeitlich w\"achst, gleichzeitig von der Str\"o"-mung wegtransportiert wird, wird als konvektiv instabil bezeichnet.
Im Gegensatz dazu bezeichnet die absolute Instabilit\"at den Zustand, da\ss\ St\"or"-ungen nach gen\"ugend langer Zeit an jedem Ort anwachsen.
Ob die Unterscheidung notwendig ist, h\"angt von der Betrachtung ab.
Jede konvektive Instabilit\"at, kann mittels einer Galilei-\!Transformation in eine absolute umgewandelt werden und umgekehrt.
Betrachtet man hingegen ein begrenztes System, aus dem eine wachsende St\"orung hinausflie\ss t, bevor sie einen gewissen Wert erreicht hat, dann ist die Unterscheidung essentiell.\\
Beim vorliegenden Problem mu\ss\ zwischen konvektiven und absoluten In"-sta"-bi"-li"-t\"at"-en nicht unterschieden werden.
Zum einen erstreckt sich das System \"uber ein unendlich gro\ss es Gebiet, zum anderen oszilliert die Grundstr\"o"-mung harmonisch, soda\ss\ in einem Zeitraum \"uber mehrere Perioden keine Konvektion stattfindet.\\

Im Gegensatz zu den zahlreichen Arbeiten, welche station\"are Str\"o"-mungen behandeln, sind Theorien zur Stabilit\"at periodischer oder instation\"arer Str\"o"-mungen seltener.
\textsl{Davis} gibt eine \"Ubersicht \"uber einige behandelte Arbeiten.\footnote{\label{bib:davis_review}\textsl{Davis}: \textit{Ann.\ Rev.\ Fluid Mech.} \textbf{8} (1976)}\\
Von den ebenen Str\"o"-mungen wurden die oszillierende Str\"o"-mung zwischen zwei Platten und die Stokessche Grenzschicht untersucht.\footnote{\label{bib:grosch_salwen}\textsl{Grosch, Salwen}: \textit{J.\ Fluid.\ Mech.} \textbf{34} (1968)\\\bibspace\label{bib:hall}\textsl{Hall}: \textit{Proc.\ Roy.\ Soc.} \textbf{359} (1978)\\\bibspace\label{bib:blennerhassett_bassom}\textsl{Blennerhassett, Bassom}: \textit{J.\ Fluid Mech.} \textbf{464} (2002)\\\bibspace\label{bib:hall3}\textsl{Hall}: \textit{J.\ Fluid Mech.} \textbf{482} (2003)}
Diese sind aufgrund ihrer einfachen Geometrie leichter zug\"anglich als Str\"omungen in zylindrischen Koordinaten.
In Grenzf\"allen kann die Aussage von Theorien ebener Str\"omungen auch auf jene mit gekr\"ummter Umrandung angewandt werden.\\
\textsl{Ghidaoui und Kolyshkin}\footnote{\label{bib:ghidaoui_kolyshkin}\textsl{Ghidaoui, Kolyshkin}: \textit{J.\ Fluid.\ Mech.} \textbf{465} (2002)} behandeln eine instation\"are Str\"o"-mung im Rohr mit der quasista"-tisch"-en Methode. Allerdings untersuchen sie im Gegensatz zur vorliegenden Arbeit keine oszillierenden Geschwindigkeiten, sondern eine Anlauf-Abbremsstr\"o"-mung.\\
\textsl{Tozzi und von Kerczek}\footnote{\label{bib:tozzi_kerczek}\textsl{Tozzi, von Kerczek}: \textit{J.\ Appl.\ Mech.} \textbf{53} (1986)} wenden die Floquet-\!Theorie auf eine Rohrstr\"o"-mung an, welcher zus\"atzlich zum station\"aren Anteil kleine Oszillationen \"uberlagert werden.
Sie ziehen den Schlu\ss, da\ss\ die Schwingungen die Hauptstr\"o"-mung leicht stabilisieren.\\

Die rein oszillierende Rohrstr\"o"-mung wird von \textsl{Yang und Yih}\footnote{\label{bib:yang_yih}\textsl{Yang, Yih}: \textit{J.\ Fluid.\ Mech.} \textbf{82} (1977)} mit einer Floquet-\!Theorie mit finiten Differenzen untersucht.
Es stellt sich heraus, da\ss\ die Str\"o"-mung gegen\"uber axialsymmetrischen St\"or"-ungen keine Instabilit\"at aufweist.
Auch \textsl{Fedele, Hitt und Prabhu}\footnote{\label{bib:fedele_hitt}\textsl{Fedele, Hitt, Prabhu}: \textit{Euro.\ J.\ Mech.} \textbf{24} (2005)} beschr\"anken sich auf axialsymmetrische St\"or"-ungen und kommen mit einer \"ahnlichen Theorie zum gleichen Ergebnis.
\textsl{Zhao, Ghidaoui, Kolyshkin und Vaillancourt}\footnote{\label{bib:zhao_ghidaoui}\textsl{Zhao, Ghidaoui, Kolyshkin, Vaillancourt}: \textit{Tech.\ Mech.} \textbf{24} (2004)}
l\"osen das linearisierte Anfangswertproblem exemplarisch f\"ur wenige Parameter.
Auch sie best\"atigen ein Abklingen der St\"orungsenergie nach mehreren Perioden.\\

Das Ziel der vorliegenden Arbeit ist die Charakterisierung des Str\"omungsverhaltens der oszillierenden Rohrstr\"omung in Abh\"angigkeit ihrer Parameter.
Welche Parameter n\"otig sind, ermitteln wir im ersten Kapitel, in dem wir eine analytische Darstellung der laminaren Grundstr\"omung herleiten und ihre dimensionslosen Kenngr\"o\ss en identifizieren.\\
Anschlie\ss end f\"uhren wir eine Stabilit\"atsrechnung durch.
Hierf\"ur vergleichen wir mehrere Theorien und beurteilen ihre Aussagef\"ahigkeiten.
F\"ur das vorliegende zeitlich periodische Problem kommen quasistatische Methoden und eine Floquet-\!Theorie in Frage.
Wir entscheiden uns f\"ur eine quasistatische Methode mit Hilfe einer Galerkin-Entwicklung.
Diese erm\"oglicht es, kurzzeitige Instabilit\"aten vorauszusagen, welche im Laufe der Periode wieder ged\"ampft werden.
Eine Floquet-\!Theorie beurteilt das Langzeitverhalten von St\"orungen \"uber mehrere Perioden und kommt zu dem Ergebnis, da\ss\ die oszillierende Rohrstr\"omung gegen\"uber allen St\"orungen stabil f\"ur lange Zeiten ist.\\
Mit Hilfe der quasistatischen Methode f\"uhren wir Parameterstudien \"uber viele Kombinationen von Grundstr\"omungen und St\"orungen durch.
Die Linearit\"at der Theorie erm\"oglicht es, da\ss\ wir mittles Fourierreihen und Koordinatenseparation eine sehr allgemeine Darstellung von St\"orungen ber\"ucksichtigen k\"onnen.
Die benutzten Methoden sind so weit wie m\"oglich analytisch aufbereitet.
Im letzten Schritt werden die Rechnungen numerisch durchgef\"uhrt.\\
Die Berechnungsergebnisse vergleichen wir im letzten Kapitel mit Experimenten vorheriger Arbeiten.\\

%% Oszillierende Rohrstroemung
\chapter{Oszillierende Rohrstr\"omung}\label{sec:rohrstroemung}
\pagestyle{main}

In diesem Kapitel leiten wir die Grundstr\"omung her, deren Stabilit\"at im folgenden untersucht wird.
Wir geben eine dimensionslose Form an, die sich durch zwei unabh\"anige Kenngr\"o\ss en parametrisieren l\"a\ss t.\\

\begin{center}
\begin{tabular}{rl}	
	\hline\hline
	Radius & $L$ \\
	Oszillationsfrequenz & $\Omega$ \\
	Amplitude & $V/2$ \\
	kinematische Viskosit\"at & $\nu$ \\
	\hline
	Grenzschichtdicke & $\delta = (\nu/\Omega)^{\frac{1}{2}}$\\
	diffusive Zeiteinheit & $L^2/\nu$\\
	konvektive Zeiteinheit & $L/V$\\
	aufgepr\"agte Zeiteinheit & $1/\Omega$\\
	\hline
	Rey"-noldszahl & $R = VL/\nu$\\
	Womersleyzahl & $\alpha = L(\Omega/\nu)^{\frac{1}{2}} = L/\delta$\\
	Grenzschicht-Rey"-noldszahl & $R/\alpha = V\delta/\nu$\\
	\hline\hline
\end{tabular}
\end{center}

Wir betrachten die Bewegung einer viskosen Fl\"ussigkeit in einem unendlich ausgestreckten Rohr verursacht durch einen harmonisch oszillierenden axialen Druckgradienten.
Die eingeschwungene laminare L\"osung nennen wir die \textit{oszillierende Rohrstr\"o"-mung}.\footnote{\label{bib:sexl2}\textsl{Sexl}: \textit{Z.\ Phys.} \textbf{61} (1930)\\\bibspace\label{bib:uchida}\textsl{Uchida}: \textit{ZAMM} \textbf{7} (1956)} Sie hat nur eine Geschwindigkeitskomponente in axialer Richtung, und geh\"ort daher zur Klasse der Parallelstr\"o"-mungen.\\
Die r\"aumlich \"uber den Querschnitt gemittelte Geschwindigkeit oszilliert harmonisch mit der Kreisfrequenz $\Omega$ und der Amplitude $V/2$.
Den Faktor $1/2$ w\"ahlen wir, damit die Formulierung im station\"aren Grenzfall mit der Hagen-Poiseuillschen L\"osung vergleichbar ist.\\
Aus diesen Parametern sowie dem Radius $L$ und der kinematischen Viskosit\"at $\nu$ bilden wir die Kenngr\"o\ss en
\begin{equation}
	R = VL/\nu \qquad \alpha = L \left(\Omega/\nu\right)^{\frac{1}{2}}.
\end{equation}
Es sind dies die Rey"-noldszahl und die Womersleyzahl.\footnote{\label{bib:womersley_rep}\textsl{Womersley:} \textit{WADC Tech.\ Rep.\ } \textbf{TR 56-614} (1959)}\\
Wir benutzen dimensionslose Gr\"o\ss en: Geschwindigkeiten beziehen sich auf $V\!$, L\"angen auf $L$, die Zeit auf $L/V$ und der Druck auf das Produkt aus der Dichte und $V^2$.
Die dimensionslose r\"aumlich \"uber den Querschnitt gemittelte Geschwindigkeit sei
\begin{equation}\label{eq:baseflow-barV}
	\overline W = \frac{1}{2} \sin \tau \qquad \tau = \frac{\alpha^2}{R}t.
\end{equation}
Die Bewegungsgleichung lautet im zylindrischen Koordinatensystem mit der radialen Koordinate $r$, der azimuthalen Koordinate $\varphi$ und der axialen Koordiante $z$
\begin{equation}
	  \frac{\partial{W}}{\partial t} +  \frac{\partial{p}}{\partial z} = \frac{1}{R}\left(  \frac{\partial^{2}{W}}{\partial r^2} + \frac{1}{r}  \frac{\partial{W}}{\partial r} \right)
\end{equation}
mit der axialen laminaren Geschwindigkeit $W(r,t)$.
Aus formalen Gr\"unden formulieren wir die Schwingungsgleichungen mit Hilfe komplexer Gr\"o\ss en.
Als physikalisch relevant ist hierbei nur der Realteil zu betrachten.\\
Der axiale Druckgradient oszilliert harmonisch
\begin{equation}
	 \frac{\partial{p}}{\partial z} = -\frac{\alpha^2}{R} a e^{i(\tau - \tau_0)}.
\end{equation}
Als Randbedingungen fordern wir die Haftung der Fl\"ussigkeit an der Rohrwand und Regularit\"at der Geschwindigkeit auf der Achse
\begin{equation}\label{eq:randi}
	W(1,t) = 0\qquad W(0,t):\textrm{ regul\"ar}.
\end{equation}
Die L\"osung folgt mit Hilfe des Gleichtaktansatzes zu
\begin{equation}\label{eq:baseflow}
	W(r,t) = a \left[ 1 - \frac{J_0 \left(\sqrt i \alpha r\right)}{J_0\left(\sqrt i \alpha\right) } \right] e^{i(\tau-\tau_0)}.
\end{equation}
Hierbei bezeichnet $J_0$ die nullte Besselsche Funktion erster Art.
Der Amplitudenfaktor $a(\alpha)$ und die Phase $\tau_0(\alpha)$ sind so zu w\"ahlen, da\ss\ die mittlere Geschwindigkeit
\begin{equation}
	\overline{W} = 2\int_0^1 W(r,t) \,r \,\mathrm{d} r
\end{equation}
gem\"a\ss\ (\mbox{\ref{eq:baseflow-barV}}) gegeben ist.\\
Im Grenzfall langsamer Oszillation (und damit kleiner Womersleyscher Zahl) geht das Str\"o"-mungs"-pro"-fil in die Hagen-Poiseuille"-sche Parabel \"uber
\begin{equation}\label{eq:hagen-poiseuille}
	\lim_{\alpha\to 0} W(r,t) = \left(1-r^2\right) \sin \tau.
\end{equation}
Im Hochfrequenzlimes zeigt die Str\"o"-mung Grenzschichtcharakter.
Sie hat einen reibungsfreien Kern, welcher sich durch eine \"uber den Querschnitt konstante Geschwindigkeit auszeichnet, und eine viskose Grenzschicht nahe der Wand
\begin{equation}
	\lim_{\alpha \gg 1} W(r,t) = \frac{1}{2} \left[ \sin\tau - e^{-\eta}\sin(\tau-\eta) \right]
\end{equation}
mit der mit der Grenzschichtdicke skalierten Koordinate $\eta = \alpha(1-r)/\sqrt{2}$.
Der mit dem Kern mitbewegte Beobachter stellt eine Geschwindigkeitsverteilung in der Grenzschicht fest, welche der \textit{Stokesschen Grenzschicht}\footnote{\label{bib:stokes}\textsl{Stokes}: \textit{Trans.\ Cambridge Phil.\ Soc.\ } \textbf{ix} (1851)} entspricht.
Die Geschwindigkeitsverteilung hat zu einem festen Zeitpunkt die Form $e^{-\eta} \sin \eta$.
Daraus folgt, da\ss\ die Grenzschicht f\"ur h\"ohere Anregungsfrequenzen enger wird, ihre Form sich jedoch nicht charakteristisch \"andert.\\
%Der Abklingfaktor der \"ortlichen Schwingung des Grenzschichtprofils gleicht der Kreisfrequenz, daher wird die Schwingungsamplitude nach einer Wellenl\"ange durch Multiplikation mit dem Faktor $e^{-2\pi}$ reduziert.\\

Die radiale Verteilung der Geschwindigkeit h\"angt also ma\ss geblich vom Womersleyschen Parameter $\alpha$ ab.
Wir k\"onnen $\alpha$ als den Quotienten aus dem Radius und der Grenzschichtdicke interpretieren oder als die Wurzel aus dem Verh\"altnis der \textit{diffusiven Zeitskala} $L^2/\nu$ zur \textit{aufgepr\"agten Zeitskala} $1/\Omega$.\\
Der Parameter $R$ \"andert nichts an der Form der Geschwindigkeitsverteilung, sondern stellt ihren charakteristischen Amplitudenfaktor $V$ ins Verh\"altnis zur Diffusionsgeschwindigkeit $\nu/L$.\\

Abbildung \mbox{\ref{fig:baseflow}} zeigt die Verteilung der Geschwindigkeit $W$ \"uber der radialen Koordinate $r$ w\"ahrend eines halben Zyklus.
Im oberen Bild erkennen wir, da\ss\ f\"ur kleine Womersleysche Zahlen $\alpha$ die Geschwindigkeitsverteilung in die Hagen-Poiseuillsche Parabel (\mbox{\ref{eq:hagen-poiseuille}}) \"ubergeht.
In der quadratischen Skalierung der radialen Koordinate erscheint die Parabel als Gerade.\\
F\"ur gr\"o\ss ere Womersleysche Zahlen oszilliert der Kern mit konstanter Geschwindigkeit, hingegen bildet sich im Gebiet nahe der Wand eine viskose Grenzschicht aus.\\

\def\ploth{248}
\def\plotoffx{60}
\def\plottoy{250}
\begin{figure}[ htbp ]	% Grundstroemung
	\begin{center}
	\begin{picture}(275,\ploth)(0,0)
	\thinlines

	\drawline(\plotoffx,10)(\plottoy,10)
	\drawline(\plotoffx,78)(\plottoy,78)
	\drawline(\plotoffx,84)(\plottoy,84)
	\drawline(\plotoffx,162)(\plottoy,162)
	\drawline(\plotoffx,168)(\plottoy,168)
	\drawline(\plotoffx,246)(\plottoy,246)

	\drawline(\plotoffx,10)(\plotoffx,78)
	\drawline(\plotoffx,84)(\plotoffx,162)
	\drawline(\plotoffx,168)(\plotoffx,246)

	\drawline(\plottoy,10)(\plottoy,78)
	\drawline(\plottoy,84)(\plottoy,162)
	\drawline(\plottoy,168)(\plottoy,246)

	\drawline(71.875,10)(71.875,78) % (1/4)^2
	\drawline(166.875,10)(166.875,78) % (3/4)^2
	\drawline(107.5,10)(107.5,78)
	\drawline(71.875,84)(71.875,162) % (1/4)^2
	\drawline(166.875,84)(166.875,162) % (3/4)^2
	\drawline(107.5,84)(107.5,162)
	\drawline(71.875,168)(71.875,246) % (1/4)^2
	\drawline(166.875,168)(166.875,246) % (3/4)^2
	\drawline(107.5,168)(107.5,246)
	\put(58,4){\tiny $0$}
	\put(67,4){\tiny $1/4$}
	\put(102,4){\tiny $1 / 2$}
	\put(162,4){\tiny $3/4$}
	\put(248,4){\tiny $1$}
	\put(205,-4){$r$}

	\put(42,8){\makebox(15,4)[ r ]{\tiny -1}}
	\put(42,76){\makebox(15,4)[ r ]{\tiny 1}}
	%\thicklines \input{fig/baseflow/base-30.tripath} \thinlines
		\thicklines
		\drawline(60,61.8448)(174.682,61.783)(198.709,62.5455)(207.74,63.0812)(216.839,62.8545)(222.463,62.1539)(227.813,60.6909)(233.3,57.9091)(238.038,54.9006)(242.434,51.4182)(250,44)
		\drawline(60,56.0545)(137.602,56.137)(172.637,55.7455)(189.507,56.0545)(196.425,56.3636)(212.75,58.1358)(224.371,58.9188)(229.995,58.4448)(234.084,57.1467)(238.685,55.2921)(242.945,52.0364)(250,44)
		\drawline(60,43.0727)(128.161,43.0727)(148.099,43.2376)(180.305,42.9285)(199.731,43.0727)(213.295,44.783)(225.257,47.4824)(230.37,48.6364)(234.596,49.1721)(238.924,49.1721)(243.15,48.4097)(246.49,46.6994)(250,44)
		\drawline(60,30.8739)(119.062,30.8739)(150.382,31.0182)(165.991,30.7091)(187.7,30.4824)(197.277,30.0909)(206.104,30.3176)(216.362,31.6364)(224.099,33.8)(228.972,35.6545)(234.323,38.2097)(242.639,42.1455)(246.217,43.3818)(250,44)
		\drawline(60,26.2376)(149.121,26.2376)(177.476,26.1552)(191.04,25.7636)(200.719,25.1455)(210.057,24.9188)(216.328,25.063)(220.657,25.4545)(224.371,26.2994)(229.858,28.2364)(235.004,31.0182)(238.822,33.6558)(243.286,37.3648)(250,44)
		\thinlines
	\drawline(60,44)(250,44)\put(42,42){\makebox(15,4)[ r ]{\tiny 0}}
	\put(28,47){$W$}
	\put(215,68){$\alpha = 30$}

	\put(42,82){\makebox(15,4)[ r ]{\tiny -1}}
	\put(42,160){\makebox(15,4)[ r ]{\tiny 1}}
	%\thicklines \input{fig/baseflow/base-10.tripath} \thinlines
		\thicklines
		\drawline(60,145.171)(69.6448,145.076)(79.0852,144.982)(114.87,146.235)(142.1,147.393)(155.699,147.298)(164.73,146.849)(173.932,146.045)(182.622,144.982)(195.778,142.595)(212.204,138.34)(232.789,130.54)(250,123)
		\drawline(60,100.947)(84.47,100.947)(104.032,100.144)(118.21,99.5291)(144.657,98.6309)(154.71,98.7255)(164.798,99.08)(178.533,100.404)(189.064,102.011)(200.242,104.304)(213.295,108.109)(229.381,113.971)(240.389,118.58)(250,123)
		\drawline(60,105.273)(75.1318,104.918)(95.2054,104.304)(111.36,103.855)(129.184,103.855)(140.124,104.209)(150.109,104.753)(160.879,105.982)(171.58,107.495)(180.884,108.913)(190.665,111.04)(209.955,115.744)(231.937,120.518)(240.457,121.865)(250,123)
		\drawline(60,119.738)(94.9327,119.1)(115.552,119.195)(134.705,120.258)(147.383,121.322)(160.811,122.929)(170.865,124.418)(199.22,127.609)(209.648,128.413)(220.657,128.602)(230.983,127.538)(239.094,126.191)(250,123)
		\drawline(60,136.307)(81.13,136.118)(93.2969,136.213)(105.668,136.473)(118.619,137.016)(133.683,137.985)(142.407,138.695)(155.187,139.664)(164.969,140.113)(174.17,140.562)(190.665,140.373)(200.31,139.404)(208.387,138.245)(218.475,136.047)(229.143,132.762)(236.743,129.736)(242.877,126.735)(250,123)
		\thinlines
	\drawline(60,123)(250,123)\put(42,121){\makebox(15,4)[ r ]{\tiny 0}}
	\put(28,126){$W$}
	\put(215,152){$\alpha = 10$}

	\put(42,166){\makebox(15,4)[ r ]{\tiny -1}}
	\put(42,244){\makebox(15,4)[ r ]{\tiny 1}}
	\drawline(60,207)(250,207)\put(42,205){\makebox(15,4)[ r ]{\tiny 0}}
	%\thicklines \input{fig/baseflow/base-03.tripath} \thinlines
		\thicklines
		\drawline(60,243.684)(97.3525,237.751)(150.246,228.178)(211.727,215.415)(250,207)
		\drawline(60,170.411)(88.0143,174.736)(110.644,178.471)(136.204,183.175)(182.486,192.44)(217.385,199.909)(250,207)
		\drawline(60,200.524)(80.0735,203.005)(101.817,205.227)(126.73,207.26)(152.461,208.418)(171.103,209.104)(190.529,209.198)(216.09,208.844)(250,207)
		\drawline(60,176.698)(116.267,187.145)(139.647,191.4)(163.912,195.3)(183.509,198.467)(203.548,201.304)(233.743,205.227)(250,207)
		\drawline(60,228.367)(80.7211,227.918)(110.03,226.5)(134.671,224.467)(166.639,221.087)(197.039,216.644)(220.759,212.578)(250,207)
		\thinlines
	\put(28,210){$W$}
	\put(220.5,236){$\alpha = 3$}

	\put(120,20){\tiny $\tau=-\pi/2$}
	\put(120,64){\tiny $\tau=\pi/2$}
	\put(120,92){\tiny $\tau=-\pi/2$}
	\put(120,150){\tiny $\tau=\pi/2$}
	\put(120,175){\tiny $\tau=-\pi/2$}
	\put(120,235){\tiny $\tau=\pi/2$}

	\end{picture}
	\caption{\label{fig:baseflow}Die oszillierende Grundstr\"omung $W(r;\tau)$ zu den Phasen $\tau/\pi = \{-1/2,-1/4,0,1/4,1/2\}$ f\"ur verschiedene Werte der Womersleyschen Zahl $\alpha$. Die Zuordnung der Zeitwerte erfolgt von unten nach oben gem\"a\ss\ (\mbox{\ref{eq:baseflow-barV}}). Die Geschwindigkeitsverteilung in der zweiten H\"alfte der Periode, kann aus den angegebenen Kurven durch Spiegelung an der Horizontalen $W=0$ gewonnen werden.}
	\end{center}
\end{figure}

%% Stabilitaetstheorie
\chapter{Einf\"uhrung in die Stabilit\"atstheorie}\label{sec:stabilitaetstheorie}
\pagestyle{main}

%% Einleitung (Stabilitaetstheorie)
%\section{Einleitung}\label{sec:stabil-intro}
Die Frage der Stabilit\"at der Grundstr\"o"-mung gegen\"uber St\"or"-ungen beantwortet die Dynamik der Evolutionsgleichung der St\"orungen.\\
Wir bezeichnen die Relativgeschwindigkeit zwischen der ge"-st\"or"-ten Geschwindigkeit und der Grundgeschwindigkeit mit $\mathfrak{v}$, die entsprechende Druckdifferenz mit $p$.
Sowohl die Grundstr\"o"-mung $\mathfrak{v}_0,\,p_0$ als auch die ge"-st\"or"-te Str\"omung $\mathfrak{v}_0 + \mathfrak{v},\,p_0+p$ gehorchen den \textit{Navier-Stokes-Glei"-chung"-en}.
Substrahieren wir von den ge"-st\"or"-ten Glei"-chung"-en die unge"-st\"or"-ten Glei"-chung"-en, erhalten wir die Evolutionsgleichnungen der St\"or"-ungsgeschwindigkeit $\mathfrak{v}$
\begin{equation}\label{eq:stoerevolution}
	\begin{split}
	  \frac{\partial{\mathfrak{v}}}{\partial t} &+ (\mathfrak{v}_0 \,\mathrm{grad}) \mathfrak{v} + (\mathfrak{v} \,\mathrm{grad}) \mathfrak{v}_0 + \,\mathrm{grad}\,p \\
	 	&= - (1/R) \,\mathrm{rot}\,\,\mathrm{rot}\,\mathfrak{v}
	\end{split}
\end{equation}
mit der Nebenbedingung der inkompressiblen Fl\"ussigkeit
\begin{equation}\label{eq:konti}
	\mathrm{div}\,\mathfrak{v} = 0.
\end{equation}
Als Randbedingung fordern wir das Haften der Fl\"ussigkeit an der festen Begrenzung des Str\"omungsraumes
\begin{equation}\label{eq:haft}
	\quad\mathfrak{v} = 0
\end{equation}
und im Falle der Rohrstr\"omung unter Annahme axial periodischer St\"orungen.\\
Die Evolutionsgleichung ist linearisiert unter der Voraussetzung, da\ss\ die St\"or"-ungsgeschwindigkeiten klein sind gegen\"uber der Geschwindigkeit der Grundstr\"o"-mung.
Im allgemeinen h\"atten wir auf der linken Seite von (\mbox{\ref{eq:stoerevolution}}) den zus\"atzlichen nichtlinearen Term 
\begin{equation}
	(\mathfrak{v} \,\mathrm{grad})\mathfrak{v}
\end{equation}
zu ber\"ucksichtigen.\\
Aus dieser relativen Impulsbilanz bekommen wir durch Multiplikation mit $\mathfrak{v}$ und Integration \"uber den Str\"o"-mungsraum $\mathcal{B}$ die Bilanz der \textit{relativen St\"or"-ungsenergie} $e = \frac{1}{2}|\mathfrak{v}|^2$
\begin{equation}\label{eq:energy}
	\int  \frac{\partial{e}}{\partial t} \, \mathrm{d} \mathcal{B} = -\int \left( \mathfrak{v} \cdot D_0 \mathfrak{v} + (1/R) | \,\mathrm{rot}\,\mathfrak{v}|^2 \right) \mathrm{d} \mathcal{B}
\end{equation}
mit dem Verzerrungsgeschwindigkeitstensor der Grundstr\"o"-mung $D_0$.
In axialer Richtung erstreckt sich die Integration \"uber eine Wellenl\"ange der St\"orungen.
Die bei der partiellen Integration auftretenden Randintegrale verschwinden unter Ber\"ucksichtigung der Randbedingungen.\\

Die Gleichung ist bekannt unter dem Namen \textsl{Rey"-nolds-Orr-Energieglei"-chung}.\footnote{\label{bib:reynolds_criterion:2}\textsl{Osborne Reynolds}: \textit{Phil.\ Trans.} \textbf{186} (1895)\\\bibspace\label{bib:orr:2}\textsl{Orr}: \textit{Proc.\ Roy.\ Irish Acad.} \textbf{XXVII} (1907)}
Stabilit\"atsbetrachtungen, die auf dieser Gleichung aufbauen, bezeichnet man als \textit{Energiemethode}.
Sie vergleicht die Terme auf der rechten Seite.
Den ersten Term k\"onnen wir als Energietransfer pro Zeit von der Grundstr\"o"-mung in die St\"or"-ung betrachten.
Der zweite Term beinhaltet die Reibungskr\"afte und ist bei newtonschen Fl\"ussigkeiten stets dissipierend.\\
Wir werden die Energiemethode nicht weiter verfolgen, doch zwei wichtige Eigenschaften wollen wir (\mbox{\ref{eq:energy}}) entnehmen.\\
Zum einen wird der Mechanismus, Energie von der Grundstr\"o"-mung in die St\"or"-ungen zu transportieren, durch den Term $\mathfrak{v} D \mathfrak{v}$ beschrieben, welcher in zylindrischen Koordinaten f\"ur das betrachtete Problem keine Abh\"angigkeit der Winkelkomponente $\mathfrak{v}\cdot\mathfrak{e}_\varphi$ aufweist 
\begin{equation}\label{eq:interaktion}
	 (\mathfrak{v}\cdot\mathfrak{e}_r) (\mathfrak{v}\cdot\mathfrak{e}_z)  \frac{\partial{W}}{\partial r} .
\end{equation}
Zum anderen ist die Energiegleichung dieselbe, w\"urde man sie aus der nichtlinearen Form von (\mbox{\ref{eq:stoerevolution}}) herleiten.
Nichtlineare Effekte spielen beim prim\"aren Energietransfer von der Grundstr\"o"-mung in die St\"or"-ungen keine Rolle.\\


Analog zur Orr-Sommerfeldschen Theorie untersuchen wir die Gleichungen (\mbox{\ref{eq:stoerevolution}}) bis (\mbox{\ref{eq:haft}}) direkt.
In zylindrischen Koordinaten stellen wir den Vektor der Geschwindigkeitsst\"orung durch
\begin{equation}
	\mathfrak{v} = u \mathfrak{e}_r + v \mathfrak{e}_\varphi + w \mathfrak{e}_z
\end{equation}
dar.
Hiermit lautet die Koordinatendarstellung der Evolutionsgleichungen (\mbox{\ref{eq:stoerevolution}})
\begin{equation}\label{eq:evo:koord}
	\begin{split}
	& R  \frac{\partial{u}}{\partial t} + R W  \frac{\partial{u}}{\partial z} + R \frac{\partial{p}}{\partial r} \\
		&\qquad   =\frac{\partial^{2}{u}}{\partial r^2} + \frac{1}{r}  \frac{\partial{u}}{\partial r} + \frac{1}{r^2}  \frac{\partial^{2}{u}}{\partial \varphi^2} +  \frac{\partial^{2}{u}}{\partial z^2} -\frac{u}{r^2} - \frac{2}{r^2}  \frac{\partial{v}}{\partial \varphi} \\
	& R  \frac{\partial{v}}{\partial t} + R W  \frac{\partial{v}}{\partial z} + \frac{R}{r}  \frac{\partial{p}}{\partial \varphi} \\
		&\qquad  =\frac{\partial^{2}{v}}{\partial r^2} + \frac{1}{r}  \frac{\partial{v}}{\partial r} + \frac{1}{r^2}  \frac{\partial^{2}{v}}{\partial \varphi^2} +  \frac{\partial^{2}{v}}{\partial z^2} -\frac{v}{r^2} + \frac{2}{r^2}  \frac{\partial{u}}{\partial \varphi} \\
	& R  \frac{\partial{w}}{\partial t} + R W  \frac{\partial{w}}{\partial z} + R  \frac{\partial{W}}{\partial r} u + R \frac{\partial{p}}{\partial z} \\
		&\qquad  =\frac{\partial^{2}{w}}{\partial r^2} + \frac{1}{r}  \frac{\partial{w}}{\partial r} + \frac{1}{r^2}  \frac{\partial^{2}{w}}{\partial \varphi^2} +  \frac{\partial^{2}{w}}{\partial z^2}
	\end{split}	
\end{equation}
unter der Nebenbedingung
\begin{equation}
	\frac{\partial{u}}{\partial r} + \frac{u}{r} + \frac{1}{r} \frac{\partial{v}}{\partial \varphi} +  \frac{\partial{w}}{\partial z} = 0.\\
\end{equation}

Aufgrund der Linearit\"at k\"onnen wir die St\"or"-ungen in modale Komponenten zerlegen und aus dem Einzelverhalten der Komponenten auf die Gesamtheit aller m\"oglichen St\"or"-ungen schlie\ss en.
Wir spalten die Axial- und Winkelabh\"angigkeiten des St\"or"-ungsgeschwindgkeitsvektors ab
\begin{equation}\label{eq:separation}
	\mathfrak{v} = \bar{\mathfrak{v}}(r,t)\, e^{i(kz + n\varphi)}
\end{equation}
und nennen $k$ die \textit{axiale Wellenzahl} und $n$ die \textit{Winkelzahl} der St\"or"-ung.
Da der Str\"omungsraum in axialer Richtung unbegrenzt ist, handelt es sich bei der Wellenzahl um eine \textit{reelle} Zahl $k\in\mathbb{R}$.
Die Winkelzahl hingegen ist eine nat\"urliche Zahl $n\in\mathbb{N}$, da das Gebiet in azimuthaler Richtung $2\pi$-periodisch ist.\\
F\"ur alle St\"or"-ungen, die sich durch (\mbox{\ref{eq:separation}}) mit $n=0$ ausdr\"ucken lassen, sprechen wir von \textit{axialsymmetrischen} oder kurz \textit{symmetrischen} St\"or"-ungen.
Die Moden mit $n=1$ bezeichnen wir als \textit{antimetrisch}.\\%, alle h\"oheren Winkelabh\"angigkeiten bezeichnen wir als $unsymmetrisch$.\\

Im folgenden unterscheiden wir zwei Herangehensweisen an das instation\"are Problem: die \textit{quasistatische Methode} und die \textit{Floquet-\!Theorie}.
Erstere gilt nur eingeschr\"ankt und trifft Aussagen \"uber die kurzzeitige Entwicklung von St\"or"-ungen.
Letztere gilt f\"ur einen beliebig langen Zeitraum.
Aus ihr folgt ein asymptotischer Stabilit\"atsbegriff.

%% Quasistatische Methode
\chapter{Quasistatische Methode}\label{sec:quasistatik}
Die quasistatische Methode beruht auf der Annahme, da\ss\ charakteristische Zeitkonstanten der St\"or"-ungen kurz gegen\"uber der Zeit sind, in der sich die Grundgeschwindigkeit \"andert.
Wir bezeichnen den Quotienten zwischen den Zeitkonstanten der St\"orungen und der Grundstr\"omung als $\varepsilon$ und fordern, da\ss\ dieser ein kleiner Parameter sei.
Eine genaue Definition des Quasistatikparameters $\varepsilon$ geben wir an sp\"aterer Stelle an.
Die Forderung ist erf\"ullt, falls die Grundgeschwindigkeit sich langsam gegen\"uber den St\"orungsgeschwindigkeiten \"andert.
Allerdings verlangen wir von der Grundgeschwindigkeit, da\ss\ sie sich be"-z\"ug"-lich der diffusiven Zeitskala $L^2/\nu$ schnell genug \"andert, damit sich das Grundstr\"o"-mungsprofil infolge der Impulskr\"afte von der station\"aren Hagen-Poiseuilleschen Parabel unterscheidet.
Die Betrachtungen, f\"ur die die quasistatische Methode angebracht ist, wird durch die Ungleichungen
\begin{equation}\label{eq:quasisteady}
	\varepsilon \ll 1 \ll \alpha^2
\end{equation}
beschrieben.
Der linke Teil tr\"agt der Forderung nach schnellen St\"or"-ungen Rechnung, der rechte ber\"ucksichtigt dynamische Str\"o"-mungs"-pro"-file.\\

Mit der Annahme (\mbox{\ref{eq:quasisteady}}) betrachten wir die kurzzeitige Entwicklung der St\"or"-ungsgeschwindigkeit.
Dadurch hat das partielle Differentialgleichungssystem zeitlich \textit{konstante Koeffizienten} und die Zeit"-ab"-h\"ang"-ig"-keit aus (\mbox{\ref{eq:separation}}) kann separiert werden
\begin{equation}\label{eq:zeitansatz}
	\bar{\mathfrak{v}}(r,t) = \mathfrak{v}^{*}(r) e^{(\sigma + i\omega)t}.
\end{equation}
Hierbei sind $\sigma$ die Aufklingrate und $\omega$ die Kreisfrequenz der betrachteten St\"or"-ungskomponente.
Es sind die gesuchten Gr\"o\ss en, die eine Aussage \"uber die Stabilit\"at der Grundstr\"o"-mung liefern.
Es gilt
\begin{equation}\label{eq:quasistatisch:kriterium}
	\begin{split}
	\sigma &< 0:\quad\textrm{\textit{quasistatisch stabil}}\\
	\sigma &> 0:\quad\textrm{\textit{quasistatisch instabil}}.
	\end{split}
\end{equation}
Mit Hilfe der Aufklingrate und Frequenz einer St\"orungskomponente definieren wir den Quasistatikparameter als den Quotienten zwischen der Oszillationsfrequenz der Grundstr\"omung und dem Betrag der komplexen Frequenz der St\"orung.
In dimensionsloser Form lautet der Parameter
\begin{equation}\label{eq:epsilon}
	\varepsilon = \frac{\alpha^2}{R|\sigma+i\omega|}.
\end{equation}
Da die zeitliche Entwicklung der St\"orungen und damit die Parameter $\sigma$ und $\omega$ von vorneherein nicht bekannt sind, kann die Forderung $\varepsilon\ll1$ nicht sofort \"uberpr\"uft werden.
Deshalb setzen wir sie a priori als erf\"ullt voraus und entscheiden nach der L\"osung, ob dies zul\"assig war.\\

Indem wir (\mbox{\ref{eq:zeitansatz}}) und (\mbox{\ref{eq:separation}}) in (\mbox{\ref{eq:stoerevolution}}) und (\mbox{\ref{eq:konti}}) einsetzen, reduzieren wir die partiellen Differentialgleichungen auf ein System gew\"ohnlicher Differentialgleichungen in $r$.
Wir benutzen die Komponentendarstellung des radialen Anteils des St\"orungsgeschwindigkeitsvektors zweckm\"a\ss igerweise in der Form
\begin{equation}\label{eq:eigenfunktion}
	\mathfrak{v}^*(r) = u^* \mathfrak{e}_r + iv^* \mathfrak{e}_\varphi - iw^* \mathfrak{e}_z
\end{equation}
und erhalten
\begin{equation}\label{eq:rwp}
        \begin{aligned}
        \frac{\mathrm{d} u^*}{\mathrm{d} r} + \frac{u^*}{r} - \frac{nv^*}{r} + kw^* &=&  0 \\
        \frac{\mathrm{d} ^2 u^*}{\mathrm{d} r^2} + \frac{1}{r}\frac{\mathrm{d} u^*}{\mathrm{d} r} + \left( b - \frac{1+n^2}{r^2}\right)u^* + \frac{2nv^*}{r^2} -R\frac{\mathrm{d} p^*}{\mathrm{d} r}&=& 0 \\
        \frac{\mathrm{d} ^2 v^*}{\mathrm{d} r^2} + \frac{1}{r}\frac{\mathrm{d} v^*}{\mathrm{d} r} + \left( b - \frac{1+n^2}{r^2}\right)v^* + \frac{2nu^*}{r^2} -\frac{nRp^*}{r}&=& 0 \\
        \frac{\mathrm{d} ^2 w^*}{\mathrm{d} r^2} + \frac{1}{r}\frac{\mathrm{d} w^*}{\mathrm{d} r} + \left( b - \frac{n^2}{r^2}\right)w^* -iR\frac{\mathrm{d} W}{\mathrm{d} r} u^* + kRp^* &=& 0\\
        \end{aligned}
\end{equation}
mit der Abk\"urzung $ b \equiv -R[\sigma + i(\omega + kW)] - k^2 $ und den Randbedingungen
\begin{equation}\label{eq:rwp:rand}
	\begin{split}
	u^* = v^* = w^* = 0 &\qquad\textrm{f\"ur } r=1\\
	u^*,\,v^*,\,w^*: \textrm{regul\"ar}&\qquad\textrm{f\"ur } r=0.
	\end{split}
\end{equation}

Die Aufgabe besteht darin, f\"ur jede Kombination der Parameter der Grundstr\"o"-mung $\alpha$ und $R$ sowie der Parameter der St\"or"-ungsgeschwindigkeit $n$ und $k$ zu jedem Zeitpunkt $\tau$ das Randwertproblem zu l\"osen, indem passende Eigenwerte $\sigma+i\omega$ gefunden werden
\begin{equation}\label{eq:parameter}
	(R,\alpha,\tau,k,n) \to (\sigma+i\omega).
\end{equation}
Zu jeder Kombination der Parameter existieren unendlich viele Eigenwerte.
F\"ur die Frage der Instabilit\"at gen\"ugt es, den Eigenwert mit dem gr\"o\ss ten Realteil $\sigma$ zu finden.\\

Im folgenden werden zwei Methoden zur L\"osung des Rand-Ei"-gen"-wert"-pro"-blems vorgestellt.
Beim \textit{Schie\ss verfahren} werden Eigenwerte angenommen und das Randwertproblem von innen nach au\ss en integriert.
Anhand der Randbedingung am \"au\ss eren Rand, die im allgemeinen nicht erf\"ullbar ist, k\"onnen wir ein G\"utema\ss\ formulieren, welches die Wahl der Eigenwerte beurteilt.
Mit einem geeigneten Optimierungsverfahren wird der Vorgang wiederholt, bis das G\"utema\ss\ einen zufriedenstellenden Wert liefert.\\
Die Alternative ist eine \textit{Galerkin-Entwicklung}, bei der die Radialfunktionen durch eine Summe von Ansatzfunktionen angen\"ahert werden.
Die L\"osung wird anschlie\ss end im verkleinerten L\"osungsraum gesucht.
Statt eines Rand-Ei"-gen"-wert"-pro"-blems mu\ss\ dadurch nur noch ein algebraisches Ei"-gen"-wert"-pro"-blem gel\"ost werden.
\newpage

%% Schiessverfahren
\section{Schie\ss verfahren}\label{sec:schiessverfahren}
Beim Schie\ss verfahren geben wir Sch\"atzungen der gesuchten Eigenwerte $\sigma+i\omega$ vor und integrieren das gew\"ohnliche Randwertproblem (\mbox{\ref{eq:rwp}}) \"uber die radiale Koordinate $r$.
Die \"au\ss ere Randbedingung (\mbox{\ref{eq:rwp:rand}}) wird dadurch im allgemeinen nicht erf\"ullt sein.
Aus ihr leiten wir eine Zielbedingung her, welche uns zu entscheiden hilft, ob die Sch\"atzungen mit den Eigenwerten \"ubereinstimmen.
Dieser Vorgang wird mittles eines Optimierungsverfahrens wiederholt, bis die Zielbedingung erf\"ullt ist.\\

%Im Prinzip m\"u\ss te die Optimierung f\"ur jede Kombination aus dem f\"unfdimensionalen Parameterraum (\mbox{\ref{eq:parameter}}) durchgef\"uhrt werden.
Das Verfahren konvergiert nur dann gegen den gesuchten instabilsten Eigenwert, falls wir einen \glqq guten\grqq\ Startwert w\"ahlen.
Dies gelingt unter der Annhame, da\ss\ sich die Eigenwerte stetig mit kleinen \"Anderungen der Parameter \"andern.
Kennen wir eine L\"osung, k\"onnen wir daraus auf den Startwert f\"ur das n\"achste Optimierungsproblem mit geringf\"ugig ge\"anderten Parametern schlie\ss en und uns somit sequentiell im Parameterraum vorantasten.
Die Startwerte f\"ur den ersten Schritt sind bekannt.
Im Grenzfall langwelliger St\"or"-ungen $k\to0$ kann das Ei"-gen"-wert"-pro"-blem analytisch gel\"ost werden.\footnote{\label{bib:gill2}\textsl{Gill}: \textit{J.\ Fluid.\ Mech.} \textbf{61} (1973)}
Die Eigenwerte sind in Tabelle \mbox{\ref{tab:start}} angegeben.
Alle Eigenwerte sind negativ reell, daher sind langwellige St\"or"-ungen unabh\"angig von der Grundstr\"o"-mung immer aperiodisch ged\"ampft.\\

\begin{table}[htbp]	% Tabelle Startwerte
	\begin{center}
	\caption{\label{tab:start}Am schw\"achsten ged\"ampfte Eigenwerte im Grenzfall langwelliger St\"or"-ungen $k\to0$ f\"ur eine beliebige Grundstr\"o"-mung.  Die axialsymmetrischen St\"or"-ungen entkoppeln die $r,z$ und die (stets stabile) $\varphi$ Komponente.
	Es bezeichnet $j_{n,m}$ die $m$-te Nullstelle gr\"o\ss er null der $n$-ten Besselschen Funktion erster Art.}
	\begin{tabular}{l|cl}
        $n$     & $-R\sigma$ & \\\hline
        $0$ ($r,z$)     & $j_{2,1}^2$ & 26.37\\
        $0$ ($\varphi$) & $j_{1,1}^2$ & 14.68\\
        $1$     & $j_{1,1}^2$ & 14.68\\
        $n>1$   & $j_{n,1}^2$ & gr\"o\ss er\\
	\end{tabular}
	\end{center}
\end{table}

Im folgenden stellen wir eine Methode zur Integration des Randwertproblems dar.\\
Im Falle symmetrischer St\"orungen ($n=0$) zerf\"allt das System (\mbox{\ref{eq:rwp}}) in zwei Teilsysteme.
Eines ent"-h\"alt die Azimuthalkomponente der St\"or"-ungsgeschwindigkeit, das andere ist eine Kopplung aus den radialen und axialen Anteilen.\\
Alle St\"or"-ungen der Azimuthalkomponente sind in der linearen Betrachtung ged\"ampft.
Da im Interaktionsterm (\mbox{\ref{eq:interaktion}}), welcher den Energietransport von der Grundstr\"o"-mung in die St\"or"-ung ausdr\"uckt, die azimuthale St\"or"-ungsgeschwindigkeitskomponente nicht auftaucht, k\"onnen diese St\"or"-ungsanteile nicht angefacht werden.
Wir werden das erste Teilsystem deshalb nicht weiter behandeln.\\

Das zweite Teilsystem \"uberf\"uhren wir in die Standardform
\begin{equation}\label{eq:shoot:system}
	\frac{\mathrm{d} \mathfrak{y}(s)}{\mathrm{d} s} = A(s) \mathfrak{y}(s).
\end{equation}
Dies gelingt durch die Wahl des komplexen Vektors $\mathfrak{y}$ mit den Komponenten
\begin{equation}
	\begin{split}
	\mathfrak{y}^{(1)}(s) = \frac{u^*(r)}{r} \qquad &\mathfrak{y}^{(2)}(s) = \frac{\mathrm{d} u^*}{\mathrm{d} r} + \frac{u^*}{r} - Rp^*\\
	\mathfrak{y}^{(3)}(s) = w^*(r) \qquad & \mathfrak{y}^{(4)}(s) = \frac{1}{2r} \frac{\mathrm{d} w^*}{\mathrm{d} r}.
	\end{split}
\end{equation}
Hierbei nutzen wir die Symmetrie aus, indem wir die Variable $s = r^2$ einf\"uhren.
Die Systemmatrix folgt zu
\begin{equation}
	A = \left(
	\begin{array}{cccc}
		-1/s & 0 & -k/2s & 0\\
		-b/2 & 0 & 0 & 0\\
		0 & 0 & 0 & 1\\
		iR\mathscr{W}'/2 & k/4s & (k^2-b)/4s & -1/s
	\end{array}
	\right)
\end{equation}
mit der Darstellung der Grundstr\"omung $\mathscr{W}(s)$.\\

Das System ist von vierter Ordnung und be"-n\"o"-tigt daher f\"ur die Integration des Schie\ss verfahrens vier Randbedingungen auf der Rohrachse $s=0$.
Wir besitzen lediglich die Vorgabe, da\ss\ alle physikalischen Gr\"o\ss en regul\"ar bleiben sollen.
Diesen Forderungen m\"ussen wir mit besonderen Einschr\"ankungen in den Randbedingungen nachkommen.\\
Wir erreichen dies, indem wir alle abh\"angigen Variablen in eine regul\"are Taylorreihe um $r=0$ entwickeln, und diese in das Differentialgleichungssystem einsetzen.
Dadurch bekommen wir ein lineares Gleichungssystem f\"ur die niedrigsten Koeffizienten.
Das Gleichungssystem ist allerdings nicht vollst\"andig l\"osbar.
Es verbleiben zwei Parameter, welche auf linear unabh\"angige L\"osungen f\"uhren.
Wir k\"onnen eine Integralbasis konstruieren, indem wir wechselweise einen der Parameter \textit{null} setzen und das System integrieren.\\
Die mit dieser Methode gewonnenen Randbedingungen f\"ur die erste Basisl\"osung lauten
\begin{equation}\label{eq:shoot:ab1}
	\mathfrak{y}_1(0) = \left( \begin{array}{c}
		0 \\
		1 \\
		0 \\
		k/4
	\end{array} \right).
\end{equation}
Au\ss erdem ben\"otigen wir die dazugeh\"origen Ableitungen an der Stelle der Singularit\"at
\begin{equation}\label{eq:shoot:sin1}
	\left.\frac{\mathrm{d} \mathfrak{y}_1}{\mathrm{d} s}\right|_{s=0} = \left( \begin{array}{c}
		-k^2/16 \\
		0 \\
		k/4 \\
		k(k^2-b)/32
	\end{array} \right).
\end{equation}
F\"ur die zweite Basisl\"osung erhalten wir die Randbedingungen entsprechend
\begin{equation}\label{eq:shoot:ab2}
	\mathfrak{y}_2(0) = \left( \begin{array}{c}
		-k/2 \\
		0 \\
		1 \\
		(k^2-b)/4
	\end{array} \right)
\end{equation}
\begin{equation}\label{eq:shoot:sin2}
	\left.\frac{\mathrm{d} \mathfrak{y}_2}{\mathrm{d} s}\right|_{s=0} = \left( \begin{array}{c}
		k(b-k^2)/16 \\
		kb/4 \\
		(k^2-b)/4 \\
		(k^4-bk^2+b^2)/32-ikR\mathscr{W}'(0)/8
	\end{array} \right).
\end{equation}
Die Integration des Systems (\mbox{\ref{eq:shoot:system}}) mit den Randbedingungen (\mbox{\ref{eq:shoot:ab1}}) und (\mbox{\ref{eq:shoot:sin1}}) von $s=0$ bis $s=1$ f\"uhrt auf die erste Basisl\"osung $\mathfrak{y}_1$.
Entsprechend f\"uhrt die Integration mit den Randbedingungen (\mbox{\ref{eq:shoot:ab2}}) und (\mbox{\ref{eq:shoot:sin2}}) auf die zweite Basisl\"osung $\mathfrak{y}_2$.
Die \"ubrigen zwei unabh\"angigen L\"osungen des Systems vierter Ordnung sind auf der Rohrachse singul\"ar und daher aus physikalischen Gr\"unden ausgeschlossen.\\

F\"ur das Schie\ss verfahren ben\"otigen wir eine Zielbedingung am \"au\ss eren Rand, um die G\"ute der Sch\"atzung des Eigenwertes $\sigma+i\omega$ zu beurteilen.
Da es sich bei $\mathfrak{y}_1$ und $\mathfrak{y}_2$ um ein Fundamentalsystem handelt, erhalten wir die allgemeine L\"osung aus einer Linearkombination beider Basisl\"osungen.
Daher lauten die Geschwindigkeitskomponenten an der Rohrwand
\begin{equation}
	\begin{split}
	u^*(1) &= C_1 \mathfrak{y}_1^{(1)}(1) + C_2 \mathfrak{y}_2^{(1)}(1) \\
	w^*(1) &= C_1 \mathfrak{y}_1^{(3)}(1) + C_2 \mathfrak{y}_2^{(3)}(1),
	\end{split}
\end{equation}
welche laut (\mbox{\ref{eq:rwp:rand}}) verschwinden sollen.
Dies gelingt nur dann auf nichttriviale Weise, wenn die Determinatenbedingung
\begin{equation}\label{eq:target}
        |\mathscr{D}| = 0\qquad \mathscr{D} = \left|
                \begin{array}{ccc}
                        \mathfrak{y}_1^{(1)}(1) & \mathfrak{y}_2^{(1)}(1) \\ 
                        \mathfrak{y}_1^{(3)}(1) & \mathfrak{y}_2^{(3)}(1)
                \end{array}
        \right|
\end{equation}
erf\"ullt ist.
Diese skalare Gleichung fassen wir als G\"utefunktion auf.
Das Optimierungsverfahren hat die Aufgabe unter Variation der Sch\"atzungen der Eigenwerte, (\mbox{\ref{eq:target}}) zu minimieren.\\
In der Implementation benutzen wir dazu das Verfahren nach \textsl{Hooke und Jeeves}.\footnote{\texttt{www.netlib.org/opt/hooke.c}\label{bib:hooke_jeeves}\\\bibspace\textsl{Hooke, Jeeves}: \textit{J.\ ACM} \textbf{8} (1961)}\\

F\"ur jede Kombination aus den Parametern $\alpha$, $R$, $\tau$ und $k$ konstruieren wir eine Integralbasis f\"ur (\mbox{\ref{eq:shoot:system}}), indem wir das System numerisch integrieren und den Eigenwert $\sigma+i\omega$ nach der Vorschrift
\begin{equation}
	\min_{\sigma,\omega}|\mathscr{D}|
\end{equation}
optimieren.
Als Startwert dient nach Tabelle \mbox{\ref{tab:start}}
\begin{equation}\label{eq:startwert}
	k=0 \qquad R\sigma = -j_{2,1}^2.
\end{equation}
Der Startwert gilt unabh\"angig von der Form der Grundstr\"o"-mung und damit f\"ur alle Womersleyzahlen $\alpha$ und jede Zeit $\tau$.
Der n\"achste Schritt besteht darin, die Wellenzahl $k$ stufenweise zu erh\"ohen.
Als Startwerte f\"ur die Eigenwerte k\"onnen lineare Sch\"atzungen aus den vorhergehenden L\"osungen genommen werden.\\

Statt einer Minimierung gibt es eine alternative Methode zur L\"osung von (\mbox{\ref{eq:target}}).
In der Arbeit von \textsl{Mack}\footnote{\label{bib:mack}\textsl{Mack}: \textit{J.\ Fluid.\ Mech.} \textbf{73} (1976)}
wird ein \"ahnliches Problem graphisch gel\"ost, indem die Kurven der Gleichungen
\begin{equation}
	\Re\mathscr{D} = 0\qquad \Im\mathscr{D} = 0
\end{equation}
erzeugt werden und die Nullstellen $|\mathscr{D}|=0$ als Schnittpunkte dieser Kurven identifiziert werden.
Das Verfahren ist zuverl\"assiger als eine Minimierung.
Es l\"a\ss t sich jedoch schwer automatisieren und wird daher nicht verwendet.\\

F\"ur h\"ohere Winkelzahlen $n>0$ ist eine analoge Vorgehensweise zu dem beschriebenen Schie\ss verfahren m\"oglich.
Dazu empfielt es sich, das System zu transformieren, indem neue Variable als Produkte aus $r^{1-n}$ und den alten Variablen, eingef\"uhrt werden.
Man erh\"alt ein System sechster Ordnung mit drei unabh\"angigen nichtsingul\"aren Basisl\"os"-ungen.
Es wird an dieser Stelle auf eine Darstellung des Systems verzichtet, da sie zum einen umfangreich ist, zum anderen werden wir im n\"achsten Abschnitt eine Methode kennenlernen, welche sich als vorteilhafter erweist.\\

Zu den Nachteilen des Schie\ss verfahrens geh\"ort die fehlende Robustheit des Optimierungsverfahrens.
Es ist schwer zu entscheiden, ob es sich bei einem Minimum tats\"achlich um eine Nullstelle handelt, und ob diese global den gr\"o\ss ten Realteil besitzt.
Im Grenzfall langwelliger St\"orungen sind die gesuchten Eigenwerte bekannt.
Nach dem beschriebenen Verfahren wird angenommen, da\ss\ der Eigenwert mit maximaler Aufklingrate im Grenzfall aus dem ersten Startwert hervorgeht.
Es ist jedoch auch denkbar, da\ss\ bei Variation der Wellenzahl $k$ ein Eigenwert den gr\"o\ss ten Realteil besitzt, welcher f\"ur langwellige St\"orungen einem st\"arker ged\"ampften Eigenwert entspricht.
Um dies auszuschliessen, m\"u\ss te das Verfahren f\"ur mehrere Startwerte durchgef\"uhrt werden.\\
Die steigende Komplexit\"at des Systems f\"ur h\"ohere Winkelzahlen $n>0$ macht das Verfahren f\"ur nichtsymmetrische St\"orungen unbrauchbar.
Es mu\ss\ f\"ur jede Winkelzahl $n$ ein eigenes System einschlie\ss lich ihrer Anfangs- und Singularit\"atsbedingungen hergeleitet werden.
Au\ss erdem hat die Determinantenbedingung (\mbox{\ref{eq:target}}) eine um eins erh\"ohte Dimension, was eine erheblich schlechtere Konditionierung zur Folge hat.\\

Die Galerkin-Entwicklung, welche wir im folgenden Absch"-nitt behandeln, besitzt diese Unzul\"anglichkeiten nicht.
Eine genauere Gegen\"uberstellung der Methoden erfolgt an sp\"aterer Stelle.
\newpage

%% Galerkin-Entwicklung
\section{Galerkin-Entwicklung}\label{sec:galerkin}
In diesem Abschnitt wird eine alternative Methode zur L\"osung des quasistatischen Stabilit\"atsproblems dargestellt.
Die Berechnungsergebnisse der vorliegenden Arbeit basieren auf dieser Theorie.
Die Methode ist einer Arbeit von \textsl{Salwen und Grosch}\footnote{\label{bib:salwen_grosch:2}\textsl{Salwen, Grosch}: \textit{J.\ Fluid.\ Mech.} \textbf{54} (1972)} entnommen, welche sich dem station\"aren Problem widmet.\\
Im Gegensatz zu den Berechnungen von \textsl{Salwen und Grosch}, die ausschlie\ss lich die Stabilit\"at der Geschwindigkeitsverteilung $W=1-r^2$ untersuchen, behandeln wir die oszillierende Str\"om"-ung (\mbox{\ref{eq:baseflow}}).
Durch diese Erweiterung erh\"oht sich die Dimension des Parameterraums um zwei: die Womersleyzahl und die Zeit.\\

Nach der Galerkinschen Methode diskretisieren wir das partielle Differentialgleichungssystem, und m\"ussen folglich statt eines Rand-Eigenwertpro"-blems nur noch ein algebraisches Eigenwertproblem l\"osen.\\

Dazu entwickeln wir die St\"orungsgeschwindigkeiten der Gleichungen (\mbox{\ref{eq:stoerevolution}}) und (\mbox{\ref{eq:konti}}) in eine Reihe aus skalaren Zeitfunktionen $g_\mathbf{i}$ und vektoriellen Ortsfunktionen $\mathbf{v_i}$\footnote{Wir bezeichnen die Indizes in diesem Abschnitt mit $\mathbf{i}$ und $\mathbf{k}$, um eine Verwechlung mit den anderweitig verwendenten Buchstaben $i,k$ zu vermeiden.}
\begin{equation}\label{eq:galerkin:ansatz}
	\mathfrak{v}(r,\varphi,z,t) = \sum_{\mathbf{i}=1}^N g_\mathbf{i}(t) \mathbf{v_i}(r,\varphi,z).
\end{equation}
Die Ansatzfunktionen $\mathbf{v_i}$ sollen eine orthonormale Basis f\"ur die N\"aherungs"-l\"osung der partiellen Differentialgleichungen der St\"orungsentwicklung darstellen.
Wir fordern daher
\begin{equation}
	\langle \mathbf{v_i},  \mathbf{v_k} \rangle = \delta_{\mathbf{ik}}
\end{equation}
mit der Definition des \textit{inneren Produktes} zweier komplexer Vektorfunkionen $\mathbf{f}(r,\varphi,z)$ und $\mathbf{g}(r,\varphi,z)$
\begin{equation}\label{eq:skalarprodukt}
	\langle \mathbf{f},  \mathbf{g} \rangle \equiv \int_0^{2\pi/k}\!\!\!\int_0^{2\pi}\!\!\int_0^1 (\mathbf{f}^*\cdot\mathbf{g})\,r \,\textrm{d}r \,\textrm{d}\varphi \,\textrm{d}z.
\end{equation}
Hierbei bezeichnet $(\mathbf{f}^*\cdot\mathbf{g})$ das Skalarprodukt des komplex Konjugierten von $\mathbf{f}$ mit $\mathbf{g}$.
Die Integration erstreckt sich \"uber das Str\"omungsgebiet, beziehungsweise in axialer Richtung \"uber eine Wellenl\"ange.\\

Insbesondere die Anforderung bez\"uglich der Regularit\"at auf der Rohr"-achse im zylindrischen Koordinatensystem und die Tatsache, da\ss\ es sich um Vektorfunktionen handelt, macht die Wahl der Basisfunktionen zu einem schwierigen Unterfangen.
Eine m\"ogliche Wahl, mit der alle Schwierigkeiten auf nat\"urliche Weise gel\"ost sind, sind die Eigenfunktionen $\mathbf{\overline{v}}_\mathbf{i}$ des vereinfachten Systems
\begin{equation}\label{eq:stokessystem}
	\begin{split}
	\,\mathrm{div}\,\mathbf{v}_\mathbf{i} &= 0\\
	\lambda_\mathbf{i} \mathbf{v}_\mathbf{i} + \,\mathrm{grad}\,p_\mathbf{i} &= - (1/R) \,\mathrm{rot}\,\mathrm{rot}\,\mathbf{v}_\mathbf{i}\\
	\mathbf{v}_\mathbf{i}:\textrm{ regul\"ar }&\textrm{ f\"ur } r=0\\
	\mathbf{v}_\mathbf{i} = 0\quad&\textrm{ f\"ur } r=1
	\end{split}
\end{equation}
mit den zugeh\"origen Eigenwerten $\lambda_\mathbf{i}$.\\
Bei dem System handelt es sich um das Eigenwertproblem des \textsl{Stokesschen} Operators in Zylinderkoordinaten, dessen L\"osung von \textsl{Rummler} angegeben wird.\footnote{\label{bib:rummler}\textsl{Rummler}: \textit{ZAMM} \textbf{77(8)} (1997)}
Au\ss erdem k\"onnen wir das System als das Stabilit\"atsproblem f\"ur den Fall der ruhenden Fl\"us"-sigkeit interpretieren.
In diesem Falle denken wir uns die Zeitab"-h\"ang"-igkeit durch den Faktor $e^{\lambda t}$ separiert.\\

Die Eigenfunkionen des Systems (\mbox{\ref{eq:stokessystem}}) setzen sich analog zu (\mbox{\ref{eq:separation}}) aus einem radialen Anteil $\mathbf{\overline{v}_i}(r)$ und exponentiellen Axial- und Winkelabh\"angigkeiten zusammen
\begin{equation}
	\mathbf{v_i}(r,\varphi,z) = \mathbf{\overline{v}_i}(r) e^{i(kz+n\varphi)}.
\end{equation}
Der von der radialen Komponente abh\"angige Anteil lautet f\"ur Winkelzahlen $n>0$ und Wellenzahlen $k>0$
\begin{equation}\label{eq:ansatzfunktionen}
	\begin{split}
	\mathbf{\overline{v}}_\mathbf{i} & =\\
	& -i^n \frac{k}{2} \bigg( \frac{I_{n+1}(k)}{J_{n+1}(\beta_\mathbf{i})} J_{n+1}(\beta_\mathbf{i} r) + \frac{I_{n-1}(k)}{J_{n-1}(\beta_\mathbf{i})} J_{n-1}(\beta_\mathbf{i} r) \\
	&\qquad\qquad\qquad\qquad\qquad\quad - I_{n-1}(kr) - I_{n+1}(kr) \bigg) \mathfrak{e}_r\\
	&+i^{n+1} \frac{k}{2}\bigg( \frac{I_{n+1}(k)}{J_{n+1}(\beta_\mathbf{i})} J_{n+1}(\beta_\mathbf{i} r) - \frac{I_{n-1}(k)}{J_{n-1}(\beta_\mathbf{i})} J_{n-1}(\beta_\mathbf{i} r)\\
	&\qquad\qquad\qquad\qquad\qquad\quad +I_{n-1}(kr) - I_{n+1}(kr) \bigg) \mathfrak{e}_\varphi\\
	&+i^{n+1} \bigg(k I_n(kr) \\
	&\qquad\qquad\,\, -\frac{1}{2}\beta_\mathbf{i} \bigg[ \frac{I_{n+1}(k)}{J_{n+1}(\beta_\mathbf{i})} - \frac{I_{n-1}(k)}{J_{n-1}(\beta_\mathbf{i})} \bigg] J_n(\beta_\mathbf{i} r) \bigg) \mathfrak{e}_z.
	\end{split}
\end{equation}
$I_n$ bezeichnet die modifizierten Besselschen Funktionen $I_n(x)=i^{-n}J_n(ix)$.
Statt der Eigenwerte $\lambda_\mathbf{i}$ benutzen wir die Hilfsgr\"o\ss en $\beta_{\mathbf{i}}$, mit
\begin{equation}
	\beta_\mathbf{i}^2 = -(k^2+R\lambda_\mathbf{i})
\end{equation}
als L\"osung der charakteristischen Gleichung 
\begin{equation}
	\begin{split}
	&2k I_n(k) J_{n-1}(\beta) J_{n+1}(\beta) - \beta I_{n+1} (k) J_{n-1} (\beta) J_n(\beta)\\
	&\quad+ \beta I_{n-1} (k) J_{n} (\beta) J_{n+1}(\beta) = 0.
	\end{split}
\end{equation}
Die Gleichung hat rein reelle Wurzeln $\beta_\mathbf{i}>0$, welche numerisch ermittelt werden m\"ussen. 
Hierbei benutzen wir als Startwert den asymptotischen Grenzfall f\"ur $k\to 0$
\begin{equation}
	\beta J_n(\beta) J_{n+1}(\beta) = 0,
\end{equation}
mit den L\"osungen (geordnet in aufsteigender Ordnung)
\begin{equation}
	\beta = \{ j_{n,1};\, j_{n+1,1};\, j_{n,2};\, j_{n+1,2};\, \cdots \}.
\end{equation}
Es ist $j_{n,m}$ die $m$-te Nullstelle gr\"o\ss er null, der $n$-ten Besselschen Funktion erster Art.\\

Der symmetrische Anteil der Eigenfunktionen mit $n=0$ lautet
\begin{equation}
	\begin{split}
	\mathbf{\overline{v}}_\mathbf{i} =& \bigg( kI_1(kr) - k\frac{I_1(k)}{J_1(\beta_\mathbf{i})} J_1(\beta_\mathbf{i} r)\bigg) \mathfrak{e}_r\\
		&+i\bigg( k I_0(kr) - \beta_\mathbf{i} \frac{I_1(k)}{J_1(\beta)} J_0(\beta_\mathbf{i} r)\bigg) \mathfrak{e}_z
	\end{split}	
\end{equation}
mit $\beta_\mathbf{i}$ aus der Eigenwertgleichung
\begin{equation}
	k I_0(k) J_1(\beta) - \beta I_1(k) J_0(\beta) = 0,
\end{equation}
dessen rein reelle L\"osungen $\beta_\mathbf{i} > 0$ im Grenzfall langwelliger axialer Abh\"ang"-ig"-keit $k\to0$ in
\begin{equation}
	\beta_\mathbf{i} = j_{2,\mathbf{i}}
\end{equation}
\"ubergehen.\\

\begin{figure}[ htbp ] % Ansatzfunktionen
	\begin{center}
	\begin{picture}(275,290)

	\put(50,280){$\mathbf{i}=1$}\put(138,280){$\mathbf{i}=2$}\put(226,280){$\mathbf{i}=3$}

	\thicklines\drawline(60,267)(85,267)\put(90,265){$-\mathbf{\overline{v}}_\mathbf{i}\!\cdot\!\mathfrak{e}_r$}\thinlines
	\thinlines\drawline(170,267)(195,267)\put(200,265){$-i\,\mathbf{\overline{v}}_\mathbf{i}\!\cdot\!\mathfrak{e}_z$}

	\thicklines\drawline(40,177)(65,177)\put(70,175){$i\,\mathbf{\overline{v}}_\mathbf{i}\!\cdot\!\mathfrak{e}_r$}\thinlines
	\dottedline{2}(115,177)(140,177)\put(145,175){$-\mathbf{\overline{v}}_\mathbf{i}\!\cdot\!\mathfrak{e}_\varphi$}
	\thinlines\drawline(190,177)(215,177)\put(220,175){$-\mathbf{\overline{v}}_\mathbf{i}\!\cdot\!\mathfrak{e}_z$}

	\thicklines\drawline(40,87)(65,87)\put(70,85){$\mathbf{\overline{v}}_\mathbf{i}\!\cdot\!\mathfrak{e}_r$}\thinlines
	\dottedline{2}(115,87)(140,87)\put(145,85){$i\,\mathbf{\overline{v}}_\mathbf{i}\!\cdot\!\mathfrak{e}_\varphi$}
	\thinlines\drawline(190,87)(215,87)\put(220,85){$i\,\mathbf{\overline{v}}_\mathbf{i}\!\cdot\!\mathfrak{e}_z$}

	\put(0,265){$n=0$}
	\drawline(20,190)(20,260)(98,260)(98,190)(20,190)\drawline(20,225)(98,225)
	\drawline(108,190)(108,260)(186,260)(186,190)(108,190)\drawline(108,225)(186,225)
	\drawline(196,190)(196,260)(274,260)(274,190)(196,190)\drawline(274,225)(196,225)

	\put(0,175){$n=1$}
	\drawline(20,100)(20,170)(98,170)(98,100)(20,100)\drawline(20,135)(98,135)
	\drawline(108,100)(108,170)(186,170)(186,100)(108,100)\drawline(108,135)(186,135)
	\drawline(196,100)(196,170)(274,170)(274,100)(196,100)\drawline(196,135)(274,135)

	\put(0,85){$n=2$}
	\drawline(20,10)(20,80)(98,80)(98,10)(20,10)\drawline(20,45)(98,45)
	\drawline(108,10)(108,80)(186,80)(186,10)(108,10)\drawline(108,45)(186,45)
	\drawline(196,10)(196,80)(274,80)(274,10)(196,10)\drawline(196,45)(274,45)
	
	\put(19,3){\tiny 0}\put(96.5,3){\tiny 1}
	\put(107,3){\tiny 0}\put(184.5,3){\tiny 1}
	\put(195,3){\tiny 0}\put(272.5,3){\tiny 1}
	\put(57,0){$r$}\put(145,0){$r$}\put(233,0){$r$}
	\put(12,8){\tiny -3} \put(14,43){\tiny 0} \put(14,78){\tiny 3}
	\put(12,98){\tiny -3} \put(14,133){\tiny 0} \put(14,168){\tiny 3}
	\put(12,188){\tiny -6} \put(14,223){\tiny 0} \put(14,258){\tiny 6}

	%\input{/home/elmar/src/floquet/ansatz/P.0-1}
		\thicklines\drawline(20,225)(22.4375,223.996)(24.875,223.007)(27.3125,222.05)(29.75,221.141)(32.1875,220.292)(34.625,219.517)(37.0625,218.828)(39.5,218.234)(41.9375,217.743)(44.375,217.363)(46.8125,217.096)(49.25,216.944)(51.6875,216.906)(54.125,216.981)(56.5625,217.163)(59,217.445)(61.4375,217.819)(63.875,218.273)(66.3125,218.796)(68.75,219.373)(71.1875,219.991)(73.625,220.634)(76.0625,221.285)(78.5,221.929)(80.9375,222.549)(83.375,223.13)(85.8125,223.656)(88.25,224.113)(90.6875,224.487)(93.125,224.766)(95.5625,224.94)(98,225)\thinlines
		\thinlines\drawline(20,246.485)(22.4375,246.371)(24.875,246.03)(27.3125,245.468)(29.75,244.696)(32.1875,243.726)(34.625,242.576)(37.0625,241.265)(39.5,239.816)(41.9375,238.255)(44.375,236.607)(46.8125,234.902)(49.25,233.167)(51.6875,231.433)(54.125,229.729)(56.5625,228.083)(59,226.522)(61.4375,225.071)(63.875,223.753)(66.3125,222.59)(68.75,221.599)(71.1875,220.794)(73.625,220.187)(76.0625,219.786)(78.5,219.594)(80.9375,219.612)(83.375,219.839)(85.8125,220.266)(88.25,220.885)(90.6875,221.683)(93.125,222.647)(95.5625,223.759)(98,225)\thinlines
	%\input{/home/elmar/src/floquet/ansatz/P.0-2}
		\thicklines\drawline(108,225)(110.438,226.294)(112.875,227.523)(115.312,228.624)(117.75,229.543)(120.188,230.235)(122.625,230.666)(125.062,230.818)(127.5,230.687)(129.938,230.284)(132.375,229.634)(134.812,228.775)(137.25,227.755)(139.688,226.629)(142.125,225.458)(144.562,224.3)(147,223.215)(149.438,222.255)(151.875,221.461)(154.312,220.868)(156.75,220.494)(159.188,220.346)(161.625,220.416)(164.062,220.683)(166.5,221.118)(168.938,221.677)(171.375,222.314)(173.812,222.977)(176.25,223.615)(178.688,224.176)(181.125,224.617)(183.562,224.901)(186,225)\thinlines
		\thinlines\drawline(108,197.153)(110.438,197.623)(112.875,199.011)(115.312,201.248)(117.75,204.225)(120.188,207.798)(122.625,211.796)(125.062,216.028)(127.5,220.298)(129.938,224.409)(132.375,228.181)(134.812,231.454)(137.25,234.099)(139.688,236.025)(142.125,237.179)(144.562,237.555)(147,237.186)(149.438,236.145)(151.875,234.539)(154.312,232.502)(156.75,230.187)(159.188,227.757)(161.625,225.375)(164.062,223.193)(166.5,221.347)(168.938,219.947)(171.375,219.073)(173.812,218.77)(176.25,219.049)(178.688,219.882)(181.125,221.213)(183.562,222.955)(186,225)\thinlines
	%\input{/home/elmar/src/floquet/ansatz/P.0-3}
		\thicklines\drawline(196,225)(198.438,223.392)(200.875,221.936)(203.312,220.765)(205.75,219.984)(208.188,219.658)(210.625,219.802)(213.062,220.383)(215.5,221.325)(217.938,222.516)(220.375,223.821)(222.812,225.099)(225.25,226.214)(227.688,227.054)(230.125,227.543)(232.562,227.643)(235,227.362)(237.438,226.748)(239.875,225.885)(242.312,224.882)(244.75,223.856)(247.188,222.922)(249.625,222.18)(252.062,221.7)(254.5,221.52)(256.938,221.637)(259.375,222.014)(261.812,222.581)(264.25,223.248)(266.688,223.915)(269.125,224.482)(271.562,224.864)(274,225)\thinlines
		\thinlines\drawline(196,259.844)(198.438,258.75)(200.875,255.574)(203.312,250.618)(205.75,244.35)(208.188,237.355)(210.625,230.275)(213.062,223.738)(215.5,218.298)(217.938,214.379)(220.375,212.234)(222.812,211.923)(225.25,213.318)(227.688,216.119)(230.125,219.896)(232.562,224.14)(235,228.321)(237.438,231.951)(239.875,234.633)(242.312,236.104)(244.75,236.256)(247.188,235.147)(249.625,232.981)(252.062,230.082)(254.5,226.848)(256.938,223.703)(259.375,221.045)(261.812,219.196)(264.25,218.372)(266.688,218.654)(269.125,219.99)(271.562,222.199)(274,225)\thinlines
	%\input{/home/elmar/src/floquet/ansatz/P.1-1}
		\thicklines\drawline(20,113.428)(22.4375,113.475)(24.875,113.616)(27.3125,113.849)(29.75,114.172)(32.1875,114.584)(34.625,115.08)(37.0625,115.657)(39.5,116.311)(41.9375,117.035)(44.375,117.824)(46.8125,118.672)(49.25,119.571)(51.6875,120.515)(54.125,121.495)(56.5625,122.503)(59,123.531)(61.4375,124.569)(63.875,125.609)(66.3125,126.641)(68.75,127.656)(71.1875,128.643)(73.625,129.594)(76.0625,130.499)(78.5,131.348)(80.9375,132.131)(83.375,132.839)(85.8125,133.463)(88.25,133.993)(90.6875,134.421)(93.125,134.737)(95.5625,134.933)(98,135)\thinlines
		\dottedline{2}(20,156.572)(22.4375,156.533)(24.875,156.419)(27.3125,156.229)(29.75,155.964)(32.1875,155.627)(34.625,155.22)(37.0625,154.746)(39.5,154.206)(41.9375,153.606)(44.375,152.949)(46.8125,152.24)(49.25,151.482)(51.6875,150.682)(54.125,149.843)(56.5625,148.973)(59,148.076)(61.4375,147.159)(63.875,146.227)(66.3125,145.287)(68.75,144.344)(71.1875,143.405)(73.625,142.477)(76.0625,141.564)(78.5,140.674)(80.9375,139.811)(83.375,138.982)(85.8125,138.192)(88.25,137.446)(90.6875,136.749)(93.125,136.107)(95.5625,135.522)(98,135)\thinlines
		\thinlines\drawline(20,135)(22.4375,133.906)(24.875,132.821)(27.3125,131.752)(29.75,130.709)(32.1875,129.7)(34.625,128.733)(37.0625,127.815)(39.5,126.955)(41.9375,126.158)(44.375,125.433)(46.8125,124.785)(49.25,124.22)(51.6875,123.744)(54.125,123.361)(56.5625,123.076)(59,122.893)(61.4375,122.815)(63.875,122.845)(66.3125,122.986)(68.75,123.238)(71.1875,123.603)(73.625,124.081)(76.0625,124.674)(78.5,125.381)(80.9375,126.2)(83.375,127.132)(85.8125,128.175)(88.25,129.328)(90.6875,130.588)(93.125,131.955)(95.5625,133.426)(98,135)\thinlines
	%\input{/home/elmar/src/floquet/ansatz/P.1-2}
		\thicklines\drawline(108,113.576)(110.438,113.627)(112.875,113.779)(115.312,114.03)(117.75,114.379)(120.188,114.822)(122.625,115.355)(125.062,115.974)(127.5,116.671)(129.938,117.442)(132.375,118.279)(134.812,119.173)(137.25,120.118)(139.688,121.102)(142.125,122.118)(144.562,123.156)(147,124.205)(149.438,125.256)(151.875,126.299)(154.312,127.322)(156.75,128.317)(159.188,129.274)(161.625,130.182)(164.062,131.034)(166.5,131.821)(168.938,132.535)(171.375,133.168)(173.812,133.716)(176.25,134.172)(178.688,134.531)(181.125,134.791)(183.562,134.947)(186,135)\thinlines
		\dottedline{2}(108,156.424)(110.438,156.195)(112.875,155.515)(115.312,154.397)(117.75,152.868)(120.188,150.963)(122.625,148.724)(125.062,146.202)(127.5,143.454)(129.938,140.543)(132.375,137.532)(134.812,134.49)(137.25,131.484)(139.688,128.581)(142.125,125.845)(144.562,123.335)(147,121.106)(149.438,119.205)(151.875,117.671)(154.312,116.536)(156.75,115.82)(159.188,115.536)(161.625,115.686)(164.062,116.26)(166.5,117.242)(168.938,118.603)(171.375,120.31)(173.812,122.318)(176.25,124.578)(178.688,127.036)(181.125,129.633)(183.562,132.308)(186,135)\thinlines
		\thinlines\drawline(108,135)(110.438,135.815)(112.875,136.615)(115.312,137.385)(117.75,138.11)(120.188,138.777)(122.625,139.375)(125.062,139.892)(127.5,140.321)(129.938,140.654)(132.375,140.888)(134.812,141.02)(137.25,141.05)(139.688,140.982)(142.125,140.819)(144.562,140.569)(147,140.239)(149.438,139.842)(151.875,139.389)(154.312,138.893)(156.75,138.369)(159.188,137.831)(161.625,137.296)(164.062,136.777)(166.5,136.29)(168.938,135.849)(171.375,135.468)(173.812,135.158)(176.25,134.931)(178.688,134.795)(181.125,134.758)(183.562,134.825)(186,135)\thinlines
	%\input{/home/elmar/src/floquet/ansatz/P.1-3}
		\thicklines\drawline(196,150.235)(198.438,150.031)(200.875,149.426)(203.312,148.438)(205.75,147.101)(208.188,145.457)(210.625,143.558)(213.062,141.465)(215.5,139.244)(217.938,136.965)(220.375,134.696)(222.812,132.508)(225.25,130.463)(227.688,128.62)(230.125,127.029)(232.562,125.731)(235,124.752)(237.438,124.111)(239.875,123.812)(242.312,123.845)(244.75,124.19)(247.188,124.816)(249.625,125.681)(252.062,126.735)(254.5,127.921)(256.938,129.179)(259.375,130.445)(261.812,131.658)(264.25,132.757)(266.688,133.686)(269.125,134.396)(271.562,134.845)(274,135)\thinlines
		\dottedline{2}(196,119.765)(198.438,119.9)(200.875,120.304)(203.312,120.963)(205.75,121.86)(208.188,122.969)(210.625,124.259)(213.062,125.695)(215.5,127.237)(217.938,128.843)(220.375,130.471)(222.812,132.079)(225.25,133.627)(227.688,135.077)(230.125,136.396)(232.562,137.555)(235,138.534)(237.438,139.316)(239.875,139.893)(242.312,140.264)(244.75,140.432)(247.188,140.411)(249.625,140.218)(252.062,139.875)(254.5,139.409)(256.938,138.851)(259.375,138.232)(261.812,137.585)(264.25,136.943)(266.688,136.337)(269.125,135.796)(271.562,135.343)(274,135)\thinlines
		\thinlines\drawline(196,135)(198.438,140.07)(200.875,144.985)(203.312,149.594)(205.75,153.759)(208.188,157.357)(210.625,160.282)(213.062,162.454)(215.5,163.818)(217.938,164.348)(220.375,164.044)(222.812,162.937)(225.25,161.083)(227.688,158.564)(230.125,155.482)(232.562,151.959)(235,148.13)(237.438,144.136)(239.875,140.125)(242.312,136.241)(244.75,132.621)(247.188,129.391)(249.625,126.662)(252.062,124.523)(254.5,123.043)(256.938,122.265)(259.375,122.208)(261.812,122.865)(264.25,124.203)(266.688,126.169)(269.125,128.687)(271.562,131.665)(274,135)\thinlines
	%\input{/home/elmar/src/floquet/ansatz/P.2-1}
		\thicklines\drawline(20,45)(22.4375,43.421)(24.875,41.8665)(27.3125,40.3606)(29.75,38.9265)(32.1875,37.586)(34.625,36.3592)(37.0625,35.2641)(39.5,34.3162)(41.9375,33.5284)(44.375,32.9103)(46.8125,32.4687)(49.25,32.2069)(51.6875,32.1249)(54.125,32.2191)(56.5625,32.483)(59,32.9065)(61.4375,33.4767)(63.875,34.1778)(66.3125,34.9912)(68.75,35.8965)(71.1875,36.871)(73.625,37.8907)(76.0625,38.9305)(78.5,39.9643)(80.9375,40.9661)(83.375,41.9096)(85.8125,42.7694)(88.25,43.5208)(90.6875,44.1404)(93.125,44.6065)(95.5625,44.8989)(98,45)\thinlines
		\dottedline{2}(20,45)(22.4375,46.5794)(24.875,48.1365)(27.3125,49.6493)(29.75,51.0966)(32.1875,52.4585)(34.625,53.7163)(37.0625,54.8533)(39.5,55.8548)(41.9375,56.7084)(44.375,57.4045)(46.8125,57.9358)(49.25,58.2982)(51.6875,58.4901)(54.125,58.5133)(56.5625,58.372)(59,58.0733)(61.4375,57.6272)(63.875,57.0458)(66.3125,56.3436)(68.75,55.5372)(71.1875,54.6446)(73.625,53.6853)(76.0625,52.68)(78.5,51.6499)(80.9375,50.6164)(83.375,49.601)(85.8125,48.6248)(88.25,47.7082)(90.6875,46.8704)(93.125,46.1292)(95.5625,45.501)(98,45)\thinlines
		\thinlines\drawline(20,45)(22.4375,44.9047)(24.875,44.6207)(27.3125,44.1545)(29.75,43.5162)(32.1875,42.7199)(34.625,41.7835)(37.0625,40.7277)(39.5,39.5764)(41.9375,38.3555)(44.375,37.0931)(46.8125,35.8184)(49.25,34.5614)(51.6875,33.3524)(54.125,32.2211)(56.5625,31.1964)(59,30.3056)(61.4375,29.5741)(63.875,29.0246)(66.3125,28.6772)(68.75,28.5484)(71.1875,28.6514)(73.625,28.9954)(76.0625,29.5858)(78.5,30.4241)(80.9375,31.5076)(83.375,32.83)(85.8125,34.3812)(88.25,36.1477)(90.6875,38.113)(93.125,40.258)(95.5625,42.5614)(98,45)\thinlines
	%\input{/home/elmar/src/floquet/ansatz/P.2-2}
		\thicklines\drawline(108,45)(110.438,42.5857)(112.875,40.2129)(115.312,37.9223)(117.75,35.7526)(120.188,33.74)(122.625,31.9173)(125.062,30.3132)(127.5,28.9515)(129.938,27.851)(132.375,27.0247)(134.812,26.4798)(137.25,26.2176)(139.688,26.2334)(142.125,26.5166)(144.562,27.0513)(147,27.8164)(149.438,28.7862)(151.875,29.931)(154.312,31.2181)(156.75,32.6122)(159.188,34.0762)(161.625,35.5724)(164.062,37.063)(166.5,38.5113)(168.938,39.882)(171.375,41.1423)(173.812,42.2627)(176.25,43.2168)(178.688,43.9829)(181.125,44.5433)(183.562,44.885)(186,45)\thinlines
		\dottedline{2}(108,45)(110.438,47.404)(112.875,49.7049)(115.312,51.804)(117.75,53.6101)(120.188,55.0437)(122.625,56.0401)(125.062,56.5516)(127.5,56.5496)(129.938,56.026)(132.375,54.9932)(134.812,53.484)(137.25,51.5503)(139.688,49.261)(142.125,46.6998)(144.562,43.9615)(147,41.149)(149.438,38.369)(151.875,35.7281)(154.312,33.3288)(156.75,31.2653)(159.188,29.6205)(161.625,28.4623)(164.062,27.8412)(166.5,27.7885)(168.938,28.315)(171.375,29.4107)(173.812,31.0451)(176.25,33.1682)(178.688,35.7129)(181.125,38.5971)(183.562,41.7268)(186,45)\thinlines
		\thinlines\drawline(108,45)(110.438,45.0727)(112.875,45.2879)(115.312,45.6371)(117.75,46.1064)(120.188,46.6774)(122.625,47.3274)(125.062,48.0306)(127.5,48.759)(129.938,49.4836)(132.375,50.1754)(134.812,50.8064)(137.25,51.3512)(139.688,51.7874)(142.125,52.097)(144.562,52.2666)(147,52.2886)(149.438,52.1608)(151.875,51.8873)(154.312,51.4778)(156.75,50.9476)(159.188,50.3172)(161.625,49.6112)(164.062,48.8579)(166.5,48.0879)(168.938,47.3334)(171.375,46.6271)(173.812,46.0006)(176.25,45.484)(178.688,45.1045)(181.125,44.8854)(183.562,44.8459)(186,45)\thinlines
	%\input{/home/elmar/src/floquet/ansatz/P.2-3}
		\thicklines\drawline(196,45)(198.438,47.0861)(200.875,49.0626)(203.312,50.8258)(205.75,52.2838)(208.188,53.3611)(210.625,54.003)(213.062,54.1783)(215.5,53.8808)(217.938,53.1294)(220.375,51.967)(222.812,50.458)(225.25,48.6838)(227.688,46.7388)(230.125,44.7249)(232.562,42.745)(235,40.8982)(237.438,39.2734)(239.875,37.9451)(242.312,36.9688)(244.75,36.3785)(247.188,36.1851)(249.625,36.3756)(252.062,36.9149)(254.5,37.7474)(256.938,38.8012)(259.375,39.9916)(261.812,41.2269)(264.25,42.4129)(266.688,43.4589)(269.125,44.2826)(271.562,44.8141)(274,45)\thinlines
		\dottedline{2}(196,45)(198.438,42.9115)(200.875,40.9184)(203.312,39.1111)(205.75,37.5709)(208.188,36.365)(210.625,35.5442)(213.062,35.1398)(215.5,35.1622)(217.938,35.601)(220.375,36.4257)(222.812,37.5873)(225.25,39.0217)(227.688,40.653)(230.125,42.3976)(232.562,44.1692)(235,45.8832)(237.438,47.4609)(239.875,48.8341)(242.312,49.9481)(244.75,50.7645)(247.188,51.2626)(249.625,51.4403)(252.062,51.3136)(254.5,50.9151)(256.938,50.2919)(259.375,49.5021)(261.812,48.6118)(264.25,47.6906)(266.688,46.8079)(269.125,46.0289)(271.562,45.4108)(274,45)\thinlines
		\thinlines\drawline(196,45)(198.438,45.4444)(200.875,46.7494)(203.312,48.8321)(205.75,51.5601)(208.188,54.759)(210.625,58.2232)(213.062,61.728)(215.5,65.0433)(217.938,67.9471)(220.375,70.2391)(222.812,71.7527)(225.25,72.3652)(227.688,72.0053)(230.125,70.6573)(232.562,68.363)(235,65.219)(237.438,61.3715)(239.875,57.0083)(242.312,52.348)(244.75,47.6272)(247.188,43.0878)(249.625,38.9625)(252.062,35.4623)(254.5,32.7641)(256.938,31.0014)(259.375,30.2567)(261.812,30.5576)(264.25,31.8756)(266.688,34.1284)(269.125,37.1855)(271.562,40.8762)(274,45)\thinlines

	\end{picture}
	\end{center}
	\caption{\label{fig:ansatz}Beispiele der ersten drei Ansatzfunktionen $\mathbf{\overline{v}}_\mathbf{i}$ mit $k=3$ und den Winkelzahlen $n=0,1,2$. Die Funktionen sind gem\"a\ss\ $\int_0^1\mathbf{\overline{v}_i}^*\cdot\mathbf{\overline{v}_i}\,r\,\textrm{d}r=1$ skaliert.}
\end{figure}
Hiermit sind die Ansatzfunktionen $\mathbf{v_i}$ vollst\"andig definiert, die wir ben\"otigen, um auf die L\"osung des Stabilit\"atsproblems ($\mbox{\ref{eq:rwp}}$) zur\"uckzukommen.\\ 
\newpage

Setzen wir den Ansatz (\mbox{\ref{eq:galerkin:ansatz}}) in das System (\mbox{\ref{eq:stoerevolution}}) bis (\mbox{\ref{eq:haft}}), multiplizieren mit $\mathbf{v_k}$, und integrieren \"uber den Str\"omungsraum, so bekommen wir das gew\"ohnliche lineare Differentialgleichungssystem
\begin{equation}\label{eq:galerkin:system}
	\frac{\mathrm{d} g_\mathbf{i}}{\mathrm{d} t} = B_{\mathbf{ik}} g_\mathbf{k} 
\end{equation}
mit der quadratischen Matrix
\begin{equation}\label{eq:matrixB}
	B_{\mathbf{ik}} = \langle (\mathfrak{v}_0 \,\mathrm{grad}) \mathbf{v_i} + (\mathbf{v_i} \,\mathrm{grad}) \mathfrak{v}_0 , \mathbf{v_k}\rangle + \lambda_\mathbf{i} \delta_{\mathbf{ik}}.
\end{equation}
Das System hat die gleiche Dimension wie die Anzahl der Ansatzfunktionen $N$.
Der Druck $p$ ist in der Matrix $B$ nicht enthalten, da der Term $\langle\mathrm{grad}\,p,\mathbf{v_k}\rangle$ keinen Beitrag leistet.\\
Da die Grundstr\"omung $\mathfrak{v}_0$ zeitabh\"angig ist, ist auch die Matrix $B$ nicht konstant.
Im Rahmen der quasista"-tisch"-en Asymptotik entkoppeln wir die Zeitabh\"angigkeit der Matrix $B$ von jener der Zeitfunktionen $g$.
Wir fassen die Zeit in $B$ als zus\"atzlichen Parameter auf und erzwingen dadurch ein System mit konstanten Koeffizienten.
Dieses hat exponentielle L\"osungen, deren Aufklingrate $\sigma$ und Kreisfrequenz $\omega$ als Eigenwerte $\sigma + i\omega$ der Matrix $B$ gegeben sind.\\

Anhand von (\mbox{\ref{eq:matrixB}}) erkennen wir die Verwandtschaft des Eigenwertproblems, welches wir zur Definition der Ansatzfunktionen benutzt haben, mit dem Eigenwertproblem der St\"orungsgleichungen.
In zwei Grenzf\"allen sind die Systeme die gleichen:
\begin{enumerate}
	\item Die Str\"omung ist ruhend ($\mathfrak{v}_0 = 0$)
	\item Die St\"orungen sind langwellig ($k\to 0$)
\end{enumerate}
Falls einer der F\"alle auftritt, sind die Eigenwerte von (\mbox{\ref{eq:matrixB}}) identisch mit den Eigenwerten der Ansatzfunktionen $\lambda_\mathbf{i}$, welche in $B$ auf der Hauptdiagonalen stehen.\\
In diesen F\"allen sind die Eigenwerte
\begin{equation}
	\sigma_\mathbf{i} = \lambda_{\mathbf{i}} = -(1/R)(\beta_\mathbf{i}^2+k^2) \qquad \omega_\mathbf{i} = 0
\end{equation}
negativ reell, und alle St\"orungen klingen exponentiell aperiodisch ab.\\

Im allgemeinen mu\ss\ f\"ur jede Kombination von $R$, $\alpha$, $n$, $k$ und dem Zeitparameter $\tau$ die vollbesetzte komplexe Matrix $B$ aufgestellt werden.
Aus ihr folgen die Eigenwerte $\sigma+i\omega$ durch eine numerische Eigenwertroutine.
Die Berechnungen k\"onnen unabh\"anig voneinander durchgef\"uhrt werden.
Im Gegensatz zum Schie\ss verfahren sind keine Startwerte n\"otig.
Es empfiehlt sich jedoch aus Zeitgr\"unden nach einem bestimmten Schema zu verfahren, welches wir kurz erl\"autern wollen.
Die folgende \"Ubersicht skizziert die Verschachtelung der Schleifen \"uber die Parameter, welche f\"ur die Berechnung der Eigenwerte verwandt wurde:
\begin{center}
\begin{tabular}{l}
	$R,\alpha$-Schleife (1)\\
		\hspace*{3em}Grundstr\"omung (2)\\
		\hspace*{3em}$n$-Schleife (3)\\
			\hspace*{6em}$k$-Schleife (4)\\
				\hspace*{9em}Ansatzfunktionen (5)\\
				\hspace*{9em}$\tau$-Schleife (6)\\
					\hspace*{12em}Grundstr\"omung anpassen (7)\\
					\hspace*{12em}Berechnung von $B$ (8)\\
					\hspace*{12em}Eigenwertproblem l\"osen (9)\\
\end{tabular}
\end{center}
F\"ur jeden Punkt in der Reynoldszahl-Womersleyzahl-Ebene f\"ur $R$ zwischen 0 und 3000 und $\alpha$ zwischen 0 und 20 iterieren wir in der \"au\ss ersten Schleife (1).\\
Durch die Ver\"anderung der Womersleyzahl ist es notwendig, die Grundstr\"omung (\mbox{\ref{eq:baseflow}}) neu zu berechnen (2).
Hierbei speichern wir sowohl den Real- und den Imagin\"arteil der Ortsfunktion ab, ohne den Faktor $e^{i\tau}$.
Zur Berechnung der Grundstr\"omung ben\"otigen wir Besselfunktionen mit komplexen Argumenten, welche von der Bibliothek \texttt{amos}\footnote{\label{bib:amos}\textsl{Amos}: \textit{ACM Trans.\ Math.\ Soft.} \textbf{12} (1986)}
bereitgestellt werden.\\
Die Schleife \"uber die Winkelzahl $n$ l\"auft \"uber die Werte 0 und 1 f\"ur symmetrische und antimetrische St\"orungen (3).
%In einer unabh\"angigen Rechnung werden auch Winkelzahlen bis $n=5$ ber\"ucksichtigt
Nachdem in (4) die Wellenzahlen variiert werden, m\"ussen im n\"achsten Schritt die Ansatzfunktionen (\mbox{\ref{eq:ansatzfunktionen}}) berechnet werden (5), welche von $n$, $k$ und $R$ abh\"angen.
Die ben\"otigten reellen modifizierten Besselschen Funktionen $I_n$ entnehmen wir der \texttt{cephes}-Bibliothek von \textsl{Moshier}.\footnote{\label{bib:moshier}\textsl{Steve Moshier:} www.moshier.net, www.netlib.org/cephes}\\
In der innersten Schleife l\"auft der Zeitparameter $\tau$ mit Werten zwischen 0 und der halben Periode $\pi$ (6).
Die Grundstr\"omung, welche in (2) gespeichert wurde, wird nun an den Zeitparameter $\tau$ durch Multiplikation mit $e^{i\tau}$ und Projektion auf die reelle Achse angepa\ss t (7).\\
Im n\"achsten Schritt stellen wir die Matrix $B$ auf (8).
Die Integration zur Berechnung des Skalarproduktes (\mbox{\ref{eq:skalarprodukt}}) in (\mbox{\ref{eq:matrixB}}) f\"uhren wir nur \"uber die radiale Koordinate durch, da sich die Beitr\"age der Integrationen \"uber die \"ubrigen Raumdimensionen aufheben.
Als Quadrationsverfahren benutzen wir die Simpson-Methode.\\
Hiermit haben wir alle Voraussetzungen erf\"ullt, um die Eigenwerte der komplexen Matrix $B$ zu berechnen (9).
Zu diesem Zweck bedienen wir uns der Routine \texttt{zgeev} aus der Bibliothek \texttt{LAPACK}.\footnote{\label{bib:lapack}\textsl{Anderson}: \textit{Soc.\ Indust.\ Appl.\ Math.\ } (1999)}\\
Es hat sich herausgestellt, da\ss\ wir f\"ur die Anzahl von Ansatzfunktionen $N=80$ stabile Werte erhalten.\\

Im \"ubern\"achsten Kapitel stellen wir die Berechnungsergebnisse detailliert dar.
Vorher stellen wir mit der Floquet-\!Theorie eine allgemeinere Stabilit\"atstheorie f\"ur periodische Systeme vor, welche nicht auf die Annahme der Quasistatik angewiesen ist.

%% Floquet-Theorie
\chapter{Aussage der Floquet-\!Theorie}\label{sec:floquet}
In diesem Kapitel erl\"autern wir die Stabilit\"atsanalyse nach der Floquet-\!Theorie.
Diese erfordert im Gegensatz zur quasistatischen Theorie des vorangegangenen Kapitels keine zeitliche Skalenseparation.\\

Die Floquet-\!Theorie\footnote{\label{bib:floquet}\textsl{Floquet}: \textit{Ann.\ Sci.\ {\'E}.\ Norm.\ Sup.} \textbf{12} (1883)\\\bibspace\label{bib:nayfeh_mook}\textsl{Nayfeh, Mook}: (1979)}
gilt f\"ur Systeme gew\"ohnlicher linearer Differentialgleichungen mit periodischen Koeffizienten.
Das Stabilit\"atsproblem der oszillierenden Rohrstr\"omung f\"uhrt auf periodische Koeffizienten in der Zeit und ist daher eine m\"ogliche Anwendung der Floquet-\!Theorie.
Aus der linearisierten Evolutionsgleichung (\mbox{\ref{eq:stoerevolution}}) k\"onnen wir mit (\mbox{\ref{eq:separation}}) die Abh\"angigkeiten von $\varphi$ und $z$ separieren, die Abh\"angigkeit von $r$ bleibt bestehen.
Daher k\"onnen wir die Floquet-\!Theorie nicht direkt anwenden, sondern m\"ussen auf die Galkerkin-Entwicklung des vorangegangenen Abschnittes zur\"uckgreifen.\\

Das lineare System (\mbox{\ref{eq:galerkin:system}}) erf\"ullt alle Voraussetzungen.
Ohne die quasistatische Separation hat die Matrix $B$ periodische Koeffizienten mit der Periodendauer 
\begin{equation}
	T = 2\pi R/\alpha^2.
\end{equation}
F\"ur das System (\mbox{\ref{eq:galerkin:system}}) l\"a\ss t sich eine Fundamentalbasis konstruieren, indem wir die homogene Matrix-Differentialgleichung 
\begin{equation}
	\frac{\mathrm{d} F}{\mathrm{d} t} = B(t) F(t) \qquad B(t+T) = B(t)\qquad F(0) = I
\end{equation}
mit der Einheitsmatrix $I$ als Anfangsbedingung \"uber eine Periodendauer integrieren.
Die Integration wird numerisch durchgef\"uhrt.
Den Endzustand nennen wir die \textit{Floquetmatrix} $F(T)$.\\

Die Eigenwerte $\gamma_i$ der Floquetmatrix $F(T)$ spielen bei der Stabilit\"atsbetrachtung die entscheidende Rolle.
Falls alle Eigenwerte betragsm\"a\ss ig kleiner eins sind, dann ist die Grundstr\"o"-mung (\mbox{\ref{eq:baseflow}}) \textit{asymptotisch stabil}.
Im Umkehrschlu\ss\ ist die Grundstr\"omung \textit{asymptotisch instabil}, falls mindestens einer der Eigenwerte betragsm\"a\ss ig gr\"o\ss er eins ist.
\begin{equation}\label{eq:floq:stabilitaet}
	| \gamma_i | \, \begin{cases} \,< 1 \quad \textrm{ f\"ur alle } i:& \textrm{stabil}\\
		\,> 1 \quad \textrm{ f\"ur mindestens ein } i: & \textrm{instabil}
		\end{cases}
\end{equation}
Nach dem Floquetschen Satz gibt es eine Wahl von Fundamentall\"os"-ungen, die sich als Produkt aus einer Exponentialfunktion und einer periodischen Funktion darstellen lassen.
Asymptotische Instabilit\"at bedeutet, da\ss\ eine dieser Exponentialfunktionen ansteigt, und damit die Amplitude der L\"osung nach langer Zeit gegen unendlich strebt.\\ 

%Auf dieser Theorie basiert die Arbeit von 
\textsl{Yang und Yih}\footnote{\label{bib:yang_yih:2}\textsl{Yang, Yih}: \textit{J.\ Fluid Mech.} \textbf{82} (1977)}
wenden die Floquet-\!Theorie erstmals auf das Stabilit\"atsproblem der oszillierenden Rohrstr\"omung an.
Statt der Diskretisierung nach der Galerkin-Entwicklung benutzen sie eine Diskretisierung nach der Methode der finiten Elemente.
In ihrer Arbeit untersuchen sie nur symmetrische St\"orungen.
F\"ur diese werden keine instabilen L\"osungen gefunden.\\

Diese Aussage k\"onnen wir auch f\"ur antimetrische St\"orungen best\"atigen.
Es konnte f\"ur keine Parameterkombination eine Instabilit\"at im Sinne der Floquetschen Theorie nachgewiesen werden.
Wir betrachten daher die oszillierende Rohrstr\"omung als asymptotisch stabil.\\
Diese Stablilit\"at bezieht sich auf das Langzeitverhalten der St\"orungen \"uber eine Dauer, welche mehrere Perioden umfa\ss t.\\

Es ist dennoch m\"oglich, da\ss\ St\"orungen w\"ahrend einer Periode wesentlich verst\"arkt werden k\"onnen, obwohl sie asymptotisch stabil sind.
Um diese kurzzeitigen Verst\"arkungen zu quantifizieren, empfiehlt sich die quastistatische Theorie.\\

In Experimenten, auf welche wir in einem sp\"ateren Kapitel eingehen, zeigt es sich, da\ss\ unter bestimmten Parameterkombinationen ein turbulentes Str\"omungsbild zu verzeichnen ist, welches jedoch nie \"uber eine komplette Periodendauer Bestand hat.
In dieser Hinsicht stimmen die Stabilit\"atsergebnisse nach der Floquetschen Theorie mit den Experimenten \"uberein.

%% Berechnungen
\chapter{Berechnungen}\label{sec:berechnungen}
In diesem Kapitel stellen wir die Berechnungsergebnisse der Parameterstudien zur Untersuchung des Stabilit\"atsverhaltens der oszillierenden Rohrstr\"omung zusammen.
Hierzu wurden Rechnungen beruhend auf den folgenden Theorien durchgef\"uhrt:
\begin{itemize}
	\item Floquet-\!Theorie
	\item Quasistatische Theorie
	\begin{itemize}
		\item Schie\ss verfahren
		\item Galerkin-Entwicklung
	\end{itemize}
	\item Anfangswertproblem
\end{itemize}
Die Berechnungen zur Galkerin-Entwicklung nehmen hiervon den Hauptteil ein.
Rechnungen zur Floquet-\!Theorie wurden nur stichpunktartig durchgef\"uhrt, um bereits bestehende Arbeiten zu best\"atigen.
Das Schie\ss verfahren dient dazu, die Ergebnisse nach der Galkerin-Entwicklung zu kontrollieren.
Au\ss erdem wurde das System (\mbox{\ref{eq:galerkin:system}}), welches aus der Diskretisierung mit der Galerkinschen Methode resultiert, ohne die Einschr\"ankungen der quasistatischen Annahme als Anfangswertproblem integriert.
Dies soll helfen, Aufschlu\ss\ \"uber die Aussagekraft der quasistatischen Theorie zu erhalten.\\

Der untersuchte Parameterbereich der $R$-$\alpha$-Ebene umfa\ss t alle Kombinationen von Womersleyschen Zahlen zwischen 0 und 20 mit Reynoldsschen Zahlen zwischen 0 und 3000.
Der Womersleysche Paramter deckt den Bereich langsamer Oszillationen von der station\"aren Grenze bis zu hochfrequenten Oszillationen mit ausgepr\"agtem Grenzschichtcharakter ab.
Das Intervall der Reynoldsschen Zahl st\"o\ss t bis in einen Bereich vor, in dem eine lineare Theorie ihre Zul\"assigkeit verliert.
Es ist bekannt, da\ss\ die station\"are Rohrstr\"omung f\"ur Reynoldssche Zahlen gr\"o\ss er $2000$ Instabilit\"atsph\"anomene aufweist, welche nicht mit einer linearen Theorie erkl\"art werden k\"onnen.
Daher erwarten wir auch f\"ur eine oszillierenden Str\"omung---zumindest zu Zeiten maximalen Flusses und kleiner Womersleyscher Zahlen---einen entsprechenden Instabilit\"atsmechanismus, der in dieser einfachen Theorie nicht mit einbegriffen ist.
Es erscheint deshalb nicht sinnvoll, lineare Untersuchungen mit gr\"o\ss eren Reynoldsschen Zahlen durchzuf\"uhren.\\

Die Untersuchungen fu\ss end auf der Floquetschen Theorie haben ergeben, da\ss\ sich die oszillierende Rohrstr\"o"-mung als \textit{asymptotisch stabil} gem\"a\ss\ des Kriteriums (\mbox{\ref{eq:floq:stabilitaet}}) erweist.
Dieses Ergebnis steht im Einklang mit den in der Einleitung erw\"ahnten Untersuchungen von \textsl{Yang und Yih} sowie \textsl{Fedele, Hitt und Prabhu}.\\

Trotz der asymptotischen Stabilit\"at reagiert die Str\"o"-mung f\"ur bestimmte Parameterkombinationen quasistatisch instabil, jedoch nur \"uber einen kurzen Zeitraum w\"ahrend der Schwingungsperiode.
Diese In"-sta"-bi"-li"-t\"at"-en treten zu Zeitpunkten der Flu\ss umkehr auf und werden nach der linearen Theorie zu Zeiten des maximalen Flusses wieder stabilisiert.
Der Kern der vorliegenden Arbeit liegt darin, diese Ph\"anomene zu ergr\"unden.
Hierf\"ur benutzen wir die Methoden der quasistatischen Theorie.\\

Es wurden beide Verfahren der quasistatischen Stabilit\"atstheorie implementiert und Rechnungen durchgef\"uhrt.
Im Falle des Schie\ss verfahrens wurden nur symmetrische St\"orungen untersucht.
In den Bereichen, in denen das Schie\ss verfahren verl\"a\ss liche Ergebnisse liefert, stimmen diese mit den Ergebnissen der Galerkin-Entwicklung \"uberein.
Dies ist eine beruhigende Ausgangssituation, da hiermit Fehler in der Herleitung und Implementierung der Gleichungen ausgeschlossen werden k\"onnen.
Im Vergleich zwischen dem Schie\ss verfahren und der Galerkin-Methode erweist sich letztere als vorteilhafter.
Das Schie\ss verfahren erm\"oglicht prinzipiell eine genauere Berechnung der Eigenwerte.
Der Unterschied ist jedoch bei einer ausreichenden Anzahl $N$ an Ansatzfunktionen vernachl\"assigbar.\\
F\"ur jede Winkelzahl $n$ m\"ussen f\"ur das Schie\ss verfahren eigene Glei"-chung"-en hergeleitet werden, deren Zielbedingung f\"ur steigende Winkelzahlen schlechter konditioniert ist.
F\"ur die Galer"-kin-Methode ist dies nicht n\"otig.
In ihren Ansatzfunktionen ist die Winkelzahl ein Parameter.
Dies ist ein bedeutender Vorteil der Methode f\"ur alle Winkelzahlen $n>0$.\\
Die Galerkin-Methode kann Rechnungen f\"ur beliebige Wellenzahlen $k$ in einem Schritt durchf\"uhren.
Das ist ein Vorteil gegen\"uber dem Schie\ss verfahren, welches L\"osungen zu einer Wellenzahl $k$ sequentiell mit dem Startwert $k=0$ berechnen mu\ss.
Au\ss erdem findet man mit der Galerkin-Methode nicht nur den gr\"o\ss ten Eigenwert, sondern eine N\"aherung der ersten $N$ Eigenwerte auf einmal.\\
Ein gro\ss er Nachteil des Schie\ss verfahrens ist es, da\ss\ das Einflu\ss gebiet der Zielbedingung (\mbox{\ref{eq:target}}) des gr\"o\ss ten Eigenwertes mit steigenden Wellenzahlen $k$ und Womersleyschen Zahlen $\alpha$ enger wird und schlie\ss lich vom Optimierungsverfahren nicht mehr als Nullstelle erkannt wird.
%An dieser Stelle k\"onnte die Methode von \textsl{Mack}\footnote{\label{bib:mack}\textsl{Mack}: \textit{J.\ Fluid.\ Mech.} \textbf{73} (1976)} hilfreich sein, bei der Nullstellen der Form (\mbox{\ref{eq:target}}) mit einer graphischen Methode bestimmt werden.
Bei der Galerkin-Methode kann es zu diesem Fehler nicht kommen.\\
Methodisch unterscheiden sich die Verfahren in den be"-n\"o"-tigten Algorithmen.
Das Schie\ss verfahren erfordert einen In"-te"-gra"-tions- und einen Optimierungsalgorithmus, die Galerkin-Methode hingegen eine Quadratur, einen komplexen Eigenwertl\"oser und eine Implementierung der reellen Versionen der Besselfunktionen erster Art $J_n$ und der modifizierten Besselfunktionen $I_n$.\\

Die Abbildungen \ref{fig:neutral} bis \ref{fig:epsilon} zeigen die Ergebnisse der quasista"-tisch"-en Berechnungen nach der Galerkin-Methode.\\
Die Ergebnisse sind als Kontourlinien \"uber der $R$-$\alpha$-Ebene aufgetragen.
Die Eigenwerte $\sigma+i\omega$ werden in dieser Ebene auf $\sim 5000$ St\"utzstellen berechnet.
Die linke Spalte der Abbildungen \ref{fig:sigma} bis \ref{fig:epsilon} umfa\ss t symmetrische St\"orungen ($n=0$), die rechte Spalte antimetrische ($n=1$).
Nicht direkt ersichtlich sind aus dem Parameterraum (\mbox{\ref{eq:parameter}}) die Wellenzahl $k$ und der Zeitparameter $\tau$.\\

Die Abbildungen sind wie folgt zu interpretieren:\\
Es werden in jedem Berechnungspunkt mit festen Werten f\"ur $R$, $\alpha$ und $n$ die $N$ Eigenwerte $\sigma_\mathbf{i}+i\omega_\mathbf{i}$ in Abh\"angigkeit von $k$ und $\tau$ ermittelt.
Von Interesse sind die Kombinationen, welche auf eine maximale Verst\"arkungsrate f\"uhren, also die maximale Aufklingrate
\begin{equation}\label{eq:sigmamax}
	\sigma = \max_{k,\tau,\mathbf{i}} \sigma_\mathbf{i},
\end{equation}
die zum entsprechenden Eigenwert zugeh\"orige Kreisfrequenz $\omega$, sowie die Wellenzahl $k$ und der Zeitparameter $\tau$, unter welchen die Aufklingrate maximal wird.
Diese Gr\"o\ss en sind den Abbildungen \ref{fig:sigma} bis \ref{fig:epsilon} aufgetragen.\\
In der Maximierung (\mbox{\ref{eq:sigmamax}}) laufen die Wellenzahlen $k$ in 40 Schritten zwischen 0 und 10, der Zeitparameter in 120 Schritten \"uber eine halbe Period (0 bis $\pi$).
Da die Str\"o"-mung periodisch ist, gen\"ugen die Eigenwerte der Periodizit\"atsbedingung
\begin{equation}
	\sigma(\tau+\pi) = \sigma(\tau) \qquad \omega(\tau+\pi) = -\omega(\tau).
\end{equation}
In zwei weiteren Schritten wird die Diskretisierung von $k$ und $\tau$ um die jeweils aktuell zu maximaler Verst\"arkung f\"uhrenden Werte verfeinert.
Hiermit erreichen wir eine Schrittweite in der Wellenzahl von $1/100$ und im Zeitparameter von $1/3000$.
Alle Rechnungen wurden mit 80 Ansatzfunktionen durchgef\"uhrt.\\

Abbildung \ref{fig:eigenvalue} zeigt schematisch den Verlauf des Eigenwertes $\sigma+i\omega$ mit der gr\"o\ss ten Verst\"arkung in Abh\"angigkeit der Wellenzahl $k$ und des Zeitparameters $\tau$.
Die qualitative Darstellung ist charakteristisch f\"ur die auftretenden Eigenwertverteilungen.
Langwellige St\"or"-ungen haben nach Tabelle \ref{tab:start} ged\"ampfte Eigenwerte $\sigma<0$ ohne schwingenden Anteil $\omega=0$.
Daher starten die Kurven des linken Bildes an einem Punkt mit negativer Aufklingrate.
Durch Erh\"ohung der Wellenzahl f\"allt die Eigenwertkurve in den meisten F\"allen ab, das bedeutet St\"or"-ungen mit k\"urzeren Wellenl\"angen werden st\"arker ged\"ampft.
In manchen F\"allen steigt die Kurve an bevor sie f\"ur kurzwellige St\"or"-ungen ab"-f\"allt.
Falls die Kurve die $\omega$-Achse schneidet, ist die Str\"o"-mung quasistatisch instabil.
Der Maximalwert dieser Kurve liefert die maximale Verst\"arkung $\sigma$.\\
Das rechte Bild zeigt die zeitliche Variation des gleichen Eigenwertes bei jener Wellenzahl, welche im linken Bild zu einer maximalen Verst\"arkung f\"uhrt.\\
Charakteristisch am Verlauf der Eigenwerte ist, da\ss\ es nur f\"ur einen engen Bereich von Wellenzahlen zu einem Ansteigen in den instabilen Bereich kommt.
Im linken Bild ist eine solche Kurve dargestellt, die andere entspricht dem Normalfall eines stabilen Verhaltens.
Die Eigenwerte mit diesen Wellenzahlen sind au\ss erdem nur zeitlich begrenzt instabil, wie aus dem rechten Bild ersichtlich ist.
F\"ur keine Parameterkombination existieren St\"orungen, welche w\"ahrend des gesamten Zyklus instabil sind.\\

\begin{figure}
	\begin{center}
	\begin{picture}(275,117)(0,0)
	%\input{fig/eigvals/eigvals.path}
		\thinlines
		\drawline(-0.0493274,55.4318)(85.7803,55.4318)
		\drawline(100.825,55.2273)(274.951,55.2273)
		\drawline(11.7892,116.877)(11.7892,0.286364)
		\drawline(191.587,116.877)(191.587,0.286364)
		\thicklines
		\drawline(223.798,31.5)(234.157,28.6364)(243.283,26.5909)(252.161,25.5682)(258.821,24.75)(265.233,23.5227)(267.7,21.2727)(266.466,18.4091)(260.794,18)(251.422,18.4091)(237.117,20.0455)(220.839,22.9091)(209.987,27)(204.314,30.8864)(203.574,32.3182)(193.709,78.5455)(192.229,84.6818)(189.022,93.0682)(186.063,96.9545)(184.336,97.9773)(182.117,98.3864)(181.87,98.3864)(179.65,97.9773)(177.924,96.9545)(174.964,93.0682)(171.758,84.6818)(170.278,78.5455)(160.413,32.3182)(159.426,31.5)(149.067,28.6364)(139.942,26.5909)(131.063,25.5682)(124.404,24.75)(117.991,23.5227)(115.525,21.2727)(116.758,18.4091)(122.43,18)(131.803,18.4091)(146.108,20.0455)(162.386,22.9091)(173.238,27)(178.91,30.8864)(179.256,32.3182)(189.121,78.5455)(190.601,84.6818)(193.807,93.0682)(196.767,96.9545)(198.493,97.9773)(200.713,98.3864)(200.96,98.3864)(203.179,97.9773)(204.906,96.9545)(207.865,93.0682)(211.072,84.6818)(212.552,78.5455)(222.417,32.3182)(223.798,31.5)
		\drawline(11.6906,42.9545)(13.417,43.9773)(14.8969,51.3409)(16.6233,82.0227)(18.1031,94.2955)(19.3363,97.3636)(20.8161,98.5909)(23.0359,97.5682)(24.5157,93.8864)(25.7489,83.25)(27.9686,49.0909)(29.6457,13.2136)
		\drawline(11.9372,43.1591)(17.8565,42.9545)(25.7489,41.5227)(32.1614,38.8636)(40.7937,35.3864)(49.9193,30.6818)(63.2377,22.7045)(75.0762,14.1136)
		\thinlines
	\put(2,110){$\sigma$} 
	\put(182,110){$\sigma$}
	\put(78,46){$\omega$}
	\put(267,46){$\omega$}
	\put(70,28){\begin{rotate}{-33}{$k$}\end{rotate}}
	\put(60,26){\begin{rotate}{-33}{$\longrightarrow$}\end{rotate}}
	\put(242,34){\begin{rotate}{-10}{$\tau$}\end{rotate}}
	\put(235,30){\begin{rotate}{-10}{$\longrightarrow$}\end{rotate}}
	\end{picture}
	\caption{\label{fig:eigenvalue}Eigenwerte in Abh\"angigkeit der Wellenzahl $k$ und der Zeit $\tau$. $\,$ Links: Beide Kurven geh\"oren zu zwei verschiedenen Zeitpunkten. Hierbei ist $k$ der Kurvenparameter. Alle Kurven gleicher Winkelzahl haben den gleichen Startwert f\"ur $k\to0$. $\,$ Rechts: Der zeitliche Verlauf eines Eigenwertes w\"ahrend einer Periode bei konstanter Wellenzahl $k$. Die Symmetrie be"-z\"ug"-lich der $\sigma$-Achse folgt aus der Symmetrie der Hin- und R\"uckstr\"o"-mung w\"ahrend einer Periode.}
	\end{center}
\end{figure}

\paragraph{Kurven neutraler Stabilit\"at}
Abbildung \ref{fig:neutral} zeigt die Kurven neutraler Stabilit\"at in der $R$-$\alpha$-Ebene f\"ur verschiedene Winkelzahlen $n$.
Dies sind die Grenzkurven mit $\sigma=0$ nach (\mbox{\ref{eq:sigmamax}}), welche die Bereiche negativer und positiver maximaler Anfachungsraten $\sigma$ separieren.
Unterhalb dieser Kurven ist die Str\"o"-mung gegen\"uber St\"or"-ungen aller Wellenzahlen zu jedem Zeitpunkt quasistatisch stabil.
Oberhalb der Kurven existiert eine Parameterkombination mit einer maximalen Verst\"arkung $\sigma > 0$.\\
Wir erkennen, da\ss\ antimetrische St\"or"-ungen das gr\"o\ss te Instabilit\"atsgebiet umfassen.
St\"or"-ungen mit gr\"o\ss eren Winkelzahlen haben ein kleineres Instabilit\"atsgebiet.
Daher beschr\"anken wir die Berechnungen auf die wichtigen Winkelzahlen symmetrischer und antimetrischer St\"orungen $n=0$ und $n=1$.\\
\begin{figure}[ htbp ] % Neutrale Stabilitaetskurven
	\begin{center}
	%\input{fig/neutral}
		\begin{picture}(188,132)(0,0)
		%\drawline(0,0)(0,130)(188,130)(188,0)(0,0)
		\put(161,-3){$\alpha$}
		\put(-12,96){$R$}
		\thinlines
		\drawline(16,10)(16,130)
		\drawline(58.5,10)(58.5,130)
		\drawline(101,10)(101,130)
		\drawline(143.5,10)(143.5,130)
		\drawline(186,10)(186,130)
		\drawline(16,10)(186,10)
		\drawline(16,30)(186,30)
		\drawline(16,50)(186,50)
		\drawline(16,70)(186,70)
		\drawline(16,90)(186,90)
		\drawline(16,110)(186,110)
		\drawline(16,130)(186,130)
		\put(14,4){\tiny 0}
		\put(56.5,4){\tiny 5}
		\put(97,4){\tiny 10}
		\put(139.5,4){\tiny 15}
		\put(182,4){\tiny 20}
		\put(0,8){\makebox(15,4)[r]{\tiny 0}}
		\put(0,28){\makebox(15,4)[r]{\tiny 500}}
		\put(0,48){\makebox(15,4)[r]{\tiny 1000}}
		\put(0,68){\makebox(15,4)[r]{\tiny 1500}}
		\put(0,88){\makebox(15,4)[r]{\tiny 2000}}
		\put(0,108){\makebox(15,4)[r]{\tiny 2500}}
		\put(0,128){\makebox(15,4)[r]{\tiny 3000}}
		\thicklines
		\drawline(69.0888,130)(69.6681,109.636)(70.2475,97.018)(70.7659,87.9636)(71.4367,78.218)(72.687,69.0908)(74.2422,60.8)(76.3157,54.2544)(78.1758,50.7636)(80.7372,47.9273)(83.3901,46.1454)(86.9273,45.2)(90.6475,45.0546)(94.4592,45.6727)(97.6609,46.4727)(103.547,48.5091)(108.242,50.3636)(113.67,52.7636)(128.855,60.182)(146.969,69.6)(186,90.8)
		\drawline(28.7157,130)(29.8745,110.909)(32.1004,86.4364)(34.3875,69.2364)(35.6376,63.018)(36.9488,57.2364)(38.2296,52.582)(39.6323,48.5818)(41.3704,44.3273)(43.4134,40.9454)(46.4628,37.3091)(49.5426,35.2)(52.8053,34.1818)(56.1901,34.4)(60.0018,35.6)(64.0574,37.4546)(68.5399,38.6909)(73.3273,39.7818)(78.6941,40.7273)(86.0736,42.8364)(92.9955,45.3091)(102.54,49.3818)(114.31,54.982)(125.075,60.4)(136.143,65.9272)(160.141,78.4364)(172.522,85.0544)(186,92.1456)
		\drawline(63.5391,130)(64.9112,117.382)(67.0152,105.454)(68.5094,96.9456)(71.0404,87.5272)(74.5165,78.5092)(77.9928,72.4728)(81.0422,69.0544)(84.122,66.7272)(88.1776,64.6908)(93.087,63.4544)(98.5758,63.2)(104.583,63.782)(114.066,65.8544)(127.819,70.3272)(142.486,76.1456)(158.099,83.0908)(186,96.582)
		\drawline(84.9758,130)(86.348,123.782)(88.1165,117.164)(90.4951,110.546)(92.5076,105.964)(95.374,100.982)(98.4538,96.7636)(103.211,92.2544)(108.303,89.3456)(113.213,87.4908)(117.848,86.6544)(123.092,86.5456)(132.209,87.1636)(141.48,88.9092)(152.701,91.9636)(165.448,96.3272)(186,104.509)
		\drawline(112.176,130)(115.896,124.436)(119.433,120.218)(122.787,117.273)(126.05,115.054)(130.106,112.873)(134.771,111.164)(139.65,110.073)(144.651,109.454)(149.804,109.309)(156.178,109.782)(163.954,110.8)(176.517,113.454)(186,116.254)

		\put(74.5,68){\begin{rotate}{-73}{\whiten\vrule width 19pt height 4pt}\end{rotate}}
		\put(74.5,68){\begin{rotate}{-73}{\tiny $n = 0$}\end{rotate}}
		\put(38,58){\begin{rotate}{-70}{\whiten\vrule width 16pt height 4pt}\end{rotate}}
		\put(38,58){\begin{rotate}{-70}{\tiny $n = 1$}\end{rotate}}
		\put(90,66){\begin{rotate}{0}{\whiten\vrule width 19pt height 4pt}\end{rotate}}
		\put(90,66){\begin{rotate}{0}{\tiny $n = 2$}\end{rotate}}
		\put(112,89){\begin{rotate}{0}{\whiten\vrule width 19pt height 4pt}\end{rotate}}
		\put(112,89){\begin{rotate}{0}{\tiny $n = 3$}\end{rotate}}
		\put(139,112){\begin{rotate}{0}{\whiten\vrule width 19pt height 4pt}\end{rotate}}
		\put(139,112){\begin{rotate}{0}{\tiny $n = 4$}\end{rotate}}
		\end{picture}
	\caption{\label{fig:neutral}Kurven neutraler Stabilit{\"at} f{\"ur} verschiedene Werte der Winkelzahl $n$. Unterhalb der Kurven herrscht zu jedem Zeitpunkt vollst\"andige Stabilit\"at f\"ur jede Wellenzahl der St\"orungen. Symmetrische und antimetrische St\"orungen dominieren das Instabilit\"atsverhalten gegen\"uber St\"orungen h\"oherer Winkelzahlen.}
	\end{center}
\end{figure}


\paragraph{Maximale Anfachungsraten}
Die maximalen Anfachungsraten und die zugeh\"origen Frequenzen symmetrischer und antimetrischer St\"or"-ungen zeigen die Abbildungen \ref{fig:sigma} und \ref{fig:sigmad} in verschiedenen Zeitskalen.
Da sich die dimensionslosen Zeitvariablen auf die konvektive Zeitskala $L/V$ beziehen, k\"onnen wir Aufklingraten und Frequenzen in der diffusiven Zeitskala $L^2/\nu$ darstellen, indem wir sie mit der Rey"-noldsschen Zahl multiplizieren.
Entsprechend erhalten wir die Gr\"o\ss en in der Einheit der Grenzschichtzeit $\delta/V$ durch Division mit $\alpha$.\\
Am aussagekr\"aftigsten ist der Bezug der Zeitkonstante auf die Oszillationszeit $1/\Omega$.
Dies gelingt, indem wir die Anfachungsrate $\sigma$ mit $R/\alpha^2$ multiplizieren.
Das untere Bild der Abbildung \ref{fig:sigmad} zeigt die Aufklingrate in diesem Bezug.
Da die maximale Verst\"arkung nur zu einem kurzen Zeitpunkt auftritt, k\"onnen wir mit dieser Darstellung absch\"atzen, um welchen Faktor St\"orungen w\"ahrend dieser Zeit verst\"arkt werden.
Wenn wir annehmen, die Verst\"arkung sei w\"ahrend eines Achtels der Periode konstant dem maximalen Wert, dann w\"achst eine St\"orung w\"ahrend des Zeitintervalls $\Delta\tau$ um den Faktor $e^{\Delta\tau R\sigma/\alpha^2}$.
In Zahlenwerten ergeben sich maximale Verst\"arkungsfaktoren von 2.2, 4.8, 10.5 und 23 f\"ur die Kurven der unteren Abbildung mit Werten von 1, 2, 3 und 4 w\"ahrend des Zeitintervalls $\Delta\tau=\pi/4$.\\

Wir erkennen an den Abbildungen, da\ss\ f\"ur steigende Rey"-noldssche Zahlen $R$ die maximalen Anfachungsraten erwartungsgem\"a\ss\ steigen.\\
F\"ur kleine Womersleysche Zahlen $\alpha<2$ steigt die Grenzkurve der quasista"-tisch"-en Instabilit\"at in der $R$-$\alpha$-Ebene stark an.
Im Grenzfall $\alpha\to0$ entspricht die Theorie dem klassischen hydrodynamischen Problem der station\"aren Rohrstr\"omung, welche nach der linearen Stabilit\"atstheorie keinen instabilen Bereich besitzt.\\
Im Hochfrequenzlimes f\"ur gro\ss e Womersleysche Zahlen ab etwa $\alpha>15$ haben die Eigenfunktionen mit maximaler Verst\"arkung Grenzschichtcharakter.
Die Anfachungsrate h\"angt hierbei nur von der Grenzschicht-Rey"-noldszahl $R/\alpha$ ab.\\
Die Kontourlinien der maximalen Aufklingrate in der Oszillationszeitskala haben f\"ur Werte der Womersleyschen Zahl $\alpha$ von 8 f\"ur symmetrische und 4 f\"ur antimetrische St\"orungen ein Minimum.
Das bedeutet, da\ss\ bei konstanter Reynoldscher Zahl St\"orungen mit diesen Womersleyschen Zahlen am deutlichsten angefacht werden.\\

\paragraph{Wellenzahl}
Die Wellenzahl $k$, welche in (\mbox{\ref{eq:sigmamax}}) zu einer maximalen Anfachungsrate f\"uhrt, ist in Abbildung \ref{fig:k} dargestellt.
Im oberen Bild erkennen wir den relevanten Bereich der Wellenzahlen, welche mit zunehmender Womersleyscher Zahl ansteigen.
Im untersuchten Parameterbereich liegt er zwischen 1 und 7.\\
W\"ahlen wir als L\"angenskala statt des Radius $L$ die Grenzschichtdicke $\delta$, dann erkennen wir, da\ss\ f\"ur gro\ss e Womersleysche Zahlen $\alpha>10$ die Wellenzahl weitgehend konstant ist.
Die Wellenl\"ange, welche zu maximalen Verst\"arkungen f\"uhrt, ist in diesem Fall proportional zur Grenzschichtdicke.
Das untere Bild zeigt diesen Zusammenhang.
Die Wellenzahl in der L\"angenskala der Grenzschichtdicke $k/\alpha$ nimmt im Hochfrequenzlimes $\alpha\to\infty$ etwa den Wert $1/\sqrt8$ an.
Damit hat die Wellenl\"ange einen Wert von ungef\"ahr $4\sqrt{2}\pi\delta$.\\

\paragraph{Aktionsradius}
Den Grenzschichtcharakter der St\"orungen erkennen wir auch an der Verteilung der Eigenfunktionen.
Die zu dem Eigenwert mit maximaler Anfachung geh\"orende Eigenfunktion des Systems (\mbox{\ref{eq:rwp}}) bezeichnen wir mit $\mathfrak{v}^*$.
Wir definieren den Aktionsradius $\varrho$ als \glqq Schwerpunkt\grqq{} ihres Betrages
\begin{equation}
	\varrho = \int_0^1 r|\mathfrak{v}^*|^2\,\textrm{d}r\,\Big/\int_0^1|\mathfrak{v}^*|^2\,\textrm{d}r.
\end{equation}
In Abbildung \ref{fig:rho} erkennen wir, da\ss\ sich der Aktionsradius mit steigenden Womersleyschen Zahlen dem Wert 1 n\"ahert.
Das bedeutet da\ss\ f\"ur diese F\"alle wandnahe St\"orungsgeschwindigkeiten am meisten verst\"arkt werden.\\
Das untere Bild zeigt den Abstand dieses Schwerpunktes von der Wand, gemessen in Einheiten der Grenzschichtdicke.
Die Umrechnung auf die Grenzschichtdicke erfolgt durch Multiplikation mit $\alpha$.
Der Wandabstand $(1-\varrho)\alpha$ nimmt f\"ur gro\ss e Womersleysche Zahlen Werte zwischen 3 und 4 ein.\\

\paragraph{Zeitparameter}
In der Definition der maximalen Aufklingrate (\mbox{\ref{eq:sigmamax}}) wird neben der Wellenzahl auch die Zeit $\tau$ variiert.
Der Zeitparameter, der zu einer maximalen Verst\"arkung f\"uhrt, wird in Abbildung \ref{fig:t} im oberen Bild dargestellt.
Die st\"arksten Instabilt\"aten treten zu den Zeitpunkten der Flu\ss umkehr auf, welche nach (\mbox{\ref{eq:baseflow-barV}}) bei ganzzahligen Vielfachen von $\tau=\pi$ stattfinden.
Mit steigender Rey"-noldsscher Zahl, verschiebt sich der Zeitpunkt etwas vor die Zeit der Flu\ss umkehr.\\

W\"ahrend einer Periode hat keiner der Eigenwerte dauerhaft einen positiven Realteil $\sigma>0$.
Im rechten Bild der Abbildung \ref{fig:eigenvalue} verfolgen wir den zeitlichen Verlauf des Eigenwertes mit der Wellenzahl, welche zur maximalen Verst\"arkung geh\"ort.
Wir definieren die Gr\"o\ss e $\Delta t$, indem wir die Zeitdauer, in welcher der Eigenwert eine positive Aufklingrate besitzt, zur Periodendauer ins Verh\"altnis setzen.
Die Gr\"o\ss e ist im unteren Bild der Abbildung \ref{fig:t} aufgetragen.
Im untersuchten Parameterbereich liegt ihr Wert unter 50\%.\\

\paragraph{Quasistatikparameter}
Die Berechnungen der quasistatisch"-en Theorie fu\ss en auf der a priori getroffenen Annahme $\varepsilon\ll1$ mit dem Parameter $\varepsilon$ aus (\mbox{\ref{eq:epsilon}}).
Abbildung \ref{fig:epsilon} zeigt den Kehrwert des Quasistatikparameters im Berechnungsgebiet.
F\"ur Womersleysche Zahlen zwischen 5 und 10 nimmt $\varepsilon$ Werte um 1 ein.
Daher sind die Annahmen, unter welchen die Berechnungen durchgef\"uhrt werden, in diesem Bereich schlecht erf\"ullt.\\
Der Zeitskala der St\"orungen im Quasistatikparameter setzt sich nach (\mbox{\ref{eq:epsilon}}) aus einer Kombination der Aukfklingrate und der Frequenz zusammen.
Dieses Ma\ss\ wird mit der Zeitskala der Grundstr\"omung ins Verh\"altnis gesetzt.
F\"ur gro\ss e Womersleysche Zahlen $\alpha>15$ dominiert der Frequenzanteil der St\"orungen.
Die Quasistatikbedingung wird mit steigenden Womersleyschen Zahlen besser erf\"ullt.\\
Den Einflu\ss\ der Aufklingrate an der zeitlichen \"Anderung der St\"orungen haben wir bereits im unteren Bild der Abbildung \ref{fig:sigmad} dargestellt.
Die dort aufgetragene Gr\"o\ss e (die Aufklingrate in der Oszillationszeitskala) entspricht dem Quasistatikparameter, falls wir die Frequenz der St\"orung $\omega$ nicht ber\"ucksichtigen.
In den Bereichen der Maxima der Aufklingrate steigt auch die Gr\"o\ss e $1/\varepsilon$.\\
Wir k\"onnen hiermit drei Gebiete des Quasistatikparameters in der $R$-$\alpha$-Ebene ausmachen.
Im langsam oszillierenden Bereich des Instabilit\"atsgebietes steigt die Aufklingrate mit steigender Reynoldszahl, und sichert damit die Annahme der Quasistatik.
Im hochfrequenten Bereich gro\ss er Womersleyscher Zahlen \"ubertrifft die Frequenz der instabilsten St\"orungen die Oszillationsfrequenz der Grundstr\"omung, was ebenfalls zu einer Verfestigung der Annahme f\"uhrt.
Die Aufklingrate der St\"orungen spielt in diesem Bereich eine untergeordnete Rolle.
Im dazwischen liegenden Bereich \"andern sich weder Amplitude noch Frequenz der instabilsten St\"orung schneller als die Grundstr\"omung selbst.
Daher ist die quasistatische Theorie in diesem Bereich nur bedingt aussagekr\"aftig.\\

\begin{figure} % Sigma, Omega
	\begin{center}
	\begin{picture}(275,135)(0,0)
	\put(0,78){$R$}
	\put(112,0){$\alpha$}
	\put(250,0){$\alpha$}
	\put(16,114.5){$n=0$}
	\put(244,114.5){$n=1$}
	\put(132,116){ $R\sigma$ }
	%\input{fig/Sigma.l}
		\thinlines
		\drawline(16,10)(16,112)
		\drawline(56,10)(56,112)
		\drawline(96,10)(96,112)
		\drawline(136,10)(136,112)
		\drawline(16,10)(136,10)
		\drawline(16,30.4)(136,30.4)
		\drawline(16,50.8)(136,50.8)
		\drawline(16,71.2)(136,71.2)
		\drawline(16,91.6)(136,91.6)
		\drawline(16,112)(136,112)
		\put(14,4){\tiny 5}
		\put(52,4){\tiny 10}
		\put(92,4){\tiny 15}
		\put(132,4){\tiny 20}
		\put(0,8){\makebox(15,4)[r]{\tiny 500}}
		\put(0,28.4){\makebox(15,4)[r]{\tiny 1000}}
		\put(0,48.8){\makebox(15,4)[r]{\tiny 1500}}
		\put(0,69.2){\makebox(15,4)[r]{\tiny 2000}}
		\put(0,89.6){\makebox(15,4)[r]{\tiny 2500}}
		\put(0,110){\makebox(15,4)[r]{\tiny 3000}}
		\thicklines
		\drawline(35.2933,112)(36.4413,104.545)(37.9336,96.6074)(39.3973,90.6726)(41.3776,84.664)(43.5301,80.0275)(46.4287,75.5766)(49.2986,72.7949)(52.6852,70.7549)(55.8995,69.6423)(59.0856,69.1229)(63.7344,68.752)(70.8808,69.5309)(79.4328,71.5337)(85.948,73.3514)(94.6152,76.4669)(106.238,80.9177)(121.363,87.5571)(136,94.5303)
		\drawline(41.9516,112)(43.6449,106.177)(45.7399,100.947)(47.7776,97.3863)(49.7866,94.6045)(51.9677,91.9709)(54.6368,89.8937)(58.54,87.7424)(62.5576,86.4816)(67.1784,85.5914)(72.5456,85.4429)(78.8016,85.9994)(89.076,88.0023)(101.015,91.4515)(117.718,97.2749)(136,104.878)
		\drawline(30.5005,112)(31.5336,97.3863)(33.6861,79.0634)(34.7193,73.3143)(36.1542,67.6394)(37.9336,61.8903)(39.9713,57.736)(42.7838,53.9155)(45.3668,51.8755)(48.2654,50.4291)(51.2215,49.8726)(54.4646,49.6503)(58.7408,49.9097)(62.816,50.7257)(68.728,52.2464)(74.4392,54.1383)(91.1712,61.1114)(114.045,72.2755)(136,83.6256)
		\drawline(52.8287,112)(55.4117,108.922)(58.1672,106.399)(62.6728,103.618)(66.8344,102.022)(70.536,101.17)(74.784,100.613)(79.4328,100.576)(86.2064,100.835)(96.8536,102.579)(106.985,105.064)(119.928,109.181)(127.935,111.963)
		\drawline(25.9659,112)(26.5112,91.2291)(27.0565,78.3584)(27.5444,69.1229)(28.1758,59.1824)(29.3525,49.8726)(30.8162,41.416)(32.7677,34.7395)(34.5184,31.1789)(36.9291,28.2858)(39.426,26.4683)(42.7551,25.504)(46.2565,25.3557)(49.8439,25.9862)(52.8574,26.8022)(58.3968,28.8793)(62.816,30.7709)(67.9248,33.2189)(82.2168,40.7856)(99.2648,50.392)(136,72.016)
	%\input{fig/Sigma.r}
		\thinlines
		\drawline(149,10)(149,112)
		\drawline(179,10)(179,112)
		\drawline(209,10)(209,112)
		\drawline(239,10)(239,112)
		\drawline(269,10)(269,112)
		\drawline(149,10)(269,10)
		\drawline(149,30.4)(269,30.4)
		\drawline(149,50.8)(269,50.8)
		\drawline(149,71.2)(269,71.2)
		\drawline(149,91.6)(269,91.6)
		\drawline(149,112)(269,112)
		\put(147,4){\tiny 0}
		\put(177,4){\tiny 5}
		\put(205,4){\tiny 10}
		\put(235,4){\tiny 15}
		\put(265,4){\tiny 20}
		\thicklines
		\drawline(168.781,112)(170.352,100.205)(171.881,90.6726)(174.076,82.0675)(176.724,74.5383)(180.921,66.1926)(184.193,60.4806)(187.314,55.9184)(189.725,53.248)(192.889,50.8)(195.816,49.5389)(199.325,48.9456)(203.221,49.3536)(207.353,50.1696)(211.83,51.7274)(220.354,55.3251)(232.3,61.6675)(245.473,69.6051)(257.118,77.0605)(269,85.0349)
		\drawline(181.524,112)(184.021,101.874)(186.582,93.3805)(188.153,88.7069)(189.961,84.4784)(192.501,79.6195)(195.321,75.9846)(197.99,73.5366)(200.724,72.016)(203.845,71.0515)(206.901,70.4954)(213.251,70.6806)(218.568,71.6823)(226.489,73.9817)(234.754,77.3943)(245.452,82.4755)(257.915,89.3005)(269,95.9766)
		\drawline(191.124,112)(192.738,107.29)(195.321,101.874)(198.205,97.2749)(200.422,94.864)(202.532,93.0835)(205.093,91.3776)(208.731,89.7085)(213.036,88.744)(217.836,88.3731)(223.346,88.5955)(231.87,90.2275)(239.64,92.7126)(247.712,95.9024)(255.87,99.5376)(262.758,103.024)(269,106.288)
		\drawline(204.943,112)(207.676,109.441)(210.948,107.438)(215.317,105.472)(219.343,104.322)(223.82,103.691)(228.06,103.543)(232.063,103.877)(238.435,105.027)(244.828,106.807)(251.156,108.847)(259.271,112)
		\drawline(157.976,112)(158.794,92.5274)(160.365,67.5651)(161.979,50.0211)(162.862,43.6784)(163.787,37.7811)(164.691,33.0336)(165.682,28.9534)(166.909,24.6138)(168.351,21.1643)(170.503,17.4553)(172.677,15.304)(174.98,14.2654)(177.369,14.488)(180.06,15.712)(182.923,17.6037)(186.087,18.8647)(189.466,19.9774)(193.255,20.9418)(198.464,23.0931)(203.35,25.6153)(210.087,29.7694)(218.395,35.4816)(225.994,41.008)(233.807,46.6457)(250.747,59.4051)(259.486,66.1555)(269,73.3885)
	%\input{fig/Sigma.label}
		\put(80,38){\begin{rotate}{25}{\whiten\vrule width 4pt height 4pt}\end{rotate}}
		\put(80,38){\begin{rotate}{25}{\tiny 0}\end{rotate}}
		\put(80,55){\begin{rotate}{23}{\whiten\vrule width 8pt height 4pt}\end{rotate}}
		\put(80,55){\begin{rotate}{23}{\tiny 50}\end{rotate}}
		\put(80,70){\begin{rotate}{20}{\whiten\vrule width 12pt height 4pt}\end{rotate}}
		\put(80,70){\begin{rotate}{20}{\tiny 100}\end{rotate}}
		\put(80,84){\begin{rotate}{14}{\whiten\vrule width 12pt height 4pt}\end{rotate}}
		\put(80,84){\begin{rotate}{14}{\tiny 150}\end{rotate}}
		\put(80,99){\begin{rotate}{5}{\whiten\vrule width 12pt height 4pt}\end{rotate}}
		\put(80,99){\begin{rotate}{5}{\tiny 200}\end{rotate}}
		\put(220,35){\begin{rotate}{30}{\whiten\vrule width 4pt height 4pt}\end{rotate}}
		\put(220,35){\begin{rotate}{30}{\tiny 0}\end{rotate}}
		\put(220,53){\begin{rotate}{25}{\whiten\vrule width 8pt height 4pt}\end{rotate}}
		\put(220,53){\begin{rotate}{25}{\tiny 50}\end{rotate}}
		\put(220,70){\begin{rotate}{20}{\whiten\vrule width 12pt height 4pt}\end{rotate}}
		\put(220,70){\begin{rotate}{20}{\tiny 100}\end{rotate}}
		\put(220,86){\begin{rotate}{8}{\whiten\vrule width 12pt height 4pt}\end{rotate}}
		\put(220,86){\begin{rotate}{8}{\tiny 150}\end{rotate}}
		\put(220,102){\begin{rotate}{00}{\whiten\vrule width 12pt height 4pt}\end{rotate}}
		\put(220,102){\begin{rotate}{00}{\tiny 200}\end{rotate}}
	\end{picture}
	\begin{picture}(275,135)(0,0)
	\put(0,78){$R$}
	\put(112,0){$\alpha$}
	\put(250,0){$\alpha$}
	\put(16,114.5){$n=0$}
	\put(244,114.5){$n=1$}
	\put(132,116){ $R\omega$ }
	%\input{fig/Omega.l}
		\thinlines
		\drawline(16,10)(16,112)
		\drawline(56,10)(56,112)
		\drawline(96,10)(96,112)
		\drawline(136,10)(136,112)
		\drawline(16,10)(136,10)
		\drawline(16,30.4)(136,30.4)
		\drawline(16,50.8)(136,50.8)
		\drawline(16,71.2)(136,71.2)
		\drawline(16,91.6)(136,91.6)
		\drawline(16,112)(136,112)
		\put(14,4){\tiny 5}
		\put(52,4){\tiny 10}
		\put(92,4){\tiny 15}
		\put(132,4){\tiny 20}
		\put(0,8){\makebox(15,4)[r]{\tiny 500}}
		\put(0,28.4){\makebox(15,4)[r]{\tiny 1000}}
		\put(0,48.8){\makebox(15,4)[r]{\tiny 1500}}
		\put(0,69.2){\makebox(15,4)[r]{\tiny 2000}}
		\put(0,89.6){\makebox(15,4)[r]{\tiny 2500}}
		\put(0,110){\makebox(15,4)[r]{\tiny 3000}}
		\thicklines
		\drawline(76.0752,112)(66.2312,88.4103)(56.0144,64.7091)(46.4,41.1194)(40.9758,25.912)
		\drawline(85.144,112)(76.7064,90.8211)(63.0744,55.9555)(55.6413,35.2589)(52.7426,26.8393)
		\drawline(100.556,112)(92.176,87.6314)(83.1064,60.0726)(79.1752,46.8685)(76.8792,37.9296)
		\drawline(126.874,112)(117.919,80.0275)(114.16,64.264)(112.983,58.6263)
		\drawline(40.5453,112)(34.6619,97.8685)(31.7632,91.7485)(30.2709,89.3005)(29.4673,88.8925)(28.4914,89.152)(27.5157,90.6726)(26.913,93.232)(26.7695,94.9011)(26.339,98.0537)
		\drawline(53.8619,112)(42.8986,87.5943)(37.0726,72.7949)(32.5668,62.0755)(31.1031,59.1456)(30.0699,57.6246)(29.3238,57.0685)(28.4628,56.7715)
		\drawline(62.3568,112)(50.6475,85.8509)(43.7022,69.6794)(33.3704,44.4576)(31.0744,40.192)
		\drawline(69.1872,112)(49.5857,67.4537)(44.1902,54.9914)(38.7946,40.9337)(34.8054,30.9194)
		\drawline(65.9728,112)(57.9944,94.456)(51.4511,80.5097)(42.7265,59.2937)(38.5363,48.1664)(34.6045,38.56)(32.7677,34.8137)
		\drawline(25.9659,112)(26.5112,91.2291)(27.0565,78.3584)(27.5444,69.1229)(28.1758,59.1824)(29.3525,49.8726)(30.8162,41.416)(32.7677,34.7395)(34.5184,31.1789)(36.9291,28.2858)(39.426,26.4683)(42.7551,25.504)(46.2565,25.3557)(49.8439,25.9862)(52.8574,26.8022)(58.3968,28.8793)(62.816,30.7709)(67.9248,33.2189)(82.2168,40.7856)(99.2648,50.392)(136,72.016)
	%\input{fig/Omega.r}
		\thinlines
		\drawline(149,10)(149,112)
		\drawline(179,10)(179,112)
		\drawline(209,10)(209,112)
		\drawline(239,10)(239,112)
		\drawline(269,10)(269,112)
		\drawline(149,10)(269,10)
		\drawline(149,30.4)(269,30.4)
		\drawline(149,50.8)(269,50.8)
		\drawline(149,71.2)(269,71.2)
		\drawline(149,91.6)(269,91.6)
		\drawline(149,112)(269,112)
		\put(147,4){\tiny 0}
		\put(177,4){\tiny 5}
		\put(205,4){\tiny 10}
		\put(235,4){\tiny 15}
		\put(265,4){\tiny 20}
		\thicklines
		\drawline(228.039,112)(213.703,63.7075)(208.15,44.1977)(204.448,30.5856)(203.436,25.6523)
		\drawline(240.394,112)(230.428,72.7577)(226.08,53.7303)(224.164,44.0125)(223.174,39.0051)
		\drawline(260.562,112)(256.257,90.5616)(252.383,69.7904)(250.747,59.4423)
		\drawline(214.241,112)(204.491,86.0736)(199.152,71.9789)(196.247,64.7834)(193.083,57.5504)(190.607,53.1737)(189.725,52.0611)(188.691,51.208)(187.486,50.6886)(186.216,51.0224)(184.796,52.1725)(183.612,53.5817)(181.352,57.9955)(179.35,62.8915)(178.209,65.896)(177.52,67.1943)(176.702,67.8989)(176.035,68.0845)(175.518,68.1216)(174.98,67.936)(174.421,67.3795)(173.969,66.712)(173.603,65.6736)(172.828,61.4451)(172.117,57.4023)(171.558,53.6189)(171.3,52.1354)(171.063,50.9485)(170.632,50.0583)(169.943,50.2806)(169.233,51.1337)(168.695,52.432)(168.006,54.3235)(166.715,58.0326)(165.208,66.8234)(163.163,83.4029)(162.065,97.2377)(161.183,112)
		\drawline(220.29,112)(209.893,79.9904)(199.712,50.0211)(196.57,40.6371)(193.319,31.5497)(191.167,26.7651)(189.122,23.464)(187.206,20.9789)(185.872,19.7178)(184.537,19.3469)(183.224,19.5323)(181.567,20.0887)(180.168,20.2742)(178.769,20.4226)(177.865,20.3113)(177.283,20.0517)(176.767,19.4582)(176.638,18.6422)(176.616,17.4923)(176.681,16.7134)(176.853,15.4894)(177.176,14.2654)
		\drawline(207.741,112)(200.272,94.9011)(195.945,85.5171)(193.083,80.3245)(192.071,78.952)(191.145,77.9504)(190.069,77.2086)(189.036,77.0976)(187.723,77.5424)(186.647,78.5069)(184.968,81.7337)(183.31,88.2246)(181.459,96.9411)(179.587,112)
		\drawline(195.945,112)(194.589,109.181)(193.298,107.771)(192.114,107.067)(190.457,107.215)(189.143,108.179)(188.11,109.626)(186.948,112)
		\drawline(211.293,112)(204.254,94.4931)(197.56,78.6183)(195.149,73.0176)(193.276,69.4566)(191.662,66.8976)(190.284,65.3395)(189.531,64.5606)(188.799,64.0416)(188.239,63.8931)(187.615,64.0045)(186.754,64.3754)(186.216,65.0057)(184.903,67.0086)(183.633,69.6423)(181.976,75.0205)(179.759,85.2576)(177.8,93.5657)(176.3,102.171)(174.92,112)
		\drawline(157.976,112)(158.794,92.5274)(160.365,67.5651)(161.979,50.0211)(162.862,43.6784)(163.787,37.7811)(164.691,33.0336)(165.682,28.9534)(166.909,24.6138)(168.351,21.1643)(170.503,17.4553)(172.677,15.304)(174.98,14.2654)(177.369,14.488)(180.06,15.712)(182.923,17.6037)(186.087,18.8647)(189.466,19.9774)(193.255,20.9418)(198.464,23.0931)(203.35,25.6153)(210.087,29.7694)(218.395,35.4816)(225.994,41.008)(233.807,46.6457)(250.747,59.4051)(259.486,66.1555)(269,73.3885)
	%\input{fig/Omega.label}
		\put(53,80){\begin{rotate}{68}{\whiten\vrule width 4pt height 4pt}\end{rotate}}
		\put(53,80){\begin{rotate}{68}{\tiny 0}\end{rotate}}
		\put(41,60){\begin{rotate}{68}{\whiten\vrule width 10pt height 4pt}\end{rotate}}
		\put(41,60){\begin{rotate}{68}{\tiny -30}\end{rotate}}
		\put(47,57){\begin{rotate}{68}{\whiten\vrule width 8pt height 4pt}\end{rotate}}
		\put(47,57){\begin{rotate}{68}{\tiny 30}\end{rotate}}
		\put(63,76){\begin{rotate}{68}{\whiten\vrule width 11pt height 4pt}\end{rotate}}
		\put(63,76){\begin{rotate}{68}{\tiny 100}\end{rotate}}
		\put(40,76){\begin{rotate}{68}{\whiten\vrule width 14pt height 4pt}\end{rotate}}
		\put(40,76){\begin{rotate}{68}{\tiny -100}\end{rotate}}
		\put(73,76){\begin{rotate}{68}{\whiten\vrule width 11pt height 4pt}\end{rotate}}
		\put(73,76){\begin{rotate}{68}{\tiny 200}\end{rotate}}
		\put(35,95){\begin{rotate}{68}{\whiten\vrule width 14pt height 4pt}\end{rotate}}
		\put(35,95){\begin{rotate}{68}{\tiny -200}\end{rotate}}
		\put(90,76){\begin{rotate}{72}{\whiten\vrule width 12pt height 4pt}\end{rotate}}
		\put(90,76){\begin{rotate}{72}{\tiny 400}\end{rotate}}
		\put(119,76){\begin{rotate}{76}{\whiten\vrule width 12pt height 4pt}\end{rotate}}
		\put(119,76){\begin{rotate}{76}{\tiny 800}\end{rotate}}
		\put(198,75){\begin{rotate}{68}{\whiten\vrule width 4pt height 4pt}\end{rotate}}
		\put(198,75){\begin{rotate}{68}{\tiny 0}\end{rotate}}
		\put(198,65){\begin{rotate}{68}{\whiten\vrule width 8pt height 4pt}\end{rotate}}
		\put(198,65){\begin{rotate}{68}{\tiny 30}\end{rotate}}
		\put(197,85){\begin{rotate}{68}{\whiten\vrule width 10pt height 4pt}\end{rotate}}
		\put(197,85){\begin{rotate}{68}{\tiny -30}\end{rotate}}
		\put(185,102){\begin{rotate}{0}{\whiten\vrule width 14pt height 4pt}\end{rotate}}
		\put(185,102){\begin{rotate}{0}{\tiny -100}\end{rotate}}
		\put(210,76){\begin{rotate}{70}{\whiten\vrule width 12pt height 4pt}\end{rotate}}
		\put(210,76){\begin{rotate}{70}{\tiny 100}\end{rotate}}
		\put(220,76){\begin{rotate}{73}{\whiten\vrule width 12pt height 4pt}\end{rotate}}
		\put(220,76){\begin{rotate}{73}{\tiny 200}\end{rotate}}
		\put(233,76){\begin{rotate}{74}{\whiten\vrule width 12pt height 4pt}\end{rotate}}
		\put(233,76){\begin{rotate}{74}{\tiny 400}\end{rotate}}
		\put(255.5,76){\begin{rotate}{80}{\whiten\vrule width 12pt height 4pt}\end{rotate}}
		\put(255.5,76){\begin{rotate}{80}{\tiny 800}\end{rotate}}
	\end{picture}
	\caption{\label{fig:sigma}Maximaler Aufklingfaktor $\sigma$ und zugeh\"orige Frequenz $\omega$ in diffusiver Zeitskala.}
	\end{center}
\end{figure}

\begin{figure} % Sigma/alpha
	\begin{center}
	\begin{picture}(275,135)(0,0)
	\put(0,78){$R$}
	\put(112,0){$\alpha$}
	\put(250,0){$\alpha$}
	\put(16,114.5){$n=0$}
	\put(244,114.5){$n=1$}
	\put(132,116){ $\sigma/\alpha$ }
	%\input{fig/sigmad.l}
		\thinlines
		\drawline(16,10)(16,112)
		\drawline(56,10)(56,112)
		\drawline(96,10)(96,112)
		\drawline(136,10)(136,112)
		\drawline(16,10)(136,10)
		\drawline(16,30.4)(136,30.4)
		\drawline(16,50.8)(136,50.8)
		\drawline(16,71.2)(136,71.2)
		\drawline(16,91.6)(136,91.6)
		\drawline(16,112)(136,112)
		\put(14,4){\tiny 5}
		\put(52,4){\tiny 10}
		\put(92,4){\tiny 15}
		\put(132,4){\tiny 20}
		\put(0,8){\makebox(15,4)[r]{\tiny 500}}
		\put(0,28.4){\makebox(15,4)[r]{\tiny 1000}}
		\put(0,48.8){\makebox(15,4)[r]{\tiny 1500}}
		\put(0,69.2){\makebox(15,4)[r]{\tiny 2000}}
		\put(0,89.6){\makebox(15,4)[r]{\tiny 2500}}
		\put(0,110){\makebox(15,4)[r]{\tiny 3000}}
		\thicklines
		\drawline(29.6108,112)(30.0986,98.3504)(30.8449,80.8435)(32.079,65.8217)(33.3417,55.7331)(34.4897,50.7257)(36.0394,45.6074)(37.1587,42.9737)(38.6798,40.3034)(40.7749,37.9664)(42.4394,37.1136)(44.104,36.6314)(46.3713,36.2234)(49.1265,36.4457)(52.4556,37.336)(56.732,39.0423)(63.1032,42.4176)(79.2608,52.7657)(101.532,69.0486)(136,95.272)
		\drawline(31.5623,112)(32.0215,99.3891)(32.6242,88.8925)(33.2269,78.8035)(34.4036,68.9376)(35.1785,63.1885)(36.4699,57.2909)(37.4458,53.656)(38.7946,50.5405)(40.6027,47.2765)(43.1857,45.0509)(45.3955,44.0125)(47.6628,43.864)(50.131,44.1977)(52.0825,44.5686)(54.9812,45.7555)(60.2048,48.3891)(66.1736,52.0983)(75.6736,59.0714)(92.4344,72.6096)(110.974,88.1875)(136,110.442)
		\drawline(27.7166,112)(28.7498,84.5155)(29.6682,67.4909)(30.4718,58.5149)(31.5623,51.3194)(32.9112,44.0496)(34.3462,39.4131)(36.2117,35.5184)(38.6798,32.6624)(40.6601,31.2903)(43.4153,30.3629)(45.4529,30.1033)(49.2699,30.5856)(53.8045,31.9577)(59.4008,34.2576)(66.6616,37.9664)(76.8504,44.0125)(92.6064,53.8416)(136,82.624)
		\drawline(36.9866,112)(37.7902,100.873)(38.9381,91.1177)(39.9713,84.256)(41.1193,79.4343)(41.8942,76.2816)(44.2475,71.0886)(45.6825,69.0486)(47.9785,67.4166)(49.9301,66.7491)(52.226,66.8976)(54.2062,67.4166)(57.2488,68.5664)(61.4384,71.2743)(68.068,76.7635)(74.2384,82.5126)(83.8816,92.6016)(90.3104,99.5005)(95.9928,106.066)(101.274,112)
		\drawline(41.3776,112)(42.5542,103.84)(44.2475,96.1623)(45.1946,93.1577)(46.6583,90.0051)(47.6628,88.4103)(49.0117,86.7411)(50.791,85.7024)(52.6852,85.2944)(54.8664,85.48)(57.0472,86.1104)(61.5248,88.7069)(66.7768,93.3805)(71.656,98.2023)(76.2192,103.432)(83.336,112)
		\drawline(33.9731,112)(34.6045,99.1296)(35.7812,84.1075)(37.3597,72.424)(38.1919,68.0103)(39.1677,64.5235)(40.6601,60.1469)(42.3821,57.5504)(43.8458,55.6217)(45.9982,54.2864)(48.3516,53.7303)(50.791,53.8045)(52.7713,54.2125)(54.8664,54.9171)(57.5928,56.2896)(61.6104,58.8857)(70.2208,65.1914)(80.8104,74.4269)(94.328,87.2605)(119.01,112)
		\drawline(25.9659,112)(26.5112,91.2291)(27.0565,78.3584)(27.5444,69.1229)(28.1758,59.1824)(29.3525,49.8726)(30.8162,41.416)(32.7677,34.7395)(34.5184,31.1789)(36.9291,28.2858)(39.426,26.4683)(42.7551,25.504)(46.2565,25.3557)(49.8439,25.9862)(52.8574,26.8022)(58.3968,28.8793)(62.816,30.7709)(67.9248,33.2189)(82.2168,40.7856)(99.2648,50.392)(136,72.016)
	%\input{fig/sigmad.r}
		\thinlines
		\drawline(149,10)(149,112)
		\drawline(179,10)(179,112)
		\drawline(209,10)(209,112)
		\drawline(239,10)(239,112)
		\drawline(269,10)(269,112)
		\drawline(149,10)(269,10)
		\drawline(149,30.4)(269,30.4)
		\drawline(149,50.8)(269,50.8)
		\drawline(149,71.2)(269,71.2)
		\drawline(149,91.6)(269,91.6)
		\drawline(149,112)(269,112)
		\put(147,4){\tiny 0}
		\put(177,4){\tiny 5}
		\put(205,4){\tiny 10}
		\put(235,4){\tiny 15}
		\put(265,4){\tiny 20}
		\thicklines
		\drawline(179.048,112.037)(180.663,106.066)(182.083,100.613)(183.999,92.5645)(185.14,88.2617)(187.68,80.584)(189.251,77.6909)(190.715,75.4656)(192.2,74.2784)(193.535,73.648)(195.343,73.5737)(196.634,74.056)(198.873,75.3171)(201.369,77.6166)(204.77,81.5856)(210.625,89.4486)(218.374,101.318)(224.982,111.963)
		\drawline(159.547,112.037)(160.021,101.911)(160.882,84.1075)(161.57,72.5354)(162.69,59.6646)(163.508,51.4304)(164.67,43.4189)(165.768,36.8166)(167.016,31.8096)(168.178,27.7666)(169.707,24.0203)(170.891,22.0546)(172.505,20.3854)(174.033,19.4211)(175.97,19.0873)(179.479,20.5338)(183.074,22.2029)(188.369,23.7978)(192.566,24.9847)(197.517,26.8763)(204.146,30.6966)(211.12,35.7783)(219.3,42.0835)(229.007,49.984)(244.612,63.1514)(268.957,84.1817)
		\drawline(168.523,111.926)(169.19,102.505)(170.245,90.784)(171.644,81.1405)(172.548,76.6525)(173.56,73.3143)(174.27,71.4595)(175.217,69.976)(176.25,68.8634)(177.305,68.4183)(179.931,67.0457)(181.868,65.4509)(183.913,62.9657)(186.905,59.6646)(189.1,57.736)(190.779,57.0314)(192.437,56.6234)(194.546,56.8829)(197.323,58.0697)(201.477,61.408)(206.019,66.2669)(213.854,75.9475)(220.914,85.48)(229.395,97.1635)(239.576,112.037)
		\drawline(165.402,112.037)(166.155,99.6857)(166.93,87.2234)(167.942,76.096)(168.674,68.8634)(169.814,61.8531)(170.804,56.8086)(172.333,51.4675)(173.108,49.6874)(174.399,47.5731)(175.691,46.6086)(177.305,46.2006)(181.115,46.6457)(184.43,46.0525)(187.249,45.3475)(189.531,44.9395)(191.447,44.9766)(193.771,45.3104)(196.247,46.2749)(199.023,47.944)(201.778,50.2806)(204.857,53.0256)(207.805,56.104)(216.307,65.3766)(223.54,74.056)(230.385,82.3274)(240.157,94.3075)(253.846,112.037)
		\drawline(161.398,111.889)(162.065,96.7184)(162.711,85.2205)(163.465,73.5737)(164.541,60.5549)(165.94,49.984)(166.844,44.68)(167.813,39.6726)(169.104,34.7395)(169.922,32.1434)(171.257,29.5098)(172.333,27.9149)(173.474,26.8022)(175.454,25.8378)(177.628,25.8378)(179.608,26.4313)(182.062,27.3586)(184.451,28.1003)(186.604,28.6567)(190.457,29.3614)(194.245,30.6224)(197.474,32.1434)(201.994,34.7395)(207.16,38.8195)(211.529,42.4915)(222.033,51.616)(235.68,64.264)(251.393,79.5456)(269.107,96.9783)
		\drawline(163.185,112.111)(164.476,88.5955)(165.832,70.0131)(167.554,54.9914)(169.169,47.3875)(169.836,44.0496)(171.321,39.6726)(172.419,37.4474)(173.839,35.4074)(175.239,34.4429)(176.659,33.9606)(178.101,34.1834)(183.848,35.5555)(188.11,35.7783)(193.405,36.5943)(197.151,38.3376)(200.659,40.5629)(203.694,42.9366)(210.109,48.9085)(217.255,55.8816)(227.608,66.4154)(236.971,76.6896)(252.146,93.3805)(268.462,111.926)
		\drawline(189.38,112.111)(190.586,107.809)(191.447,105.25)(192.351,103.21)(193.513,101.281)(194.912,99.9456)(196.613,99.5376)(198.335,99.8714)(199.949,100.984)(202.467,103.543)(205.179,107.178)(208.473,112.037)
		\drawline(157.976,112)(158.794,92.5274)(160.365,67.5651)(161.979,50.0211)(162.862,43.6784)(163.787,37.7811)(164.691,33.0336)(165.682,28.9534)(166.909,24.6138)(168.351,21.1643)(170.503,17.4553)(172.677,15.304)(174.98,14.2654)(177.369,14.488)(180.06,15.712)(182.923,17.6037)(186.087,18.8647)(189.466,19.9774)(193.255,20.9418)(198.464,23.0931)(203.35,25.6153)(210.087,29.7694)(218.395,35.4816)(225.994,41.008)(233.807,46.6457)(250.747,59.4051)(259.486,66.1555)(269,73.3885)
	%\input{fig/sigmad.label}
		\put(103,51){\begin{rotate}{30}{\whiten\vrule width 4pt height 4pt}\end{rotate}}
		\put(103,51){\begin{rotate}{30}{\tiny 0}\end{rotate}}
		\put(233,44){\begin{rotate}{33}{\whiten\vrule width 4pt height 4pt}\end{rotate}}
		\put(233,44){\begin{rotate}{33}{\tiny 0}\end{rotate}}
		\put(100,57){\begin{rotate}{33}{\whiten\vrule width 16pt height 4pt}\end{rotate}}
		\put(100,57){\begin{rotate}{33}{\tiny 0.001}\end{rotate}}
		\put(95,62){\begin{rotate}{36}{\whiten\vrule width 17pt height 4pt}\end{rotate}}
		\put(95,62){\begin{rotate}{36}{\tiny 0.002}\end{rotate}}
		\put(90,69){\begin{rotate}{40}{\whiten\vrule width 17pt height 4pt}\end{rotate}}
		\put(90,69){\begin{rotate}{40}{\tiny 0.003}\end{rotate}}
		\put(83,74){\begin{rotate}{44}{\whiten\vrule width 17pt height 4pt}\end{rotate}}
		\put(83,74){\begin{rotate}{44}{\tiny 0.004}\end{rotate}}
		\put(77,82){\begin{rotate}{48}{\whiten\vrule width 17pt height 4pt}\end{rotate}}
		\put(77,82){\begin{rotate}{48}{\tiny 0.005}\end{rotate}}
		\put(68,92){\begin{rotate}{48}{\whiten\vrule width 17pt height 4pt}\end{rotate}}
		\put(68,92){\begin{rotate}{48}{\tiny 0.006}\end{rotate}}
		\put(230,49){\begin{rotate}{40}{\whiten\vrule width 16pt height 4pt}\end{rotate}}
		\put(230,49){\begin{rotate}{40}{\tiny 0.001}\end{rotate}}
		\put(225,52){\begin{rotate}{43}{\whiten\vrule width 17pt height 4pt}\end{rotate}}
		\put(225,52){\begin{rotate}{43}{\tiny 0.002}\end{rotate}}
		\put(220,56){\begin{rotate}{46}{\whiten\vrule width 17pt height 4pt}\end{rotate}}
		\put(220,56){\begin{rotate}{46}{\tiny 0.003}\end{rotate}}
		\put(215,62){\begin{rotate}{48}{\whiten\vrule width 17pt height 4pt}\end{rotate}}
		\put(215,62){\begin{rotate}{48}{\tiny 0.004}\end{rotate}}
		\put(210,69){\begin{rotate}{49}{\whiten\vrule width 17pt height 4pt}\end{rotate}}
		\put(210,69){\begin{rotate}{49}{\tiny 0.005}\end{rotate}}
		\put(205,80){\begin{rotate}{50}{\whiten\vrule width 17pt height 4pt}\end{rotate}}
		\put(205,80){\begin{rotate}{50}{\tiny 0.006}\end{rotate}}
		\put(202,95){\begin{rotate}{50}{\whiten\vrule width 17pt height 4pt}\end{rotate}}
		\put(202,95){\begin{rotate}{50}{\tiny 0.007}\end{rotate}}
	\end{picture}
	\begin{picture}(275,135)(0,0)
	\put(0,78){$R$}
	\put(112,0){$\alpha$}
	\put(250,0){$\alpha$}
	\put(16,114.5){$n=0$}
	\put(244,114.5){$n=1$}
	\put(124,116){ $R\sigma/\alpha^2$ }
	%\input{fig/sigmaw.l}
		\thinlines
		\drawline(16,10)(16,112)
		\drawline(56,10)(56,112)
		\drawline(96,10)(96,112)
		\drawline(136,10)(136,112)
		\drawline(16,10)(136,10)
		\drawline(16,30.4)(136,30.4)
		\drawline(16,50.8)(136,50.8)
		\drawline(16,71.2)(136,71.2)
		\drawline(16,91.6)(136,91.6)
		\drawline(16,112)(136,112)
		\put(14,4){\tiny 5}
		\put(52,4){\tiny 10}
		\put(92,4){\tiny 15}
		\put(132,4){\tiny 20}
		\put(0,8){\makebox(15,4)[r]{\tiny 500}}
		\put(0,28.4){\makebox(15,4)[r]{\tiny 1000}}
		\put(0,48.8){\makebox(15,4)[r]{\tiny 1500}}
		\put(0,69.2){\makebox(15,4)[r]{\tiny 2000}}
		\put(0,89.6){\makebox(15,4)[r]{\tiny 2500}}
		\put(0,110){\makebox(15,4)[r]{\tiny 3000}}
		\thicklines
		\drawline(37.3022,112)(39.1103,107.586)(40.7462,104.693)(43.243,102.802)(45.8834,102.282)(49.0117,103.024)(52.2834,105.027)(55.7274,107.809)(59.4296,112)
		\drawline(32.9973,112)(33.6,106.362)(34.8054,100.576)(35.5802,96.1251)(36.7857,92.5645)(38.1632,88.8925)(40.2296,85.5543)(42.8126,83.7737)(45.3094,83.3286)(48.2368,83.9965)(51.8529,85.888)(56.072,89.1149)(60.6352,93.3434)(68.2976,101.466)(77.08,112)
		\drawline(30.0699,112)(31.878,92.008)(33.6,81.3257)(35.4942,74.4269)(37.7327,69.4195)(40.4879,66.0816)(44.0179,64.4125)(48.0646,65.1914)(52.9722,67.528)(62.1848,74.4269)(77.5968,89.4486)(87.9856,99.9085)(99.0064,112)
		\drawline(28.0036,112)(28.8933,91.4144)(30.2134,76.096)(32.1938,62.9657)(33.8296,56.4006)(35.896,51.2823)(38.6511,47.7216)(40.5453,46.2749)(42.8126,45.3846)(47.5193,45.9411)(51.9964,47.4989)(56.3304,49.9469)(64.0504,54.8429)(71.2248,60.184)(93.8976,78.6183)(112.438,94.2336)(132.498,112)
		\drawline(25.9659,112)(26.5112,91.2291)(27.0565,78.3584)(27.5444,69.1229)(28.1758,59.1824)(29.3525,49.8726)(30.8162,41.416)(32.7677,34.7395)(34.5184,31.1789)(36.9291,28.2858)(39.426,26.4683)(42.7551,25.504)(46.2565,25.3557)(49.8439,25.9862)(52.8574,26.8022)(58.3968,28.8793)(62.816,30.7709)(67.9248,33.2189)(82.2168,40.7856)(99.2648,50.392)(136,72.016)
	%\input{fig/sigmaw.r}
		\thinlines
		\drawline(149,10)(149,112)
		\drawline(179,10)(179,112)
		\drawline(209,10)(209,112)
		\drawline(239,10)(239,112)
		\drawline(269,10)(269,112)
		\drawline(149,10)(269,10)
		\drawline(149,30.4)(269,30.4)
		\drawline(149,50.8)(269,50.8)
		\drawline(149,71.2)(269,71.2)
		\drawline(149,91.6)(269,91.6)
		\drawline(149,112)(269,112)
		\put(147,4){\tiny 0}
		\put(177,4){\tiny 5}
		\put(205,4){\tiny 10}
		\put(235,4){\tiny 15}
		\put(265,4){\tiny 20}
		\thicklines
		\drawline(161.527,112)(163.249,98.6474)(164.864,89.9309)(165.875,86.0365)(167.124,83.0691)(168.415,81.2144)(169.427,80.6583)(170.675,80.6583)(172.225,82.1417)(174.55,86.1475)(176.595,91.1549)(179.221,96.904)(181.761,101.355)(184.279,104.137)(189.23,108.402)(192.2,112)
		\drawline(163.938,112)(164.476,108.587)(165.337,105.621)(166.952,101.726)(168.846,100.242)(170.568,101.058)(171.859,103.024)(173.732,107.178)(175.454,112)
		\drawline(160.021,112)(160.989,99.6857)(162.281,87.8166)(163.723,78.0617)(165.768,68.344)(167.813,63.04)(169.104,61.3709)(170.675,60.5549)(172.182,60.7405)(174.162,62.4835)(178.962,69.9017)(180.684,72.4983)(182.406,74.2784)(184.774,76.5783)(188.11,78.7296)(191.382,81.6966)(194.503,84.8496)(199.992,92.824)(204.792,100.984)(210.754,112)
		\drawline(158.88,112)(159.741,95.3091)(160.989,79.1744)(162.496,65.5623)(164.691,53.0256)(166.263,47.4617)(168.308,43.1223)(170.675,40.1549)(173.086,39.6726)(175.884,40.5257)(179.931,44.3091)(183.913,47.4617)(189.94,50.8)(193.922,53.7674)(197.151,56.4749)(200.422,60.0726)(206.88,67.6023)(213.574,76.3926)(221.818,87.6314)(239.446,112)
		\drawline(157.976,112)(158.794,92.5274)(160.365,67.5651)(161.979,50.0211)(162.862,43.6784)(163.787,37.7811)(164.691,33.0336)(165.682,28.9534)(166.909,24.6138)(168.351,21.1643)(170.503,17.4553)(172.677,15.304)(174.98,14.2654)(177.369,14.488)(180.06,15.712)(182.923,17.6037)(186.087,18.8647)(189.466,19.9774)(193.255,20.9418)(198.464,23.0931)(203.35,25.6153)(210.087,29.7694)(218.395,35.4816)(225.994,41.008)(233.807,46.6457)(250.747,59.4051)(259.486,66.1555)(269,73.3885)
	%\input{fig/sigmaw.label}
		\put(80,38){\begin{rotate}{25}{\whiten\vrule width 4pt height 4pt}\end{rotate}}
		\put(80,38){\begin{rotate}{25}{\tiny 0}\end{rotate}}
		\put(220,35){\begin{rotate}{30}{\whiten\vrule width 4pt height 4pt}\end{rotate}}
		\put(220,35){\begin{rotate}{30}{\tiny 0}\end{rotate}}
		\put(72,58){\begin{rotate}{40}{\whiten\vrule width 9pt height 4pt}\end{rotate}}
		\put(72,58){\begin{rotate}{40}{\tiny 0.5}\end{rotate}}
		\put(70,80){\begin{rotate}{40}{\whiten\vrule width 4pt height 4pt}\end{rotate}}
		\put(70,80){\begin{rotate}{40}{\tiny 1}\end{rotate}}
		\put(60,90){\begin{rotate}{40}{\whiten\vrule width 9pt height 4pt}\end{rotate}}
		\put(60,90){\begin{rotate}{40}{\tiny 1.5}\end{rotate}}
		\put(50,101){\begin{rotate}{40}{\whiten\vrule width 4pt height 4pt}\end{rotate}}
		\put(50,101){\begin{rotate}{40}{\tiny 2}\end{rotate}}
		\put(190,49){\begin{rotate}{35}{\whiten\vrule width 4pt height 4pt}\end{rotate}}
		\put(190,49){\begin{rotate}{35}{\tiny 1}\end{rotate}}
		\put(186,75){\begin{rotate}{40}{\whiten\vrule width 4pt height 4pt}\end{rotate}}
		\put(186,75){\begin{rotate}{40}{\tiny 2}\end{rotate}}
		\put(177,88){\begin{rotate}{60}{\whiten\vrule width 4pt height 4pt}\end{rotate}}
		\put(177,88){\begin{rotate}{60}{\tiny 3}\end{rotate}}
		\put(172,100){\begin{rotate}{50}{\whiten\vrule width 4pt height 4pt}\end{rotate}}
		\put(172,100){\begin{rotate}{50}{\tiny 4}\end{rotate}}
	\end{picture}
	\caption{\label{fig:sigmad}Aufklingfaktor in Grenzschichtzeit $\delta/V$ (oben) und in der Oszillationszeit $1/\Omega$ (unten).}
	\end{center}
\end{figure}

\begin{figure} % Wellenzahl k, k/alpha
	\begin{center}
	\begin{picture}(275,135)(0,0)
	\put(0,78){$R$}
	\put(112,0){$\alpha$}
	\put(250,0){$\alpha$}
	\put(16,114.5){$n=0$}
	\put(244,114.5){$n=1$}
	\put(136,116){ $k$ }
	%\input{fig/k.l}
		\thinlines
		\drawline(16,10)(16,112)
		\drawline(56,10)(56,112)
		\drawline(96,10)(96,112)
		\drawline(136,10)(136,112)
		\drawline(16,10)(136,10)
		\drawline(16,30.4)(136,30.4)
		\drawline(16,50.8)(136,50.8)
		\drawline(16,71.2)(136,71.2)
		\drawline(16,91.6)(136,91.6)
		\drawline(16,112)(136,112)
		\put(14,4){\tiny 5}
		\put(52,4){\tiny 10}
		\put(92,4){\tiny 15}
		\put(132,4){\tiny 20}
		\put(0,8){\makebox(15,4)[r]{\tiny 500}}
		\put(0,28.4){\makebox(15,4)[r]{\tiny 1000}}
		\put(0,48.8){\makebox(15,4)[r]{\tiny 1500}}
		\put(0,69.2){\makebox(15,4)[r]{\tiny 2000}}
		\put(0,89.6){\makebox(15,4)[r]{\tiny 2500}}
		\put(0,110){\makebox(15,4)[r]{\tiny 3000}}
		\thicklines
		\drawline(34.1166,112)(33.8296,87.2605)(33.3991,64.1897)(32.5381,35.4074)
		\drawline(50.9346,112)(50.2744,89.5229)(49.7578,70.1984)(49.2699,48.5005)(49.6143,38.7824)(50.3318,32.2915)(50.9919,28.768)(51.6233,26.5426)
		\drawline(71.0816,112)(70.4504,83.5514)(70.5936,61.9274)(71.1392,52.3206)(72.2296,46.4976)(73.6648,41.1936)(75.5016,37.0394)
		\drawline(113.298,112)(113.557,100.835)(114.533,90.4874)(115.91,81.2886)(116.829,76.1703)(118.494,71.2371)(120.215,66.9715)(121.938,63.7817)
		\drawline(91.4584,112)(91.8888,87.6685)(92.4056,76.7264)(93.2376,69.4566)(94.0984,63.2624)(95.2176,58.6634)(96.8824,53.8045)(98.7768,50.2435)
		\drawline(25.9659,112)(26.5112,91.2291)(27.0565,78.3584)(27.5444,69.1229)(28.1758,59.1824)(29.3525,49.8726)(30.8162,41.416)(32.7677,34.7395)(34.5184,31.1789)(36.9291,28.2858)(39.426,26.4683)(42.7551,25.504)(46.2565,25.3557)(49.8439,25.9862)(52.8574,26.8022)(58.3968,28.8793)(62.816,30.7709)(67.9248,33.2189)(82.2168,40.7856)(99.2648,50.392)(136,72.016)
	%\input{fig/k.r}
		\thinlines
		\drawline(149,10)(149,112)
		\drawline(179,10)(179,112)
		\drawline(209,10)(209,112)
		\drawline(239,10)(239,112)
		\drawline(269,10)(269,112)
		\drawline(149,10)(269,10)
		\drawline(149,30.4)(269,30.4)
		\drawline(149,50.8)(269,50.8)
		\drawline(149,71.2)(269,71.2)
		\drawline(149,91.6)(269,91.6)
		\drawline(149,112)(269,112)
		\put(147,4){\tiny 0}
		\put(177,4){\tiny 5}
		\put(205,4){\tiny 10}
		\put(235,4){\tiny 15}
		\put(265,4){\tiny 20}
		\thicklines
		\drawline(190.779,112)(190.155,84.0336)(189.402,62.5206)(187.917,32.1805)(187.077,19.1614)
		\drawline(206.299,112)(205.933,87.5943)(205.115,58.6263)(205.05,45.2736)(205.373,36.7056)(205.847,31.7725)(206.665,27.7294)
		\drawline(221,112)(220.44,81.5485)(220.44,68.9744)(220.893,58.4035)(221.538,51.0595)(222.528,44.4576)(223.82,39.4503)
		\drawline(235.83,112)(235.83,98.2765)(236.239,80.6583)(237.531,65.08)(238.456,59.4423)(239.619,55.3994)(240.802,51.9126)
		\drawline(251.716,112)(252.577,91.8224)(253.761,80.5469)(255.138,72.7577)(256.602,68.4554)(257.893,64.9315)
		\drawline(157.976,112)(158.794,92.5274)(160.365,67.5651)(161.979,50.0211)(162.862,43.6784)(163.787,37.7811)(164.691,33.0336)(165.682,28.9534)(166.909,24.6138)(168.351,21.1643)(170.503,17.4553)(172.677,15.304)(174.98,14.2654)(177.369,14.488)(180.06,15.712)(182.923,17.6037)(186.087,18.8647)(189.466,19.9774)(193.255,20.9418)(198.464,23.0931)(203.35,25.6153)(210.087,29.7694)(218.395,35.4816)(225.994,41.008)(233.807,46.6457)(250.747,59.4051)(259.486,66.1555)(269,73.3885)
	%\input{fig/k.label}
		\put(36,80){\begin{rotate}{90}{\whiten\vrule width 4pt height 4pt}\end{rotate}}
		\put(36,80){\begin{rotate}{90}{\tiny 2}\end{rotate}}
		\put(52,80){\begin{rotate}{90}{\whiten\vrule width 4pt height 4pt}\end{rotate}}
		\put(52,80){\begin{rotate}{90}{\tiny 3}\end{rotate}}
		\put(72,80){\begin{rotate}{90}{\whiten\vrule width 4pt height 4pt}\end{rotate}}
		\put(72,80){\begin{rotate}{90}{\tiny 4}\end{rotate}}
		\put(94,80){\begin{rotate}{92}{\whiten\vrule width 4pt height 4pt}\end{rotate}}
		\put(94,80){\begin{rotate}{92}{\tiny 5}\end{rotate}}
		\put(118,80){\begin{rotate}{95}{\whiten\vrule width 4pt height 4pt}\end{rotate}}
		\put(118,80){\begin{rotate}{95}{\tiny 6}\end{rotate}}
		\put(192,80){\begin{rotate}{90}{\whiten\vrule width 4pt height 4pt}\end{rotate}}
		\put(192,80){\begin{rotate}{90}{\tiny 2}\end{rotate}}
		\put(208,80){\begin{rotate}{90}{\whiten\vrule width 4pt height 4pt}\end{rotate}}
		\put(208,80){\begin{rotate}{90}{\tiny 3}\end{rotate}}
		\put(222,80){\begin{rotate}{90}{\whiten\vrule width 4pt height 4pt}\end{rotate}}
		\put(222,80){\begin{rotate}{90}{\tiny 4}\end{rotate}}
		\put(238,80){\begin{rotate}{90}{\whiten\vrule width 4pt height 4pt}\end{rotate}}
		\put(238,80){\begin{rotate}{90}{\tiny 5}\end{rotate}}
		\put(256,80){\begin{rotate}{95}{\whiten\vrule width 4pt height 4pt}\end{rotate}}
		\put(256,80){\begin{rotate}{95}{\tiny 6}\end{rotate}}
	\end{picture}
	\begin{picture}(275,135)(0,0)
	\put(0,78){$R$}
	\put(112,0){$\alpha$}
	\put(250,0){$\alpha$}
	\put(16,114.5){$n=0$}
	\put(244,114.5){$n=1$}

	\put(133,116){ $k/\alpha$ }
	%\input{fig/kappa.l}
		\thinlines
		\drawline(16,10)(16,112)
		\drawline(56,10)(56,112)
		\drawline(96,10)(96,112)
		\drawline(136,10)(136,112)
		\drawline(16,10)(136,10)
		\drawline(16,30.4)(136,30.4)
		\drawline(16,50.8)(136,50.8)
		\drawline(16,71.2)(136,71.2)
		\drawline(16,91.6)(136,91.6)
		\drawline(16,112)(136,112)
		\put(14,4){\tiny 5}
		\put(52,4){\tiny 10}
		\put(92,4){\tiny 15}
		\put(132,4){\tiny 20}
		\put(0,8){\makebox(15,4)[r]{\tiny 500}}
		\put(0,28.4){\makebox(15,4)[r]{\tiny 1000}}
		\put(0,48.8){\makebox(15,4)[r]{\tiny 1500}}
		\put(0,69.2){\makebox(15,4)[r]{\tiny 2000}}
		\put(0,89.6){\makebox(15,4)[r]{\tiny 2500}}
		\put(0,110){\makebox(15,4)[r]{\tiny 3000}}
		\thicklines
		\drawline(31.1031,112)(30.6726,93.5286)(30.2422,69.3824)(29.4673,48.7971)
		\drawline(40.2296,112)(39.1964,89.0777)(37.8188,64.9315)(36.4413,46.2377)(35.9821,35.4816)(36.3551,28.9906)
		\drawline(50.705,112)(47.1749,81.5114)(44.4198,53.6189)(43.817,42.7143)(43.817,37.4845)(44.0754,34.9251)(44.7354,33.1075)(45.539,31.0304)(47.0027,29.584)(50.7623,28.6567)(53.9767,28.4713)(58.1096,29.3614)(63.3904,30.8823)
		\drawline(60.6064,112)(56.9328,87.2605)(55.096,75.0944)(53.9193,65.0057)(52.9435,57.8474)(53.0296,51.9497)(53.2018,48.5005)(53.8045,46.3863)(54.5794,44.0496)(55.9282,42.1949)(58.1096,40.9337)(59.4872,40.6)(63.62,40.7114)(71.6272,42.6029)(80.3232,46.0896)(104,56.2896)(135.971,72.2016)
		\drawline(76.276,112)(73.148,92.2675)(72.488,83.1063)(72.4016,76.9863)(72.66,73.648)(73.6072,70.5325)(74.9848,68.0845)(77.3952,66.0816)(80.2368,64.9686)(84.1112,64.6349)(88.7608,64.8576)(101.13,69.4566)(121.794,77.6166)(135.886,84.1075)
		\drawline(114.446,112)(114.963,109.811)(115.652,108.699)(117.202,107.661)(120.732,106.696)(124.52,107.03)(129.944,108.81)(135.857,110.627)

		\drawline(25.9659,112)(26.5112,91.2291)(27.0565,78.3584)(27.5444,69.1229)(28.1758,59.1824)(29.3525,49.8726)(30.8162,41.416)(32.7677,34.7395)(34.5184,31.1789)(36.9291,28.2858)(39.426,26.4683)(42.7551,25.504)(46.2565,25.3557)(49.8439,25.9862)(52.8574,26.8022)(58.3968,28.8793)(62.816,30.7709)(67.9248,33.2189)(82.2168,40.7856)(99.2648,50.392)(136,72.016)
	%\input{fig/kappa.r}
		\thinlines
		\drawline(149,10)(149,112)
		\drawline(179,10)(179,112)
		\drawline(209,10)(209,112)
		\drawline(239,10)(239,112)
		\drawline(269,10)(269,112)
		\drawline(149,10)(269,10)
		\drawline(149,30.4)(269,30.4)
		\drawline(149,50.8)(269,50.8)
		\drawline(149,71.2)(269,71.2)
		\drawline(149,91.6)(269,91.6)
		\drawline(149,112)(269,112)
		\put(147,4){\tiny 0}
		\put(177,4){\tiny 5}
		\put(205,4){\tiny 10}
		\put(235,4){\tiny 15}
		\put(265,4){\tiny 20}
		\thicklines
		\drawline(167.899,112)(168.114,84.3674)(168.996,52.1354)(170.417,29.6582)(171.45,16.5651)
		\drawline(175.454,112)(175.712,80.5097)(176.616,60.9629)(177.391,47.1651)(178.618,35.2589)(179.845,25.1331)(181.395,16.6763)
		\drawline(176.552,112)(177.197,79.8794)(177.843,65.08)(178.338,54.1383)(179.393,41.4903)(180.921,30.7709)(183.289,17.9746)
		\drawline(177.456,112)(178.295,79.768)(178.876,67.4909)(179.78,53.1737)(180.706,44.8656)(181.782,35.3703)(183.203,28.1374)(185.204,18.6793)
		\drawline(179.78,112)(180.297,92.6384)(181.352,70.0874)(183.935,54.1383)(185.334,51.0595)(186.948,49.6874)(188.304,50.8371)(189.531,53.9526)(192.437,66.1926)(195.45,85.8509)(198.055,112)
		\drawline(181.524,112)(182.169,94.864)(183.332,81.9565)(183.784,77.7651)(184.946,74.0931)(185.786,73.7223)(186.948,74.0931)(188.562,79.2857)(190.564,92.5274)(191.791,103.654)(192.652,112)
		\drawline(178.553,112)(179.07,86.9635)(180.039,67.528)(181.244,53.5817)(182.622,43.6045)(184.473,35.9635)(186.044,31.4755)(187.465,28.4713)(188.562,27.4327)(189.789,27.3586)(191.081,28.6938)(191.791,29.9178)(192.824,32.9223)(193.664,36.3715)(196.053,48.6115)(198.141,64.3383)(202.231,94.5674)(204.125,112)
		\drawline(210.517,111.889)(207.289,83.0691)(205.545,69.4195)(204.512,56.7344)(204.189,48.7229)(204.577,44.0496)(204.835,40.4886)(206.062,38.152)(208.903,36.4829)(211.809,36.3715)(214.973,36.8166)(222.679,38.3744)
		\drawline(217.685,112)(213.961,72.1274)(214.069,61.2966)(214.65,57.4023)(216.135,53.8416)(218.073,51.2823)(220.72,49.8355)(223.368,49.3904)(225.477,49.5017)(240.415,55.6944)(257.463,64.4864)
		\drawline(234.97,112)(235.335,101.355)(235.981,95.3463)(237.144,90.1165)(238.371,87.0006)(240.114,84.664)(242.439,83.2176)(244.893,82.9949)(249.542,83.44)(253.266,84.9606)(269,92.1936)
		\drawline(157.976,112)(158.794,92.5274)(160.365,67.5651)(161.979,50.0211)(162.862,43.6784)(163.787,37.7811)(164.691,33.0336)(165.682,28.9534)(166.909,24.6138)(168.351,21.1643)(170.503,17.4553)(172.677,15.304)(174.98,14.2654)(177.369,14.488)(180.06,15.712)(182.923,17.6037)(186.087,18.8647)(189.466,19.9774)(193.255,20.9418)(198.464,23.0931)(203.35,25.6153)(210.087,29.7694)(218.395,35.4816)(225.994,41.008)(233.807,46.6457)(250.747,59.4051)(259.486,66.1555)(269,73.3885)
	%\input{fig/kappa.label}
		\put(32,75){\begin{rotate}{87}{\whiten\vrule width 13pt height 4pt}\end{rotate}}
		\put(32,75){\begin{rotate}{87}{\tiny 0.25}\end{rotate}}
		\put(41,74){\begin{rotate}{87}{\whiten\vrule width 9pt height 4pt}\end{rotate}}
		\put(41,74){\begin{rotate}{87}{\tiny 0.3}\end{rotate}}
		\put(48,75){\begin{rotate}{80}{\whiten\vrule width 13pt height 4pt}\end{rotate}}
		\put(48,75){\begin{rotate}{80}{\tiny 0.32}\end{rotate}}
		\put(57,75){\begin{rotate}{80}{\whiten\vrule width 13pt height 4pt}\end{rotate}}
		\put(57,75){\begin{rotate}{80}{\tiny 0.33}\end{rotate}}
		\put(114,73){\begin{rotate}{20}{\whiten\vrule width 13pt height 4pt}\end{rotate}}
		\put(114,73){\begin{rotate}{20}{\tiny 0.34}\end{rotate}}
		\put(120,99){\begin{rotate}{20}{\whiten\vrule width 13pt height 4pt}\end{rotate}}
		\put(120,99){\begin{rotate}{20}{\tiny 0.35}\end{rotate}}
		\put(171,52){\begin{rotate}{90}{\whiten\vrule width 9pt height 4pt}\end{rotate}}
		\put(171,52){\begin{rotate}{90}{\tiny 0.5}\end{rotate}}
		\put(174,92){\begin{rotate}{90}{\whiten\vrule width 16pt height 4pt}\end{rotate}}
		\put(174,92){\begin{rotate}{90}{\tiny 0.345}\end{rotate}}
		\put(181,32){\begin{rotate}{-80}{\whiten\vrule width 13pt height 4pt}\end{rotate}}
		\put(181,32){\begin{rotate}{-80}{\tiny 0.32}\end{rotate}}
		\put(196,35){\begin{rotate}{80}{\whiten\vrule width 13pt height 4pt}\end{rotate}}
		\put(196,35){\begin{rotate}{80}{\tiny 0.31}\end{rotate}}
		\put(191,55){\begin{rotate}{80}{\whiten\vrule width 13pt height 4pt}\end{rotate}}
		\put(191,55){\begin{rotate}{80}{\tiny 0.30}\end{rotate}}
		\put(192,90){\begin{rotate}{83}{\whiten\vrule width 13pt height 4pt}\end{rotate}}
		\put(192,90){\begin{rotate}{83}{\tiny 0.29}\end{rotate}}
		\put(206,55){\begin{rotate}{85}{\whiten\vrule width 13pt height 4pt}\end{rotate}}
		\put(206,55){\begin{rotate}{85}{\tiny 0.32}\end{rotate}}
		\put(230,49){\begin{rotate}{25}{\whiten\vrule width 13pt height 4pt}\end{rotate}}
		\put(230,49){\begin{rotate}{25}{\tiny 0.33}\end{rotate}}
		\put(250,82){\begin{rotate}{24}{\whiten\vrule width 16pt height 4pt}\end{rotate}}
		\put(250,82){\begin{rotate}{24}{\tiny 0.345}\end{rotate}}
	\end{picture}
	\caption{\label{fig:k}Wellenzahl $k$ mit maximaler Verst\"arkung in L\"angeneinheiten des Radius (oben) und der Grenzschichtdicke (unten).}
	\end{center}
\end{figure}

\begin{figure} % Aktionsradius rho, Wandabstand
	\begin{center}
	\begin{picture}(275,135)(0,0)
	\put(0,78){$R$}
	\put(112,0){$\alpha$}
	\put(250,0){$\alpha$}
	\put(16,114.5){$n=0$}
	\put(244,114.5){$n=1$}
	\put(136,116){ $\varrho$ }
	%\input{fig/rho.l}
		\thinlines
		\drawline(16,10)(16,112)
		\drawline(56,10)(56,112)
		\drawline(96,10)(96,112)
		\drawline(136,10)(136,112)
		\drawline(16,10)(136,10)
		\drawline(16,30.4)(136,30.4)
		\drawline(16,50.8)(136,50.8)
		\drawline(16,71.2)(136,71.2)
		\drawline(16,91.6)(136,91.6)
		\drawline(16,112)(136,112)
		\put(14,4){\tiny 5}
		\put(52,4){\tiny 10}
		\put(92,4){\tiny 15}
		\put(132,4){\tiny 20}
		\put(0,8){\makebox(15,4)[r]{\tiny 500}}
		\put(0,28.4){\makebox(15,4)[r]{\tiny 1000}}
		\put(0,48.8){\makebox(15,4)[r]{\tiny 1500}}
		\put(0,69.2){\makebox(15,4)[r]{\tiny 2000}}
		\put(0,89.6){\makebox(15,4)[r]{\tiny 2500}}
		\put(0,110){\makebox(15,4)[r]{\tiny 3000}}
		\thicklines
		\drawline(42.1812,112)(42.8986,99.9456)(44.3336,82.3274)(45.5677,70.1984)(46.6009,61.7789)(48.6386,51.9126)(49.4422,48.0183)(50.9346,43.0109)(52.0251,39.4503)(53.9767,35.2589)(55.4403,32.44)(57.8512,28.7309)
		\drawline(57.1912,112)(58.7696,89.2634)(61.4096,68.4183)(62.9024,60.5177)(64.5672,53.7674)(65.916,49.7245)(67.2072,45.9783)(68.872,42.232)(71.024,38.8937)(73.1192,35.9635)
		\drawline(77.1088,112)(79.0024,95.3463)(80.8968,85.2576)(82.504,77.0234)(84.8288,67.936)(87.0096,62.0384)(89.2768,57.1056)(92.6064,51.5417)(95.1608,48.0183)
		\drawline(119.928,112)(121.22,107.29)(123.372,100.873)(125.639,94.9754)(128.28,89.0035)(130.576,84.6269)(132.901,81.0663)(136,76.7635)
		\drawline(25.9659,112)(26.5112,91.2291)(27.0565,78.3584)(27.5444,69.1229)(28.1758,59.1824)(29.3525,49.8726)(30.8162,41.416)(32.7677,34.7395)(34.5184,31.1789)(36.9291,28.2858)(39.426,26.4683)(42.7551,25.504)(46.2565,25.3557)(49.8439,25.9862)(52.8574,26.8022)(58.3968,28.8793)(62.816,30.7709)(67.9248,33.2189)(82.2168,40.7856)(99.2648,50.392)(136,72.016)
	%\input{fig/rho.r}
		\thinlines
		\drawline(149,10)(149,112)
		\drawline(179,10)(179,112)
		\drawline(209,10)(209,112)
		\drawline(239,10)(239,112)
		\drawline(269,10)(269,112)
		\drawline(149,10)(269,10)
		\drawline(149,30.4)(269,30.4)
		\drawline(149,50.8)(269,50.8)
		\drawline(149,71.2)(269,71.2)
		\drawline(149,91.6)(269,91.6)
		\drawline(149,112)(269,112)
		\put(147,4){\tiny 0}
		\put(177,4){\tiny 5}
		\put(205,4){\tiny 10}
		\put(235,4){\tiny 15}
		\put(265,4){\tiny 20}
		\thicklines
		\drawline(252.211,112)(253.804,104.545)(256.387,95.0125)(257.958,90.0051)(259.529,85.9251)(261.832,80.4355)(264.329,75.4285)(265.491,73.4624)(266.848,71.6451)
		\drawline(217.276,112)(218.395,98.8697)(219.579,88.7069)(221.323,76.4669)(222.636,69.4937)(224.185,63.1885)(225.778,57.6989)(227.242,53.5817)(228.361,50.8743)(229.653,48.2406)(231.59,44.9395)
		\drawline(195.795,112)(196.849,98.4989)(198.464,83.5514)(200.917,65.896)(202.51,57.7731)(204.318,50.0583)(205.696,45.0509)(207.353,40.1177)(208.343,37.5955)(209.764,34.7395)(210.776,32.8109)(211.873,30.9194)
		\drawline(181.158,112)(181.674,94.048)(182.794,75.1686)(184.193,58.8486)(185.377,48.6486)(187.077,38.9309)(188.605,33.0336)(189.897,28.768)(191.124,25.6523)(192.415,23.0189)(193.664,21.0531)
		\drawline(157.976,112)(158.794,92.5274)(160.365,67.5651)(161.979,50.0211)(162.862,43.6784)(163.787,37.7811)(164.691,33.0336)(165.682,28.9534)(166.909,24.6138)(168.351,21.1643)(170.503,17.4553)(172.677,15.304)(174.98,14.2654)(177.369,14.488)(180.06,15.712)(182.923,17.6037)(186.087,18.8647)(189.466,19.9774)(193.255,20.9418)(198.464,23.0931)(203.35,25.6153)(210.087,29.7694)(218.395,35.4816)(225.994,41.008)(233.807,46.6457)(250.747,59.4051)(259.486,66.1555)(269,73.3885)
	%\input{fig/rho.label}
		\put(47,78){\begin{rotate}{98}{\whiten\vrule width 10pt height 4pt}\end{rotate}}
		\put(47,78){\begin{rotate}{98}{\tiny 0.5}\end{rotate}}
		\put(62,78){\begin{rotate}{95}{\whiten\vrule width 10pt height 4pt}\end{rotate}}
		\put(62,78){\begin{rotate}{95}{\tiny 0.6}\end{rotate}}
		\put(84,78){\begin{rotate}{105}{\whiten\vrule width 10pt height 4pt}\end{rotate}}
		\put(84,78){\begin{rotate}{105}{\tiny 0.7}\end{rotate}}
		\put(128,95){\begin{rotate}{110}{\whiten\vrule width 10pt height 4pt}\end{rotate}}
		\put(128,95){\begin{rotate}{110}{\tiny 0.8}\end{rotate}}
		\put(185,78){\begin{rotate}{92}{\whiten\vrule width 10pt height 4pt}\end{rotate}}
		\put(185,78){\begin{rotate}{92}{\tiny 0.5}\end{rotate}}
		\put(201,78){\begin{rotate}{95}{\whiten\vrule width 10pt height 4pt}\end{rotate}}
		\put(201,78){\begin{rotate}{95}{\tiny 0.6}\end{rotate}}
		\put(223,78){\begin{rotate}{95}{\whiten\vrule width 10pt height 4pt}\end{rotate}}
		\put(223,78){\begin{rotate}{95}{\tiny 0.7}\end{rotate}}
		\put(258,95){\begin{rotate}{105}{\whiten\vrule width 10pt height 4pt}\end{rotate}}
		\put(258,95){\begin{rotate}{105}{\tiny 0.8}\end{rotate}}
	\end{picture}
	\begin{picture}(275,135)(0,0)
	\put(0,78){$R$}
	\put(112,0){$\alpha$}
	\put(250,0){$\alpha$}
	\put(16,114.5){$n=0$}
	\put(244,114.5){$n=1$}
	\put(120,116){ $(1-\varrho)\alpha$ }
	%\input{fig/rhod.l}
		\thinlines
		\drawline(16,10)(16,112)
		\drawline(56,10)(56,112)
		\drawline(96,10)(96,112)
		\drawline(136,10)(136,112)
		\drawline(16,10)(136,10)
		\drawline(16,30.4)(136,30.4)
		\drawline(16,50.8)(136,50.8)
		\drawline(16,71.2)(136,71.2)
		\drawline(16,91.6)(136,91.6)
		\drawline(16,112)(136,112)
		\put(14,4){\tiny 5}
		\put(52,4){\tiny 10}
		\put(92,4){\tiny 15}
		\put(132,4){\tiny 20}
		\put(0,8){\makebox(15,4)[r]{\tiny 500}}
		\put(0,28.4){\makebox(15,4)[r]{\tiny 1000}}
		\put(0,48.8){\makebox(15,4)[r]{\tiny 1500}}
		\put(0,69.2){\makebox(15,4)[r]{\tiny 2000}}
		\put(0,89.6){\makebox(15,4)[r]{\tiny 2500}}
		\put(0,110){\makebox(15,4)[r]{\tiny 3000}}
		\thicklines
		\drawline(43.1282,25.6523)(47.7776,29.1018)(52.5991,31.1046)(57.5928,31.7725)(61.5536,31.9949)(65.6,32.2176)
		\drawline(33.9731,112)(31.6484,92.9725)(29.6108,76.2816)(27.5444,65.8217)
		\drawline(33.3991,33.4416)(36.2403,37.336)(38.9094,43.2336)(40.8036,46.7943)(43.4726,51.0224)(46.3139,53.9155)(49.069,54.9171)(51.3938,55.0285)(53.6323,54.3606)(56.6456,53.248)(58.9704,51.9126)(65.428,49.3536)(71.8856,47.6845)(79.376,46.5715)(85.3168,46.6829)(93.496,47.0166)
		\drawline(29.9551,45.7926)(32.2798,53.0256)(34.0018,61.3709)(38.565,82.2531)(40.287,89.1891)(42.009,93.64)(43.9031,97.6457)(45.6825,99.0925)(47.835,99.0925)(49.2126,97.9795)(51.279,94.864)(54.4072,88.4103)(58.884,80.5097)(61.812,75.8365)(65.256,71.3856)(69.388,67.4909)(73.2624,64.8205)(76.9648,62.8176)(82.2168,60.8144)(91.688,59.368)(102.881,59.8131)(111.405,60.8144)(120.186,62.4835)
		\drawline(61.1232,112)(65.6576,97.8685)(70.0768,88.744)(74.8128,82.1789)(78.5144,78.7296)(82.1312,76.3926)(85.3168,74.5011)(89.708,72.832)(97.7144,71.608)(108.075,71.3856)(126.586,74.1674)(136,76.9491)
		\drawline(76.1904,112)(80.9256,102.43)(86.1776,96.5331)(91.2864,92.5274)(95.7344,90.1904)(100.814,88.6326)(106.325,88.0765)(112.38,87.8166)(118.809,88.1875)(125.869,89.1891)(136,91.4144)
		\drawline(25.9659,112)(26.5112,91.2291)(27.0565,78.3584)(27.5444,69.1229)(28.1758,59.1824)(29.3525,49.8726)(30.8162,41.416)(32.7677,34.7395)(34.5184,31.1789)(36.9291,28.2858)(39.426,26.4683)(42.7551,25.504)(46.2565,25.3557)(49.8439,25.9862)(52.8574,26.8022)(58.3968,28.8793)(62.816,30.7709)(67.9248,33.2189)(82.2168,40.7856)(99.2648,50.392)(136,72.016)
	%\input{fig/rhod.r}
		\thinlines
		\drawline(149,10)(149,112)
		\drawline(179,10)(179,112)
		\drawline(209,10)(209,112)
		\drawline(239,10)(239,112)
		\drawline(269,10)(269,112)
		\drawline(149,10)(269,10)
		\drawline(149,30.4)(269,30.4)
		\drawline(149,50.8)(269,50.8)
		\drawline(149,71.2)(269,71.2)
		\drawline(149,91.6)(269,91.6)
		\drawline(149,112)(269,112)
		\put(147,4){\tiny 0}
		\put(177,4){\tiny 5}
		\put(205,4){\tiny 10}
		\put(235,4){\tiny 15}
		\put(265,4){\tiny 20}
		\thicklines
		\drawline(169.169,112)(168.717,79.6937)(168.071,57.6617)(167.167,40.5257)(166.715,25.9491)
		\drawline(192.48,112)(190.413,87.2605)(189.38,76.6154)(187.658,65.0057)(185.248,52.024)(183.698,46.1264)(182.019,39.6726)(179.436,28.6567)(178.467,24.8734)(177.24,21.424)(176.401,17.0843)(176.013,13.8574)
		\drawline(268.354,112)(255.956,104.767)(245.107,99.7229)(237.961,97.3863)(227.113,94.7155)(220.72,94.048)(216.781,92.6016)(213.337,90.4874)(210.001,86.3703)(207.461,81.3629)(205.481,76.5783)(203.328,69.9017)(200.293,58.7744)(195.709,40.6371)(192.609,30.1774)(190.284,24.5026)(187.443,19.1614)
		\drawline(269,85.3686)(259.507,80.0275)(244.204,72.4611)(229.309,66.1184)(221.861,63.2256)(215.425,59.7017)(211.788,56.4749)(208.946,52.8771)(205.072,46.4605)(200.229,35.5555)(194.611,21.8691)
		\drawline(200.745,24.3913)(206.169,32.7737)(208.946,36.3344)(212.175,39.5616)(215.404,42.1206)(219.084,44.68)(227.027,48.7971)(236.326,52.9143)(249.972,58.8857)
		\drawline(157.976,112)(158.794,92.5274)(160.365,67.5651)(161.979,50.0211)(162.862,43.6784)(163.787,37.7811)(164.691,33.0336)(165.682,28.9534)(166.909,24.6138)(168.351,21.1643)(170.503,17.4553)(172.677,15.304)(174.98,14.2654)(177.369,14.488)(180.06,15.712)(182.923,17.6037)(186.087,18.8647)(189.466,19.9774)(193.255,20.9418)(198.464,23.0931)(203.35,25.6153)(210.087,29.7694)(218.395,35.4816)(225.994,41.008)(233.807,46.6457)(250.747,59.4051)(259.486,66.1555)(269,73.3885)
	%\input{fig/rhod.label}
		\put(34,98){\begin{rotate}{80}{\whiten\vrule width 4pt height 4pt}\end{rotate}}
		\put(34,98){\begin{rotate}{80}{\tiny 4}\end{rotate}}
		\put(68,66){\begin{rotate}{-30}{\whiten\vrule width 10pt height 4pt}\end{rotate}}
		\put(68,66){\begin{rotate}{-30}{\tiny 4.2}\end{rotate}}
		\put(66,47){\begin{rotate}{-10}{\whiten\vrule width 10pt height 4pt}\end{rotate}}
		\put(66,47){\begin{rotate}{-10}{\tiny 4.5}\end{rotate}}
		\put(53,30){\begin{rotate}{3}{\whiten\vrule width 4pt height 4pt}\end{rotate}}
		\put(53,30){\begin{rotate}{3}{\tiny 5}\end{rotate}}
		\put(100,70){\begin{rotate}{0}{\whiten\vrule width 4pt height 4pt}\end{rotate}}
		\put(100,70){\begin{rotate}{0}{\tiny 4}\end{rotate}}
		\put(110,87){\begin{rotate}{0}{\whiten\vrule width 9.5pt height 4pt}\end{rotate}}
		\put(110,87){\begin{rotate}{0}{\tiny 3.8}\end{rotate}}
		\put(170,80){\begin{rotate}{85}{\whiten\vrule width 4pt height 4pt}\end{rotate}}
		\put(170,80){\begin{rotate}{85}{\tiny 2}\end{rotate}}
		\put(191,80){\begin{rotate}{80}{\whiten\vrule width 4pt height 4pt}\end{rotate}}
		\put(191,80){\begin{rotate}{80}{\tiny 3}\end{rotate}}
		\put(234,95){\begin{rotate}{18}{\whiten\vrule width 10pt height 4pt}\end{rotate}}
		\put(234,95){\begin{rotate}{18}{\tiny 3.5}\end{rotate}}
		\put(224,62){\begin{rotate}{25}{\whiten\vrule width 13pt height 4pt}\end{rotate}}
		\put(224,62){\begin{rotate}{25}{\tiny 3.75}\end{rotate}}
		\put(220,44){\begin{rotate}{30}{\whiten\vrule width 4pt height 4pt}\end{rotate}}
		\put(220,44){\begin{rotate}{30}{\tiny 4}\end{rotate}}
	\end{picture}
	\caption{\label{fig:rho}Aktionsradius $\varrho$ (oben) und sein Wandabstand bezogen auf die Grenzschichtdicke $(1-\varrho)\alpha$ des Eigenmodes mit maximaler Verst\"arkung $\mathfrak{v}^*$ (unten).}
	\end{center}
\end{figure}

\begin{figure} % Dauer Delta_t, Zeitpunkt
	\begin{center}
	\begin{picture}(275,135)(0,0)
	\put(0,78){$R$}
	\put(112,0){$\alpha$}
	\put(250,0){$\alpha$}
	\put(16,114.5){$n=0$}
	\put(244,114.5){$n=1$}
	\put(132,116){ $\tau/\pi$ }
	%\input{fig/t.l}
		\thinlines
		\drawline(16,10)(16,112)
		\drawline(56,10)(56,112)
		\drawline(96,10)(96,112)
		\drawline(136,10)(136,112)
		\drawline(16,10)(136,10)
		\drawline(16,30.4)(136,30.4)
		\drawline(16,50.8)(136,50.8)
		\drawline(16,71.2)(136,71.2)
		\drawline(16,91.6)(136,91.6)
		\drawline(16,112)(136,112)
		\put(14,4){\tiny 5}
		\put(52,4){\tiny 10}
		\put(92,4){\tiny 15}
		\put(132,4){\tiny 20}
		\put(0,8){\makebox(15,4)[r]{\tiny 500}}
		\put(0,28.4){\makebox(15,4)[r]{\tiny 1000}}
		\put(0,48.8){\makebox(15,4)[r]{\tiny 1500}}
		\put(0,69.2){\makebox(15,4)[r]{\tiny 2000}}
		\put(0,89.6){\makebox(15,4)[r]{\tiny 2500}}
		\put(0,110){\makebox(15,4)[r]{\tiny 3000}}
		\thicklines
		\drawline(30.7587,42.0835)(31.6198,41.3417)(33.0547,40.6)(37.0726,40.7856)(48.2654,43.0109)(65.916,46.72)(91.7456,55.8074)(113.701,65.2656)(136,75.0944)
		\drawline(26.9417,78.9891)(28.5202,74.1674)(29.3812,71.8304)(30.4718,70.4583)(32.2511,69.2343)(35.2072,68.6777)(38.3642,68.2326)(45.3955,66.9344)(51.0494,66.3783)(58.224,66.6377)(70.9384,69.2343)(81.5568,73.0544)(104.23,83.2544)(120.531,91.2291)(136,99.76)
		\drawline(30.2422,112)(34.2601,100.947)(40.6601,92.5274)(46.687,87.7056)(50.5614,85.48)(55.4403,84.0704)(64.108,83.9965)(71.7992,85.2944)(80.1216,88.6326)(93.324,93.8256)(104.947,99.9456)(115.853,105.694)(126.758,112)
		\drawline(55.3542,112)(58.052,109.774)(60.836,108.477)(64.7104,107.438)(68.6424,107.363)(74.0376,108.106)(79.6912,109.589)(85.5752,112)
		\drawline(25.9659,112)(26.5112,91.2291)(27.0565,78.3584)(27.5444,69.1229)(28.1758,59.1824)(29.3525,49.8726)(30.8162,41.416)(32.7677,34.7395)(34.5184,31.1789)(36.9291,28.2858)(39.426,26.4683)(42.7551,25.504)(46.2565,25.3557)(49.8439,25.9862)(52.8574,26.8022)(58.3968,28.8793)(62.816,30.7709)(67.9248,33.2189)(82.2168,40.7856)(99.2648,50.392)(136,72.016)
	%\input{fig/t.r}
		\thinlines
		\drawline(149,10)(149,112)
		\drawline(179,10)(179,112)
		\drawline(209,10)(209,112)
		\drawline(239,10)(239,112)
		\drawline(269,10)(269,112)
		\drawline(149,10)(269,10)
		\drawline(149,30.4)(269,30.4)
		\drawline(149,50.8)(269,50.8)
		\drawline(149,71.2)(269,71.2)
		\drawline(149,91.6)(269,91.6)
		\drawline(149,112)(269,112)
		\put(147,4){\tiny 0}
		\put(177,4){\tiny 5}
		\put(205,4){\tiny 10}
		\put(235,4){\tiny 15}
		\put(265,4){\tiny 20}
		\thicklines
		\drawline(167.038,112)(166.521,96.3104)(165.961,82.8096)(165.466,78.9149)(164.907,76.6896)(164.132,75.7251)(163.206,73.9075)(162.819,69.0857)(162.604,59.8131)(163.249,40.9709)
		\drawline(178.553,112)(179.436,95.6057)(181.373,75.3914)(184.731,56.92)(186.969,49.7984)(188.885,46.3491)(191.016,43.9011)(192.867,42.9366)(194.804,42.4544)(197.086,42.3805)(204.232,43.9754)(213.789,47.3875)(227.823,53.2109)(248.853,64.4496)(259.185,70.384)(269,76.3926)
		\drawline(184.774,112)(185.829,98.3875)(186.496,91.7856)(187.874,84.4784)(189.079,79.1376)(190.822,74.1303)(191.877,72.016)(194.288,69.2343)(196.225,67.9731)(198.313,67.2314)(203.694,67.2314)(212.196,68.9005)(221.775,72.2384)(234.905,79.1744)(248.466,87.0749)(254.385,91.1549)(264.071,97.6457)(269,101.429)
		\drawline(189.402,112)(190.779,102.282)(192.674,96.6445)(193.836,93.7885)(195.235,91.8966)(197.43,89.0777)(198.894,87.7795)(200.788,86.8154)(203.414,85.9251)(205.847,85.7024)(211.486,85.9251)(218.826,87.4457)(226.554,90.7097)(235.83,96.0137)(245.517,102.245)(252.663,107.104)(259.077,112)
		\drawline(207.439,112)(211.96,110.405)(214.822,110.034)(216.329,110.034)(218.331,110.331)(220.204,110.887)(222.205,112)
		\drawline(157.976,112)(158.794,92.5274)(160.365,67.5651)(161.979,50.0211)(162.862,43.6784)(163.787,37.7811)(164.691,33.0336)(165.682,28.9534)(166.909,24.6138)(168.351,21.1643)(170.503,17.4553)(172.677,15.304)(174.98,14.2654)(177.369,14.488)(180.06,15.712)(182.923,17.6037)(186.087,18.8647)(189.466,19.9774)(193.255,20.9418)(198.464,23.0931)(203.35,25.6153)(210.087,29.7694)(218.395,35.4816)(225.994,41.008)(233.807,46.6457)(250.747,59.4051)(259.486,66.1555)(269,73.3885)
	%\input{fig/t.label}
		\put(65,45){\begin{rotate}{17}{\whiten\vrule width 13pt height 4pt}\end{rotate}}
		\put(65,45){\begin{rotate}{17}{\tiny 1.04}\end{rotate}}
		\put(65,66){\begin{rotate}{16}{\whiten\vrule width 4pt height 4pt}\end{rotate}}
		\put(65,66){\begin{rotate}{16}{\tiny 1}\end{rotate}}
		\put(65,82){\begin{rotate}{13}{\whiten\vrule width 13pt height 4pt}\end{rotate}}
		\put(65,82){\begin{rotate}{13}{\tiny 0.98}\end{rotate}}
		\put(65,105){\begin{rotate}{5}{\whiten\vrule width 13pt height 4pt}\end{rotate}}
		\put(65,105){\begin{rotate}{5}{\tiny 0.96}\end{rotate}}
		\put(215,46){\begin{rotate}{22}{\whiten\vrule width 13pt height 4pt}\end{rotate}}
		\put(215,46){\begin{rotate}{22}{\tiny 1.04}\end{rotate}}
		\put(215,68){\begin{rotate}{22}{\whiten\vrule width 4pt height 4pt}\end{rotate}}
		\put(215,68){\begin{rotate}{22}{\tiny 1}\end{rotate}}
		\put(215,85){\begin{rotate}{18}{\whiten\vrule width 13pt height 4pt}\end{rotate}}
		\put(215,85){\begin{rotate}{18}{\tiny 0.98}\end{rotate}}
		\put(210,104.5){\begin{rotate}{0}{\whiten\vrule width 13pt height 4pt}\end{rotate}}
		\put(210,104.5){\begin{rotate}{0}{\tiny 0.96}\end{rotate}}
		\put(168,85){\begin{rotate}{88}{\whiten\vrule width 13pt height 4pt}\end{rotate}}
		\put(168,85){\begin{rotate}{88}{\tiny 1.04}\end{rotate}}
	\end{picture}

	\begin{picture}(275,135)(0,0)
	\put(0,78){$R$}
	\put(112,0){$\alpha$}
	\put(250,0){$\alpha$}
	\put(16,114.5){$n=0$}
	\put(244,114.5){$n=1$}
	\put(134,116){ $\Delta t$ }
	%\input{fig/dt.l}
		\thinlines
		\drawline(16,10)(16,112)
		\drawline(56,10)(56,112)
		\drawline(96,10)(96,112)
		\drawline(136,10)(136,112)
		\drawline(16,10)(136,10)
		\drawline(16,30.4)(136,30.4)
		\drawline(16,50.8)(136,50.8)
		\drawline(16,71.2)(136,71.2)
		\drawline(16,91.6)(136,91.6)
		\drawline(16,112)(136,112)
		\put(14,4){\tiny 5}
		\put(52,4){\tiny 10}
		\put(92,4){\tiny 15}
		\put(132,4){\tiny 20}
		\put(0,8){\makebox(15,4)[r]{\tiny 500}}
		\put(0,28.4){\makebox(15,4)[r]{\tiny 1000}}
		\put(0,48.8){\makebox(15,4)[r]{\tiny 1500}}
		\put(0,69.2){\makebox(15,4)[r]{\tiny 2000}}
		\put(0,89.6){\makebox(15,4)[r]{\tiny 2500}}
		\put(0,110){\makebox(15,4)[r]{\tiny 3000}}
		\thicklines
		\drawline(53.8045,112)(55.0673,105.917)(55.6986,104.099)(56.9328,102.542)(58.3968,101.503)(60.5488,101.17)(62.2712,101.651)(63.8208,102.505)(67.0064,105.323)(69.044,108.254)(71.512,112)
		\drawline(35.9534,112)(36.4413,98.7216)(37.0153,88.7811)(38.3067,78.0617)(39.3973,69.6423)(39.8278,67.2685)(41.0045,63.8931)(42.9847,60.5549)(45.1659,58.4406)(47.7202,56.5863)(51.7668,55.7331)(55.383,56.9943)(58.3968,59.4794)(68.6992,65.1543)(78.3136,74.0189)(95.476,91.0435)(114.16,112)
		\drawline(29.1516,112)(30.2422,80.1389)(31.9642,57.5136)(33.4565,49.6874)(35.896,43.0109)(38.7946,38.152)(40.4018,36.4086)(42.296,35.1104)(44.879,34.3315)(46.974,34.1091)(49.3847,34.48)(54.2062,36.5571)(64.768,41.5274)(83.164,53.9897)(95.5336,62.8915)(108.018,72.3497)(121.306,81.6966)(136,93.64)
		\drawline(25.9659,112)(26.5112,91.2291)(27.0565,78.3584)(27.5444,69.1229)(28.1758,59.1824)(29.3525,49.8726)(30.8162,41.416)(32.7677,34.7395)(34.5184,31.1789)(36.9291,28.2858)(39.426,26.4683)(42.7551,25.504)(46.2565,25.3557)(49.8439,25.9862)(52.8574,26.8022)(58.3968,28.8793)(62.816,30.7709)(67.9248,33.2189)(82.2168,40.7856)(99.2648,50.392)(136,72.016)
	%\input{fig/dt.r}
		\thinlines
		\drawline(149,10)(149,112)
		\drawline(179,10)(179,112)
		\drawline(209,10)(209,112)
		\drawline(239,10)(239,112)
		\drawline(269,10)(269,112)
		\drawline(149,10)(269,10)
		\drawline(149,30.4)(269,30.4)
		\drawline(149,50.8)(269,50.8)
		\drawline(149,71.2)(269,71.2)
		\drawline(149,91.6)(269,91.6)
		\drawline(149,112)(269,112)
		\put(147,4){\tiny 0}
		\put(177,4){\tiny 5}
		\put(205,4){\tiny 10}
		\put(235,4){\tiny 15}
		\put(265,4){\tiny 20}
		\thicklines
		\drawline(175.368,112)(175.002,88.7069)(175.325,54.1754)(176.186,44.8656)(177.477,39.5616)(178.596,36.3715)(180.254,32.8851)(181.352,31.3645)(182.923,30.1033)(184.86,29.0647)(186.668,28.8793)(189.854,28.6197)(195.902,30.2517)(205.352,35.7411)(215.705,44.7171)(236.605,63.7075)(249.37,75.9104)(269,95.8656)
		\drawline(182.342,112)(182.729,79.4343)(183.074,66.5264)(183.526,61.1114)(184.408,56.4377)(185.183,53.656)(186.108,51.3194)(187.034,49.9097)(188.648,48.2777)(190.134,47.2765)(192.566,46.7943)(195.579,47.2765)(199.755,49.5017)(205.309,54.3606)(216.458,66.8605)(227.005,81.5856)(238.629,96.4217)(248.551,112)
		\drawline(191.124,112)(192.092,103.989)(193.19,97.4605)(194.309,93.8995)(195.816,91.192)(197.301,89.6714)(198.873,88.9664)(200.724,88.744)(202.295,89.0406)(204.491,90.376)(206.901,92.5274)(209.893,96.1994)(213.187,101.429)(215.748,106.77)(217.513,112)
		\drawline(157.976,112)(158.794,92.5274)(160.365,67.5651)(161.979,50.0211)(162.862,43.6784)(163.787,37.7811)(164.691,33.0336)(165.682,28.9534)(166.909,24.6138)(168.351,21.1643)(170.503,17.4553)(172.677,15.304)(174.98,14.2654)(177.369,14.488)(180.06,15.712)(182.923,17.6037)(186.087,18.8647)(189.466,19.9774)(193.255,20.9418)(198.464,23.0931)(203.35,25.6153)(210.087,29.7694)(218.395,35.4816)(225.994,41.008)(233.807,46.6457)(250.747,59.4051)(259.486,66.1555)(269,73.3885)
	%\input{fig/dt.label}
		\put(74,34){\begin{rotate}{30}{\whiten\vrule width 4pt height 4pt}\end{rotate}}
		\put(74,34){\begin{rotate}{30}{\tiny 0}\end{rotate}}
		\put(69,42){\begin{rotate}{32}{\whiten\vrule width 8pt height 4pt}\end{rotate}}
		\put(69,42){\begin{rotate}{32}{\tiny 20}\end{rotate}}
		\put(68,63){\begin{rotate}{38}{\whiten\vrule width 7pt height 4pt}\end{rotate}}
		\put(68,63){\begin{rotate}{38}{\tiny 30}\end{rotate}}
		\put(66,101){\begin{rotate}{50}{\whiten\vrule width 8pt height 4pt}\end{rotate}}
		\put(66,101){\begin{rotate}{50}{\tiny 40}\end{rotate}}
		\put(226,38){\begin{rotate}{40}{\whiten\vrule width 4pt height 4pt}\end{rotate}}
		\put(226,38){\begin{rotate}{40}{\tiny 0}\end{rotate}}
		\put(220,46){\begin{rotate}{45}{\whiten\vrule width 8pt height 4pt}\end{rotate}}
		\put(220,46){\begin{rotate}{45}{\tiny 20}\end{rotate}}
		\put(213,60){\begin{rotate}{52}{\whiten\vrule width 8pt height 4pt}\end{rotate}}
		\put(213,60){\begin{rotate}{52}{\tiny 30}\end{rotate}}
		\put(213,97){\begin{rotate}{57}{\whiten\vrule width 8pt height 4pt}\end{rotate}}
		\put(213,97){\begin{rotate}{57}{\tiny 40}\end{rotate}}
	\end{picture}
	\caption{\label{fig:t}Zeitpunkt maximaler Verst\"arkung $\tau$. Zeitdauer positiver Aufklingrate von St\"orungen der Wellenzahl mit maximaler Verst\"arkung bezogen auf die Periodendauer $\Delta t$ in \% des Zyklus.}
	\end{center}
\end{figure}

\begin{figure} % Quasistatikparameter 1/epsilon
	\begin{center}
	\begin{picture}(275,135)(0,0)
	\put(0,78){$R$}
	\put(112,0){$\alpha$}
	\put(250,0){$\alpha$}
	\put(16,114.5){$n=0$}
	\put(244,114.5){$n=1$}
	\put(132,116){ $1/\varepsilon$ }
	%\input{fig/epsilon.l}
		\thinlines
		\drawline(16,10)(16,112)
		\drawline(56,10)(56,112)
		\drawline(96,10)(96,112)
		\drawline(136,10)(136,112)
		\drawline(16,10)(136,10)
		\drawline(16,30.4)(136,30.4)
		\drawline(16,50.8)(136,50.8)
		\drawline(16,71.2)(136,71.2)
		\drawline(16,91.6)(136,91.6)
		\drawline(16,112)(136,112)
		\put(14,4){\tiny 5}
		\put(52,4){\tiny 10}
		\put(92,4){\tiny 15}
		\put(132,4){\tiny 20}
		\put(0,8){\makebox(15,4)[r]{\tiny 500}}
		\put(0,28.4){\makebox(15,4)[r]{\tiny 1000}}
		\put(0,48.8){\makebox(15,4)[r]{\tiny 1500}}
		\put(0,69.2){\makebox(15,4)[r]{\tiny 2000}}
		\put(0,89.6){\makebox(15,4)[r]{\tiny 2500}}
		\put(0,110){\makebox(15,4)[r]{\tiny 3000}}
		\thicklines
		\drawline(30.2709,43.1965)(32.9973,48.76)(35.5802,54.3235)(39.0242,59.1456)(42.0377,62.6691)(45.3381,64.5235)(47.6341,62.4835)(47.7776,61.1856)(47.6341,56.92)(46.4861,53.2109)(44.6206,47.6474)(41.8942,39.3017)(39.0242,32.44)(37.0153,28.3229)
		\drawline(28.8933,53.248)(30.1847,55.1395)(31.4762,58.1069)(33.4278,63.04)(37.3022,73.24)(41.0619,81.5485)(46.3426,90.8583)(53.661,101.8)(60.9216,112.037)
		\drawline(48.0646,25.6153)(54.522,34.8509)(75.6448,64.0045)(91.1712,86.704)(99.3504,99.9824)(106.497,112)
		\drawline(75.3864,37.2246)(87.6696,50.6144)(98.0592,62.1869)(110.486,76.2074)(122.913,90.6726)(136,106.288)
		\drawline(28.2045,58.9971)(29.2377,59.8874)(30.1847,61.5565)(31.6484,65.4137)(33.6861,71.0144)(37.1587,81.0291)(40.3157,88.8925)(44.3336,97.5344)(48.2081,104.73)(52.3695,112)
		\drawline(29.7542,48.1664)(31.9642,52.6176)(33.6861,57.0685)(36.1256,62.5577)(38.7086,68.2697)(41.3776,72.8691)(45.9695,79.4714)(49.7866,83.6256)(54.7229,88.1504)(58.54,90.6726)(61.8688,92.8983)(63.9928,93.3805)(65.8584,93.5657)(67.092,93.1577)(68.0104,91.9709)(68.4416,90.3017)(68.3264,87.8166)(67.896,85.3686)(67.2072,81.7709)(65.0544,75.4285)(61.984,68.0474)(56.9904,56.92)(48.495,40.6)(40.8897,26.1717)
		\drawline(25.9659,112)(26.5112,91.2291)(27.0565,78.3584)(27.5444,69.1229)(28.1758,59.1824)(29.3525,49.8726)(30.8162,41.416)(32.7677,34.7395)(34.5184,31.1789)(36.9291,28.2858)(39.426,26.4683)(42.7551,25.504)(46.2565,25.3557)(49.8439,25.9862)(52.8574,26.8022)(58.3968,28.8793)(62.816,30.7709)(67.9248,33.2189)(82.2168,40.7856)(99.2648,50.392)(136,72.016)
	%\input{fig/epsilon.r}
		\thinlines
		\drawline(149,10)(149,112)
		\drawline(179,10)(179,112)
		\drawline(209,10)(209,112)
		\drawline(239,10)(239,112)
		\drawline(269,10)(269,112)
		\drawline(149,10)(269,10)
		\drawline(149,30.4)(269,30.4)
		\drawline(149,50.8)(269,50.8)
		\drawline(149,71.2)(269,71.2)
		\drawline(149,91.6)(269,91.6)
		\drawline(149,112)(269,112)
		\put(147,4){\tiny 0}
		\put(177,4){\tiny 5}
		\put(205,4){\tiny 10}
		\put(235,4){\tiny 15}
		\put(265,4){\tiny 20}
		\thicklines
		\drawline(267.902,112)(262.392,102.875)(254.535,90.4503)(250.015,83.4029)(237.876,65.7104)(229.998,54.9914)(221.323,43.5674)(214.349,35.2217)(208.666,28.9534)
		\drawline(196.268,71.1629)(193.793,68.6777)(190.779,66.1555)(187.335,63.9303)(184.989,62.0016)(184.343,60.8144)(183.935,59.0714)(183.784,56.8829)(183.741,54.2125)(184.021,51.4304)(184.559,48.0183)(185.398,44.9024)(186.173,42.5286)(187.787,39.9696)(188.993,39.0051)(190.478,39.0051)(191.662,39.5616)(193.556,41.5645)(194.718,43.5674)(197.71,48.7971)(199.604,53.8784)(201.305,57.1424)(202.812,61.4451)(204.706,67.3056)(205.631,71.7565)(205.847,73.5737)(205.847,74.9463)(205.61,75.7994)(205.309,76.3555)(204.663,76.8377)(203.996,76.8377)(203.091,76.3555)(202.058,75.7994)(198.851,73.9075)(196.053,71.1629)
		\drawline(210.798,112)(205.739,102.505)(200.573,92.8611)(194.503,82.7354)(191.102,78.3955)(189.423,76.7635)(187.206,75.5024)(185.506,75.0205)(184.387,74.9463)(183.332,74.9463)(182.686,74.7977)(182.083,74.3155)(181.545,73.6851)(181.179,72.6464)(180.749,70.7177)(180.448,68.9376)(179.952,66.0445)(179.909,62.4464)(180.06,54.7686)(180.986,47.0537)(181.761,42.9366)(183.224,37.9296)(185.721,32.44)(187.099,30.0291)(188.993,28.2117)(190.198,27.5811)(191.425,27.7666)(194.008,28.6938)(196.376,30.6595)(198.528,33.7011)(203.371,40.192)(209.076,49.2794)(217.534,63.0771)(222.636,72.2384)(227.802,82.1046)(232.429,92.1194)(235.615,98.3136)(238.371,104.693)(241.427,112)
		\drawline(203.5,112)(201.09,106.399)(197.452,99.1664)(194.546,93.5286)(191.102,88.2246)(188.627,85.6285)(187.766,84.8864)(186.324,84.7011)(184.257,85.3315)(183.052,85.9994)(181.954,86.1846)(181.072,85.9994)(180.448,85.4429)(179.651,84.6269)(179.285,83.5885)(178.683,82.1046)(178.295,80.0646)(177.52,76.4297)(176.961,72.4983)(176.465,63.3737)(176.444,58.8115)(176.788,51.8755)(177.154,47.3504)(177.929,42.7514)(178.769,39.1165)(179.716,36.2976)(182.944,27.952)(185.506,23.1302)(187.938,19.5323)
		\drawline(193.879,112)(191.447,107.178)(189.918,104.656)(188.541,102.987)(187.056,102.282)(185.592,102.282)(184.214,103.061)(183.009,104.582)(181.567,106.139)(180.06,107.178)(179.07,106.955)(178.166,106.251)(177.326,104.73)(175.927,101.392)(174.421,96.2365)(172.806,91.6)(171.881,89.3744)(171.386,88.4474)(170.675,87.9651)(169.814,88.1875)(169.126,88.5955)(168.286,89.2263)(167.554,90.8211)(166.672,93.8995)(165.94,95.68)(165.38,96.8669)(164.928,97.2377)(164.605,97.3491)(164.326,96.904)(164.089,95.6429)(164.089,93.4176)(164.218,90.4131)(164.605,83.44)(166.026,66.8976)(166.952,58.9971)(168.243,51.0595)(169.987,43.3446)(171.709,37.5216)(174.528,30.7337)(178.532,23.5382)(182.966,17.6407)
		\drawline(157.976,112)(158.794,92.5274)(160.365,67.5651)(161.979,50.0211)(162.862,43.6784)(163.787,37.7811)(164.691,33.0336)(165.682,28.9534)(166.909,24.6138)(168.351,21.1643)(170.503,17.4553)(172.677,15.304)(174.98,14.2654)(177.369,14.488)(180.06,15.712)(182.923,17.6037)(186.087,18.8647)(189.466,19.9774)(193.255,20.9418)(198.464,23.0931)(203.35,25.6153)(210.087,29.7694)(218.395,35.4816)(225.994,41.008)(233.807,46.6457)(250.747,59.4051)(259.486,66.1555)(269,73.3885)
	%\input{fig/epsilon.label}
		\put(44,40){\begin{rotate}{70}{\whiten\vrule width 4pt height 4pt}\end{rotate}}
		\put(44,40){\begin{rotate}{70}{\tiny 1}\end{rotate}}
		\put(54,47){\begin{rotate}{60}{\whiten\vrule width 9pt height 4pt}\end{rotate}}
		\put(54,47){\begin{rotate}{60}{\tiny 1.5}\end{rotate}}
		\put(52,96){\begin{rotate}{55}{\whiten\vrule width 4pt height 4pt}\end{rotate}}
		\put(52,96){\begin{rotate}{55}{\tiny 2}\end{rotate}}
		\put(75,60){\begin{rotate}{55}{\whiten\vrule width 4pt height 4pt}\end{rotate}}
		\put(75,60){\begin{rotate}{55}{\tiny 2}\end{rotate}}
		\put(37,76){\begin{rotate}{70}{\whiten\vrule width 10pt height 4pt}\end{rotate}}
		\put(37,76){\begin{rotate}{70}{\tiny 2.5}\end{rotate}}
		\put(112,76){\begin{rotate}{46}{\whiten\vrule width 10pt height 4pt}\end{rotate}}
		\put(112,76){\begin{rotate}{46}{\tiny 2.5}\end{rotate}}
		\put(202,54){\begin{rotate}{68}{\whiten\vrule width 10pt height 4pt}\end{rotate}}
		\put(202,54){\begin{rotate}{68}{\tiny 1.5}\end{rotate}}
		\put(217.5,60){\begin{rotate}{60}{\whiten\vrule width 4pt height 4pt}\end{rotate}}
		\put(217.5,60){\begin{rotate}{60}{\tiny 2}\end{rotate}}
		\put(194,90){\begin{rotate}{60}{\whiten\vrule width 10pt height 4pt}\end{rotate}}
		\put(194,90){\begin{rotate}{60}{\tiny 2.5}\end{rotate}}
		\put(247,76){\begin{rotate}{60}{\whiten\vrule width 10pt height 4pt}\end{rotate}}
		\put(247,76){\begin{rotate}{60}{\tiny 2.5}\end{rotate}}
		\put(175,92){\begin{rotate}{70}{\whiten\vrule width 10pt height 4pt}\end{rotate}}
		\put(175,92){\begin{rotate}{70}{\tiny 3.5}\end{rotate}}
	\end{picture}
	\caption{\label{fig:epsilon}\"Uberpr\"ufung der Voraussetzung der Quasistatik: F\"ur Parameter mit $\varepsilon\ll1$, ist die quasistatische Theorie eine gute Approximation.}
	\end{center}
\end{figure}

\clearpage
\paragraph{Hochfrequenzlimes}
Wir haben festgestellt, da\ss\ im Bereich gro\ss er Womersleyscher Zahlen die Str\"omung sowie die maximal verst\"arkten St\"orungen Grenzschichtcharakter besitzen.
St\"orungen in einer engen Zone nahe der Wand werden am st\"arksten angefacht.
Dies hat zur Folge, da\ss\ der Radius $L$ nicht mehr zu den Einflu\ss parametern der instabilsten St\"orungen z\"ahlt.
Vielmehr h\"angen diese nun von der Grenzschichtdicke $\delta$ ab.
Hiermit haben wir die M\"oglichkeit, die Zahl der unabh\"angigen Parameter zu reduzieren.
Statt der beiden Parametern $R$ und $\alpha$, betrachten wir nunmehr die Grenzschicht-Reynoldszahl $R/\alpha=V\delta/\nu$.\\
Analog zu den vorgestellten Ergebnissen \"uber der $R$-$\alpha$-Ebene wurden Berechnungen in Abh\"angigkeit der Grenzschicht-Rey"-nolds"-zahl durchgef\"uhrt.
Die Ergebnisse sind in den Abbildungen \ref{fig:lim:sigma} und \ref{fig:lim:t} dargestellt.\\
In den Rechnungen nimmt die Womersleyzahl Werte zwischen 30 und 50 ein.
In diesem Bereich n\"ahern sich die Ergebnisse ihren Grenzwerten f\"ur hochfrequente Oszillationen an.
Die Winkelzahl verliert im Grenz\"ubergang ebenfalls an Einflu\ss.
Die Ergebnisse f\"ur symmetrische und antimetrische St\"orungen n\"ahern sich einander an.\\
Das obere Bild der Abbildung \ref{fig:lim:sigma} zeigt die maximale Aufklingrate gem\"a\ss\  (\mbox{\ref{eq:sigmamax}}) bezogen auf die Oszillationsfrequenz.
Sie steigt mit steigender Grenz"-schicht-Reynoldszahl ann\"ahrend linear an.
Das mittlere Bild zeigt die gleiche Gr\"o\ss e bez\"uglich der Zeiteinheit $\delta/V$.
Aus dem unteren Bild k\"onnen wir ablesen, da\ss\ die Kreisfrequenz der maximal verst\"arkten St\"orungen mit steigender Reynoldszahl abnimmt.\\
Abbildung \ref{fig:lim:t} zeigt im oberen Bild die Wellenzahl bez\"uglich der Grenzschichtdicke, welche in diesem Grenzwert wie bereits erw\"ahnt einen konstanten Wert einnimmt.
Das mittlere Bild zeigt den Zeitpunkt maximaler Verst\"arkung.
Er verschiebt sich mit steigender Reynoldszahl kurz vor den Zeitpunkt der Flu\ss umkehr.\\
Der Kehrwert des Quasistatikparameters $1/\varepsilon$ ist im unteren Bild dargestellt.
F\"ur kleine Reynoldszahlen dominiert der Anteil welcher mit der Kreisfrequenz gebildet wird.
Daher f\"allt der Wert bis $R/\alpha=260$ leicht ab.
F\"ur gr\"o\ss ere Reynoldszahlen steigt er aufgrund der steigenden Aufklingrate wieder an.\\
Bei den Darstellungen sollte beachtet werden, da\ss\ die Reynoldssche Zahl sehr gro\ss e Werte annimmt.
Da wir in diesen Bereichen auch andere Instabilit\"atsmechanismen als den linearen Mechnismus vermuten, sollten die Ergebnisse f\"ur gro\ss e $R/\alpha$ nicht \"uberbewertet werden.\\

\begin{figure} % alpha->infty (1) sigma/alpha, Rsigma/alpha^2, omega/alpha
	\begin{picture}(275,80)(0,0)
	\thinlines\drawline(60,10)(250,10)(250,78)(60,78)(60,10)
	\drawline(107.5,10)(107.5,78) \drawline(155,10)(155,78) \drawline(202.5,10)(202.5,78)
	\put(42,8){\makebox(15,4)[ r ]{\tiny 0}}
	\put(42,76){\makebox(15,4)[ r ]{\tiny 5}}
	\drawline(60,23.6)(250,23.6)\put(42,21.6){\makebox(15,4)[ r ]{\tiny 1}}
	\drawline(60,37.2)(250,37.2)\put(42,35.2){\makebox(15,4)[ r ]{\tiny 2}}
	\drawline(60,50.8)(250,50.8)\put(42,48.8){\makebox(15,4)[ r ]{\tiny 3}}
	\drawline(60,64.4)(250,64.4)\put(42,62.4){\makebox(15,4)[ r ]{\tiny 4}}
	\put(20,43){$R\sigma/\alpha^2$}
	%\input{/home/elmar/dat/floquet/job-08/newplots/rsa2.path}
	\thicklines\drawline(62.2491,10)(70.0196,12.0212)(74.8932,13.3997)(82.7658,15.5781)(90.2637,17.7927)(98.0342,20.026)(106.145,22.5284)(115.552,25.1999)(130.07,29.7484)(143.055,33.5949)(157.028,37.8502)(172.841,42.8867)(189.302,48.2112)(206.411,53.7595)(233.13,62.584)(250,68.0058)
	\end{picture}
	\begin{picture}(275,80)(0,0)
	\thinlines\drawline(60,10)(250,10)(250,78)(60,78)(60,10)
	\drawline(107.5,10)(107.5,78) \drawline(155,10)(155,78) \drawline(202.5,10)(202.5,78)
	\put(42,8){\makebox(15,4)[ r ]{\tiny 0}}
	\put(42,76){\makebox(15,4)[ r ]{\tiny 0.01}}
	\drawline(60,44)(250,44)\put(42,42){\makebox(15,4)[ r ]{\tiny 0.005}}
	\drawline(60,27)(250,27) \drawline(60,61)(250,61)
	\put(22,55){$\sigma/\alpha$}
	%\input{/home/elmar/dat/floquet/job-08/newplots/sigma.path}
	\thicklines\drawline(62.2493,10)(70.0197,18.3457)(74.8933,22.9409)(82.7659,28.8543)(90.2637,33.8)(98.0341,37.839)(106.145,41.7743)(115.552,45.0305)(130.07,49.8933)(143.055,52.9228)(157.028,55.7657)(172.841,58.7124)(189.302,61.3295)(206.411,63.5962)(233.13,66.6048)(250,68.0057)
	\end{picture}
	\begin{picture}(275,80)(0,0)
	\thinlines\drawline(60,10)(250,10)(250,78)(60,78)(60,10)
	\drawline(107.5,10)(107.5,78) \drawline(155,10)(155,78) \drawline(202.5,10)(202.5,78)
	\put(42,8){\makebox(15,4)[ r ]{\tiny -0.01}}
	\put(42,76){\makebox(15,4)[ r ]{\tiny 0.03}}
	\drawline(60,27)(250,27)\put(42,25){\makebox(15,4)[ r ]{\tiny 0}}
	\drawline(60,44)(250,44)\put(42,42){\makebox(15,4)[ r ]{\tiny 0.01}}
	\drawline(60,61)(250,61)\put(42,59){\makebox(15,4)[ r ]{\tiny 0.02}}
	\put(22,49){$\omega/\alpha$}
	%\input{/home/elmar/dat/floquet/job-08/newplots/omega.path}
	\thicklines \drawline(63.1695,78)(66.5776,71.7152)(70.9399,67.4905)(78.165,62.0305)(89.0027,55.3333)(103.044,48.9457)(117.29,43.5876)(132.183,39.879)(154.267,35.2428)(176.079,31.8428)(193.63,29.679)(214.215,27.721)(250,25.5572)
	\put(56,4){\tiny 100}
	\put(102,4){\tiny 200}
	\put(149.5,4){\tiny 300}
	\put(197,4){\tiny 400}
	\put(241,4){\tiny 500}
	\put(169,-4){$R/\alpha$}
	\end{picture}
	\caption{\label{fig:lim:sigma}Hochfrequenzlimes $\alpha\to\infty$: Aufklingrate mit maximaler Verst\"arkung $\sigma$ in der Oszillationszeit $1/\Omega$ und in der Grenzschichtzeit $\delta/V$, sowie die dazugeh\"orige Kreisfrequenz $\omega$ in Abh\"angigkeit der Grenzschicht-Reynoldszahl $R/\alpha$.}
\end{figure}

\begin{figure} % alpha->infty (2) k/alpha, tau/pi, 1/epsilon ueber R/alpha
	\begin{picture}(275,80)(0,0)
	\thinlines\drawline(60,10)(250,10)(250,78)(60,78)(60,10)
	\drawline(107.5,10)(107.5,78) \drawline(155,10)(155,78) \drawline(202.5,10)(202.5,78)
	\put(42,8){\makebox(15,4)[ r ]{\tiny 0.33}}
	\put(42,76){\makebox(15,4)[ r ]{\tiny 0.37}}
	\drawline(60,27)(250,27)\put(42,25){\makebox(15,4)[ r ]{\tiny 0.34}}
	\drawline(60,44)(250,44)\put(42,42){\makebox(15,4)[ r ]{\tiny 0.35}}
	\drawline(60,61)(250,61)\put(42,59){\makebox(15,4)[ r ]{\tiny 0.36}}
	\put(22,45){$k/\alpha$}
	%\input{/home/elmar/dat/floquet/job-08/newplots/k.path}
	\thicklines \drawline(60,10.7624)(65.078,27.3091)(69.9516,40.6)(74.6888,48.1624)(79.1534,53.1285)(84.061,59.1867)(88.8323,63.1018)(95.6825,67.017)(102.635,69.9018)(111.155,71.9624)(121.039,73.0958)(132.796,73.4048)(152.734,72.7867)(169.944,71.2412)(184.292,69.9018)(197.55,68.3564)(218.373,66.7079)(236.606,65.3479)(250,64.4412)
	\end{picture}
	\begin{picture}(275,80)(0,0)
	\thinlines\drawline(60,10)(250,10)(250,78)(60,78)(60,10)
	\put(42,8){\makebox(15,4)[ r ]{\tiny 0.9}}
	\put(42,76){\makebox(15,4)[ r ]{\tiny 1.1}}
	\drawline(107.5,10)(107.5,78) \drawline(155,10)(155,78) \drawline(202.5,10)(202.5,78)
	\drawline(60,44)(250,44)\put(42,42){\makebox(15,4)[ r ]{\tiny 1}}
	\drawline(60,27)(250,27)\put(42,25){\makebox(15,4)[ r ]{\tiny 0.95}}
	\drawline(60,61)(250,61)\put(42,59){\makebox(15,4)[ r ]{\tiny 1.05}}
	\put(22,45){$\tau/\pi$}
	%\input{/home/elmar/dat/floquet/job-08/newplots/t.path}
	\thicklines \drawline(60,51.0267)(66.4413,46.9057)(73.4278,43.1962)(85.1516,39.0752)(96.2619,35.9838)(109.042,33.2028)(125.026,30.6267)(148.065,28.0505)(167.457,26.1962)(197.448,23.9091)(226.621,22.4872)(250,21.4364)
	\end{picture}
	\begin{picture}(275,80)(0,0)
	\thinlines\drawline(60,10)(250,10)(250,78)(60,78)(60,10)
	\drawline(107.5,10)(107.5,78) \drawline(155,10)(155,78) \drawline(202.5,10)(202.5,78)
	\put(42,8){\makebox(15,4)[ r ]{\tiny 1}}
	\put(42,76){\makebox(15,4)[ r ]{\tiny 5}}
	\drawline(60,27)(250,27)\put(42,25){\makebox(15,4)[ r ]{\tiny 2}}
	\drawline(60,44)(250,44)\put(42,42){\makebox(15,4)[ r ]{\tiny 3}}
	\drawline(60,61)(250,61)\put(42,59){\makebox(15,4)[ r ]{\tiny 4}}
	\put(22,45){$1/\varepsilon$}
	%\input{/home/elmar/dat/floquet/job-08/newplots/ep.path}
	\thicklines \drawline(60,47.4)(67.191,45.2364)(77.4152,42.7636)(89.5139,39.8788)(101.783,37.2)(114.563,34.7273)(125.333,33.3879)(135.012,33.1818)(141.828,33.5939)(148.303,34.0061)(156.823,35.0364)(165.718,36.8909)(178.805,40.1879)(193.971,43.7939)(204.843,47.0909)(220.759,53.8909)(239.844,61.6182)(250,65.7394)
	\put(56,4){\tiny 100}
	\put(102,4){\tiny 200}
	\put(149.5,4){\tiny 300}
	\put(197,4){\tiny 400}
	\put(241,4){\tiny 500}
	\put(169,-4){$R/\alpha$}
	\end{picture}
	\caption{\label{fig:lim:t}Hochfrequenzlimes $\alpha\to\infty$: Zur maximalen Verst\"arkung geh\"orende Wellenzahl $k$ in der L\"angeneinheit der Grenzschichtdicke $\delta$, der Zeitpunkt der Instabilit\"at $\tau$ und der Quasistatikparameter $1/\varepsilon = R|\sigma + i\omega|/\alpha^2$ in Abh\"angigkeit der Grenzschicht-Reynoldszahl $R/\alpha$.}
\end{figure}
\newpage

\paragraph{Eigenfunktionen}
In den vorherigen Abschnitten haben wir Eigenschaften von St\"orungen diskutiert, welche unter bestimmten Paramterkombinationen zu einer Instabilit\"at der Grundstr\"omung f\"uhren.
Hierbei sind die Aufklingrate und die Frequenz der St\"orungen in der Kombination $\sigma+i\omega$ Eigenwerte des Eigenwertproblems (\mbox{\ref{eq:rwp}}).
Die zugeh\"origen Eigenfunktionen $\mathfrak{v}^*$ geben Aufschlu\ss\ \"uber die radiale Verteilung der St\"orungsgeschwindigkeitskomponenten, welche sich zeitlich mit der Aufklingrate $\sigma$ und der Kreisfrequenz $\omega$ entwickeln.
Nach  (\mbox{\ref{eq:separation}}) und  (\mbox{\ref{eq:zeitansatz}}) setzt sich eine Fourierkomponente der St\"orungsgeschwindigkeit aus
\begin{equation}
	\mathfrak{v}(r,\varphi,z,t) = \mathfrak{v}^*(r) \, e^{i(kz+n\varphi)}\, e^{(\sigma+i\omega)t}
\end{equation}
zusammen.
Eine allgemeine St\"orung wird nur dann mit der vorausgesagten maximalen Aufklingrate $\sigma$ ansteigen, falls sich in ihrer spektralen Zusammensetzung ein Anteil der zugeh\"origen Eigenfunktion $\mathfrak{v}^*$ mit hinreichend gro\ss er Amplitude befindet.\\
Die Abbildungen \ref{fig:eigenfunktion:sym} und \ref{fig:eigenfunktion:asy} zeigen den physikalisch relevanten Realteil der Geschwindigkeitsverteilung
\begin{equation}
	\Re \, \mathfrak{v}(r,0,z,0) = \Re\{\mathfrak{v}^*(r) \, e^{ikz}\},
\end{equation}
welcher aus den Eigenfunktionen mit maximaler Aufklingrate abgeleitet ist.
In der Darstellung sind die Geschwindigkeitskomponenten an verschiedenen Stellen \"uber eine halbe Wellenl\"ange entlang der Rohrachse aufgetragen.\\
Es ist deutlich erkennbar, da\ss\ St\"orungen mit maximalen Aufklingraten f\"ur gro\ss e Womersleysche Zahlen Grenzschichtcharakter besitzen.
Bei Ann\"aherung an die Rohrachse $r=0$ verschwinden alle Geschwindigkeitskomponenten.
Hierbei ist zu beachten, da\ss\ bei symmetrischen St\"orungen die radiale und bei antimetrischen St\"orungen die axiale Komponente aufgrund der Stetigkeitsbedingung f\"ur $r=0$ ohnehin verschwinden m\"ussen.\\
Bei kleineren Womersleyschen Zahlen spielt die Grenzschicht eine untergeordnete Rolle.
Beispielsweise verst\"arkte eine Grundstr\"omung mit $R=2000$ und $\alpha=10$ symmetrische St\"orungen haupts\"achlich in der Rohrmitte.

\begin{figure} % Eigenfunktionen n=0
	\begin{picture}(275,3)
	\thicklines\drawline(70,2)(85,2)\put(90,0){\footnotesize$\mathfrak{v}\!\cdot\!\mathfrak{e}_r$}
	\thinlines\drawline(165,2)(180,2)\put(185,0){\footnotesize$\mathfrak{v}\!\cdot\!\mathfrak{e}_z$}
	\end{picture}
	\begin{picture}(275,50)
	\drawline(30,5)(235,5)\drawline(30,45)(235,45)
	\drawline(55,5)(55,45) \drawline(91.25,5)(91.25,45) \drawline(127.5,5)(127.5,45) \drawline(163.75,5)(163.75,45) \drawline(200,5)(200,45)
	\put(230,28){$R=2500$} \put(230,15){$\alpha=10$}
	\put(10,22){\footnotesize$r$} \put(22,3){\tiny 0} \put(22,43){\tiny 1}
		%\input{/home/elmar/dat/floquet/eig/2500-50-0.path}
		\thicklines\drawline(55,5)(55.6313,6.25)(56.2677,7.5)(56.9143,8.75)(57.5766,10)(58.2602,11.25)(58.9708,12.5)(59.7143,13.75)(60.4967,15)(61.3231,16.25)(62.1977,17.5)(63.1227,18.75)(64.0973,20)(65.1161,21.25)(66.1679,22.5)(67.2331,23.75)(68.2812,25)(69.2676,26.25)(70.13,27.5)(70.788,28.75)(71.1488,30)(71.1246,31.25)(70.6615,32.5)(69.7648,33.75)(68.4987,35)(66.9487,36.25)(65.1689,37.5)(63.1671,38.75)(60.9589,40)(58.6781,41.25)(56.6632,42.5)(55.3721,43.75)(55,45)
		\thinlines\drawline(55.7151,5)(55.7389,6.25)(55.7552,7.5)(55.7639,8.75)(55.7733,10)(55.7933,11.25)(55.8147,12.5)(55.8283,13.75)(55.8404,15)(55.8552,16.25)(55.864,17.5)(55.8641,18.75)(55.8656,20)(55.8725,21.25)(55.8799,22.5)(55.8931,23.75)(55.9311,25)(56.0202,26.25)(56.2278,27.5)(56.7151,28.75)(57.735,30)(59.534,31.25)(62.1534,32.5)(65.1747,33.75)(67.6061,35)(68.1539,36.25)(65.7913,37.5)(60.1646,38.75)(51.6647,40)(41.7454,41.25)(34.1045,42.5)(35.7411,43.75)(55,45)
		\thicklines\drawline(91.25,5)(91.6695,6.25)(92.0923,7.5)(92.5216,8.75)(92.9616,10)(93.4157,11.25)(93.888,12.5)(94.3826,13.75)(94.904,15)(95.456,16.25)(96.0417,17.5)(96.663,18.75)(97.3194,20)(98.0071,21.25)(98.7178,22.5)(99.4374,23.75)(100.143,25)(100.802,26.25)(101.363,27.5)(101.757,28.75)(101.891,30)(101.658,31.25)(100.965,32.5)(99.7732,33.75)(98.14,35)(96.2186,36.25)(94.2225,37.5)(92.3765,38.75)(90.9035,40)(90.0416,41.25)(89.9953,42.5)(90.6759,43.75)(91.25,45)
		\thinlines\drawline(100.443,5)(100.478,6.25)(100.557,7.5)(100.682,8.75)(100.858,10)(101.092,11.25)(101.38,12.5)(101.718,13.75)(102.107,15)(102.543,16.25)(103.009,17.5)(103.487,18.75)(103.952,20)(104.368,21.25)(104.678,22.5)(104.813,23.75)(104.687,25)(104.195,26.25)(103.241,27.5)(101.8,28.75)(99.9893,30)(98.096,31.25)(96.4844,32.5)(95.3219,33.75)(94.2512,35)(92.3474,36.25)(88.5647,37.5)(82.4355,38.75)(74.6421,40)(67.4548,41.25)(64.9855,42.5)(72.0906,43.75)(91.25,45)
		\thicklines\drawline(127.5,5)(127.462,6.25)(127.423,7.5)(127.384,8.75)(127.344,10)(127.303,11.25)(127.26,12.5)(127.216,13.75)(127.171,15)(127.125,16.25)(127.079,17.5)(127.032,18.75)(126.986,20)(126.94,21.25)(126.893,22.5)(126.846,23.75)(126.796,25)(126.741,26.25)(126.672,27.5)(126.572,28.75)(126.4,30)(126.095,31.25)(125.577,32.5)(124.789,33.75)(123.745,35)(122.578,36.25)(121.535,37.5)(120.926,38.75)(121.051,40)(122.113,41.25)(124.062,42.5)(126.316,43.75)(127.5,45)
		\thinlines\drawline(139.785,5)(139.811,6.25)(139.907,7.5)(140.075,8.75)(140.315,10)(140.625,11.25)(141.011,12.5)(141.476,13.75)(142.014,15)(142.615,16.25)(143.266,17.5)(143.941,18.75)(144.598,20)(145.179,21.25)(145.611,22.5)(145.788,23.75)(145.571,25)(144.787,26.25)(143.229,27.5)(140.705,28.75)(137.124,30)(132.648,31.25)(127.749,32.5)(123.084,33.75)(119.138,35)(115.898,36.25)(112.911,37.5)(109.87,38.75)(107.348,40)(107.103,41.25)(111.252,42.5)(119.663,43.75)(127.5,45)
		\thicklines\drawline(163.75,5)(163.277,6.25)(162.799,7.5)(162.314,8.75)(161.818,10)(161.305,11.25)(160.772,12.5)(160.216,13.75)(159.631,15)(159.014,16.25)(158.363,17.5)(157.676,18.75)(156.954,20)(156.201,21.25)(155.424,22.5)(154.637,23.75)(153.861,25)(153.124,26.25)(152.466,27.5)(151.93,28.75)(151.553,30)(151.355,31.25)(151.316,32.5)(151.393,33.75)(151.55,35)(151.821,36.25)(152.342,37.5)(153.327,38.75)(154.976,40)(157.34,41.25)(160.143,42.5)(162.65,43.75)(163.75,45)
		\thinlines\drawline(171.931,5)(171.933,6.25)(171.989,7.5)(172.102,8.75)(172.265,10)(172.47,11.25)(172.728,12.5)(173.047,13.75)(173.419,15)(173.833,16.25)(174.287,17.5)(174.765,18.75)(175.228,20)(175.634,21.25)(175.934,22.5)(176.05,23.75)(175.87,25)(175.252,26.25)(174.004,27.5)(171.874,28.75)(168.622,30)(164.184,31.25)(158.868,32.5)(153.433,33.75)(148.924,35)(146.245,36.25)(145.803,37.5)(147.632,38.75)(151.859,40)(158.7,41.25)(167.036,42.5)(171.827,43.75)(163.75,45)
		\thicklines\drawline(200,5)(199.369,6.25)(198.732,7.5)(198.086,8.75)(197.423,10)(196.74,11.25)(196.029,12.5)(195.286,13.75)(194.503,15)(193.677,16.25)(192.802,17.5)(191.877,18.75)(190.903,20)(189.884,21.25)(188.832,22.5)(187.767,23.75)(186.719,25)(185.732,26.25)(184.87,27.5)(184.212,28.75)(183.851,30)(183.875,31.25)(184.339,32.5)(185.235,33.75)(186.501,35)(188.051,36.25)(189.831,37.5)(191.833,38.75)(194.041,40)(196.322,41.25)(198.337,42.5)(199.628,43.75)(200,45)
		\thinlines\drawline(199.285,5)(199.261,6.25)(199.245,7.5)(199.236,8.75)(199.227,10)(199.207,11.25)(199.185,12.5)(199.172,13.75)(199.16,15)(199.145,16.25)(199.136,17.5)(199.136,18.75)(199.134,20)(199.127,21.25)(199.12,22.5)(199.107,23.75)(199.069,25)(198.98,26.25)(198.772,27.5)(198.285,28.75)(197.265,30)(195.466,31.25)(192.847,32.5)(189.825,33.75)(187.394,35)(186.846,36.25)(189.209,37.5)(194.835,38.75)(203.335,40)(213.254,41.25)(220.896,42.5)(219.259,43.75)(200,45)
	\end{picture}
	\begin{picture}(275,50)
	\drawline(30,5)(235,5)\drawline(30,45)(235,45)
	\drawline(55,5)(55,45) \drawline(91.25,5)(91.25,45) \drawline(127.5,5)(127.5,45) \drawline(163.75,5)(163.75,45) \drawline(200,5)(200,45)
	\put(230,28){$R=2000$} \put(230,15){$\alpha=10$}
	\put(10,22){\footnotesize$r$} \put(22,3){\tiny 0} \put(22,43){\tiny 1}
		%\input{/home/elmar/dat/floquet/eig/2000-50-0.path}
		\thicklines\drawline(55,5)(53.7572,6.25)(52.5464,7.5)(51.3995,8.75)(50.3487,10)(49.4255,11.25)(48.6599,12.5)(48.079,13.75)(47.7048,15)(47.552,16.25)(47.626,17.5)(47.9213,18.75)(48.4213,20)(49.0987,21.25)(49.9173,22.5)(50.8343,23.75)(51.8018,25)(52.7682,26.25)(53.6787,27.5)(54.4767,28.75)(55.1091,30)(55.5361,31.25)(55.7445,32.5)(55.759,33.75)(55.6418,35)(55.4731,36.25)(55.3204,37.5)(55.2142,38.75)(55.1485,40)(55.1026,41.25)(55.0606,42.5)(55.0203,43.75)(55,45)
		\thinlines\drawline(62.69,5)(62.5368,6.25)(62.0906,7.5)(61.3999,8.75)(60.5408,10)(59.61,11.25)(58.7125,12.5)(57.9505,13.75)(57.4087,15)(57.1383,16.25)(57.1428,17.5)(57.3708,18.75)(57.7171,20)(58.0336,21.25)(58.154,22.5)(57.9287,23.75)(57.264,25)(56.1565,26.25)(54.7168,27.5)(53.1726,28.75)(51.8372,30)(51.0346,31.25)(50.9839,32.5)(51.6771,33.75)(52.8292,35)(53.9855,36.25)(54.7693,37.5)(55.0897,38.75)(55.1168,40)(55.0601,41.25)(54.9936,42.5)(54.8967,43.75)(55,45)
		\thicklines\drawline(91.25,5)(90.0925,6.25)(88.9742,7.5)(87.9323,8.75)(87.0003,10)(86.2063,11.25)(85.5706,12.5)(85.1061,13.75)(84.8156,15)(84.694,16.25)(84.7278,17.5)(84.8987,18.75)(85.1861,20)(85.5705,21.25)(86.0368,22.5)(86.5755,23.75)(87.1812,25)(87.8473,26.25)(88.5604,27.5)(89.2931,28.75)(90.0004,30)(90.6245,31.25)(91.1078,32.5)(91.4137,33.75)(91.5444,35)(91.5442,36.25)(91.4794,37.5)(91.4067,38.75)(91.3516,40)(91.3129,41.25)(91.2815,42.5)(91.2571,43.75)(91.25,45)
		\thinlines\drawline(79.641,5)(79.6778,6.25)(79.7989,7.5)(80.039,8.75)(80.4511,10)(81.0965,11.25)(82.0329,12.5)(83.3017,13.75)(84.9144,15)(86.8413,16.25)(89.0032,17.5)(91.2729,18.75)(93.4842,20)(95.4486,21.25)(96.9789,22.5)(97.9158,23.75)(98.1503,25)(97.6439,26.25)(96.4461,27.5)(94.7098,28.75)(92.7017,30)(90.785,31.25)(89.3508,32.5)(88.6874,33.75)(88.8297,35)(89.5058,36.25)(90.2762,37.5)(90.8017,38.75)(91.0121,40)(91.0279,41.25)(90.9707,42.5)(90.9401,43.75)(91.25,45)
		\thicklines\drawline(127.5,5)(127.106,6.25)(126.735,7.5)(126.409,8.75)(126.141,10)(125.942,11.25)(125.808,12.5)(125.732,13.75)(125.696,15)(125.676,16.25)(125.65,17.5)(125.597,18.75)(125.503,20)(125.369,21.25)(125.21,22.5)(125.055,23.75)(124.944,25)(124.92,26.25)(125.018,27.5)(125.256,28.75)(125.624,30)(126.079,31.25)(126.554,32.5)(126.972,33.75)(127.275,35)(127.443,36.25)(127.504,37.5)(127.507,38.75)(127.495,40)(127.486,41.25)(127.484,42.5)(127.49,43.75)(127.5,45)
		\thinlines\drawline(103.392,5)(103.598,6.25)(104.215,7.5)(105.245,8.75)(106.687,10)(108.531,11.25)(110.753,12.5)(113.309,13.75)(116.131,15)(119.127,16.25)(122.18,17.5)(125.162,18.75)(127.943,20)(130.404,21.25)(132.448,22.5)(133.998,23.75)(134.994,25)(135.386,26.25)(135.131,27.5)(134.22,28.75)(132.716,30)(130.808,31.25)(128.83,32.5)(127.199,33.75)(126.248,35)(126.048,36.25)(126.354,37.5)(126.776,38.75)(127.047,40)(127.126,41.25)(127.111,42.5)(127.165,43.75)(127.5,45)
		\thicklines\drawline(163.75,5)(164.35,6.25)(164.944,7.5)(165.524,8.75)(166.078,10)(166.59,11.25)(167.037,12.5)(167.394,13.75)(167.633,15)(167.727,16.25)(167.656,17.5)(167.41,18.75)(166.99,20)(166.416,21.25)(165.725,22.5)(164.967,23.75)(164.204,25)(163.503,26.25)(162.929,27.5)(162.533,28.75)(162.346,30)(162.366,31.25)(162.555,32.5)(162.84,33.75)(163.137,35)(163.375,36.25)(163.526,37.5)(163.604,38.75)(163.642,40)(163.668,41.25)(163.696,42.5)(163.728,43.75)(163.75,45)
		\thinlines\drawline(141.266,5)(141.519,6.25)(142.271,7.5)(143.488,8.75)(145.115,10)(147.077,11.25)(149.283,12.5)(151.629,13.75)(154.008,15)(156.317,16.25)(158.473,17.5)(160.42,18.75)(162.142,20)(163.658,21.25)(165.019,22.5)(166.274,23.75)(167.449,25)(168.508,26.25)(169.347,27.5)(169.794,28.75)(169.675,30)(168.893,31.25)(167.53,32.5)(165.887,33.75)(164.4,35)(163.441,36.25)(163.103,37.5)(163.175,38.75)(163.347,40)(163.443,41.25)(163.48,42.5)(163.586,43.75)(163.75,45)
		\thicklines\drawline(200,5)(201.243,6.25)(202.453,7.5)(203.6,8.75)(204.652,10)(205.575,11.25)(206.339,12.5)(206.921,13.75)(207.295,15)(207.449,16.25)(207.375,17.5)(207.079,18.75)(206.579,20)(205.901,21.25)(205.082,22.5)(204.166,23.75)(203.199,25)(202.232,26.25)(201.321,27.5)(200.523,28.75)(199.891,30)(199.464,31.25)(199.255,32.5)(199.241,33.75)(199.358,35)(199.527,36.25)(199.68,37.5)(199.786,38.75)(199.852,40)(199.897,41.25)(199.939,42.5)(199.98,43.75)(200,45)
		\thinlines\drawline(192.31,5)(192.463,6.25)(192.909,7.5)(193.6,8.75)(194.459,10)(195.39,11.25)(196.288,12.5)(197.05,13.75)(197.591,15)(197.862,16.25)(197.857,17.5)(197.629,18.75)(197.283,20)(196.966,21.25)(196.846,22.5)(197.071,23.75)(197.736,25)(198.843,26.25)(200.283,27.5)(201.827,28.75)(203.162,30)(203.966,31.25)(204.017,32.5)(203.323,33.75)(202.171,35)(201.015,36.25)(200.231,37.5)(199.91,38.75)(199.883,40)(199.94,41.25)(200.006,42.5)(200.103,43.75)(200,45)
	\end{picture}
	\begin{picture}(275,50)
	\drawline(30,5)(235,5)\drawline(30,45)(235,45)
	\drawline(55,5)(55,45) \drawline(91.25,5)(91.25,45) \drawline(127.5,5)(127.5,45) \drawline(163.75,5)(163.75,45) \drawline(200,5)(200,45)
	\put(230,28){$R=2500$} \put(230,15){$\alpha=15$}
	\put(10,22){\footnotesize$r$} \put(22,3){\tiny 0} \put(22,43){\tiny 1}
		%\input{/home/elmar/dat/floquet/eig/2500-75-0.path}
		\thicklines\drawline(55,5)(55.2328,6.25)(55.4702,7.5)(55.7168,8.75)(55.9777,10)(56.2579,11.25)(56.563,12.5)(56.8993,13.75)(57.2737,15)(57.6944,16.25)(58.1707,17.5)(58.7141,18.75)(59.3385,20)(60.0607,21.25)(60.9012,22.5)(61.8836,23.75)(63.0336,25)(64.3757,26.25)(65.9271,27.5)(67.6869,28.75)(69.6179,30)(71.6205,31.25)(73.4994,32.5)(74.942,33.75)(75.5508,35)(74.98,36.25)(73.1294,37.5)(70.2102,38.75)(66.5584,40)(62.4816,41.25)(58.5317,42.5)(55.795,43.75)(55,45)
		\thinlines\drawline(55.1378,5)(55.1711,6.25)(55.1882,7.5)(55.19,8.75)(55.1905,10)(55.2062,11.25)(55.2263,12.5)(55.2395,13.75)(55.2584,15)(55.2917,16.25)(55.3278,17.5)(55.3623,18.75)(55.4077,20)(55.464,21.25)(55.5142,22.5)(55.5523,23.75)(55.5801,25)(55.5786,26.25)(55.5166,27.5)(55.3789,28.75)(55.1608,30)(54.8874,31.25)(54.7507,32.5)(55.3037,33.75)(57.4257,35)(61.6459,36.25)(66.8138,37.5)(69.4436,38.75)(65.492,40)(53.7526,41.25)(38.7754,42.5)(33.3731,43.75)(55,45)
		\thicklines\drawline(91.25,5)(91.4046,6.25)(91.562,7.5)(91.7255,8.75)(91.8986,10)(92.0846,11.25)(92.2869,12.5)(92.5098,13.75)(92.7581,15)(93.0367,16.25)(93.3518,17.5)(93.7111,18.75)(94.1239,20)(94.6015,21.25)(95.158,22.5)(95.8111,23.75)(96.5801,25)(97.485,26.25)(98.5425,27.5)(99.7589,28.75)(101.116,30)(102.552,31.25)(103.925,32.5)(104.969,33.75)(105.28,35)(104.404,36.25)(102.09,37.5)(98.5753,38.75)(94.6382,40)(91.3405,41.25)(89.7237,42.5)(90.2285,43.75)(91.25,45)
		\thinlines\drawline(93.3793,5)(93.4096,6.25)(93.4591,7.5)(93.529,8.75)(93.6276,10)(93.7653,11.25)(93.9392,12.5)(94.1472,13.75)(94.4006,15)(94.7101,16.25)(95.0784,17.5)(95.5146,18.75)(96.0386,20)(96.6671,21.25)(97.4083,22.5)(98.2737,23.75)(99.2682,25)(100.366,26.25)(101.494,27.5)(102.524,28.75)(103.231,30)(103.269,31.25)(102.254,32.5)(100.018,33.75)(96.9707,35)(94.1032,36.25)(92.0047,37.5)(89.3606,38.75)(83.3752,40)(73.3883,41.25)(64.7978,42.5)(68.3802,43.75)(91.25,45)
		\thicklines\drawline(127.5,5)(127.486,6.25)(127.471,7.5)(127.456,8.75)(127.44,10)(127.422,11.25)(127.403,12.5)(127.382,13.75)(127.359,15)(127.332,16.25)(127.302,17.5)(127.267,18.75)(127.226,20)(127.179,21.25)(127.126,22.5)(127.067,23.75)(127.004,25)(126.942,26.25)(126.886,27.5)(126.846,28.75)(126.835,30)(126.864,31.25)(126.926,32.5)(126.96,33.75)(126.791,35)(126.123,36.25)(124.7,37.5)(122.649,38.75)(120.733,40)(120.146,41.25)(121.81,42.5)(125.26,43.75)(127.5,45)
		\thinlines\drawline(130.374,5)(130.383,6.25)(130.436,7.5)(130.533,8.75)(130.672,10)(130.851,11.25)(131.077,12.5)(131.358,13.75)(131.697,15)(132.102,16.25)(132.586,17.5)(133.169,18.75)(133.864,20)(134.697,21.25)(135.695,22.5)(136.881,23.75)(138.259,25)(139.813,26.25)(141.47,27.5)(143.065,28.75)(144.283,30)(144.61,31.25)(143.311,32.5)(139.597,33.75)(133.165,35)(124.889,36.25)(116.754,37.5)(110.384,38.75)(105.871,40)(103.487,41.25)(106.316,42.5)(116.784,43.75)(127.5,45)
		\thicklines\drawline(163.75,5)(163.575,6.25)(163.397,7.5)(163.212,8.75)(163.016,10)(162.806,11.25)(162.577,12.5)(162.324,13.75)(162.042,15)(161.726,16.25)(161.368,17.5)(160.959,18.75)(160.488,20)(159.945,21.25)(159.313,22.5)(158.576,23.75)(157.719,25)(156.726,26.25)(155.589,27.5)(154.317,28.75)(152.944,30)(151.548,31.25)(150.263,32.5)(149.267,33.75)(148.717,35)(148.649,36.25)(148.951,37.5)(149.565,38.75)(150.792,40)(153.26,41.25)(157.229,42.5)(161.604,43.75)(163.75,45)
		\thinlines\drawline(165.684,5)(165.668,6.25)(165.693,7.5)(165.76,8.75)(165.858,10)(165.974,11.25)(166.119,12.5)(166.308,13.75)(166.535,15)(166.797,16.25)(167.115,17.5)(167.502,18.75)(167.962,20)(168.511,21.25)(169.181,22.5)(169.993,23.75)(170.948,25)(172.047,26.25)(173.263,27.5)(174.488,28.75)(175.504,30)(175.928,31.25)(175.106,32.5)(172.089,33.75)(166.04,35)(157.205,36.25)(147.798,37.5)(141.434,38.75)(141.037,40)(147.652,41.25)(160.243,42.5)(171.465,43.75)(163.75,45)
		\thicklines\drawline(200,5)(199.767,6.25)(199.53,7.5)(199.283,8.75)(199.022,10)(198.742,11.25)(198.437,12.5)(198.101,13.75)(197.726,15)(197.306,16.25)(196.829,17.5)(196.286,18.75)(195.662,20)(194.939,21.25)(194.099,22.5)(193.116,23.75)(191.966,25)(190.624,26.25)(189.073,27.5)(187.313,28.75)(185.382,30)(183.38,31.25)(181.501,32.5)(180.058,33.75)(179.449,35)(180.02,36.25)(181.871,37.5)(184.79,38.75)(188.442,40)(192.518,41.25)(196.468,42.5)(199.205,43.75)(200,45)
		\thinlines\drawline(199.862,5)(199.829,6.25)(199.812,7.5)(199.81,8.75)(199.809,10)(199.794,11.25)(199.774,12.5)(199.76,13.75)(199.742,15)(199.708,16.25)(199.672,17.5)(199.638,18.75)(199.592,20)(199.536,21.25)(199.486,22.5)(199.448,23.75)(199.42,25)(199.421,26.25)(199.483,27.5)(199.621,28.75)(199.839,30)(200.113,31.25)(200.249,32.5)(199.696,33.75)(197.574,35)(193.354,36.25)(188.186,37.5)(185.556,38.75)(189.508,40)(201.247,41.25)(216.224,42.5)(221.627,43.75)(200,45)
	\end{picture}
	\begin{picture}(275,50)
	\drawline(30,5)(235,5)\drawline(30,45)(235,45)
	\drawline(55,5)(55,45) \drawline(91.25,5)(91.25,45) \drawline(127.5,5)(127.5,45) \drawline(163.75,5)(163.75,45) \drawline(200,5)(200,45)
	\put(230,28){$R=2000$} \put(230,15){$\alpha=15$}
	\put(10,22){\footnotesize$r$} \put(22,3){\tiny 0} \put(22,43){\tiny 1}
		%\input{/home/elmar/dat/floquet/eig/2000-75-0.path}
		\thicklines\drawline(55,5)(55.2022,6.25)(55.4083,7.5)(55.6222,8.75)(55.848,10)(56.0903,11.25)(56.3535,12.5)(56.643,13.75)(56.9647,15)(57.3252,16.25)(57.7327,17.5)(58.1968,18.75)(58.7294,20)(59.345,21.25)(60.0614,22.5)(60.8988,23.75)(61.8795,25)(63.0237,26.25)(64.3433,27.5)(65.83,28.75)(67.4364,30)(69.0518,31.25)(70.4812,32.5)(71.4485,33.75)(71.6546,35)(70.8939,36.25)(69.1672,37.5)(66.6722,38.75)(63.6714,40)(60.4466,41.25)(57.4876,42.5)(55.5518,43.75)(55,45)
		\thinlines\drawline(54.8429,5)(54.8228,6.25)(54.8113,7.5)(54.808,8.75)(54.8045,10)(54.7907,11.25)(54.7729,12.5)(54.7575,13.75)(54.7366,15)(54.7044,16.25)(54.6674,17.5)(54.6275,18.75)(54.5774,20)(54.5179,21.25)(54.4617,22.5)(54.4185,23.75)(54.3985,25)(54.4319,26.25)(54.5629,27.5)(54.8243,28.75)(55.2141,30)(55.6375,31.25)(55.785,32.5)(55.0635,33.75)(52.8266,35)(49.0577,36.25)(45.1637,37.5)(43.912,38.75)(47.8712,40)(57.1863,41.25)(67.7607,42.5)(70.4468,43.75)(55,45)
		\thicklines\drawline(91.25,5)(91.4031,6.25)(91.5593,7.5)(91.7214,8.75)(91.8925,10)(92.0761,11.25)(92.2754,12.5)(92.495,13.75)(92.739,15)(93.0128,16.25)(93.3226,17.5)(93.676,18.75)(94.082,20)(94.5516,21.25)(95.0974,22.5)(95.7331,23.75)(96.4725,25)(97.3252,26.25)(98.2922,27.5)(99.3556,28.75)(100.468,30)(101.542,31.25)(102.453,32.5)(103.068,33.75)(103.308,35)(103.199,36.25)(102.838,37.5)(102.217,38.75)(101.064,40)(98.9675,41.25)(95.889,42.5)(92.7303,43.75)(91.25,45)
		\thinlines\drawline(92.9239,5)(92.9176,6.25)(92.9426,7.5)(92.9987,8.75)(93.0798,10)(93.1785,11.25)(93.3026,12.5)(93.461,13.75)(93.6508,15)(93.8725,16.25)(94.1393,17.5)(94.4629,18.75)(94.8493,20)(95.3139,21.25)(95.8829,22.5)(96.5766,23.75)(97.4052,25)(98.3724,26.25)(99.4509,27.5)(100.532,28.75)(101.376,30)(101.56,31.25)(100.434,32.5)(97.2533,33.75)(91.6399,35)(84.2748,36.25)(77.1846,37.5)(72.9906,38.75)(73.5802,40)(79.4738,41.25)(89.2584,42.5)(96.8688,43.75)(91.25,45)
		\thicklines\drawline(127.5,5)(127.514,6.25)(127.529,7.5)(127.545,8.75)(127.561,10)(127.578,11.25)(127.597,12.5)(127.618,13.75)(127.641,15)(127.668,16.25)(127.699,17.5)(127.734,18.75)(127.776,20)(127.824,21.25)(127.88,22.5)(127.941,23.75)(128.006,25)(128.068,26.25)(128.116,27.5)(128.133,28.75)(128.1,30)(128.003,31.25)(127.862,32.5)(127.764,33.75)(127.898,35)(128.505,36.25)(129.721,37.5)(131.338,38.75)(132.708,40)(132.968,41.25)(131.573,42.5)(129.042,43.75)(127.5,45)
		\thinlines\drawline(130.024,5)(130.036,6.25)(130.082,7.5)(130.165,8.75)(130.283,10)(130.437,11.25)(130.63,12.5)(130.869,13.75)(131.159,15)(131.504,16.25)(131.919,17.5)(132.416,18.75)(133.013,20)(133.729,21.25)(134.59,22.5)(135.614,23.75)(136.806,25)(138.141,26.25)(139.535,27.5)(140.802,28.75)(141.606,30)(141.443,31.25)(139.703,32.5)(135.927,33.75)(130.225,35)(123.578,36.25)(117.445,37.5)(112.765,38.75)(109.64,40)(108.66,41.25)(111.923,42.5)(119.999,43.75)(127.5,45)
		\thicklines\drawline(163.75,5)(163.617,6.25)(163.482,7.5)(163.342,8.75)(163.193,10)(163.034,11.25)(162.861,12.5)(162.671,13.75)(162.461,15)(162.224,16.25)(161.958,17.5)(161.655,18.75)(161.308,20)(160.907,21.25)(160.44,22.5)(159.891,23.75)(159.243,25)(158.478,26.25)(157.579,27.5)(156.54,28.75)(155.38,30)(154.17,31.25)(153.059,32.5)(152.306,33.75)(152.255,35)(153.222,36.25)(155.303,37.5)(158.21,38.75)(161.301,40)(163.765,41.25)(164.871,42.5)(164.45,43.75)(163.75,45)
		\thinlines\drawline(165.646,5)(165.668,6.25)(165.709,7.5)(165.77,8.75)(165.856,10)(165.974,11.25)(166.124,12.5)(166.304,13.75)(166.523,15)(166.79,16.25)(167.11,17.5)(167.49,18.75)(167.947,20)(168.496,21.25)(169.144,22.5)(169.899,23.75)(170.756,25)(171.676,26.25)(172.569,27.5)(173.28,28.75)(173.573,30)(173.158,31.25)(171.824,32.5)(169.664,33.75)(167.214,35)(165.179,36.25)(163.595,37.5)(161.171,38.75)(156.162,40)(148.882,41.25)(143.712,42.5)(147.524,43.75)(163.75,45)
		\thicklines\drawline(200,5)(199.798,6.25)(199.592,7.5)(199.378,8.75)(199.152,10)(198.91,11.25)(198.646,12.5)(198.357,13.75)(198.035,15)(197.675,16.25)(197.267,17.5)(196.803,18.75)(196.271,20)(195.655,21.25)(194.939,22.5)(194.101,23.75)(193.12,25)(191.976,26.25)(190.657,27.5)(189.17,28.75)(187.564,30)(185.948,31.25)(184.519,32.5)(183.551,33.75)(183.345,35)(184.106,36.25)(185.833,37.5)(188.328,38.75)(191.329,40)(194.553,41.25)(197.512,42.5)(199.448,43.75)(200,45)
		\thinlines\drawline(200.157,5)(200.177,6.25)(200.189,7.5)(200.191,8.75)(200.196,10)(200.209,11.25)(200.228,12.5)(200.242,13.75)(200.264,15)(200.296,16.25)(200.332,17.5)(200.373,18.75)(200.422,20)(200.481,21.25)(200.538,22.5)(200.581,23.75)(200.602,25)(200.568,26.25)(200.436,27.5)(200.175,28.75)(199.786,30)(199.362,31.25)(199.215,32.5)(199.936,33.75)(202.174,35)(205.942,36.25)(209.837,37.5)(211.088,38.75)(207.128,40)(197.814,41.25)(187.239,42.5)(184.553,43.75)(200,45)
	\end{picture}
	\begin{picture}(275,50)
	\drawline(30,5)(235,5)\drawline(30,45)(235,45)
	\drawline(55,5)(55,45) \drawline(91.25,5)(91.25,45) \drawline(127.5,5)(127.5,45) \drawline(163.75,5)(163.75,45) \drawline(200,5)(200,45)
	\put(230,28){$R=2500$} \put(230,15){$\alpha=20$}
	\put(10,22){\footnotesize$r$} \put(22,3){\tiny 0} \put(22,43){\tiny 1}
		%\input{/home/elmar/dat/floquet/eig/2500-99-0.path}
		\thicklines\drawline(55,5)(55.0611,6.25)(55.1241,7.5)(55.1914,8.75)(55.2653,10)(55.3485,11.25)(55.4436,12.5)(55.5542,13.75)(55.6842,15)(55.8384,16.25)(56.0222,17.5)(56.2424,18.75)(56.5071,20)(56.826,21.25)(57.2113,22.5)(57.6784,23.75)(58.2474,25)(58.9451,26.25)(59.8075,27.5)(60.8819,28.75)(62.2252,30)(63.8928,31.25)(65.908,32.5)(68.1956,33.75)(70.4812,35)(72.2102,36.25)(72.6414,37.5)(71.2483,38.75)(68.1427,40)(63.9395,41.25)(59.4401,42.5)(56.0334,43.75)(55,45)
		\thinlines\drawline(54.9942,5)(54.9626,6.25)(54.9479,7.5)(54.9493,8.75)(54.9534,10)(54.944,11.25)(54.9322,12.5)(54.929,13.75)(54.9227,15)(54.9053,16.25)(54.8882,17.5)(54.8755,18.75)(54.8538,20)(54.8201,21.25)(54.7844,22.5)(54.743,23.75)(54.6804,25)(54.5984,26.25)(54.5118,27.5)(54.4301,28.75)(54.3845,30)(54.4671,31.25)(54.8031,32.5)(55.4378,33.75)(56.0855,35)(55.7411,36.25)(52.8482,37.5)(47.3699,38.75)(43.6095,40)(48.1586,41.25)(61.6612,42.5)(71.4136,43.75)(55,45)
		\thicklines\drawline(91.25,5)(91.2963,6.25)(91.3444,7.5)(91.3959,8.75)(91.452,10)(91.5152,11.25)(91.5877,12.5)(91.672,13.75)(91.7708,15)(91.888,16.25)(92.0281,17.5)(92.1955,18.75)(92.397,20)(92.6397,21.25)(92.9336,22.5)(93.2901,23.75)(93.7256,25)(94.2608,26.25)(94.9233,27.5)(95.7466,28.75)(96.7678,30)(98.0138,31.25)(99.4743,32.5)(101.055,33.75)(102.535,35)(103.587,36.25)(103.952,37.5)(103.669,38.75)(102.912,40)(101.335,41.25)(98.1039,42.5)(93.6796,43.75)(91.25,45)
		\thinlines\drawline(91.6556,5)(91.6321,6.25)(91.6332,7.5)(91.6586,8.75)(91.6973,10)(91.736,11.25)(91.7861,12.5)(91.8597,13.75)(91.9489,15)(92.0497,16.25)(92.1763,17.5)(92.3372,18.75)(92.5267,20)(92.7499,21.25)(93.0267,22.5)(93.3686,23.75)(93.7833,25)(94.3023,26.25)(94.9768,27.5)(95.8551,28.75)(96.9846,30)(98.3956,31.25)(99.9828,32.5)(101.281,33.75)(101.207,35)(98.0546,36.25)(90.5266,37.5)(80.1305,38.75)(72.229,40)(72.4576,41.25)(82.335,42.5)(95.405,43.75)(91.25,45)
		\thicklines\drawline(127.5,5)(127.504,6.25)(127.509,7.5)(127.515,8.75)(127.52,10)(127.527,11.25)(127.534,12.5)(127.542,13.75)(127.552,15)(127.564,16.25)(127.578,17.5)(127.595,18.75)(127.615,20)(127.639,21.25)(127.67,22.5)(127.707,23.75)(127.754,25)(127.813,26.25)(127.887,27.5)(127.977,28.75)(128.078,30)(128.173,31.25)(128.223,32.5)(128.171,33.75)(127.978,35)(127.737,36.25)(127.822,37.5)(128.814,38.75)(130.85,40)(132.823,41.25)(132.753,42.5)(129.903,43.75)(127.5,45)
		\thinlines\drawline(128.079,5)(128.078,6.25)(128.094,7.5)(128.129,8.75)(128.179,10)(128.243,11.25)(128.326,12.5)(128.433,13.75)(128.566,15)(128.726,16.25)(128.922,17.5)(129.162,18.75)(129.452,20)(129.801,21.25)(130.228,22.5)(130.753,23.75)(131.402,25)(132.218,26.25)(133.259,27.5)(134.582,28.75)(136.225,30)(138.138,31.25)(140.047,32.5)(141.248,33.75)(140.496,35)(136.382,36.25)(128.629,37.5)(119.405,38.75)(111.991,40)(107.765,41.25)(108.231,42.5)(116.962,43.75)(127.5,45)
		\thicklines\drawline(163.75,5)(163.71,6.25)(163.669,7.5)(163.625,8.75)(163.577,10)(163.522,11.25)(163.46,12.5)(163.388,13.75)(163.303,15)(163.202,16.25)(163.082,17.5)(162.939,18.75)(162.766,20)(162.557,21.25)(162.306,22.5)(162.002,23.75)(161.633,25)(161.182,26.25)(160.624,27.5)(159.928,28.75)(159.05,30)(157.938,31.25)(156.548,32.5)(154.894,33.75)(153.141,35)(151.748,36.25)(151.503,37.5)(153.19,38.75)(156.825,40)(161.193,41.25)(164.325,42.5)(164.718,43.75)(163.75,45)
		\thinlines\drawline(164.164,5)(164.185,6.25)(164.207,7.5)(164.23,8.75)(164.263,10)(164.315,11.25)(164.382,12.5)(164.46,13.75)(164.558,15)(164.684,16.25)(164.834,17.5)(165.013,18.75)(165.233,20)(165.504,21.25)(165.832,22.5)(166.232,23.75)(166.735,25)(167.37,26.25)(168.167,27.5)(169.161,28.75)(170.355,30)(171.649,31.25)(172.761,32.5)(173.162,33.75)(172.172,35)(169.507,36.25)(166.07,37.5)(163.421,38.75)(160.838,40)(154.633,41.25)(145.415,42.5)(144.693,43.75)(163.75,45)
		\thicklines\drawline(200,5)(199.939,6.25)(199.876,7.5)(199.809,8.75)(199.735,10)(199.652,11.25)(199.556,12.5)(199.446,13.75)(199.316,15)(199.162,16.25)(198.978,17.5)(198.758,18.75)(198.493,20)(198.174,21.25)(197.789,22.5)(197.322,23.75)(196.753,25)(196.055,26.25)(195.193,27.5)(194.118,28.75)(192.775,30)(191.107,31.25)(189.092,32.5)(186.804,33.75)(184.519,35)(182.79,36.25)(182.359,37.5)(183.752,38.75)(186.857,40)(191.06,41.25)(195.56,42.5)(198.967,43.75)(200,45)
		\thinlines\drawline(200.006,5)(200.038,6.25)(200.052,7.5)(200.051,8.75)(200.046,10)(200.057,11.25)(200.068,12.5)(200.071,13.75)(200.077,15)(200.094,16.25)(200.112,17.5)(200.125,18.75)(200.146,20)(200.18,21.25)(200.216,22.5)(200.257,23.75)(200.319,25)(200.402,26.25)(200.489,27.5)(200.57,28.75)(200.615,30)(200.534,31.25)(200.197,32.5)(199.562,33.75)(198.915,35)(199.259,36.25)(202.152,37.5)(207.63,38.75)(211.391,40)(206.841,41.25)(193.339,42.5)(183.586,43.75)(200,45)
	\end{picture}
	\begin{picture}(275,14)
	\put(53,10){\footnotesize$0$}
	\put(86,10){\footnotesize$1/4$}
	\put(122,10){\footnotesize$1/2$}
	\put(157,10){\footnotesize$3/4$}
	\put(199,10){\footnotesize$1$}
	\put(118,-2){\footnotesize$kz/\pi$}
	\end{picture}
	\caption{\label{fig:eigenfunktion:sym}Eigenfunktionen symmetrischer St\"orungen ($n=0$) zu Eigenwerten mit maximaler Verst\"arkung.}
\end{figure}

\begin{figure} % Eigenfunktionen n=1
	\begin{picture}(275,3)
	\thicklines\drawline(65,2)(80,2)\put(85,0){\footnotesize$\mathfrak{v}\!\cdot\!\mathfrak{e}_r$}
	\dottedline{2}(115,2)(130,2)\put(135,0){\footnotesize$\mathfrak{v}\!\cdot\!\mathfrak{e}_\varphi$}
	\thinlines\drawline(165,2)(180,2)\put(185,0){\footnotesize$\mathfrak{v}\!\cdot\!\mathfrak{e}_z$}
	\end{picture}
	\begin{picture}(275,50)
	\drawline(30,5)(235,5)\drawline(30,45)(235,45)
	\drawline(55,5)(55,45) \drawline(91.25,5)(91.25,45) \drawline(127.5,5)(127.5,45) \drawline(163.75,5)(163.75,45) \drawline(200,5)(200,45)
	\put(230,28){$R=2500$} \put(230,15){$\alpha=5$}
	\put(10,22){\footnotesize$r$} \put(22,3){\tiny 0} \put(22,43){\tiny 1}
		%\input{/home/elmar/dat/floquet/eig/2500-25-1.path}
		\thicklines\drawline(59.8496,5)(59.8476,6.25)(59.8357,7.5)(59.8168,8.75)(59.7908,10)(59.7591,11.25)(59.723,12.5)(59.6833,13.75)(59.6397,15)(59.5893,16.25)(59.526,17.5)(59.4414,18.75)(59.3271,20)(59.1789,21.25)(59.0005,22.5)(58.8044,23.75)(58.6094,25)(58.4345,26.25)(58.291,27.5)(58.1786,28.75)(58.0847,30)(57.9891,31.25)(57.8701,32.5)(57.7101,33.75)(57.4974,35)(57.2273,36.25)(56.9008,37.5)(56.5259,38.75)(56.1205,40)(55.7155,41.25)(55.356,42.5)(55.0982,43.75)(55,45)
		\thinlines\dottedline{3}(59.0025,5)(59.0034,6.25)(59.0256,7.5)(59.0521,8.75)(59.079,10)(59.1112,11.25)(59.1549,12.5)(59.2573,13.75)(59.5102,15)(60.0445,16.25)(61.0235,17.5)(62.5627,18.75)(64.6363,20)(67.0081,21.25)(69.1994,22.5)(70.5972,23.75)(70.6691,25)(69.178,26.25)(66.3291,27.5)(62.7219,28.75)(59.111,30)(56.1413,31.25)(54.1458,32.5)(53.1029,33.75)(52.7734,35)(52.8586,36.25)(53.1279,37.5)(53.4966,38.75)(53.9714,40)(54.553,41.25)(55.1461,42.5)(55.4421,43.75)(55,45)
		\thinlines\drawline(55,5)(54.5986,6.25)(54.1939,7.5)(53.7994,8.75)(53.4007,10)(53.0017,11.25)(52.5681,12.5)(52.0116,13.75)(51.196,15)(49.9086,16.25)(47.9292,17.5)(45.1355,18.75)(41.6049,20)(37.7369,21.25)(34.2308,22.5)(31.9047,23.75)(31.4378,25)(33.0631,26.25)(36.438,27.5)(40.775,28.75)(45.1193,30)(48.7086,31.25)(51.1998,32.5)(52.6483,33.75)(53.389,35)(53.8334,36.25)(54.2917,37.5)(54.9411,38.75)(55.7734,40)(56.5753,41.25)(57.0139,42.5)(56.6407,43.75)(55,45)
		\thicklines\drawline(91.849,5)(91.8531,6.25)(91.8535,7.5)(91.8562,8.75)(91.8612,10)(91.8705,11.25)(91.8852,12.5)(91.9053,13.75)(91.9287,15)(91.9519,16.25)(91.9693,17.5)(91.9764,18.75)(91.973,20)(91.9648,21.25)(91.9651,22.5)(91.9911,23.75)(92.0576,25)(92.1715,26.25)(92.326,27.5)(92.5035,28.75)(92.6791,30)(92.828,31.25)(92.9317,32.5)(92.9785,33.75)(92.9629,35)(92.8826,36.25)(92.7362,37.5)(92.5235,38.75)(92.2505,40)(91.9351,41.25)(91.616,42.5)(91.3582,43.75)(91.25,45)
		\thinlines\dottedline{3}(97.5094,5)(97.5165,6.25)(97.5449,7.5)(97.5898,8.75)(97.6558,10)(97.7595,11.25)(97.9329,12.5)(98.2383,13.75)(98.7607,15)(99.5814,16.25)(100.734,17.5)(102.133,18.75)(103.519,20)(104.467,21.25)(104.485,22.5)(103.19,23.75)(100.514,25)(96.7986,26.25)(92.726,27.5)(89.0899,28.75)(86.5005,30)(85.1941,31.25)(85.0156,32.5)(85.5599,33.75)(86.3825,35)(87.1523,36.25)(87.7155,37.5)(88.0918,38.75)(88.4171,40)(88.8681,41.25)(89.565,42.5)(90.4432,43.75)(91.25,45)
		\thinlines\drawline(91.25,5)(90.7048,6.25)(90.1544,7.5)(89.5953,8.75)(89.0063,10)(88.3562,11.25)(87.5783,12.5)(86.5666,13.75)(85.1954,15)(83.3495,16.25)(81.0085,17.5)(78.3379,18.75)(75.7386,20)(73.8219,21.25)(73.2467,22.5)(74.4642,23.75)(77.487,25)(81.7899,26.25)(86.447,27.5)(90.4589,28.75)(93.0967,30)(94.118,31.25)(93.7547,32.5)(92.5127,33.75)(90.9454,35)(89.4913,36.25)(88.4212,37.5)(87.8805,38.75)(87.9188,40)(88.4869,41.25)(89.4272,42.5)(90.4553,43.75)(91.25,45)
		\thicklines\drawline(123.498,5)(123.505,6.25)(123.518,7.5)(123.54,8.75)(123.574,10)(123.618,11.25)(123.675,12.5)(123.743,13.75)(123.82,15)(123.903,16.25)(123.991,17.5)(124.086,18.75)(124.195,20)(124.332,21.25)(124.511,22.5)(124.744,23.75)(125.033,25)(125.369,26.25)(125.731,27.5)(126.094,28.75)(126.436,30)(126.743,31.25)(127.008,32.5)(127.234,33.75)(127.425,35)(127.582,36.25)(127.701,37.5)(127.775,38.75)(127.794,40)(127.753,41.25)(127.662,42.5)(127.555,43.75)(127.5,45)
		\thinlines\dottedline{3}(132.35,5)(132.359,6.25)(132.377,7.5)(132.414,8.75)(132.48,10)(132.595,11.25)(132.796,12.5)(133.126,13.75)(133.611,15)(134.238,16.25)(134.889,17.5)(135.329,18.75)(135.214,20)(134.184,21.25)(132.017,22.5)(128.789,23.75)(124.933,25)(121.169,26.25)(118.258,27.5)(116.723,28.75)(116.672,30)(117.794,31.25)(119.537,32.5)(121.35,33.75)(122.843,35)(123.846,36.25)(124.374,37.5)(124.537,38.75)(124.522,40)(124.579,41.25)(124.971,42.5)(125.917,43.75)(127.5,45)
		\thinlines\drawline(127.5,5)(127.13,6.25)(126.757,7.5)(126.36,8.75)(125.926,10)(125.406,11.25)(124.739,12.5)(123.865,13.75)(122.742,15)(121.418,16.25)(120.087,17.5)(119.104,18.75)(118.959,20)(120.116,21.25)(122.809,22.5)(126.857,23.75)(131.599,25)(136.058,26.25)(139.27,27.5)(140.606,28.75)(139.992,30)(137.847,31.25)(134.843,32.5)(131.637,33.75)(128.68,35)(126.179,36.25)(124.208,37.5)(122.794,38.75)(122.016,40)(122.017,41.25)(122.908,42.5)(124.735,43.75)(127.5,45)
		\thicklines\drawline(157.491,5)(157.498,6.25)(157.515,7.5)(157.544,8.75)(157.586,10)(157.64,11.25)(157.706,12.5)(157.782,13.75)(157.867,15)(157.962,16.25)(158.069,17.5)(158.195,18.75)(158.354,20)(158.555,21.25)(158.808,22.5)(159.111,23.75)(159.453,25)(159.814,26.25)(160.172,27.5)(160.508,28.75)(160.817,30)(161.101,31.25)(161.373,32.5)(161.646,33.75)(161.931,35)(162.233,36.25)(162.548,37.5)(162.866,38.75)(163.166,40)(163.423,41.25)(163.613,42.5)(163.719,43.75)(163.75,45)
		\thinlines\dottedline{3}(164.349,5)(164.355,6.25)(164.352,7.5)(164.359,8.75)(164.387,10)(164.445,11.25)(164.557,12.5)(164.717,13.75)(164.882,15)(164.947,16.25)(164.716,17.5)(163.938,18.75)(162.391,20)(159.985,21.25)(156.904,22.5)(153.632,23.75)(150.855,25)(149.248,26.25)(149.204,27.5)(150.669,28.75)(153.187,30)(156.08,31.25)(158.724,32.5)(160.743,33.75)(162.031,35)(162.681,36.25)(162.863,37.5)(162.718,38.75)(162.372,40)(162,41.25)(161.858,42.5)(162.318,43.75)(163.75,45)
		\thinlines\drawline(163.75,5)(163.772,6.25)(163.794,7.5)(163.793,8.75)(163.768,10)(163.682,11.25)(163.517,12.5)(163.293,13.75)(163.075,15)(163.05,16.25)(163.508,17.5)(164.788,18.75)(167.182,20)(170.736,21.25)(175.119,22.5)(179.626,23.75)(183.309,25)(185.313,26.25)(185.198,27.5)(183.076,28.75)(179.57,30)(175.515,31.25)(171.629,32.5)(168.339,33.75)(165.724,35)(163.641,36.25)(161.923,37.5)(160.464,38.75)(159.325,40)(158.759,41.25)(159.079,42.5)(160.635,43.75)(163.75,45)
		\thicklines\drawline(195.15,5)(195.152,6.25)(195.164,7.5)(195.183,8.75)(195.209,10)(195.241,11.25)(195.277,12.5)(195.317,13.75)(195.36,15)(195.411,16.25)(195.474,17.5)(195.559,18.75)(195.673,20)(195.821,21.25)(196,22.5)(196.196,23.75)(196.391,25)(196.566,26.25)(196.709,27.5)(196.821,28.75)(196.915,30)(197.011,31.25)(197.13,32.5)(197.29,33.75)(197.503,35)(197.773,36.25)(198.099,37.5)(198.474,38.75)(198.879,40)(199.284,41.25)(199.644,42.5)(199.902,43.75)(200,45)
		\thinlines\dottedline{3}(195.998,5)(195.997,6.25)(195.974,7.5)(195.948,8.75)(195.921,10)(195.889,11.25)(195.845,12.5)(195.743,13.75)(195.49,15)(194.956,16.25)(193.977,17.5)(192.437,18.75)(190.364,20)(187.992,21.25)(185.801,22.5)(184.403,23.75)(184.331,25)(185.822,26.25)(188.671,27.5)(192.278,28.75)(195.889,30)(198.859,31.25)(200.854,32.5)(201.897,33.75)(202.227,35)(202.142,36.25)(201.872,37.5)(201.504,38.75)(201.028,40)(200.447,41.25)(199.854,42.5)(199.558,43.75)(200,45)
		\thinlines\drawline(200,5)(200.402,6.25)(200.806,7.5)(201.201,8.75)(201.599,10)(201.998,11.25)(202.432,12.5)(202.988,13.75)(203.803,15)(205.091,16.25)(207.07,17.5)(209.864,18.75)(213.395,20)(217.264,21.25)(220.77,22.5)(223.096,23.75)(223.562,25)(221.937,26.25)(218.561,27.5)(214.224,28.75)(209.88,30)(206.292,31.25)(203.8,32.5)(202.352,33.75)(201.611,35)(201.167,36.25)(200.709,37.5)(200.059,38.75)(199.227,40)(198.425,41.25)(197.986,42.5)(198.359,43.75)(200,45)
	\end{picture}
	\begin{picture}(275,50)
	\drawline(30,5)(235,5)\drawline(30,45)(235,45)
	\drawline(55,5)(55,45) \drawline(91.25,5)(91.25,45) \drawline(127.5,5)(127.5,45) \drawline(163.75,5)(163.75,45) \drawline(200,5)(200,45)
	\put(230,28){$R=1500$} \put(230,15){$\alpha=5$}
	\put(10,22){\footnotesize$r$} \put(22,3){\tiny 0} \put(22,43){\tiny 1}
		%\input{/home/elmar/dat/floquet/eig/1500-25-1.path}
		\thicklines\drawline(60.3737,5)(60.3743,6.25)(60.3728,7.5)(60.371,8.75)(60.3684,10)(60.3638,11.25)(60.3541,12.5)(60.3331,13.75)(60.2927,15)(60.2233,16.25)(60.1159,17.5)(59.9656,18.75)(59.7732,20)(59.5476,21.25)(59.3042,22.5)(59.0625,23.75)(58.8407,25)(58.6508,26.25)(58.4951,27.5)(58.3657,28.75)(58.2464,30)(58.1171,31.25)(57.9582,32.5)(57.7544,33.75)(57.4968,35)(57.1838,36.25)(56.8215,37.5)(56.4245,38.75)(56.0166,40)(55.6304,41.25)(55.3052,42.5)(55.0823,43.75)(55,45)
		\thinlines\dottedline{3}(59.257,5)(59.2466,6.25)(59.2312,7.5)(59.2162,8.75)(59.2298,10)(59.3235,11.25)(59.5658,12.5)(60.0522,13.75)(60.8813,15)(62.1195,16.25)(63.7686,17.5)(65.7177,18.75)(67.7231,20)(69.4376,21.25)(70.4702,22.5)(70.495,23.75)(69.3654,25)(67.1701,26.25)(64.2322,27.5)(61.0235,28.75)(58.0212,30)(55.5896,31.25)(53.9041,32.5)(52.9465,33.75)(52.5784,35)(52.6198,36.25)(52.9138,37.5)(53.3703,38.75)(53.9344,40)(54.5356,41.25)(55.0538,42.5)(55.283,43.75)(55,45)
		\thinlines\drawline(55,5)(54.733,6.25)(54.4579,7.5)(54.1624,8.75)(53.7912,10)(53.2699,11.25)(52.4793,12.5)(51.2725,13.75)(49.5198,15)(47.1391,16.25)(44.1654,17.5)(40.8024,18.75)(37.417,20)(34.5092,21.25)(32.6,22.5)(32.0817,23.75)(33.1072,25)(35.5136,26.25)(38.8603,27.5)(42.5679,28.75)(46.0701,30)(48.969,31.25)(51.1135,32.5)(52.5649,33.75)(53.5341,35)(54.2731,36.25)(54.9685,37.5)(55.7085,38.75)(56.4435,40)(56.9856,41.25)(57.0916,42.5)(56.4982,43.75)(55,45)
		\thicklines\drawline(92.0397,5)(92.0465,6.25)(92.0594,7.5)(92.0807,8.75)(92.1071,10)(92.1358,11.25)(92.1616,12.5)(92.1781,13.75)(92.1799,15)(92.1625,16.25)(92.1261,17.5)(92.0765,18.75)(92.026,20)(91.9911,21.25)(91.9896,22.5)(92.0356,23.75)(92.1352,25)(92.2837,26.25)(92.466,27.5)(92.6598,28.75)(92.8405,30)(92.9856,31.25)(93.0777,32.5)(93.1053,33.75)(93.0621,35)(92.9459,36.25)(92.7586,37.5)(92.5066,38.75)(92.205,40)(91.8805,41.25)(91.5744,42.5)(91.3425,43.75)(91.25,45)
		\thinlines\dottedline{3}(98.0599,5)(98.0694,6.25)(98.1085,7.5)(98.1965,8.75)(98.3704,10)(98.6808,11.25)(99.1824,12.5)(99.9207,13.75)(100.905,15)(102.076,16.25)(103.287,17.5)(104.296,18.75)(104.796,20)(104.492,21.25)(103.186,22.5)(100.866,23.75)(97.7517,25)(94.2564,26.25)(90.8889,27.5)(88.1128,28.75)(86.221,30)(85.2785,31.25)(85.1426,32.5)(85.5445,33.75)(86.1973,35)(86.88,36.25)(87.4797,37.5)(87.9965,38.75)(88.5047,40)(89.0933,41.25)(89.8012,42.5)(90.5647,43.75)(91.25,45)
		\thinlines\drawline(91.25,5)(90.7382,6.25)(90.1835,7.5)(89.5361,8.75)(88.7248,10)(87.6688,11.25)(86.2836,12.5)(84.5103,13.75)(82.357,15)(79.9341,16.25)(77.4811,17.5)(75.3597,18.75)(73.9904,20)(73.7546,21.25)(74.8707,22.5)(77.2929,23.75)(80.6866,25)(84.4892,26.25)(88.0554,27.5)(90.8344,28.75)(92.5042,30)(93.028,31.25)(92.6162,32.5)(91.6206,33.75)(90.4242,35)(89.3538,36.25)(88.6325,37.5)(88.3765,38.75)(88.5926,40)(89.1806,41.25)(89.9598,42.5)(90.7073,43.75)(91.25,45)
		\thicklines\drawline(123.243,5)(123.252,6.25)(123.272,7.5)(123.304,8.75)(123.344,10)(123.389,11.25)(123.435,12.5)(123.48,13.75)(123.522,15)(123.567,16.25)(123.623,17.5)(123.703,18.75)(123.824,20)(124,21.25)(124.242,22.5)(124.549,23.75)(124.911,25)(125.311,26.25)(125.724,27.5)(126.128,28.75)(126.503,30)(126.837,31.25)(127.126,32.5)(127.369,33.75)(127.566,35)(127.715,36.25)(127.812,37.5)(127.853,38.75)(127.834,40)(127.761,41.25)(127.654,42.5)(127.549,43.75)(127.5,45)
		\thinlines\dottedline{3}(132.874,5)(132.897,6.25)(132.968,7.5)(133.108,8.75)(133.34,10)(133.685,11.25)(134.152,12.5)(134.71,13.75)(135.273,15)(135.691,16.25)(135.755,17.5)(135.232,18.75)(133.934,20)(131.79,21.25)(128.91,22.5)(125.604,23.75)(122.329,25)(119.582,26.25)(117.757,27.5)(117.04,28.75)(117.367,30)(118.465,31.25)(119.959,32.5)(121.485,33.75)(122.776,35)(123.7,36.25)(124.254,37.5)(124.529,38.75)(124.683,40)(124.914,41.25)(125.397,42.5)(126.248,43.75)(127.5,45)
		\thinlines\drawline(127.5,5)(127.043,6.25)(126.534,7.5)(125.914,8.75)(125.138,10)(124.165,11.25)(122.997,12.5)(121.696,13.75)(120.404,15)(119.358,16.25)(118.862,17.5)(119.225,18.75)(120.674,20)(123.248,21.25)(126.736,22.5)(130.68,23.75)(134.454,25)(137.425,26.25)(139.122,27.5)(139.344,28.75)(138.204,30)(136.045,31.25)(133.319,32.5)(130.459,33.75)(127.798,35)(125.545,36.25)(123.83,37.5)(122.728,38.75)(122.298,40)(122.588,41.25)(123.584,42.5)(125.234,43.75)(127.5,45)
		\thicklines\drawline(156.94,5)(156.946,6.25)(156.961,7.5)(156.985,8.75)(157.015,10)(157.05,11.25)(157.09,12.5)(157.136,13.75)(157.195,15)(157.276,16.25)(157.391,17.5)(157.554,18.75)(157.776,20)(158.06,21.25)(158.403,22.5)(158.79,23.75)(159.204,25)(159.621,26.25)(160.023,27.5)(160.4,28.75)(160.749,30)(161.077,31.25)(161.394,32.5)(161.71,33.75)(162.031,35)(162.358,36.25)(162.683,37.5)(162.992,38.75)(163.267,40)(163.489,41.25)(163.643,42.5)(163.726,43.75)(163.75,45)
		\thinlines\dottedline{3}(164.54,5)(164.564,6.25)(164.625,7.5)(164.734,8.75)(164.889,10)(165.066,11.25)(165.225,12.5)(165.276,13.75)(165.088,15)(164.508,16.25)(163.387,17.5)(161.639,18.75)(159.303,20)(156.574,21.25)(153.808,22.5)(151.453,23.75)(149.936,25)(149.545,26.25)(150.333,27.5)(152.094,28.75)(154.448,30)(156.945,31.25)(159.193,32.5)(160.949,33.75)(162.122,35)(162.746,36.25)(162.93,37.5)(162.801,38.75)(162.512,40)(162.25,41.25)(162.225,42.5)(162.664,43.75)(163.75,45)
		\thinlines\drawline(163.75,5)(163.616,6.25)(163.45,7.5)(163.221,8.75)(162.934,10)(162.616,11.25)(162.348,12.5)(162.282,13.75)(162.607,15)(163.551,16.25)(165.304,17.5)(167.938,18.75)(171.356,20)(175.233,21.25)(179.049,22.5)(182.204,23.75)(184.148,25)(184.547,26.25)(183.38,27.5)(180.916,28.75)(177.633,30)(174.057,31.25)(170.613,32.5)(167.564,33.75)(164.997,35)(162.882,36.25)(161.177,37.5)(159.874,38.75)(159.051,40)(158.872,41.25)(159.502,42.5)(161.089,43.75)(163.75,45)
		\thicklines\drawline(194.626,5)(194.626,6.25)(194.627,7.5)(194.629,8.75)(194.632,10)(194.636,11.25)(194.646,12.5)(194.667,13.75)(194.707,15)(194.777,16.25)(194.884,17.5)(195.034,18.75)(195.227,20)(195.453,21.25)(195.696,22.5)(195.938,23.75)(196.159,25)(196.349,26.25)(196.505,27.5)(196.634,28.75)(196.754,30)(196.883,31.25)(197.042,32.5)(197.246,33.75)(197.503,35)(197.816,36.25)(198.179,37.5)(198.576,38.75)(198.983,40)(199.37,41.25)(199.695,42.5)(199.918,43.75)(200,45)
		\thinlines\dottedline{3}(195.743,5)(195.753,6.25)(195.769,7.5)(195.784,8.75)(195.77,10)(195.677,11.25)(195.434,12.5)(194.948,13.75)(194.119,15)(192.881,16.25)(191.231,17.5)(189.282,18.75)(187.277,20)(185.562,21.25)(184.53,22.5)(184.505,23.75)(185.635,25)(187.83,26.25)(190.768,27.5)(193.976,28.75)(196.979,30)(199.41,31.25)(201.096,32.5)(202.053,33.75)(202.421,35)(202.379,36.25)(202.087,37.5)(201.63,38.75)(201.066,40)(200.464,41.25)(199.946,42.5)(199.717,43.75)(200,45)
		\thinlines\drawline(200,5)(200.267,6.25)(200.542,7.5)(200.838,8.75)(201.209,10)(201.73,11.25)(202.52,12.5)(203.728,13.75)(205.48,15)(207.86,16.25)(210.834,17.5)(214.197,18.75)(217.583,20)(220.491,21.25)(222.4,22.5)(222.919,23.75)(221.894,25)(219.487,26.25)(216.14,27.5)(212.432,28.75)(208.931,30)(206.031,31.25)(203.886,32.5)(202.435,33.75)(201.466,35)(200.726,36.25)(200.032,37.5)(199.291,38.75)(198.557,40)(198.014,41.25)(197.908,42.5)(198.502,43.75)(200,45)
	\end{picture}
	\begin{picture}(275,50)
	\drawline(30,5)(235,5)\drawline(30,45)(235,45)
	\drawline(55,5)(55,45) \drawline(91.25,5)(91.25,45) \drawline(127.5,5)(127.5,45) \drawline(163.75,5)(163.75,45) \drawline(200,5)(200,45)
	\put(230,28){$R=2500$} \put(230,15){$\alpha=10$}
	\put(10,22){\footnotesize$r$} \put(22,3){\tiny 0} \put(22,43){\tiny 1}
		%\input{/home/elmar/dat/floquet/eig/2500-50-1.path}
		\thicklines\drawline(53.7356,5)(53.7173,6.25)(53.704,7.5)(53.6769,8.75)(53.6405,10)(53.5939,11.25)(53.5349,12.5)(53.4653,13.75)(53.3843,15)(53.2907,16.25)(53.1864,17.5)(53.0706,18.75)(52.9438,20)(52.8086,21.25)(52.6653,22.5)(52.5167,23.75)(52.3669,25)(52.2182,26.25)(52.0761,27.5)(51.9425,28.75)(51.8102,30)(51.6603,31.25)(51.4536,32.5)(51.1466,33.75)(50.7292,35)(50.2591,36.25)(49.8831,37.5)(49.8058,38.75)(50.2163,40)(51.2158,41.25)(52.7073,42.5)(54.2476,43.75)(55,45)
		\thinlines\dottedline{3}(59.2568,5)(59.2457,6.25)(59.2652,7.5)(59.2837,8.75)(59.3048,10)(59.3434,11.25)(59.3807,12.5)(59.4256,13.75)(59.4846,15)(59.5448,16.25)(59.6198,17.5)(59.7121,18.75)(59.8139,20)(59.9401,21.25)(60.0871,22.5)(60.2565,23.75)(60.4954,25)(60.8908,26.25)(61.6383,27.5)(63.0152,28.75)(65.1461,30)(67.6645,31.25)(69.4142,32.5)(68.7743,33.75)(64.903,35)(58.8213,36.25)(53.0079,37.5)(49.5647,38.75)(48.756,40)(49.5717,41.25)(51.2509,42.5)(53.4311,43.75)(55,45)
		\thinlines\drawline(55,5)(55.4913,6.25)(55.971,7.5)(56.5004,8.75)(57.01,10)(57.5588,11.25)(58.1435,12.5)(58.729,13.75)(59.3609,15)(60.0049,16.25)(60.638,17.5)(61.2775,18.75)(61.8554,20)(62.3453,21.25)(62.7155,22.5)(62.8549,23.75)(62.6919,25)(62.0688,26.25)(60.73,27.5)(58.4626,28.75)(55.1035,30)(50.8764,31.25)(46.6288,32.5)(43.4793,33.75)(42.2529,35)(42.7412,36.25)(43.7458,37.5)(44.368,38.75)(44.8264,40)(46.2459,41.25)(49.6344,42.5)(53.841,43.75)(55,45)
		\thicklines\drawline(87.3459,5)(87.3273,6.25)(87.2793,7.5)(87.1978,8.75)(87.0826,10)(86.9323,11.25)(86.745,12.5)(86.5195,13.75)(86.2543,15)(85.9476,16.25)(85.5996,17.5)(85.2106,18.75)(84.7829,20)(84.3218,21.25)(83.8348,22.5)(83.3347,23.75)(82.8393,25)(82.3721,26.25)(81.9649,27.5)(81.6564,28.75)(81.4873,30)(81.4916,31.25)(81.6799,32.5)(82.0305,33.75)(82.5046,35)(83.0817,36.25)(83.7983,37.5)(84.7476,38.75)(86.019,40)(87.6043,41.25)(89.3022,42.5)(90.688,43.75)(91.25,45)
		\thinlines\dottedline{3}(93.366,5)(93.3428,6.25)(93.3615,7.5)(93.3693,8.75)(93.3729,10)(93.3967,11.25)(93.4096,12.5)(93.426,13.75)(93.456,15)(93.4766,16.25)(93.5081,17.5)(93.5523,18.75)(93.5938,20)(93.6537,21.25)(93.7188,22.5)(93.7687,23.75)(93.82,25)(93.8949,26.25)(94.1448,27.5)(94.9321,28.75)(96.7214,30)(99.8194,31.25)(103.774,32.5)(107.001,33.75)(107.53,35)(104.509,36.25)(99.1815,37.5)(94.1045,38.75)(91.1069,40)(90.4104,41.25)(91.311,42.5)(92.3723,43.75)(91.25,45)
		\thinlines\drawline(91.25,5)(91.5017,6.25)(91.7268,7.5)(92.0227,8.75)(92.2651,10)(92.5495,11.25)(92.8672,12.5)(93.1578,13.75)(93.5027,15)(93.8512,16.25)(94.179,17.5)(94.5385,18.75)(94.8463,20)(95.1016,21.25)(95.312,22.5)(95.3742,23.75)(95.2901,25)(94.9674,26.25)(94.1744,27.5)(92.6858,28.75)(90.0964,30)(86.1838,31.25)(81.3817,32.5)(76.7535,33.75)(73.8748,35)(73.8597,36.25)(76.4773,37.5)(80.8774,38.75)(86.2932,40)(92.1402,41.25)(97.4382,42.5)(98.8919,43.75)(91.25,45)
		\thicklines\drawline(123.243,5)(123.235,6.25)(123.181,7.5)(123.092,8.75)(122.966,10)(122.8,11.25)(122.594,12.5)(122.345,13.75)(122.051,15)(121.711,16.25)(121.323,17.5)(120.888,18.75)(120.41,20)(119.893,21.25)(119.348,22.5)(118.789,23.75)(118.239,25)(117.726,26.25)(117.293,27.5)(116.99,28.75)(116.883,30)(117.039,31.25)(117.512,32.5)(118.315,33.75)(119.403,35)(120.689,36.25)(122.079,37.5)(123.498,38.75)(124.886,40)(126.128,41.25)(127.038,42.5)(127.458,43.75)(127.5,45)
		\thinlines\dottedline{3}(126.236,5)(126.214,6.25)(126.221,7.5)(126.214,8.75)(126.197,10)(126.193,11.25)(126.174,12.5)(126.152,13.75)(126.135,15)(126.104,16.25)(126.074,17.5)(126.044,18.75)(126.001,20)(125.959,21.25)(125.904,22.5)(125.805,23.75)(125.639,25)(125.35,26.25)(124.956,27.5)(124.692,28.75)(125.092,30)(126.954,31.25)(130.798,32.5)(136.001,33.75)(140.62,35)(142.429,36.25)(140.709,37.5)(136.972,38.75)(133.542,40)(131.741,41.25)(131.335,42.5)(130.656,43.75)(127.5,45)
		\thinlines\drawline(127.5,5)(127.365,6.25)(127.203,7.5)(127.093,8.75)(126.926,10)(126.779,11.25)(126.644,12.5)(126.469,13.75)(126.325,15)(126.174,16.25)(126.004,17.5)(125.873,18.75)(125.731,20)(125.602,21.25)(125.529,22.5)(125.478,23.75)(125.522,25)(125.688,26.25)(125.906,27.5)(126.068,28.75)(125.765,30)(124.459,31.25)(121.915,32.5)(118.519,33.75)(115.675,35)(115.165,36.25)(117.862,37.5)(123.463,38.75)(130.664,40)(137.513,41.25)(141.617,42.5)(139.466,43.75)(127.5,45)
		\thicklines\drawline(161.634,5)(161.641,6.25)(161.612,7.5)(161.569,8.75)(161.505,10)(161.421,11.25)(161.317,12.5)(161.19,13.75)(161.039,15)(160.865,16.25)(160.665,17.5)(160.439,18.75)(160.191,20)(159.921,21.25)(159.637,22.5)(159.347,23.75)(159.063,25)(158.806,26.25)(158.6,27.5)(158.48,28.75)(158.498,30)(158.715,31.25)(159.195,32.5)(159.98,33.75)(161.045,35)(162.286,36.25)(163.535,37.5)(164.593,38.75)(165.284,40)(165.456,41.25)(165.045,42.5)(164.252,43.75)(163.75,45)
		\thinlines\dottedline{3}(159.846,5)(159.838,6.25)(159.829,7.5)(159.811,8.75)(159.785,10)(159.754,11.25)(159.714,12.5)(159.667,13.75)(159.614,15)(159.549,16.25)(159.475,17.5)(159.388,18.75)(159.286,20)(159.167,21.25)(159.025,22.5)(158.835,23.75)(158.548,25)(158.064,26.25)(157.257,27.5)(156.097,28.75)(154.873,30)(154.409,31.25)(155.889,32.5)(160.021,33.75)(166.025,35)(171.604,36.25)(174.499,37.5)(174.291,38.75)(172.437,40)(170.587,41.25)(169.113,42.5)(167.091,43.75)(163.75,45)
		\thinlines\drawline(163.75,5)(163.307,6.25)(162.853,7.5)(162.401,8.75)(161.923,10)(161.431,11.25)(160.922,12.5)(160.384,13.75)(159.835,15)(159.273,16.25)(158.706,17.5)(158.161,18.75)(157.651,20)(157.214,21.25)(156.901,22.5)(156.766,23.75)(156.912,25)(157.471,26.25)(158.571,27.5)(160.289,28.75)(162.45,30)(164.515,31.25)(165.72,32.5)(165.546,33.75)(164.402,35)(163.696,36.25)(164.893,37.5)(168.413,38.75)(173.181,40)(177.02,41.25)(177.526,42.5)(173.031,43.75)(163.75,45)
		\thicklines\drawline(201.264,5)(201.283,6.25)(201.296,7.5)(201.322,8.75)(201.36,10)(201.406,11.25)(201.465,12.5)(201.534,13.75)(201.615,15)(201.71,16.25)(201.814,17.5)(201.93,18.75)(202.056,20)(202.191,21.25)(202.334,22.5)(202.484,23.75)(202.633,25)(202.781,26.25)(202.925,27.5)(203.058,28.75)(203.19,30)(203.339,31.25)(203.547,32.5)(203.854,33.75)(204.27,35)(204.742,36.25)(205.117,37.5)(205.194,38.75)(204.784,40)(203.784,41.25)(202.292,42.5)(200.753,43.75)(200,45)
		\thinlines\dottedline{3}(195.743,5)(195.754,6.25)(195.735,7.5)(195.716,8.75)(195.695,10)(195.657,11.25)(195.619,12.5)(195.574,13.75)(195.515,15)(195.455,16.25)(195.38,17.5)(195.288,18.75)(195.186,20)(195.06,21.25)(194.913,22.5)(194.743,23.75)(194.505,25)(194.109,26.25)(193.362,27.5)(191.985,28.75)(189.854,30)(187.335,31.25)(185.586,32.5)(186.226,33.75)(190.097,35)(196.179,36.25)(201.992,37.5)(205.435,38.75)(206.244,40)(205.429,41.25)(203.75,42.5)(201.569,43.75)(200,45)
		\thinlines\drawline(200,5)(199.509,6.25)(199.029,7.5)(198.5,8.75)(197.99,10)(197.441,11.25)(196.857,12.5)(196.271,13.75)(195.639,15)(194.995,16.25)(194.362,17.5)(193.723,18.75)(193.145,20)(192.655,21.25)(192.285,22.5)(192.145,23.75)(192.308,25)(192.931,26.25)(194.27,27.5)(196.537,28.75)(199.897,30)(204.124,31.25)(208.371,32.5)(211.52,33.75)(212.747,35)(212.258,36.25)(211.255,37.5)(210.631,38.75)(210.173,40)(208.754,41.25)(205.365,42.5)(201.159,43.75)(200,45)
	\end{picture}
	\begin{picture}(275,50)
	\drawline(30,5)(235,5)\drawline(30,45)(235,45)
	\drawline(55,5)(55,45) \drawline(91.25,5)(91.25,45) \drawline(127.5,5)(127.5,45) \drawline(163.75,5)(163.75,45) \drawline(200,5)(200,45)
	\put(230,28){$R=2000$} \put(230,15){$\alpha=15$}
	\put(10,22){\footnotesize$r$} \put(22,3){\tiny 0} \put(22,43){\tiny 1}
		%\input{/home/elmar/dat/floquet/eig/2000-75-1.path}
		\thicklines\drawline(55.2502,5)(55.2714,6.25)(55.2751,7.5)(55.2898,8.75)(55.3089,10)(55.3317,11.25)(55.3642,12.5)(55.4024,13.75)(55.448,15)(55.5056,16.25)(55.5722,17.5)(55.6522,18.75)(55.7491,20)(55.8614,21.25)(55.9952,22.5)(56.151,23.75)(56.3258,25)(56.5198,26.25)(56.7205,27.5)(56.9091,28.75)(57.0632,30)(57.1507,31.25)(57.1663,32.5)(57.1672,33.75)(57.3039,35)(57.8149,36.25)(58.8542,37.5)(60.2358,38.75)(61.3256,40)(61.2439,41.25)(59.4687,42.5)(56.6432,43.75)(55,45)
		\thinlines\dottedline{3}(56.4492,5)(56.4393,6.25)(56.4608,7.5)(56.484,8.75)(56.513,10)(56.562,11.25)(56.6145,12.5)(56.6792,13.75)(56.7622,15)(56.8511,16.25)(56.9572,17.5)(57.0808,18.75)(57.213,20)(57.3658,21.25)(57.5339,22.5)(57.7101,23.75)(57.9026,25)(58.0979,26.25)(58.2972,27.5)(58.5447,28.75)(58.9412,30)(59.7351,31.25)(61.2605,32.5)(63.5705,33.75)(65.8318,35)(66.0269,36.25)(62.3275,37.5)(55.7055,38.75)(49.9448,40)(47.8273,41.25)(48.9554,42.5)(51.8766,43.75)(55,45)
		\thinlines\drawline(55,5)(55.2341,6.25)(55.4489,7.5)(55.7208,8.75)(55.9563,10)(56.2314,11.25)(56.5432,12.5)(56.8513,13.75)(57.2267,15)(57.643,16.25)(58.1003,17.5)(58.6635,18.75)(59.3021,20)(60.057,21.25)(60.9809,22.5)(62.045,23.75)(63.3009,25)(64.7282,26.25)(66.2035,27.5)(67.5867,28.75)(68.4707,30)(68.2326,31.25)(66.1686,32.5)(61.6421,33.75)(54.9174,35)(47.5477,36.25)(41.5646,37.5)(38.0975,38.75)(36.6632,40)(37.1607,41.25)(41.5475,42.5)(49.6848,43.75)(55,45)
		\thicklines\drawline(90.4022,5)(90.4145,6.25)(90.3862,7.5)(90.3477,8.75)(90.2904,10)(90.2121,11.25)(90.1171,12.5)(89.9984,13.75)(89.8539,15)(89.6843,16.25)(89.4807,17.5)(89.2406,18.75)(88.9587,20)(88.6223,21.25)(88.2244,22.5)(87.75,23.75)(87.1796,25)(86.4995,26.25)(85.6894,27.5)(84.7404,28.75)(83.6695,30)(82.5313,31.25)(81.4666,32.5)(80.7285,33.75)(80.6628,35)(81.6198,36.25)(83.7241,37.5)(86.6761,38.75)(89.7748,40)(92.1125,41.25)(92.9136,42.5)(92.1288,43.75)(91.25,45)
		\thinlines\dottedline{3}(92.4516,5)(92.4641,6.25)(92.4715,7.5)(92.4913,8.75)(92.5219,10)(92.5557,11.25)(92.6027,12.5)(92.6594,13.75)(92.7228,15)(92.8012,16.25)(92.8895,17.5)(92.988,18.75)(93.1028,20)(93.2272,21.25)(93.3634,22.5)(93.5132,23.75)(93.6683,25)(93.8368,26.25)(94.0388,27.5)(94.3259,28.75)(94.8229,30)(95.6784,31.25)(96.8938,32.5)(97.9648,33.75)(97.5742,35)(94.2421,36.25)(88.2212,37.5)(82.5462,38.75)(80.5328,40)(82.0535,41.25)(84.4724,42.5)(87.0518,43.75)(91.25,45)
		\thinlines\drawline(91.25,5)(91.4398,6.25)(91.6479,7.5)(91.8328,8.75)(92.0595,10)(92.2883,11.25)(92.5288,12.5)(92.816,13.75)(93.1175,15)(93.4639,16.25)(93.8739,17.5)(94.3298,18.75)(94.8788,20)(95.5276,21.25)(96.2757,22.5)(97.1666,23.75)(98.17,25)(99.2476,26.25)(100.328,27.5)(101.185,28.75)(101.527,30)(100.974,31.25)(99.1873,32.5)(96.3314,33.75)(93.2403,35)(91.0616,36.25)(90.0572,37.5)(88.4022,38.75)(83.693,40)(76.4854,41.25)(71.6976,42.5)(75.9094,43.75)(91.25,45)
		\thicklines\drawline(126.051,5)(126.047,6.25)(126.003,7.5)(125.934,8.75)(125.834,10)(125.701,11.25)(125.534,12.5)(125.327,13.75)(125.078,15)(124.78,16.25)(124.426,17.5)(124.006,18.75)(123.511,20)(122.922,21.25)(122.226,22.5)(121.399,23.75)(120.418,25)(119.262,26.25)(117.916,27.5)(116.385,28.75)(114.716,30)(113.019,31.25)(111.498,32.5)(110.453,33.75)(110.224,35)(111.066,36.25)(113.002,37.5)(115.796,38.75)(119.088,40)(122.476,41.25)(125.384,42.5)(127.1,43.75)(127.5,45)
		\thinlines\dottedline{3}(127.75,5)(127.778,6.25)(127.767,7.5)(127.772,8.75)(127.786,10)(127.784,11.25)(127.799,12.5)(127.814,13.75)(127.821,15)(127.843,16.25)(127.861,17.5)(127.877,18.75)(127.907,20)(127.931,21.25)(127.955,22.5)(127.991,23.75)(128.018,25)(128.06,26.25)(128.147,27.5)(128.305,28.75)(128.612,30)(129.028,31.25)(129.221,32.5)(128.426,33.75)(125.612,35)(120.705,36.25)(115.889,37.5)(114.486,38.75)(117.399,40)(121.667,41.25)(123.96,42.5)(124.686,43.75)(127.5,45)
		\thinlines\drawline(127.5,5)(127.534,6.25)(127.614,7.5)(127.603,8.75)(127.689,10)(127.737,11.25)(127.765,12.5)(127.864,13.75)(127.914,15)(127.988,16.25)(128.11,17.5)(128.192,18.75)(128.33,20)(128.492,21.25)(128.627,22.5)(128.822,23.75)(128.985,25)(129.082,26.25)(129.135,27.5)(128.963,28.75)(128.563,30)(128.019,31.25)(127.556,32.5)(128.044,33.75)(130.397,35)(134.686,36.25)(139.249,37.5)(140.375,38.75)(135.15,40)(124.459,41.25)(113.301,42.5)(111.121,43.75)(127.5,45)
		\thicklines\drawline(162.548,5)(162.531,6.25)(162.497,7.5)(162.438,8.75)(162.354,10)(162.243,11.25)(162.102,12.5)(161.929,13.75)(161.72,15)(161.469,16.25)(161.171,17.5)(160.818,18.75)(160.399,20)(159.904,21.25)(159.317,22.5)(158.622,23.75)(157.805,25)(156.85,26.25)(155.756,27.5)(154.54,28.75)(153.252,30)(151.99,31.25)(150.903,32.5)(150.163,33.75)(149.905,35)(150.139,36.25)(150.773,37.5)(151.772,38.75)(153.329,40)(155.782,41.25)(159.094,42.5)(162.305,43.75)(163.75,45)
		\thinlines\dottedline{3}(162.902,5)(162.929,6.25)(162.906,7.5)(162.893,8.75)(162.882,10)(162.847,11.25)(162.82,12.5)(162.785,13.75)(162.731,15)(162.683,16.25)(162.622,17.5)(162.545,18.75)(162.473,20)(162.381,21.25)(162.28,22.5)(162.181,23.75)(162.064,25)(161.956,26.25)(161.876,27.5)(161.813,28.75)(161.749,30)(161.482,31.25)(160.54,32.5)(158.344,33.75)(154.756,35)(151.148,36.25)(150.359,37.5)(154.048,38.75)(160.182,40)(164.697,41.25)(165.521,42.5)(163.969,43.75)(163.75,45)
		\thinlines\drawline(163.75,5)(163.609,6.25)(163.513,7.5)(163.313,8.75)(163.207,10)(163.047,11.25)(162.846,12.5)(162.698,13.75)(162.468,15)(162.226,16.25)(161.989,17.5)(161.649,18.75)(161.295,20)(160.876,21.25)(160.317,22.5)(159.703,23.75)(158.931,25)(157.99,26.25)(156.984,27.5)(155.885,28.75)(154.977,30)(154.76,31.25)(155.893,32.5)(159.438,33.75)(165.857,35)(174.101,36.25)(181.558,37.5)(184.806,38.75)(182.125,40)(174.214,41.25)(163.222,42.5)(155.926,43.75)(163.75,45)
		\thicklines\drawline(199.75,5)(199.729,6.25)(199.725,7.5)(199.71,8.75)(199.691,10)(199.668,11.25)(199.636,12.5)(199.598,13.75)(199.552,15)(199.494,16.25)(199.428,17.5)(199.348,18.75)(199.251,20)(199.139,21.25)(199.005,22.5)(198.849,23.75)(198.674,25)(198.48,26.25)(198.28,27.5)(198.091,28.75)(197.937,30)(197.849,31.25)(197.834,32.5)(197.833,33.75)(197.696,35)(197.185,36.25)(196.146,37.5)(194.764,38.75)(193.675,40)(193.756,41.25)(195.531,42.5)(198.357,43.75)(200,45)
		\thinlines\dottedline{3}(198.551,5)(198.561,6.25)(198.539,7.5)(198.516,8.75)(198.487,10)(198.438,11.25)(198.385,12.5)(198.321,13.75)(198.238,15)(198.149,16.25)(198.043,17.5)(197.919,18.75)(197.787,20)(197.634,21.25)(197.466,22.5)(197.29,23.75)(197.097,25)(196.902,26.25)(196.703,27.5)(196.455,28.75)(196.059,30)(195.265,31.25)(193.739,32.5)(191.429,33.75)(189.168,35)(188.973,36.25)(192.673,37.5)(199.294,38.75)(205.055,40)(207.173,41.25)(206.045,42.5)(203.123,43.75)(200,45)
		\thinlines\drawline(200,5)(199.766,6.25)(199.551,7.5)(199.279,8.75)(199.044,10)(198.769,11.25)(198.457,12.5)(198.149,13.75)(197.773,15)(197.357,16.25)(196.9,17.5)(196.336,18.75)(195.698,20)(194.943,21.25)(194.019,22.5)(192.955,23.75)(191.699,25)(190.272,26.25)(188.797,27.5)(187.413,28.75)(186.529,30)(186.767,31.25)(188.831,32.5)(193.358,33.75)(200.083,35)(207.452,36.25)(213.436,37.5)(216.903,38.75)(218.337,40)(217.839,41.25)(213.453,42.5)(205.316,43.75)(200,45)
	\end{picture}
	\begin{picture}(275,50)
	\drawline(30,5)(235,5)\drawline(30,45)(235,45)
	\drawline(55,5)(55,45) \drawline(91.25,5)(91.25,45) \drawline(127.5,5)(127.5,45) \drawline(163.75,5)(163.75,45) \drawline(200,5)(200,45)
	\put(230,28){$R=2500$} \put(230,15){$\alpha=20$}
	\put(10,22){\footnotesize$r$} \put(22,3){\tiny 0} \put(22,43){\tiny 1}
		%\input{/home/elmar/dat/floquet/eig/2500-99-1.path}
		\thicklines\drawline(55.0193,5)(55.0587,6.25)(55.0516,7.5)(55.0596,8.75)(55.0687,10)(55.0727,11.25)(55.088,12.5)(55.1027,13.75)(55.118,15)(55.1447,16.25)(55.1708,17.5)(55.2039,18.75)(55.2494,20)(55.2978,21.25)(55.362,22.5)(55.4429,23.75)(55.5375,25)(55.6634,26.25)(55.8185,27.5)(56.0056,28.75)(56.2337,30)(56.4719,31.25)(56.6787,32.5)(56.7814,33.75)(56.6921,35)(56.4981,36.25)(56.627,37.5)(57.7786,38.75)(60.1759,40)(62.4348,41.25)(62.0304,42.5)(58.1453,43.75)(55,45)
		\thinlines\dottedline{3}(55.4066,5)(55.3809,6.25)(55.4013,7.5)(55.4117,8.75)(55.4202,10)(55.4509,11.25)(55.4741,12.5)(55.5047,13.75)(55.5529,15)(55.5963,16.25)(55.655,17.5)(55.7299,18.75)(55.8058,20)(55.9059,21.25)(56.0236,22.5)(56.1518,23.75)(56.3145,25)(56.4984,26.25)(56.7037,27.5)(56.9478,28.75)(57.2016,30)(57.4662,31.25)(57.7934,32.5)(58.3585,33.75)(59.6747,35)(62.1767,36.25)(64.677,37.5)(63.4707,38.75)(56.6158,40)(49.4712,41.25)(47.8458,42.5)(50.6963,43.75)(55,45)
		\thinlines\drawline(55,5)(55.0937,6.25)(55.1335,7.5)(55.2969,8.75)(55.3492,10)(55.4665,11.25)(55.6325,12.5)(55.7379,13.75)(55.9419,15)(56.1627,16.25)(56.3781,17.5)(56.7158,18.75)(57.0644,20)(57.4804,21.25)(58.0471,22.5)(58.6652,23.75)(59.4715,25)(60.5171,26.25)(61.7745,27.5)(63.4587,28.75)(65.5529,30)(67.959,31.25)(70.4705,32.5)(72.0159,33.75)(70.9259,35)(65.3374,36.25)(54.7547,37.5)(42.767,38.75)(34.2137,40)(30.1203,41.25)(31.8609,42.5)(43.367,43.75)(55,45)
		\thicklines\drawline(90.9761,5)(91.0144,6.25)(90.991,7.5)(90.9736,8.75)(90.9455,10)(90.8982,11.25)(90.8482,12.5)(90.7796,13.75)(90.6904,15)(90.5888,16.25)(90.456,17.5)(90.2947,18.75)(90.1029,20)(89.8609,21.25)(89.571,22.5)(89.2193,23.75)(88.7822,25)(88.253,26.25)(87.594,27.5)(86.7634,28.75)(85.7196,30)(84.3828,31.25)(82.7011,32.5)(80.6883,33.75)(78.5197,35)(76.7707,36.25)(76.4406,37.5)(78.5351,38.75)(83.1332,40)(88.6351,41.25)(92.4248,42.5)(92.6384,43.75)(91.25,45)
		\thinlines\dottedline{3}(91.5512,5)(91.5694,6.25)(91.5665,7.5)(91.5761,8.75)(91.5939,10)(91.6055,11.25)(91.6298,12.5)(91.6588,13.75)(91.687,15)(91.7298,16.25)(91.7769,17.5)(91.8293,18.75)(91.8993,20)(91.9756,21.25)(92.0652,22.5)(92.1767,23.75)(92.2994,25)(92.4444,26.25)(92.6139,27.5)(92.7944,28.75)(92.9967,30)(93.2317,31.25)(93.5696,32.5)(94.2332,33.75)(95.446,35)(96.781,36.25)(96.1514,37.5)(91.0405,38.75)(83.7217,40)(80.8692,41.25)(83.0839,42.5)(86.1383,43.75)(91.25,45)
		\thinlines\drawline(91.25,5)(91.3121,6.25)(91.4033,7.5)(91.4434,8.75)(91.5475,10)(91.6371,11.25)(91.7247,12.5)(91.8678,13.75)(91.9985,15)(92.1626,16.25)(92.381,17.5)(92.6044,18.75)(92.8997,20)(93.2593,21.25)(93.6677,22.5)(94.2068,23.75)(94.8619,25)(95.6747,26.25)(96.7435,27.5)(98.0518,28.75)(99.6375,30)(101.403,31.25)(102.906,32.5)(103.451,33.75)(102.025,35)(98.1595,36.25)(93.3864,37.5)(90.1773,38.75)(87.4981,40)(80.0508,41.25)(68.4462,42.5)(67.7025,43.75)(91.25,45)
		\thicklines\drawline(127.093,5)(127.108,6.25)(127.082,7.5)(127.049,8.75)(127.001,10)(126.93,11.25)(126.844,12.5)(126.732,13.75)(126.591,15)(126.42,16.25)(126.206,17.5)(125.945,18.75)(125.628,20)(125.238,21.25)(124.764,22.5)(124.185,23.75)(123.472,25)(122.598,26.25)(121.511,27.5)(120.149,28.75)(118.445,30)(116.316,31.25)(113.731,32.5)(110.782,33.75)(107.805,35)(105.525,36.25)(104.929,37.5)(106.74,38.75)(110.845,40)(116.367,41.25)(122.131,42.5)(126.318,43.75)(127.5,45)
		\thinlines\dottedline{3}(127.519,5)(127.571,6.25)(127.546,7.5)(127.549,8.75)(127.566,10)(127.552,11.25)(127.563,12.5)(127.573,13.75)(127.565,15)(127.582,16.25)(127.59,17.5)(127.589,18.75)(127.612,20)(127.62,21.25)(127.629,22.5)(127.659,23.75)(127.67,25)(127.691,26.25)(127.725,27.5)(127.736,28.75)(127.769,30)(127.836,31.25)(127.987,32.5)(128.36,33.75)(128.759,35)(128.145,36.25)(124.755,37.5)(118.733,38.75)(115.238,40)(118.348,41.25)(123.106,42.5)(124.575,43.75)(127.5,45)
		\thinlines\drawline(127.5,5)(127.494,6.25)(127.583,7.5)(127.477,8.75)(127.572,10)(127.581,11.25)(127.539,12.5)(127.636,13.75)(127.617,15)(127.628,16.25)(127.721,17.5)(127.7,18.75)(127.769,20)(127.861,21.25)(127.872,22.5)(128.016,23.75)(128.137,25)(128.24,26.25)(128.494,27.5)(128.661,28.75)(128.809,30)(128.9,31.25)(128.514,32.5)(127.74,33.75)(126.812,35)(126.934,36.25)(130.767,37.5)(138.216,38.75)(142.98,40)(136.542,41.25)(118.39,42.5)(105.832,43.75)(127.5,45)
		\thicklines\drawline(163.449,5)(163.431,6.25)(163.418,7.5)(163.389,8.75)(163.348,10)(163.295,11.25)(163.224,12.5)(163.134,13.75)(163.024,15)(162.884,16.25)(162.715,17.5)(162.506,18.75)(162.25,20)(161.94,21.25)(161.559,22.5)(161.093,23.75)(160.522,25)(159.815,26.25)(158.936,27.5)(157.841,28.75)(156.475,30)(154.801,31.25)(152.827,32.5)(150.669,33.75)(148.627,35)(147.152,36.25)(146.64,37.5)(147.106,38.75)(148.313,40)(150.621,41.25)(154.982,42.5)(160.69,43.75)(163.75,45)
		\thinlines\dottedline{3}(163.476,5)(163.531,6.25)(163.499,7.5)(163.494,8.75)(163.5,10)(163.468,11.25)(163.459,12.5)(163.445,13.75)(163.405,15)(163.387,16.25)(163.351,17.5)(163.297,18.75)(163.26,20)(163.195,21.25)(163.118,22.5)(163.048,23.75)(162.94,25)(162.825,26.25)(162.705,27.5)(162.54,28.75)(162.383,30)(162.244,31.25)(162.119,32.5)(161.984,33.75)(161.335,35)(159.132,36.25)(154.966,37.5)(151.561,38.75)(153.937,40)(161.188,41.25)(165.701,42.5)(164.725,43.75)(163.75,45)
		\thinlines\drawline(163.75,5)(163.68,6.25)(163.714,7.5)(163.524,8.75)(163.554,10)(163.477,11.25)(163.33,12.5)(163.324,13.75)(163.167,15)(163.018,16.25)(162.932,17.5)(162.678,18.75)(162.48,20)(162.252,21.25)(161.858,22.5)(161.523,23.75)(161.038,25)(160.372,26.25)(159.663,27.5)(158.589,28.75)(157.214,30)(155.576,31.25)(153.528,32.5)(151.887,33.75)(152.002,35)(156.04,36.25)(166.233,37.5)(179.977,38.75)(189.395,40)(187.736,41.25)(173.67,42.5)(156.654,43.75)(163.75,45)
		\thicklines\drawline(199.981,5)(199.941,6.25)(199.948,7.5)(199.94,8.75)(199.931,10)(199.927,11.25)(199.912,12.5)(199.897,13.75)(199.882,15)(199.855,16.25)(199.829,17.5)(199.796,18.75)(199.751,20)(199.702,21.25)(199.638,22.5)(199.557,23.75)(199.462,25)(199.337,26.25)(199.181,27.5)(198.994,28.75)(198.766,30)(198.528,31.25)(198.321,32.5)(198.219,33.75)(198.308,35)(198.502,36.25)(198.373,37.5)(197.221,38.75)(194.824,40)(192.565,41.25)(192.97,42.5)(196.855,43.75)(200,45)
		\thinlines\dottedline{3}(199.593,5)(199.619,6.25)(199.599,7.5)(199.588,8.75)(199.58,10)(199.549,11.25)(199.526,12.5)(199.495,13.75)(199.447,15)(199.404,16.25)(199.345,17.5)(199.27,18.75)(199.194,20)(199.094,21.25)(198.976,22.5)(198.848,23.75)(198.686,25)(198.502,26.25)(198.296,27.5)(198.052,28.75)(197.798,30)(197.534,31.25)(197.207,32.5)(196.642,33.75)(195.325,35)(192.823,36.25)(190.323,37.5)(191.529,38.75)(198.384,40)(205.529,41.25)(207.154,42.5)(204.304,43.75)(200,45)
		\thinlines\drawline(200,5)(199.906,6.25)(199.866,7.5)(199.703,8.75)(199.651,10)(199.534,11.25)(199.368,12.5)(199.262,13.75)(199.058,15)(198.837,16.25)(198.622,17.5)(198.284,18.75)(197.936,20)(197.52,21.25)(196.953,22.5)(196.335,23.75)(195.528,25)(194.483,26.25)(193.225,27.5)(191.541,28.75)(189.447,30)(187.041,31.25)(184.53,32.5)(182.984,33.75)(184.074,35)(189.663,36.25)(200.245,37.5)(212.234,38.75)(220.786,40)(224.879,41.25)(223.139,42.5)(211.633,43.75)(200,45)
	\end{picture}
	\begin{picture}(275,14)
	\put(53,10){\footnotesize$0$}
	\put(86,10){\footnotesize$1/4$}
	\put(122,10){\footnotesize$1/2$}
	\put(157,10){\footnotesize$3/4$}
	\put(199,10){\footnotesize$1$}
	\put(118,-2){\footnotesize$kz/\pi$}
	\end{picture}
	\caption{\label{fig:eigenfunktion:asy}Eigenfunktionen antimetrischer St\"orungen ($n=1$) zu Eigenwerten mit maximaler Verst\"arkung.}
\end{figure}
\newpage

\paragraph{Anfangswertproblem}
Die Eigenwertanalyse der quasistatischen Methode soll eine Aussage \"uber die zeitliche Entwicklung von St\"orungen liefern.
Hierzu haben wir das Randwertproblem (\mbox{\ref{eq:stoerevolution}}) bis (\mbox{\ref{eq:haft}}) mit Hilfe der Galerkin-Methode diskritisiert und erhalten das zeitliche Differentialgleichungssystem (\mbox{\ref{eq:galerkin:system}}).
Unter der Annahme langsam ver\"anderlicher Koeffizienten, besitzt das System kurzzeitig exponentielle L\"osungen, welche durch die Eigenwerte der Systemmatrix charakterisiert sind.\\
Im Gegensatz hierzu k\"onnen wir das System (\mbox{\ref{eq:galerkin:system}}) als Anfangswertproblem auffassen und direkt numerisch integrieren, ohne die Zeitabh\"angigkeit der Systemmatrix einzuschr\"anken.
Das hat den Vorteil, da\ss\ wir statt quasistatischer Eigenwerte die tats\"achliche zeitliche Entwicklung von St\"orungen in der linearen N\"aherung erhalten.\\
Allerdings kann mit jeder Integration nur eine einzelne Anfangsst\"orung untersucht werden.
Eine Aussage \"uber die Entwicklung der Gesamtheit der St\"orungen ist nicht m\"oglich.
Au\ss erdem ist die Entscheidung ob eine Instabilit\"at vorliegt komplizierter als das einfache Kriterium der quasistatischen Stabilit\"at (\mbox{\ref{eq:quasistatisch:kriterium}}).\\
Im Hinblick auf den gro\ss en Aufwand, welcher mit der Untersuchung des Parameterraumes (\mbox{\ref{eq:parameter}}) verbunden ist, l\"osen wir das Anfangswertproblem nur f\"ur einige Parameterkombinationen.
Die Abbildungen \ref{fig:awp:sym} und \ref{fig:awp:asy} zeigen die zeitliche Entwicklung symmetrischer und antimetrischer St\"orungen.
Hierbei betrachten wir die skalare Gr\"o\ss e
\begin{equation}
	\zeta(\tau) = \int_0^1 |\bar{\mathfrak{v}}(r,\tau)|\, r\,\textrm{d}r
\end{equation}
als Ma\ss\ f\"ur die Intensit\"at der St\"orung.
Als Wellenzahl $k$, w\"ahlen wir jene, welche nach der quasistatischen Theorie f\"ur die jeweilige Parameterkombination aus $R,\alpha$ und $n$ eine maximale Verst\"arkung zur Folge hat.\\
Das Anfangswertproblem (\mbox{\ref{eq:galerkin:system}}) l\"osen wir f\"ur neun zuf\"allig ge"-w\"ah"-lte Anfangsbedingungen.
Die Integration wird jeweils nach einem Intervall von $\tau=\pi/4$ mit neuen Anfangsbedingungen gestartet.
Das System hat eine Ordnung von $N=30$.
Die Komponenten der Anfangsbedingungen sind so gew\"ahlt, da\ss\ sie mit steigendem Index exponentiell abfallen
\begin{equation}
	q_\mathbf{i} = \mathtt{r}e^{-6\mathbf{i}/N}
\end{equation}
hierbei ist \texttt{r} eine Zufallszahl im Intervall $[-1,1]$.\\

Die Berechnungen best\"atigen qualitativ die quasistatische Theorie, insbesondere f\"ur Parameterkombinationen mit gro\ss en Wachstumsraten in der Oszillationszeitskala.
F\"ur $R=2500$ und $\alpha=5$ wird die Intensit\"at antimetrischer St\"orungen zu den Zeiten der Flu\ss umkehr bei $\tau=0$ und $\tau=\pi$ bis um den Faktor 10 verst\"arkt.
Nicht alle Anfangsbedingungen verhalten sich gleich.
Dennoch k\"onnen Bereiche mit vornehmlich anfachender Wirkung der Grundstr\"omung ausgemacht werden.
Hingegen gibt es in Bereichen mit kleinen quasistatischen Wachstumsraten, beispielsweise f\"ur symmetrische St\"orungen einer Str\"omung mit $R=2500$ und $\alpha=15$ keine erkennbare Verst\"arkung.

\begin{figure} % Stoerungsenergie n=0
	%\input{/home/elmar/src/floquet/integrator/dat/resrand/2500-50-0.epic}
	\begin{picture}(275,78)
		\thinlines
		\drawline(10,8)(10,73)
		\drawline(42.5,8)(42.5,73)
		\drawline(75,8)(75,73)
		\drawline(107.5,8)(107.5,73)
		\drawline(140,8)(140,73)
		\drawline(172.5,8)(172.5,73)
		\drawline(205,8)(205,73)
		\drawline(237.5,8)(237.5,73)
		\drawline(270,8)(270,73)
		\drawline(10,8)(270,8)
		\drawline(10,73)(270,73)
		\thicklines
		\drawline(10,29.0557)(12.0312,30.0701)(14.0625,31.6151)(16.0938,33.2952)(18.125,34.9106)(20.1562,36.3751)(22.1875,37.6625)(24.2188,38.7951)(26.25,39.5945)(28.2812,39.9629)(30.3125,39.863)(32.3438,39.2988)(34.375,38.3186)(36.4062,37.0143)(38.4375,35.5149)(40.4688,33.9701)(42.5,32.5262)
		\drawline(42.5,28.291)(44.5312,26.2412)(46.5625,24.4705)(48.5938,22.9338)(50.625,21.6598)(52.6562,20.5746)(54.6875,19.6809)(56.7188,19.0183)(58.75,18.6682)(60.7812,18.7068)(62.8125,18.8485)(64.8438,18.9363)(66.875,18.9371)(68.9062,18.877)(70.9375,18.8122)(72.9688,18.7984)(75,18.8521)
		\drawline(75,22.814)(77.0312,23.6239)(79.0625,23.7058)(81.0938,23.7576)(83.125,23.3595)(85.1562,21.7258)(87.1875,19.9206)(89.2188,19.3687)(91.25,18.0287)(93.2812,16.4578)(95.3125,15.7062)(97.3438,15.2397)(99.375,15.1092)(101.406,14.9981)(103.438,14.4431)(105.469,13.5632)(107.5,12.9684)
		\drawline(107.5,23.9666)(109.531,22.1851)(111.562,21.2413)(113.594,20.1477)(115.625,18.9878)(117.656,18.1657)(119.688,17.7255)(121.719,17.3501)(123.75,16.9657)(125.781,16.8765)(127.812,17.0932)(129.844,17.3219)(131.875,17.4162)(133.906,17.3738)(135.938,17.3621)(137.969,17.6613)(140,18.1412)
		\drawline(140,28.9249)(142.031,26.9345)(144.062,26.1634)(146.094,25.8842)(148.125,26.3171)(150.156,27.4565)(152.188,28.8375)(154.219,30.1678)(156.25,31.2815)(158.281,32.1137)(160.312,32.6395)(162.344,32.8627)(164.375,32.8209)(166.406,32.6053)(168.438,32.2086)(170.469,31.6726)(172.5,31.0876)
		\drawline(172.5,29.5363)(174.531,28.9622)(176.562,30.061)(178.594,31.2468)(180.625,31.8498)(182.656,31.8301)(184.688,31.4327)(186.719,30.9797)(188.75,30.7535)(190.781,30.96)(192.812,31.4494)(194.844,31.7763)(196.875,31.7885)(198.906,31.5397)(200.938,31.1645)(202.969,30.7966)(205,30.4985)
		\drawline(205,32.9882)(207.031,32.1642)(209.062,30.6989)(211.094,29.2717)(213.125,28.4053)(215.156,28.5961)(217.188,28.8665)(219.219,28.2687)(221.25,26.8397)(223.281,25.11)(225.312,23.6914)(227.344,23.2873)(229.375,23.3911)(231.406,23.1493)(233.438,22.4336)(235.469,21.4445)(237.5,20.4979)
		\drawline(237.5,29.0557)(239.531,26.3619)(241.562,25.7633)(243.594,27.6507)(245.625,29.0647)(247.656,29.8792)(249.688,31.5239)(251.719,32.8807)(253.75,33.5966)(255.781,34.0513)(257.812,35.0415)(259.844,36.436)(261.875,37.7013)(263.906,38.6722)(265.938,39.4312)(267.969,40.3026)(270,42.2188)
		\drawline(10,27.1268)(12.0312,28.5625)(14.0625,30.2976)(16.0938,32.128)(18.125,33.94)(20.1562,35.6497)(22.1875,37.1655)(24.2188,38.4018)(26.25,39.2919)(28.2812,39.7796)(30.3125,39.8253)(32.3438,39.4178)(34.375,38.5857)(36.4062,37.4006)(38.4375,35.9719)(40.4688,34.434)(42.5,32.9219)
		\drawline(42.5,31.6977)(44.5312,30.4018)(46.5625,30.2638)(48.5938,29.9531)(50.625,29.0662)(52.6562,27.7438)(54.6875,26.2423)(56.7188,24.7945)(58.75,23.629)(60.7812,23.0268)(62.8125,22.9226)(64.8438,22.6336)(66.875,22.0951)(68.9062,21.4454)(70.9375,20.8391)(72.9688,20.3732)(75,20.0916)
		\drawline(75,29.5594)(77.0312,28.4141)(79.0625,28.2152)(81.0938,27.6628)(83.125,26.7376)(85.1562,25.1342)(87.1875,23.6682)(89.2188,23.454)(91.25,23.5595)(93.2812,23.2543)(95.3125,22.5343)(97.3438,21.9708)(99.375,22.0853)(101.406,22.6814)(103.438,23.0231)(105.469,22.7743)(107.5,22.0534)
		\drawline(107.5,28.623)(109.531,26.623)(111.562,25.1107)(113.594,24.167)(115.625,23.4146)(117.656,23.1058)(119.688,23.2959)(121.719,23.6047)(123.75,23.9519)(125.781,24.5975)(127.812,25.664)(129.844,26.4779)(131.875,26.6576)(133.906,26.2377)(135.938,25.4594)(137.969,24.7104)(140,24.4759)
		\drawline(140,33.6185)(142.031,34.5872)(144.062,36.5151)(146.094,38.9957)(148.125,41.733)(150.156,44.4893)(152.188,47.0925)(154.219,49.4269)(156.25,51.4074)(158.281,52.9579)(160.312,53.9981)(162.344,54.4559)(164.375,54.3017)(166.406,53.5589)(168.438,52.2656)(170.469,50.5423)(172.5,48.557)
		\drawline(172.5,32.9145)(174.531,31.7506)(176.562,30.2796)(178.594,28.6467)(180.625,27.1313)(182.656,26.0818)(184.688,25.9401)(186.719,27.0147)(188.75,27.6725)(190.781,27.629)(192.812,27.0854)(194.844,26.3439)(196.875,25.6655)(198.906,25.2364)(200.938,25.0407)(202.969,24.7812)(205,24.3366)
		\drawline(205,29.7115)(207.031,28.6983)(209.062,28.7987)(211.094,29.2944)(213.125,29.6987)(215.156,29.6479)(217.188,29.0288)(219.219,27.9837)(221.25,26.7783)(223.281,25.6709)(225.312,24.9033)(227.344,24.2949)(229.375,23.5711)(231.406,22.7082)(233.438,21.7933)(235.469,20.9471)(237.5,20.3364)
		\drawline(237.5,34.0979)(239.531,29.8222)(241.562,27.1607)(243.594,25.2025)(245.625,24.5758)(247.656,24.9735)(249.688,25.7851)(251.719,26.8651)(253.75,27.7836)(255.781,28.0219)(257.812,27.4043)(259.844,26.137)(261.875,24.7028)(263.906,23.8182)(265.938,23.9546)(267.969,24.5054)(270,25.3236)
		\drawline(10,34.5391)(12.0312,35.9331)(14.0625,37.5626)(16.0938,39.0951)(18.125,40.5607)(20.1562,42.065)(22.1875,43.4354)(24.2188,44.3896)(26.25,44.7865)(28.2812,44.5714)(30.3125,43.7359)(32.3438,42.3107)(34.375,40.3726)(36.4062,38.0376)(38.4375,35.4549)(40.4688,32.815)(42.5,30.3323)
		\drawline(42.5,32.965)(44.5312,32.3358)(46.5625,32.5707)(48.5938,32.4866)(50.625,31.6095)(52.6562,30.0086)(54.6875,27.9952)(56.7188,25.9419)(58.75,24.1468)(60.7812,22.7815)(62.8125,21.9584)(64.8438,21.4816)(66.875,20.9544)(68.9062,20.3474)(70.9375,19.7465)(72.9688,19.2418)(75,18.9049)
		\drawline(75,28.1014)(77.0312,25.9508)(79.0625,25.0053)(81.0938,24.1418)(83.125,23.1733)(85.1562,22.8279)(87.1875,22.9076)(89.2188,22.554)(91.25,21.6618)(93.2812,20.5774)(95.3125,19.6609)(97.3438,19.2859)(99.375,19.1639)(101.406,18.7696)(103.438,18.0661)(105.469,17.2076)(107.5,16.3833)
		\drawline(107.5,32.0212)(109.531,28.669)(111.562,26.2561)(113.594,25.3299)(115.625,24.4404)(117.656,23.0786)(119.688,21.911)(121.719,22.0026)(123.75,22.8136)(125.781,23.7997)(127.812,24.7384)(129.844,25.4208)(131.875,25.6236)(133.906,25.2659)(135.938,24.4808)(137.969,23.6095)(140,23.2897)
		\drawline(140,34.7018)(142.031,32.7446)(144.062,30.8949)(146.094,29.3036)(148.125,28.2126)(150.156,27.8294)(152.188,27.8972)(154.219,28.1434)(156.25,28.4287)(158.281,28.6678)(160.312,28.8085)(162.344,28.8227)(164.375,28.7032)(166.406,28.4628)(168.438,28.1354)(170.469,27.7645)(172.5,27.3529)
		\drawline(172.5,27.472)(174.531,24.5561)(176.562,22.0251)(178.594,20.8014)(180.625,20.9513)(182.656,21.5848)(184.688,21.9508)(186.719,21.8311)(188.75,21.3215)(190.781,20.6306)(192.812,19.9722)(194.844,19.5276)(196.875,19.4477)(198.906,19.3636)(200.938,19.1075)(202.969,18.6939)(205,18.2011)
		\drawline(205,33.3602)(207.031,30.5312)(209.062,27.3406)(211.094,24.1622)(213.125,21.7432)(215.156,20.3488)(217.188,19.6192)(219.219,19.1336)(221.25,18.8)(223.281,18.8119)(225.312,19.2373)(227.344,19.6758)(229.375,19.7644)(231.406,19.4448)(233.438,18.9187)(235.469,18.5053)(237.5,18.6087)
		\drawline(237.5,32.4019)(239.531,30.1984)(241.562,27.8195)(243.594,25.7311)(245.625,25.6632)(247.656,26.5354)(249.688,27.6596)(251.719,28.9777)(253.75,29.835)(255.781,29.7696)(257.812,28.7987)(259.844,27.3806)(261.875,26.3822)(263.906,26.4787)(265.938,27.1896)(267.969,28.4435)(270,30.0357)
		\drawline(10,29.2145)(12.0312,28.9119)(14.0625,29.7139)(16.0938,30.943)(18.125,32.2903)(20.1562,33.5409)(22.1875,34.562)(24.2188,35.2686)(26.25,35.6036)(28.2812,35.5297)(30.3125,35.0294)(32.3438,34.1138)(34.375,32.8339)(36.4062,31.3066)(38.4375,29.7683)(40.4688,28.2658)(42.5,26.8449)
		\drawline(42.5,25.2623)(44.5312,21.9652)(46.5625,20.7001)(48.5938,20.333)(50.625,19.9564)(52.6562,19.4349)(54.6875,18.8532)(56.7188,18.3099)(58.75,17.9301)(60.7812,17.833)(62.8125,18.0459)(64.8438,18.1892)(66.875,18.1616)(68.9062,18.0154)(70.9375,17.8405)(72.9688,17.7165)(75,17.6683)
		\drawline(75,35.573)(77.0312,32.1165)(79.0625,29.616)(81.0938,27.4239)(83.125,25.7231)(85.1562,23.7502)(87.1875,21.6654)(89.2188,20.0458)(91.25,19.2145)(93.2812,19.0693)(95.3125,18.8211)(97.3438,18.1637)(99.375,17.2428)(101.406,16.3474)(103.438,15.7426)(105.469,15.555)(107.5,15.3886)
		\drawline(107.5,30.7695)(109.531,27.0194)(111.562,23.9914)(113.594,21.3912)(115.625,19.2565)(117.656,17.5517)(119.688,16.3905)(121.719,16.3469)(123.75,16.7372)(125.781,17.1218)(127.812,17.5888)(129.844,18.3126)(131.875,18.8344)(133.906,19.0362)(135.938,18.9732)(137.969,18.7846)(140,18.6975)
		\drawline(140,27.8129)(142.031,28.0543)(144.062,28.7393)(146.094,29.5961)(148.125,30.8442)(150.156,32.2779)(152.188,33.561)(154.219,34.5575)(156.25,35.186)(158.281,35.3938)(160.312,35.1542)(162.344,34.4721)(164.375,33.389)(166.406,31.9848)(168.438,30.376)(170.469,28.7046)(172.5,27.1219)
		\drawline(172.5,30.4469)(174.531,29.5039)(176.562,29.0777)(178.594,29.2873)(180.625,29.3778)(182.656,29.0234)(184.688,28.1919)(186.719,27.0349)(188.75,25.7937)(190.781,24.7079)(192.812,23.9796)(194.844,23.7291)(196.875,23.4971)(198.906,23.039)(200.938,22.3752)(202.969,21.6211)(205,20.9091)
		\drawline(205,34.8816)(207.031,36.1571)(209.062,36.5131)(211.094,35.2835)(213.125,32.8179)(215.156,29.993)(217.188,26.9148)(219.219,23.8538)(221.25,21.7669)(223.281,20.8233)(225.312,20.961)(227.344,21.505)(229.375,21.3882)(231.406,20.5373)(233.438,19.4068)(235.469,18.5178)(237.5,18.2921)
		\drawline(237.5,32.5067)(239.531,31.5311)(241.562,31.0494)(243.594,29.6517)(245.625,27.1877)(247.656,24.3131)(249.688,21.9138)(251.719,20.6007)(253.75,20.1446)(255.781,20.1355)(257.812,20.2215)(259.844,20.1809)(261.875,19.9226)(263.906,19.5354)(265.938,19.4091)(267.969,19.5963)(270,20.0109)
		\drawline(10,29.8253)(12.0312,31.5625)(14.0625,33.6659)(16.0938,35.7117)(18.125,37.5867)(20.1562,39.2303)(22.1875,40.5895)(24.2188,41.6095)(26.25,42.2353)(28.2812,42.4158)(30.3125,42.1172)(32.3438,41.3368)(34.375,40.1146)(36.4062,38.5398)(38.4375,36.7458)(40.4688,34.8931)(42.5,33.1427)
		\drawline(42.5,26.2826)(44.5312,25.7558)(46.5625,25.5922)(48.5938,25.2549)(50.625,24.7125)(52.6562,24.102)(54.6875,23.6056)(56.7188,23.3874)(58.75,23.4042)(60.7812,23.3623)(62.8125,23.1268)(64.8438,22.7242)(66.875,22.2456)(68.9062,21.7884)(70.9375,21.4157)(72.9688,21.1219)(75,20.8425)
		\drawline(75,29.8593)(77.0312,29.0459)(79.0625,27.7715)(81.0938,26.8953)(83.125,26.1592)(85.1562,24.2704)(87.1875,22.1029)(89.2188,21.2841)(91.25,20.2167)(93.2812,18.9168)(95.3125,18.0981)(97.3438,17.4505)(99.375,16.9963)(101.406,16.921)(103.438,16.5306)(105.469,15.7505)(107.5,14.9495)
		\drawline(107.5,32.4224)(109.531,32.7797)(111.562,33.4703)(113.594,34.1088)(115.625,33.4798)(117.656,31.3101)(119.688,28.1901)(121.719,25.2207)(123.75,23.5789)(125.781,23.2188)(127.812,23.5067)(129.844,24.0331)(131.875,24.5235)(133.906,24.7655)(135.938,24.6555)(137.969,24.2833)(140,23.9694)
		\drawline(140,32.3789)(142.031,28.1308)(144.062,25.4469)(146.094,24.3327)(148.125,24.3468)(150.156,24.9759)(152.188,25.8302)(154.219,26.6207)(156.25,27.2078)(158.281,27.521)(160.312,27.5289)(162.344,27.2312)(164.375,26.6607)(166.406,25.8893)(168.438,25.0542)(170.469,24.3064)(172.5,23.6451)
		\drawline(172.5,24.9218)(174.531,23.4515)(176.562,22.3049)(178.594,21.4425)(180.625,20.9381)(182.656,20.823)(184.688,20.9429)(186.719,21.0574)(188.75,21.0644)(190.781,20.9728)(192.812,20.8372)(194.844,20.7156)(196.875,20.6392)(198.906,20.5922)(200.938,20.5257)(202.969,20.4057)(205,20.2319)
		\drawline(205,28.9911)(207.031,27.4371)(209.062,26.893)(211.094,26.7704)(213.125,26.1801)(215.156,24.9756)(217.188,23.5345)(219.219,22.3684)(221.25,22.046)(223.281,22.3048)(225.312,22.3563)(227.344,21.9352)(229.375,21.1525)(231.406,20.2839)(233.438,19.7133)(235.469,19.753)(237.5,19.8466)
		\drawline(237.5,30.7855)(239.531,30.6382)(241.562,30.2042)(243.594,28.9976)(245.625,28.2749)(247.656,29.2564)(249.688,30.7704)(251.719,32.1397)(253.75,33.5327)(255.781,34.5079)(257.812,34.4812)(259.844,33.3899)(261.875,31.6936)(263.906,30.214)(265.938,30.1067)(267.969,31.0325)(270,32.619)
		\drawline(10,28.0444)(12.0312,26.0906)(14.0625,24.8873)(16.0938,24.4465)(18.125,24.6537)(20.1562,25.0956)(22.1875,25.5371)(24.2188,25.8773)(26.25,26.0676)(28.2812,26.0806)(30.3125,25.9037)(32.3438,25.539)(34.375,25.0061)(36.4062,24.342)(38.4375,23.5985)(40.4688,22.8343)(42.5,22.1056)
		\drawline(42.5,27.3471)(44.5312,26.6027)(46.5625,26.4097)(48.5938,26.7279)(50.625,27.5991)(52.6562,28.4817)(54.6875,29.0598)(56.7188,29.2662)(58.75,29.1562)(60.7812,28.8556)(62.8125,28.5024)(64.8438,28.1971)(66.875,27.96)(68.9062,27.7243)(70.9375,27.4134)(72.9688,27.0073)(75,26.5378)
		\drawline(75,29.3489)(77.0312,27.9437)(79.0625,26.2439)(81.0938,25.0277)(83.125,23.3562)(85.1562,21.6765)(87.1875,20.8908)(89.2188,20.4245)(91.25,20.7524)(93.2812,21.0858)(95.3125,20.6092)(97.3438,19.607)(99.375,18.756)(101.406,18.6074)(103.438,18.9182)(105.469,19.1075)(107.5,18.8308)
		\drawline(107.5,29.6738)(109.531,27.0576)(111.562,27.2571)(113.594,29.0775)(115.625,31.0596)(117.656,32.9639)(119.688,34.4236)(121.719,34.8251)(123.75,33.9842)(125.781,32.1915)(127.812,30.1893)(129.844,29.3044)(131.875,29.6616)(133.906,30.5788)(135.938,31.9906)(137.969,33.8371)(140,35.8001)
		\drawline(140,24.4102)(142.031,23.1543)(144.062,22.596)(146.094,22.6075)(148.125,23.0093)(150.156,23.7092)(152.188,24.4744)(154.219,25.1838)(156.25,25.7774)(158.281,26.2171)(160.312,26.4782)(162.344,26.5491)(164.375,26.4343)(166.406,26.1559)(168.438,25.7532)(170.469,25.2814)(172.5,24.7756)
		\drawline(172.5,34.4754)(174.531,32.3876)(176.562,30.5182)(178.594,28.8211)(180.625,27.348)(182.656,26.2913)(184.688,26.297)(186.719,26.6695)(188.75,27.0213)(190.781,27.2235)(192.812,27.2419)(194.844,27.1207)(196.875,26.9484)(198.906,26.8121)(200.938,26.7384)(202.969,26.669)(205,26.5308)
		\drawline(205,28.9593)(207.031,27.9844)(209.062,27.8722)(211.094,27.3123)(213.125,26.5671)(215.156,25.9656)(217.188,26.0547)(219.219,26.2164)(221.25,26.0058)(223.281,25.4351)(225.312,24.6597)(227.344,23.8852)(229.375,23.4119)(231.406,23.1987)(233.438,22.8872)(235.469,22.3938)(237.5,21.752)
		\drawline(237.5,23.3531)(239.531,23.197)(241.562,22.9191)(243.594,22.3448)(245.625,21.195)(247.656,19.4436)(249.688,17.5224)(251.719,16.0276)(253.75,15.5061)(255.781,15.8133)(257.812,16.2765)(259.844,16.6423)(261.875,16.8085)(263.906,16.7427)(265.938,16.4677)(267.969,16.0555)(270,15.6244)
		\drawline(10,25.5458)(12.0312,24.1352)(14.0625,25.4943)(16.0938,27.3082)(18.125,29.0345)(20.1562,30.4807)(22.1875,31.5755)(24.2188,32.3224)(26.25,32.8031)(28.2812,33.2193)(30.3125,33.4293)(32.3438,33.3296)(34.375,32.9149)(36.4062,32.2383)(38.4375,31.3961)(40.4688,30.5082)(42.5,29.6983)
		\drawline(42.5,28.4269)(44.5312,27.4253)(46.5625,26.4974)(48.5938,25.6065)(50.625,24.8412)(52.6562,24.3458)(54.6875,24.2857)(56.7188,24.7401)(58.75,25.271)(60.7812,25.6147)(62.8125,25.7288)(64.8438,25.6679)(66.875,25.5271)(68.9062,25.3969)(70.9375,25.3187)(72.9688,25.2544)(75,25.1374)
		\drawline(75,28.6938)(77.0312,26.6913)(79.0625,25.6656)(81.0938,25.0815)(83.125,24.3472)(85.1562,23.2026)(87.1875,21.847)(89.2188,20.5559)(91.25,19.5046)(93.2812,18.913)(95.3125,18.4865)(97.3438,17.8942)(99.375,17.1528)(101.406,16.3533)(103.438,15.5998)(105.469,15.0162)(107.5,14.6592)
		\drawline(107.5,30.4003)(109.531,27.1489)(111.562,23.8037)(113.594,21.4624)(115.625,21.3927)(117.656,22.4528)(119.688,23.5697)(121.719,24.2274)(123.75,24.1594)(125.781,23.3508)(127.812,22.0551)(129.844,20.7431)(131.875,20.142)(133.906,20.4227)(135.938,21.2603)(137.969,22.3246)(140,23.4861)
		\drawline(140,22.0178)(142.031,20.1645)(144.062,19.9875)(146.094,20.7125)(148.125,21.6471)(150.156,22.6978)(152.188,23.6193)(154.219,24.3675)(156.25,24.9327)(158.281,25.3057)(160.312,25.4769)(162.344,25.4403)(164.375,25.1999)(166.406,24.7762)(168.438,24.2079)(170.469,23.5486)(172.5,22.8569)
		\drawline(172.5,35.8538)(174.531,34.9208)(176.562,33.3409)(178.594,31.0386)(180.625,28.369)(182.656,25.7507)(184.688,23.4669)(186.719,21.6484)(188.75,20.4162)(190.781,19.7998)(192.812,19.0087)(194.844,18.026)(196.875,16.9726)(198.906,15.9776)(200.938,15.1254)(202.969,14.4837)(205,14.0227)
		\drawline(205,24.0396)(207.031,23.1247)(209.062,23.1787)(211.094,23.9491)(213.125,24.4471)(215.156,24.2531)(217.188,23.4944)(219.219,22.5719)(221.25,21.8869)(223.281,21.9279)(225.312,22.1315)(227.344,22.008)(229.375,21.5657)(231.406,20.9706)(233.438,20.4214)(235.469,20.3771)(237.5,20.411)
		\drawline(237.5,37.1189)(239.531,35.4257)(241.562,32.141)(243.594,30.2079)(245.625,31.6741)(247.656,32.8582)(249.688,33.3461)(251.719,34.1543)(253.75,34.9434)(255.781,34.7869)(257.812,33.8275)(259.844,33.1273)(261.875,34.0249)(263.906,35.9697)(265.938,38.3466)(267.969,40.8946)(270,43.5128)
		\drawline(10,32.5871)(12.0312,31.4184)(14.0625,31.5538)(16.0938,31.9193)(18.125,32.1401)(20.1562,32.1117)(22.1875,31.8245)(24.2188,31.3051)(26.25,30.6072)(28.2812,29.8053)(30.3125,28.948)(32.3438,28.0988)(34.375,27.1726)(36.4062,26.1678)(38.4375,25.1606)(40.4688,24.2456)(42.5,23.5097)
		\drawline(42.5,21.6658)(44.5312,20.8791)(46.5625,20.2246)(48.5938,19.1015)(50.625,17.6968)(52.6562,16.36)(54.6875,15.3268)(56.7188,14.6848)(58.75,14.5745)(60.7812,14.6344)(62.8125,14.4931)(64.8438,14.211)(66.875,13.8829)(68.9062,13.5935)(70.9375,13.4126)(72.9688,13.3706)(75,13.2935)
		\drawline(75,28.9608)(77.0312,28.7509)(79.0625,28.8029)(81.0938,27.8151)(83.125,26.5411)(85.1562,25.6204)(87.1875,23.8678)(89.2188,21.8149)(91.25,20.5971)(93.2812,20.459)(95.3125,20.8515)(97.3438,20.9152)(99.375,20.2866)(101.406,19.2264)(103.438,18.2811)(105.469,17.9718)(107.5,18.1431)
		\drawline(107.5,35.6872)(109.531,32.0175)(111.562,29.8307)(113.594,27.7901)(115.625,25.4165)(117.656,22.8718)(119.688,20.4365)(121.719,18.4254)(123.75,17.134)(125.781,16.6007)(127.812,16.5423)(129.844,16.5885)(131.875,16.554)(133.906,16.414)(135.938,16.3541)(137.969,16.6305)(140,17.109)
		\drawline(140,28.3048)(142.031,28.2337)(144.062,28.5154)(146.094,29.2266)(148.125,30.311)(150.156,31.322)(152.188,32.0691)(154.219,32.4706)(156.25,32.4935)(158.281,32.131)(160.312,31.3933)(162.344,30.3087)(164.375,28.928)(166.406,27.3286)(168.438,25.6162)(170.469,23.9226)(172.5,22.4065)
		\drawline(172.5,33.3896)(174.531,29.37)(176.562,26.7742)(178.594,26.03)(180.625,25.8315)(182.656,25.251)(184.688,24.3109)(186.719,23.2369)(188.75,22.25)(190.781,21.4942)(192.812,21.0161)(194.844,20.6773)(196.875,20.2896)(198.906,19.8173)(200.938,19.3138)(202.969,18.8485)(205,18.4679)
		\drawline(205,27.7496)(207.031,26.3389)(209.062,25.0074)(211.094,24.1681)(213.125,24.4718)(215.156,25.1281)(217.188,25.1381)(219.219,24.3026)(221.25,22.9481)(223.281,21.5701)(225.312,20.5694)(227.344,20.3482)(229.375,20.2755)(231.406,19.8694)(233.438,19.1711)(235.469,18.3643)(237.5,17.6649)
		\drawline(237.5,33.2341)(239.531,30.7517)(241.562,31.1227)(243.594,32.0288)(245.625,33.6739)(247.656,35.2068)(249.688,35.8575)(251.719,35.5769)(253.75,35.531)(255.781,37.1798)(257.812,38.3658)(259.844,38.6159)(261.875,38.1522)(263.906,37.6166)(265.938,38.5374)(267.969,40.4541)(270,42.5898)
		\drawline(10,31.4767)(12.0312,29.1195)(14.0625,27.6751)(16.0938,27.1682)(18.125,27.2349)(20.1562,27.6119)(22.1875,28.1569)(24.2188,28.9878)(26.25,29.8723)(28.2812,30.6079)(30.3125,31.0973)(32.3438,31.2808)(34.375,31.1294)(36.4062,30.646)(38.4375,29.8638)(40.4688,28.8441)(42.5,27.6691)
		\drawline(42.5,40.8278)(44.5312,38.6213)(46.5625,37.2197)(48.5938,35.1031)(50.625,32.268)(52.6562,29.2227)(54.6875,26.5834)(56.7188,24.704)(58.75,23.6045)(60.7812,23.328)(62.8125,23.6491)(64.8438,23.5159)(66.875,22.9588)(68.9062,22.1957)(70.9375,21.4487)(72.9688,20.8802)(75,20.5847)
		\drawline(75,26.4925)(77.0312,24.3273)(79.0625,23.6476)(81.0938,22.8645)(83.125,21.8372)(85.1562,20.9983)(87.1875,20.8016)(89.2188,20.9238)(91.25,20.7374)(93.2812,20.183)(95.3125,19.4591)(97.3438,18.8392)(99.375,18.7244)(101.406,18.5384)(103.438,18.1175)(105.469,17.4978)(107.5,16.7869)
		\drawline(107.5,25.6629)(109.531,25.481)(111.562,25.9906)(113.594,26.1782)(115.625,26.9738)(117.656,27.9867)(119.688,28.5541)(121.719,28.5039)(123.75,28.7167)(125.781,29.1141)(127.812,28.801)(129.844,27.9726)(131.875,27.2257)(133.906,27.3749)(135.938,28.4714)(137.969,30.1388)(140,31.9724)
		\drawline(140,27.1384)(142.031,25.646)(144.062,25.2865)(146.094,25.2625)(148.125,25.2351)(150.156,25.073)(152.188,24.7381)(154.219,24.2595)(156.25,23.6959)(158.281,23.0952)(160.312,22.5222)(162.344,21.9164)(164.375,21.2499)(166.406,20.5421)(168.438,19.8386)(170.469,19.1998)(172.5,18.6884)
		\drawline(172.5,33.1138)(174.531,32.3763)(176.562,31.5465)(178.594,30.0047)(180.625,27.9349)(182.656,25.724)(184.688,23.7819)(186.719,22.3374)(188.75,21.4271)(190.781,21.1208)(192.812,21.1665)(194.844,20.9779)(196.875,20.5515)(198.906,20.0268)(200.938,19.5483)(202.969,19.2235)(205,19.0517)
		\drawline(205,29.1172)(207.031,26.7863)(209.062,24.6667)(211.094,22.854)(213.125,21.3364)(215.156,20.0297)(217.188,19.0016)(219.219,18.2401)(221.25,17.8387)(223.281,17.6034)(225.312,17.4048)(227.344,17.1282)(229.375,16.7832)(231.406,16.452)(233.438,16.2883)(235.469,16.2241)(237.5,16.1252)
		\drawline(237.5,24.4175)(239.531,20.5792)(241.562,18.8942)(243.594,19.1996)(245.625,20.1334)(247.656,20.7798)(249.688,20.7687)(251.719,20.1985)(253.75,19.723)(255.781,19.9318)(257.812,20.3096)(259.844,20.6567)(261.875,21.1433)(263.906,21.8871)(265.938,22.5788)(267.969,23.0826)(270,23.4663)
		\put(5,7){\tiny 0}
		\put(0,40){$\zeta$}
		\put(5,70){\tiny 3}
		\put(175,75){\makebox(96,4)[r]{$R=2500,\,\alpha=10,\,n=0$}}
	\end{picture}
	%\input{/home/elmar/src/floquet/integrator/dat/resrand/2000-50-0.epic}
	\begin{picture}(275,78)
		\thinlines
		\drawline(10,8)(10,73)
		\drawline(42.5,8)(42.5,73)
		\drawline(75,8)(75,73)
		\drawline(107.5,8)(107.5,73)
		\drawline(140,8)(140,73)
		\drawline(172.5,8)(172.5,73)
		\drawline(205,8)(205,73)
		\drawline(237.5,8)(237.5,73)
		\drawline(270,8)(270,73)
		\drawline(10,8)(270,8)
		\drawline(10,73)(270,73)
		\thicklines
		\drawline(10,28.5387)(12.0312,24.1152)(14.0625,21.0218)(16.0938,18.8994)(18.125,17.7454)(20.1562,17.5842)(22.1875,17.6518)(24.2188,17.7355)(26.25,17.775)(28.2812,17.7394)(30.3125,17.6135)(32.3438,17.3913)(34.375,17.0736)(36.4062,16.6663)(38.4375,16.1802)(40.4688,15.631)(42.5,15.0383)
		\drawline(42.5,27.2819)(44.5312,26.6511)(46.5625,26.5517)(48.5938,26.3134)(50.625,25.7349)(52.6562,24.8236)(54.6875,23.6729)(56.7188,22.4118)(58.75,21.1683)(60.7812,20.0474)(62.8125,19.1242)(64.8438,18.4892)(66.875,18.1078)(68.9062,17.6899)(70.9375,17.1791)(72.9688,16.5927)(75,15.9729)
		\drawline(75,34.4303)(77.0312,32.5609)(79.0625,30.4724)(81.0938,28.6368)(83.125,27.0149)(85.1562,25.5575)(87.1875,24.6763)(89.2188,24.7149)(91.25,25.0147)(93.2812,25.0109)(95.3125,24.6101)(97.3438,23.9425)(99.375,23.2005)(101.406,22.5801)(103.438,22.3899)(105.469,22.4837)(107.5,22.458)
		\drawline(107.5,30.6618)(109.531,28.0241)(111.562,26.2726)(113.594,25.9054)(115.625,26.2214)(117.656,26.6135)(119.688,26.7553)(121.719,26.463)(123.75,25.69)(125.781,24.5272)(127.812,23.1544)(129.844,21.7769)(131.875,20.6329)(133.906,20.0307)(135.938,19.8732)(137.969,20.0025)(140,20.3194)
		\drawline(140,28.6175)(142.031,26.0064)(144.062,23.9253)(146.094,22.1327)(148.125,20.5455)(150.156,19.1435)(152.188,17.9572)(154.219,17.0508)(156.25,16.6375)(158.281,16.8151)(160.312,17.1016)(162.344,17.388)(164.375,17.6387)(166.406,17.826)(168.438,17.9292)(170.469,17.9361)(172.5,17.8457)
		\drawline(172.5,26.6389)(174.531,23.3803)(176.562,21.0488)(178.594,19.3544)(180.625,18.0829)(182.656,17.0264)(184.688,16.0891)(186.719,15.2618)(188.75,14.5499)(190.781,13.9509)(192.812,13.4588)(194.844,13.0705)(196.875,12.7955)(198.906,12.5886)(200.938,12.388)(202.969,12.1803)(205,11.9668)
		\drawline(205,33.5034)(207.031,32.4201)(209.062,31.9781)(211.094,31.7599)(213.125,30.7751)(215.156,29.1289)(217.188,27.2111)(219.219,25.448)(221.25,24.1785)(223.281,23.6522)(225.312,23.7857)(227.344,24.0959)(229.375,24.103)(231.406,23.717)(233.438,23.0568)(235.469,22.3215)(237.5,21.7262)
		\drawline(237.5,36.7467)(239.531,35.2034)(241.562,33.5073)(243.594,31.9508)(245.625,31.2245)(247.656,30.8436)(249.688,29.9416)(251.719,28.6164)(253.75,27.3475)(255.781,26.949)(257.812,27.1496)(259.844,27.4509)(261.875,27.8007)(263.906,28.38)(265.938,29.3355)(267.969,30.417)(270,31.3721)
		\drawline(10,37.13)(12.0312,36.7105)(14.0625,37.6244)(16.0938,38.8661)(18.125,40.1473)(20.1562,41.3873)(22.1875,42.5419)(24.2188,43.5572)(26.25,44.3623)(28.2812,44.8832)(30.3125,45.0606)(32.3438,44.8587)(34.375,44.2702)(36.4062,43.3175)(38.4375,42.0539)(40.4688,40.5587)(42.5,38.9311)
		\drawline(42.5,26.591)(44.5312,25.4613)(46.5625,24.8128)(48.5938,24.4438)(50.625,24.2358)(52.6562,24.0859)(54.6875,23.9207)(56.7188,23.701)(58.75,23.4153)(60.7812,23.07)(62.8125,22.6807)(64.8438,22.2635)(66.875,21.8302)(68.9062,21.3844)(70.9375,20.9233)(72.9688,20.4427)(75,19.9402)
		\drawline(75,33.6754)(77.0312,31.6008)(79.0625,29.482)(81.0938,27.2565)(83.125,25.1673)(85.1562,23.4729)(87.1875,22.3767)(89.2188,22.0804)(91.25,22.3201)(93.2812,22.4235)(95.3125,22.174)(97.3438,21.6205)(99.375,20.9155)(101.406,20.2268)(103.438,19.7342)(105.469,19.6687)(107.5,19.6821)
		\drawline(107.5,29.9154)(109.531,30.8018)(111.562,31.6702)(113.594,31.9867)(115.625,31.488)(117.656,30.2073)(119.688,28.7892)(121.719,27.3257)(123.75,25.8711)(125.781,25.0473)(127.812,24.5622)(129.844,24.29)(131.875,24.2459)(133.906,24.4131)(135.938,24.8067)(137.969,25.3551)(140,25.9686)
		\drawline(140,26.7771)(142.031,25.1101)(144.062,24.0686)(146.094,23.2773)(148.125,22.6206)(150.156,22.0585)(152.188,21.5671)(154.219,21.1224)(156.25,20.7027)(158.281,20.2938)(160.312,19.8874)(162.344,19.479)(164.375,19.0656)(166.406,18.6465)(168.438,18.2243)(170.469,17.8054)(172.5,17.4021)
		\drawline(172.5,32.9392)(174.531,31.1419)(176.562,29.3923)(178.594,27.5375)(180.625,25.6565)(182.656,23.8875)(184.688,22.3693)(186.719,21.2074)(188.75,20.4626)(190.781,20.2194)(192.812,20.459)(194.844,20.5585)(196.875,20.4363)(198.906,20.1278)(200.938,19.7087)(202.969,19.2617)(205,18.8565)
		\drawline(205,23.4241)(207.031,22.5757)(209.062,21.6442)(211.094,20.5716)(213.125,19.6162)(215.156,18.8774)(217.188,17.8215)(219.219,16.4816)(221.25,15.1734)(223.281,14.2836)(225.312,13.895)(227.344,13.685)(229.375,13.5149)(231.406,13.2661)(233.438,12.9036)(235.469,12.5009)(237.5,12.1856)
		\drawline(237.5,26.3623)(239.531,24.5497)(241.562,23.88)(243.594,23.7035)(245.625,23.5738)(247.656,23.3961)(249.688,22.9856)(251.719,22.3949)(253.75,21.794)(255.781,21.4219)(257.812,21.6431)(259.844,22.2183)(261.875,22.6785)(263.906,22.9352)(265.938,22.9854)(267.969,22.8811)(270,22.722)
		\drawline(10,30.7541)(12.0312,29.8595)(14.0625,29.9767)(16.0938,30.2118)(18.125,30.4402)(20.1562,30.6978)(22.1875,31.0343)(24.2188,31.3818)(26.25,31.6011)(28.2812,31.6082)(30.3125,31.3658)(32.3438,30.8622)(34.375,30.1063)(36.4062,29.1241)(38.4375,27.9589)(40.4688,26.669)(42.5,25.322)
		\drawline(42.5,28.113)(44.5312,27.5503)(46.5625,27.1611)(48.5938,26.8835)(50.625,26.5232)(52.6562,25.9572)(54.6875,25.2959)(56.7188,24.7023)(58.75,24.3358)(60.7812,24.4027)(62.8125,24.9884)(64.8438,25.455)(66.875,25.6893)(68.9062,25.6935)(70.9375,25.5241)(72.9688,25.2616)(75,24.9842)
		\drawline(75,26.7234)(77.0312,26.4758)(79.0625,26.9559)(81.0938,27.1827)(83.125,26.8004)(85.1562,25.8355)(87.1875,24.525)(89.2188,23.1384)(91.25,22.0071)(93.2812,21.4459)(95.3125,21.429)(97.3438,21.5319)(99.375,21.3614)(101.406,20.8738)(103.438,20.1936)(105.469,19.4913)(107.5,18.9543)
		\drawline(107.5,26.3147)(109.531,29.1229)(111.562,32.1854)(113.594,34.5811)(115.625,36.1808)(117.656,37.3174)(119.688,38.1968)(121.719,38.4365)(123.75,38.0478)(125.781,37.248)(127.812,36.3359)(129.844,35.6298)(131.875,35.3782)(133.906,35.5669)(135.938,36.1008)(137.969,36.9168)(140,37.9769)
		\drawline(140,27.8162)(142.031,24.461)(144.062,22.2018)(146.094,20.6687)(148.125,19.5837)(150.156,18.7911)(152.188,18.2011)(154.219,17.7551)(156.25,17.3907)(158.281,17.0816)(160.312,16.8151)(162.344,16.5842)(164.375,16.3851)(166.406,16.2138)(168.438,16.0655)(170.469,15.9342)(172.5,15.8121)
		\drawline(172.5,38.2116)(174.531,36.9575)(176.562,37.3512)(178.594,38.3004)(180.625,38.911)(182.656,38.8421)(184.688,38.1182)(186.719,36.9484)(188.75,35.6044)(190.781,34.3337)(192.812,33.3175)(194.844,32.6783)(196.875,32.5044)(198.906,32.5052)(200.938,32.3323)(202.969,31.9085)(205,31.2847)
		\drawline(205,33.4367)(207.031,33.0577)(209.062,32.926)(211.094,33.8076)(213.125,35.2322)(215.156,35.9355)(217.188,35.5093)(219.219,34.1257)(221.25,32.2535)(223.281,30.4441)(225.312,29.2041)(227.344,28.9696)(229.375,29.412)(231.406,29.7193)(233.438,29.5485)(235.469,28.8723)(237.5,27.8464)
		\drawline(237.5,29.2102)(239.531,27.8153)(241.562,28.406)(243.594,30.0006)(245.625,31.5677)(247.656,32.5956)(249.688,33.3797)(251.719,34.4145)(253.75,35.329)(255.781,35.8334)(257.812,35.8633)(259.844,35.6224)(261.875,35.4467)(263.906,35.6309)(265.938,36.1431)(267.969,36.7521)(270,37.3282)
		\drawline(10,29.8901)(12.0312,30.0307)(14.0625,30.7285)(16.0938,31.6446)(18.125,32.6504)(20.1562,33.6653)(22.1875,34.6164)(24.2188,35.4376)(26.25,36.0715)(28.2812,36.4685)(30.3125,36.5905)(32.3438,36.415)(34.375,35.9383)(36.4062,35.1778)(38.4375,34.1725)(40.4688,32.9825)(42.5,31.6836)
		\drawline(42.5,29.9223)(44.5312,28.8139)(46.5625,28.9094)(48.5938,29.4027)(50.625,29.5595)(52.6562,29.1688)(54.6875,28.2775)(56.7188,27.063)(58.75,25.7458)(60.7812,24.5253)(62.8125,23.5436)(64.8438,22.8874)(66.875,22.7021)(68.9062,22.7269)(70.9375,22.609)(72.9688,22.3008)(75,21.8497)
		\drawline(75,32.8116)(77.0312,32.1865)(79.0625,31.4802)(81.0938,30.5138)(83.125,28.678)(85.1562,26.0086)(87.1875,22.9014)(89.2188,19.9996)(91.25,17.5971)(93.2812,15.977)(95.3125,15.1372)(97.3438,14.8443)(99.375,14.8872)(101.406,14.8844)(103.438,14.6382)(105.469,14.1452)(107.5,13.5629)
		\drawline(107.5,33.7383)(109.531,32.4617)(111.562,31.4234)(113.594,30.1414)(115.625,28.8471)(117.656,28.3445)(119.688,28.412)(121.719,28.6459)(123.75,28.9249)(125.781,29.2407)(127.812,29.56)(129.844,29.6905)(131.875,29.4136)(133.906,28.7369)(135.938,27.8327)(137.969,26.978)(140,26.4858)
		\drawline(140,32.2405)(142.031,32.1638)(144.062,32.1555)(146.094,32.5871)(148.125,33.8223)(150.156,34.9026)(152.188,35.7277)(154.219,36.2559)(156.25,36.462)(158.281,36.3313)(160.312,35.8627)(162.344,35.068)(164.375,33.9766)(166.406,32.638)(168.438,31.1233)(170.469,29.5282)(172.5,27.9742)
		\drawline(172.5,31.8652)(174.531,29.141)(176.562,27.0821)(178.594,25.4832)(180.625,24.2411)(182.656,23.24)(184.688,22.2218)(186.719,21.1495)(188.75,20.112)(190.781,19.2081)(192.812,18.5235)(194.844,18.1847)(196.875,18.1601)(198.906,18.1416)(200.938,18.0455)(202.969,17.8634)(205,17.6242)
		\drawline(205,30.5665)(207.031,30.9883)(209.062,31.5627)(211.094,31.4312)(213.125,30.2781)(215.156,28.2121)(217.188,25.6676)(219.219,23.2162)(221.25,21.3923)(223.281,20.5566)(225.312,20.3451)(227.344,20.3861)(229.375,20.1978)(231.406,19.6525)(233.438,18.8701)(235.469,18.0905)(237.5,17.59)
		\drawline(237.5,29.6671)(239.531,27.9762)(241.562,27.5096)(243.594,27.5283)(245.625,27.2749)(247.656,26.623)(249.688,25.6682)(251.719,24.7309)(253.75,24.1122)(255.781,23.6731)(257.812,23.4426)(259.844,23.4498)(261.875,23.6389)(263.906,23.9691)(265.938,24.3428)(267.969,24.5844)(270,24.5639)
		\drawline(10,30.6555)(12.0312,27.5972)(14.0625,25.1686)(16.0938,23.0728)(18.125,21.266)(20.1562,19.7185)(22.1875,18.401)(24.2188,17.2854)(26.25,16.3468)(28.2812,15.5678)(30.3125,14.939)(32.3438,14.4401)(34.375,14.0421)(36.4062,13.7153)(38.4375,13.4548)(40.4688,13.2635)(42.5,13.1469)
		\drawline(42.5,27.999)(44.5312,27.0209)(46.5625,26.4819)(48.5938,26.5107)(50.625,27.0737)(52.6562,27.7311)(54.6875,28.2376)(56.7188,28.47)(58.75,28.4041)(60.7812,28.0868)(62.8125,27.6047)(64.8438,27.0556)(66.875,26.5253)(68.9062,26.0709)(70.9375,25.6991)(72.9688,25.3589)(75,24.9869)
		\drawline(75,26.4689)(77.0312,25.7242)(79.0625,24.8132)(81.0938,23.1693)(83.125,21.211)(85.1562,19.4859)(87.1875,18.3912)(89.2188,17.7891)(91.25,17.5869)(93.2812,17.3707)(95.3125,16.9314)(97.3438,16.3015)(99.375,15.5402)(101.406,14.7955)(103.438,14.2256)(105.469,13.8994)(107.5,13.8076)
		\drawline(107.5,30.3724)(109.531,29.7323)(111.562,29.8183)(113.594,30.4735)(115.625,31.2297)(117.656,31.543)(119.688,31.4923)(121.719,30.8375)(123.75,29.8463)(125.781,28.9857)(127.812,28.2806)(129.844,27.6372)(131.875,27.177)(133.906,27.2062)(135.938,27.9195)(137.969,28.9212)(140,29.8957)
		\drawline(140,32.1703)(142.031,29.4921)(144.062,27.725)(146.094,26.6566)(148.125,26.2242)(150.156,26.1822)(152.188,26.3075)(154.219,26.4826)(156.25,26.6291)(158.281,26.6902)(160.312,26.6259)(162.344,26.4124)(164.375,26.0422)(166.406,25.5249)(168.438,24.8864)(170.469,24.166)(172.5,23.4124)
		\drawline(172.5,29.7089)(174.531,29.332)(176.562,29.7536)(178.594,30.2482)(180.625,30.2213)(182.656,29.5195)(184.688,28.268)(186.719,26.7214)(188.75,25.1468)(190.781,23.7543)(192.812,22.6696)(194.844,21.9432)(196.875,21.6742)(198.906,21.8301)(200.938,21.8045)(202.969,21.5362)(205,21.0857)
		\drawline(205,29.9691)(207.031,27.386)(209.062,26.3391)(211.094,25.2758)(213.125,23.7498)(215.156,21.745)(217.188,19.5035)(219.219,17.3965)(221.25,15.7497)(223.281,14.6797)(225.312,14.1804)(227.344,13.9841)(229.375,13.7634)(231.406,13.4244)(233.438,12.9771)(235.469,12.4729)(237.5,11.9822)
		\drawline(237.5,32.146)(239.531,31.3298)(241.562,30.5897)(243.594,30.136)(245.625,29.9206)(247.656,29.3838)(249.688,28.3654)(251.719,27.5275)(253.75,27.4386)(255.781,27.6428)(257.812,28.0753)(259.844,28.7699)(261.875,29.6823)(263.906,30.6982)(265.938,31.6908)(267.969,32.5652)(270,33.277)
		\drawline(10,31.6973)(12.0312,28.5581)(14.0625,26.5012)(16.0938,25.0346)(18.125,23.9114)(20.1562,22.99)(22.1875,22.1962)(24.2188,21.4925)(26.25,20.8536)(28.2812,20.2572)(30.3125,19.6789)(32.3438,19.0955)(34.375,18.4882)(36.4062,17.8485)(38.4375,17.1751)(40.4688,16.4766)(42.5,15.7697)
		\drawline(42.5,29.1411)(44.5312,27.726)(46.5625,27.4376)(48.5938,27.3121)(50.625,26.9045)(52.6562,26.164)(54.6875,25.1727)(56.7188,24.063)(58.75,22.9794)(60.7812,22.0557)(62.8125,21.4013)(64.8438,21.1998)(66.875,21.3459)(68.9062,21.3814)(70.9375,21.248)(72.9688,20.9549)(75,20.5495)
		\drawline(75,25.9365)(77.0312,24.4339)(79.0625,23.8914)(81.0938,23.4896)(83.125,22.4605)(85.1562,20.6419)(87.1875,18.4509)(89.2188,16.4966)(91.25,15.1669)(93.2812,14.492)(95.3125,14.1454)(97.3438,13.8615)(99.375,13.3818)(101.406,12.6867)(103.438,11.9195)(105.469,11.2912)(107.5,10.944)
		\drawline(107.5,32.567)(109.531,30.1438)(111.562,29.9362)(113.594,30.6154)(115.625,31.4425)(117.656,32.0821)(119.688,32.2946)(121.719,31.943)(123.75,31.0068)(125.781,29.5788)(127.812,27.8729)(129.844,26.2554)(131.875,25.353)(133.906,25.3591)(135.938,25.715)(137.969,26.2082)(140,26.7856)
		\drawline(140,37.2106)(142.031,38.6546)(144.062,40.7639)(146.094,42.8875)(148.125,44.8557)(150.156,46.558)(152.188,47.9083)(154.219,48.8475)(156.25,49.3346)(158.281,49.3437)(160.312,48.8627)(162.344,47.8968)(164.375,46.4746)(166.406,44.6522)(168.438,42.5163)(170.469,40.1839)(172.5,37.7938)
		\drawline(172.5,27.1608)(174.531,26.254)(176.562,25.484)(178.594,24.4938)(180.625,23.228)(182.656,21.7644)(184.688,20.2404)(186.719,18.7967)(188.75,17.5457)(190.781,16.5629)(192.812,15.8873)(194.844,15.5615)(196.875,15.5346)(198.906,15.4648)(200.938,15.2875)(202.969,15.0167)(205,14.6926)
		\drawline(205,29.2404)(207.031,27.2237)(209.062,25.0679)(211.094,23.0123)(213.125,21.2026)(215.156,20.0676)(217.188,19.1515)(219.219,18.3168)(221.25,17.5149)(223.281,16.9156)(225.312,16.5719)(227.344,16.2966)(229.375,15.9318)(231.406,15.4545)(233.438,14.9471)(235.469,14.5265)(237.5,14.212)
		\drawline(237.5,31.2468)(239.531,30.0885)(241.562,30.7595)(243.594,31.4137)(245.625,31.6398)(247.656,31.2408)(249.688,30.1854)(251.719,28.6)(253.75,26.7655)(255.781,25.3104)(257.812,24.2683)(259.844,23.5296)(261.875,23.2049)(263.906,23.3564)(265.938,23.9064)(267.969,24.6399)(270,25.4375)
		\drawline(10,31.2022)(12.0312,30.2673)(14.0625,30.4859)(16.0938,30.8423)(18.125,31.2022)(20.1562,31.4368)(22.1875,31.4888)(24.2188,31.332)(26.25,30.9576)(28.2812,30.3693)(30.3125,29.5837)(32.3438,28.6339)(34.375,27.5837)(36.4062,26.5269)(38.4375,25.5232)(40.4688,24.5505)(42.5,23.635)
		\drawline(42.5,22.1085)(44.5312,21.3391)(46.5625,21.1411)(48.5938,20.9518)(50.625,20.6122)(52.6562,20.1003)(54.6875,19.465)(56.7188,18.7847)(58.75,18.1383)(60.7812,17.589)(62.8125,17.1803)(64.8438,16.9442)(66.875,16.8274)(68.9062,16.7011)(70.9375,16.5112)(72.9688,16.2575)(75,15.9634)
		\drawline(75,35.0836)(77.0312,33.7227)(79.0625,33.5216)(81.0938,32.9362)(83.125,31.2754)(85.1562,28.9019)(87.1875,26.4028)(89.2188,24.2953)(91.25,22.8569)(93.2812,22.2537)(95.3125,22.2311)(97.3438,22.0965)(99.375,21.5622)(101.406,20.6817)(103.438,19.6462)(105.469,18.6553)(107.5,17.8712)
		\drawline(107.5,30.1043)(109.531,29.695)(111.562,29.7141)(113.594,30.0515)(115.625,30.7652)(117.656,31.0917)(119.688,30.966)(121.719,30.5706)(123.75,30.6016)(125.781,31.0516)(127.812,31.351)(129.844,31.519)(131.875,31.6528)(133.906,31.9549)(135.938,32.6792)(137.969,33.5829)(140,34.5237)
		\drawline(140,28.5973)(142.031,27.3183)(144.062,27.3755)(146.094,27.9914)(148.125,28.9435)(150.156,30.1072)(152.188,31.3515)(154.219,32.5455)(156.25,33.5909)(158.281,34.4258)(160.312,35.0133)(162.344,35.3351)(164.375,35.3882)(166.406,35.1856)(168.438,34.7572)(170.469,34.1491)(172.5,33.4204)
		\drawline(172.5,26.4121)(174.531,23.7502)(176.562,22.3306)(178.594,21.6685)(180.625,21.3512)(182.656,21.0708)(184.688,20.7026)(186.719,20.2417)(188.75,19.7215)(190.781,19.1816)(192.812,18.6638)(194.844,18.2142)(196.875,17.88)(198.906,17.7102)(200.938,17.7684)(202.969,17.8924)(205,17.967)
		\drawline(205,29.2071)(207.031,26.5703)(209.062,24.8678)(211.094,24.2692)(213.125,24.4766)(215.156,24.8096)(217.188,24.8457)(219.219,24.3872)(221.25,23.4201)(223.281,22.1149)(225.312,20.7181)(227.344,19.4618)(229.375,18.5566)(231.406,18.2049)(233.438,18.1705)(235.469,18.0862)(237.5,17.8231)
		\drawline(237.5,29.5367)(239.531,29.0534)(241.562,28.3832)(243.594,27.6374)(245.625,27.528)(247.656,27.739)(249.688,27.4743)(251.719,26.6984)(253.75,25.6888)(255.781,24.7734)(257.812,24.3962)(259.844,24.3714)(261.875,24.4318)(263.906,24.5188)(265.938,24.856)(267.969,25.4565)(270,26.1571)
		\drawline(10,34.3989)(12.0312,32.9596)(14.0625,33.2369)(16.0938,34.0227)(18.125,34.9715)(20.1562,35.9699)(22.1875,36.948)(24.2188,37.8094)(26.25,38.4482)(28.2812,38.7868)(30.3125,38.7818)(32.3438,38.4189)(34.375,37.7065)(36.4062,36.6756)(38.4375,35.3817)(40.4688,33.8951)(42.5,32.2931)
		\drawline(42.5,28.827)(44.5312,28.5471)(46.5625,29.2027)(48.5938,30.3875)(50.625,31.3703)(52.6562,31.8804)(54.6875,31.8966)(56.7188,31.5161)(58.75,30.8923)(60.7812,30.1897)(62.8125,29.5486)(64.8438,29.0703)(66.875,28.8005)(68.9062,28.6708)(70.9375,28.5305)(72.9688,28.2879)(75,27.9316)
		\drawline(75,24.6044)(77.0312,24.1104)(79.0625,23.73)(81.0938,23.5701)(83.125,23.7222)(85.1562,24.1802)(87.1875,24.7624)(89.2188,25.0221)(91.25,24.6357)(93.2812,23.6541)(95.3125,22.3788)(97.3438,21.1512)(99.375,20.2466)(101.406,19.9143)(103.438,19.9683)(105.469,19.9226)(107.5,19.6073)
		\drawline(107.5,27.1939)(109.531,24.5397)(111.562,22.3744)(113.594,20.8158)(115.625,20.4634)(117.656,20.8688)(119.688,21.5026)(121.719,22.0948)(123.75,22.5222)(125.781,22.785)(127.812,22.8062)(129.844,22.6016)(131.875,22.2409)(133.906,21.8243)(135.938,21.4513)(137.969,21.1604)(140,20.9387)
		\drawline(140,32.5932)(142.031,32.7113)(144.062,33.254)(146.094,33.8798)(148.125,34.4175)(150.156,34.8233)(152.188,35.1139)(154.219,35.3639)(156.25,35.757)(158.281,36.1129)(160.312,36.2923)(162.344,36.2341)(164.375,35.9181)(166.406,35.3576)(168.438,34.5943)(170.469,33.6923)(172.5,32.7312)
		\drawline(172.5,25.9142)(174.531,24.4483)(176.562,23.5228)(178.594,23.1938)(180.625,23.4613)(182.656,23.9891)(184.688,24.2414)(186.719,24.1018)(188.75,23.6092)(190.781,22.8683)(192.812,22.0013)(194.844,21.117)(196.875,20.3061)(198.906,19.6365)(200.938,19.1527)(202.969,18.7695)(205,18.3553)
		\drawline(205,26.9217)(207.031,25.2705)(209.062,24.6557)(211.094,24.2996)(213.125,23.9669)(215.156,23.6876)(217.188,23.7411)(219.219,23.8797)(221.25,23.6504)(223.281,22.9216)(225.312,21.8245)(227.344,20.616)(229.375,19.5209)(231.406,18.6973)(233.438,18.3232)(235.469,18.2424)(237.5,18.0678)
		\drawline(237.5,26.182)(239.531,25.4089)(241.562,24.6933)(243.594,23.5769)(245.625,22.0967)(247.656,20.7796)(249.688,19.8271)(251.719,19.3393)(253.75,19.3705)(255.781,19.6198)(257.812,19.9369)(259.844,20.2259)(261.875,20.4328)(263.906,20.5447)(265.938,20.5934)(267.969,20.6768)(270,20.9593)
		\drawline(10,32.2032)(12.0312,33.4583)(14.0625,35.3464)(16.0938,37.2764)(18.125,39.0427)(20.1562,40.5368)(22.1875,41.7021)(24.2188,42.5059)(26.25,42.928)(28.2812,42.9585)(30.3125,42.5999)(32.3438,41.886)(34.375,40.8597)(36.4062,39.5629)(38.4375,38.0636)(40.4688,36.4453)(42.5,34.8012)
		\drawline(42.5,32.4883)(44.5312,32.3674)(46.5625,31.8988)(48.5938,30.9699)(50.625,29.6611)(52.6562,28.138)(54.6875,26.5955)(56.7188,25.2144)(58.75,24.1308)(60.7812,23.4078)(62.8125,23.0941)(64.8438,23.2412)(66.875,23.3455)(68.9062,23.2537)(70.9375,22.9731)(72.9688,22.5647)(75,22.106)
		\drawline(75,30.6317)(77.0312,28.6088)(79.0625,26.891)(81.0938,25.8445)(83.125,25.4687)(85.1562,25.1377)(87.1875,24.4538)(89.2188,23.3133)(91.25,21.8368)(93.2812,20.275)(95.3125,18.8797)(97.3438,17.8223)(99.375,17.2251)(101.406,16.9563)(103.438,16.6669)(105.469,16.2331)(107.5,15.6677)
		\drawline(107.5,30.552)(109.531,27.1803)(111.562,24.4464)(113.594,22.8205)(115.625,22.5466)(117.656,22.6896)(119.688,22.914)(121.719,23.06)(123.75,23.0147)(125.781,22.6919)(127.812,22.08)(129.844,21.2415)(131.875,20.2863)(133.906,19.3532)(135.938,18.6308)(137.969,18.3294)(140,18.2486)
		\drawline(140,31.1866)(142.031,30.9259)(144.062,31.1198)(146.094,31.3012)(148.125,31.3417)(150.156,31.2492)(152.188,31.0418)(154.219,30.7502)(156.25,30.4475)(158.281,30.2218)(160.312,29.9657)(162.344,29.6419)(164.375,29.1607)(166.406,28.52)(168.438,27.7529)(170.469,26.9134)(172.5,26.0659)
		\drawline(172.5,36.0284)(174.531,35.1798)(176.562,35.2922)(178.594,35.225)(180.625,34.6264)(182.656,33.5376)(184.688,32.1713)(186.719,30.7823)(188.75,29.587)(190.781,28.7252)(192.812,28.2826)(194.844,28.4035)(196.875,28.6042)(198.906,28.5524)(200.938,28.2257)(202.969,27.6964)(205,27.0703)
		\drawline(205,27.1578)(207.031,26.7881)(209.062,25.9453)(211.094,24.2432)(213.125,21.9291)(215.156,19.6282)(217.188,18.1846)(219.219,17.3564)(221.25,17.0461)(223.281,17.0235)(225.312,16.82)(227.344,16.2543)(229.375,15.4373)(231.406,14.5954)(233.438,14.0105)(235.469,13.7741)(237.5,13.7884)
		\drawline(237.5,32.2056)(239.531,31.6314)(241.562,32.0624)(243.594,33.0939)(245.625,34.2973)(247.656,35.3052)(249.688,36.1875)(251.719,37.1256)(253.75,37.7486)(255.781,37.757)(257.812,37.0312)(259.844,35.6213)(261.875,33.6954)(263.906,31.5018)(265.938,29.3527)(267.969,27.6281)(270,26.8892)
		\put(5,7){\tiny 0}
		\put(0,40){$\zeta$}
		\put(5,70){\tiny 3}
		\put(175,75){\makebox(96,4)[r]{$R=2000,\,\alpha=10,\,n=0$}}
	\end{picture}
	%\input{/home/elmar/src/floquet/integrator/dat/resrand/2500-75-0.epic}
	\begin{picture}(275,78)
		\thinlines
		\drawline(10,8)(10,73)
		\drawline(42.5,8)(42.5,73)
		\drawline(75,8)(75,73)
		\drawline(107.5,8)(107.5,73)
		\drawline(140,8)(140,73)
		\drawline(172.5,8)(172.5,73)
		\drawline(205,8)(205,73)
		\drawline(237.5,8)(237.5,73)
		\drawline(270,8)(270,73)
		\drawline(10,8)(270,8)
		\drawline(10,73)(270,73)
		\thicklines
		\drawline(10,38.0671)(12.0312,37.5975)(14.0625,37.6651)(16.0938,37.7655)(18.125,37.7689)(20.1562,37.617)(22.1875,37.2907)(24.2188,36.7922)(26.25,36.1342)(28.2812,35.3334)(30.3125,34.4091)(32.3438,33.3825)(34.375,32.2758)(36.4062,31.1153)(38.4375,29.9312)(40.4688,28.7596)(42.5,27.6438)
		\drawline(42.5,25.9852)(44.5312,24.7203)(46.5625,23.5591)(48.5938,22.5032)(50.625,21.5492)(52.6562,20.7224)(54.6875,19.9321)(56.7188,19.159)(58.75,18.411)(60.7812,17.7039)(62.8125,17.0601)(64.8438,16.5247)(66.875,16.0941)(68.9062,15.7234)(70.9375,15.4098)(72.9688,15.1362)(75,14.8833)
		\drawline(75,33.5168)(77.0312,31.5326)(79.0625,29.5961)(81.0938,27.753)(83.125,26.0992)(85.1562,24.7423)(87.1875,23.7639)(89.2188,23.1093)(91.25,22.6546)(93.2812,22.2478)(95.3125,21.8137)(97.3438,21.3245)(99.375,20.7755)(101.406,20.1839)(103.438,19.5838)(105.469,19.0149)(107.5,18.5136)
		\drawline(107.5,28.9771)(109.531,28.2732)(111.562,27.7106)(113.594,27.1946)(115.625,26.6207)(117.656,25.8358)(119.688,24.8694)(121.719,23.8265)(123.75,22.8513)(125.781,22.1778)(127.812,21.9014)(129.844,21.8212)(131.875,21.8499)(133.906,21.9558)(135.938,22.1288)(137.969,22.3493)(140,22.5802)
		\drawline(140,29.7275)(142.031,28.5216)(144.062,27.9121)(146.094,27.59)(148.125,27.3977)(150.156,27.2497)(152.188,27.095)(154.219,26.9006)(156.25,26.6421)(158.281,26.3024)(160.312,25.8725)(162.344,25.3511)(164.375,24.7449)(166.406,24.0683)(168.438,23.3431)(170.469,22.598)(172.5,21.8667)
		\drawline(172.5,27.0475)(174.531,26.0597)(176.562,25.1501)(178.594,24.1704)(180.625,23.1054)(182.656,21.9975)(184.688,20.9126)(186.719,19.9171)(188.75,19.0583)(190.781,18.3546)(192.812,17.8003)(194.844,17.3799)(196.875,17.0978)(198.906,16.8839)(200.938,16.6286)(202.969,16.3284)(205,16.0007)
		\drawline(205,24.2913)(207.031,22.5121)(209.062,21.1615)(211.094,20.325)(213.125,19.8971)(215.156,19.6797)(217.188,19.6045)(219.219,19.5238)(221.25,19.3546)(223.281,19.064)(225.312,18.6637)(227.344,18.1975)(229.375,17.7232)(231.406,17.2932)(233.438,16.946)(235.469,16.7154)(237.5,16.5774)
		\drawline(237.5,31.2808)(239.531,30.1462)(241.562,29.1023)(243.594,28.0479)(245.625,26.9649)(247.656,25.8755)(249.688,24.8307)(251.719,23.8996)(253.75,23.1712)(255.781,22.5949)(257.812,22.1265)(259.844,21.7727)(261.875,21.487)(263.906,21.249)(265.938,21.0541)(267.969,20.9027)(270,20.7551)
		\drawline(10,28.5435)(12.0312,26.5422)(14.0625,25.1088)(16.0938,24.0649)(18.125,23.3689)(20.1562,23.0179)(22.1875,22.8837)(24.2188,22.8254)(26.25,22.7772)(28.2812,22.7067)(30.3125,22.5952)(32.3438,22.4307)(34.375,22.2072)(36.4062,21.9232)(38.4375,21.5817)(40.4688,21.1903)(42.5,20.7601)
		\drawline(42.5,31.2219)(44.5312,30.5697)(46.5625,30.0974)(48.5938,29.6073)(50.625,29.0267)(52.6562,28.366)(54.6875,27.667)(56.7188,26.9861)(58.75,26.3817)(60.7812,25.9031)(62.8125,25.5875)(64.8438,25.4832)(66.875,25.5618)(68.9062,25.6251)(70.9375,25.6262)(72.9688,25.5617)(75,25.4463)
		\drawline(75,36.642)(77.0312,35.3468)(79.0625,34.2245)(81.0938,33.4529)(83.125,32.4796)(85.1562,31.2037)(87.1875,29.786)(89.2188,28.4636)(91.25,27.448)(93.2812,26.8828)(95.3125,26.6563)(97.3438,26.5533)(99.375,26.3678)(101.406,26.0418)(103.438,25.5794)(105.469,25.0209)(107.5,24.4344)
		\drawline(107.5,43.2413)(109.531,40.895)(111.562,38.2068)(113.594,35.3379)(115.625,32.5769)(117.656,30.1984)(119.688,28.3416)(121.719,27.1213)(123.75,26.5345)(125.781,26.361)(127.812,26.3731)(129.844,26.4026)(131.875,26.3512)(133.906,26.1605)(135.938,25.8063)(137.969,25.2943)(140,24.6545)
		\drawline(140,36.0529)(142.031,34.965)(144.062,34.0403)(146.094,33.2406)(148.125,32.5817)(150.156,32.1423)(152.188,32.0082)(154.219,31.993)(156.25,31.9742)(158.281,31.9126)(160.312,31.787)(162.344,31.5822)(164.375,31.2867)(166.406,30.893)(168.438,30.3988)(170.469,29.8064)(172.5,29.124)
		\drawline(172.5,31.7064)(174.531,30.8759)(176.562,30.1555)(178.594,29.4658)(180.625,28.7738)(182.656,28.0845)(184.688,27.419)(186.719,26.8021)(188.75,26.2585)(190.781,25.8122)(192.812,25.4905)(194.844,25.2631)(196.875,25.0278)(198.906,24.7444)(200.938,24.4079)(202.969,24.0279)(205,23.6208)
		\drawline(205,27.2637)(207.031,26.176)(209.062,25.2726)(211.094,24.8825)(213.125,24.4551)(215.156,23.7467)(217.188,22.871)(219.219,21.9287)(221.25,21.0497)(223.281,20.368)(225.312,19.8112)(227.344,19.3996)(229.375,19.1395)(231.406,18.91)(233.438,18.6193)(235.469,18.2305)(237.5,17.7584)
		\drawline(237.5,29.213)(239.531,28.6504)(241.562,28.2605)(243.594,27.8662)(245.625,27.4795)(247.656,27.2607)(249.688,27.351)(251.719,27.6219)(253.75,27.9385)(255.781,28.1878)(257.812,28.298)(259.844,28.2374)(261.875,28.0149)(263.906,27.7018)(265.938,27.4829)(267.969,27.4923)(270,27.6774)
		\drawline(10,29.4011)(12.0312,28.616)(14.0625,28.0677)(16.0938,27.7)(18.125,27.4449)(20.1562,27.2486)(22.1875,27.0677)(24.2188,26.8701)(26.25,26.6326)(28.2812,26.3385)(30.3125,25.9767)(32.3438,25.5408)(34.375,25.0289)(36.4062,24.4437)(38.4375,23.7929)(40.4688,23.0886)(42.5,22.3474)
		\drawline(42.5,32.0424)(44.5312,32.0504)(46.5625,32.0847)(48.5938,31.8784)(50.625,31.408)(52.6562,30.7207)(54.6875,29.8944)(56.7188,29.0165)(58.75,28.1343)(60.7812,27.2943)(62.8125,26.5361)(64.8438,25.885)(66.875,25.3597)(68.9062,24.9745)(70.9375,24.6348)(72.9688,24.2591)(75,23.829)
		\drawline(75,32.6079)(77.0312,32.9113)(79.0625,32.8187)(81.0938,32.2541)(83.125,31.3551)(85.1562,30.3966)(87.1875,29.5904)(89.2188,28.9056)(91.25,28.4514)(93.2812,28.293)(95.3125,28.3612)(97.3438,28.3749)(99.375,28.1866)(101.406,27.8142)(103.438,27.3436)(105.469,26.8752)(107.5,26.5213)
		\drawline(107.5,39.7176)(109.531,37.4463)(111.562,36.013)(113.594,34.8476)(115.625,33.6358)(117.656,32.281)(119.688,30.7593)(121.719,29.1042)(123.75,27.4113)(125.781,25.8167)(127.812,24.4576)(129.844,23.45)(131.875,22.911)(133.906,22.7776)(135.938,22.8361)(137.969,22.9591)(140,23.121)
		\drawline(140,39.2882)(142.031,38.059)(144.062,36.9185)(146.094,35.7713)(148.125,34.6519)(150.156,33.5881)(152.188,32.5947)(154.219,31.6862)(156.25,30.8768)(158.281,30.204)(160.312,29.6328)(162.344,29.0908)(164.375,28.5667)(166.406,28.0742)(168.438,27.6231)(170.469,27.2249)(172.5,26.8918)
		\drawline(172.5,30.3284)(174.531,29.184)(176.562,28.1964)(178.594,27.1796)(180.625,26.1025)(182.656,24.9897)(184.688,23.8978)(186.719,22.8908)(188.75,22.0207)(190.781,21.3176)(192.812,20.7889)(194.844,20.4268)(196.875,20.235)(198.906,20.1797)(200.938,20.1051)(202.969,19.9856)(205,19.8262)
		\drawline(205,29.6691)(207.031,28.9291)(209.062,28.2607)(211.094,27.67)(213.125,27.1751)(215.156,26.6895)(217.188,26.0892)(219.219,25.334)(221.25,24.4849)(223.281,23.6426)(225.312,22.8902)(227.344,22.2904)(229.375,21.9134)(231.406,21.7705)(233.438,21.6726)(235.469,21.4864)(237.5,21.1806)
		\drawline(237.5,28.9384)(239.531,28.8188)(241.562,28.8105)(243.594,28.6234)(245.625,28.2335)(247.656,27.6959)(249.688,27.2443)(251.719,26.8233)(253.75,26.3641)(255.781,25.8497)(257.812,25.2882)(259.844,24.7196)(261.875,24.2055)(263.906,23.8819)(265.938,23.9328)(267.969,24.0744)(270,24.11)
		\drawline(10,29.1898)(12.0312,29.0676)(14.0625,28.9836)(16.0938,28.8788)(18.125,28.736)(20.1562,28.5556)(22.1875,28.3512)(24.2188,28.1727)(26.25,27.9981)(28.2812,27.752)(30.3125,27.4205)(32.3438,27.0031)(34.375,26.5073)(36.4062,25.9482)(38.4375,25.3482)(40.4688,24.7412)(42.5,24.156)
		\drawline(42.5,27.1355)(44.5312,25.76)(46.5625,24.6577)(48.5938,23.8324)(50.625,23.1754)(52.6562,22.6055)(54.6875,22.08)(56.7188,21.5784)(58.75,21.0951)(60.7812,20.6283)(62.8125,20.1744)(64.8438,19.747)(66.875,19.3667)(68.9062,19.0699)(70.9375,18.9139)(72.9688,18.8159)(75,18.7132)
		\drawline(75,28.5491)(77.0312,27.2977)(79.0625,26.2838)(81.0938,25.4948)(83.125,24.9402)(85.1562,24.4791)(87.1875,23.9877)(89.2188,23.4183)(91.25,22.7727)(93.2812,22.0758)(95.3125,21.3647)(97.3438,20.6813)(99.375,20.0647)(101.406,19.5513)(103.438,19.1805)(105.469,18.992)(107.5,18.9011)
		\drawline(107.5,31.1203)(109.531,30.2833)(111.562,29.6247)(113.594,29.0431)(115.625,28.5149)(117.656,28.0519)(119.688,27.7151)(121.719,27.5545)(123.75,27.4293)(125.781,27.2754)(127.812,27.0691)(129.844,26.7994)(131.875,26.4646)(133.906,26.0697)(135.938,25.6246)(137.969,25.142)(140,24.6353)
		\drawline(140,23.5252)(142.031,22.1385)(144.062,20.9828)(146.094,20.0008)(148.125,19.1736)(150.156,18.4925)(152.188,17.9543)(154.219,17.5386)(156.25,17.2385)(158.281,17.0331)(160.312,16.8928)(162.344,16.7998)(164.375,16.7363)(166.406,16.685)(168.438,16.632)(170.469,16.5683)(172.5,16.4882)
		\drawline(172.5,28.7319)(174.531,27.6796)(176.562,26.9463)(178.594,26.434)(180.625,26.0618)(182.656,25.7899)(184.688,25.5625)(186.719,25.3355)(188.75,25.0906)(190.781,24.8317)(192.812,24.5751)(194.844,24.3408)(196.875,24.1463)(198.906,24.0034)(200.938,23.9146)(202.969,23.8567)(205,23.7941)
		\drawline(205,34.3347)(207.031,33.062)(209.062,31.8004)(211.094,30.5049)(213.125,29.114)(215.156,27.632)(217.188,26.1157)(219.219,24.6465)(221.25,23.3118)(223.281,22.1808)(225.312,21.2849)(227.344,20.6298)(229.375,20.1717)(231.406,19.8039)(233.438,19.4294)(235.469,19.0215)(237.5,18.5805)
		\drawline(237.5,26.2134)(239.531,26.2356)(241.562,26.3965)(243.594,26.5713)(245.625,26.7883)(247.656,26.9213)(249.688,26.8112)(251.719,26.3715)(253.75,25.6083)(255.781,24.594)(257.812,23.4388)(259.844,22.2744)(261.875,21.247)(263.906,20.5181)(265.938,20.2158)(267.969,20.151)(270,20.2093)
		\drawline(10,29.3776)(12.0312,28.7465)(14.0625,28.4729)(16.0938,28.3814)(18.125,28.3611)(20.1562,28.352)(22.1875,28.3307)(24.2188,28.3002)(26.25,28.3021)(28.2812,28.4135)(30.3125,28.5126)(32.3438,28.5333)(34.375,28.4518)(36.4062,28.2618)(38.4375,27.9667)(40.4688,27.5771)(42.5,27.1087)
		\drawline(42.5,23.8041)(44.5312,22.7047)(46.5625,21.7681)(48.5938,20.9631)(50.625,20.2787)(52.6562,19.7175)(54.6875,19.2819)(56.7188,18.948)(58.75,18.7014)(60.7812,18.5259)(62.8125,18.3875)(64.8438,18.2568)(66.875,18.1199)(68.9062,17.9735)(70.9375,17.8201)(72.9688,17.6652)(75,17.5145)
		\drawline(75,26.74)(77.0312,25.7222)(79.0625,24.5555)(81.0938,23.3543)(83.125,22.3532)(85.1562,21.7479)(87.1875,21.4091)(89.2188,21.173)(91.25,20.9025)(93.2812,20.53)(95.3125,20.0453)(97.3438,19.4855)(99.375,18.9157)(101.406,18.4099)(103.438,18.0364)(105.469,17.8294)(107.5,17.7042)
		\drawline(107.5,33.1999)(109.531,32.5366)(111.562,32.092)(113.594,31.6591)(115.625,31.2702)(117.656,30.792)(119.688,30.1466)(121.719,29.3718)(123.75,28.5858)(125.781,27.9412)(127.812,27.2521)(129.844,26.4899)(131.875,25.696)(133.906,24.9375)(135.938,24.3099)(137.969,23.9755)(140,23.8176)
		\drawline(140,27.5472)(142.031,26.1777)(144.062,25.0835)(146.094,24.1609)(148.125,23.399)(150.156,22.8229)(152.188,22.5309)(154.219,22.3496)(156.25,22.183)(158.281,22.0396)(160.312,21.8973)(162.344,21.7422)(164.375,21.5701)(166.406,21.383)(168.438,21.1818)(170.469,20.9709)(172.5,20.7585)
		\drawline(172.5,30.5931)(174.531,29.4369)(176.562,28.3698)(178.594,27.4035)(180.625,26.4983)(182.656,25.6623)(184.688,24.8949)(186.719,24.2317)(188.75,23.7339)(190.781,23.5242)(192.812,23.5098)(194.844,23.4049)(196.875,23.1785)(198.906,22.8431)(200.938,22.4282)(202.969,21.9694)(205,21.5002)
		\drawline(205,33.2527)(207.031,32.6391)(209.062,31.8431)(211.094,30.8044)(213.125,29.5349)(215.156,28.1263)(217.188,26.6929)(219.219,25.3792)(221.25,24.2992)(223.281,23.5288)(225.312,23.1334)(227.344,23.0189)(229.375,22.9278)(231.406,22.7261)(233.438,22.3949)(235.469,21.9668)(237.5,21.4958)
		\drawline(237.5,25.1294)(239.531,24.1247)(241.562,23.2844)(243.594,22.4728)(245.625,21.6601)(247.656,20.8554)(249.688,20.0886)(251.719,19.4063)(253.75,18.8955)(255.781,18.52)(257.812,18.169)(259.844,17.8082)(261.875,17.4293)(263.906,17.0357)(265.938,16.6364)(267.969,16.2436)(270,15.8712)
		\drawline(10,22.8624)(12.0312,22.2717)(14.0625,22.0269)(16.0938,21.9146)(18.125,21.8673)(20.1562,21.865)(22.1875,21.9357)(24.2188,22.0544)(26.25,22.123)(28.2812,22.1121)(30.3125,22.0116)(32.3438,21.8214)(34.375,21.5492)(36.4062,21.21)(38.4375,20.8266)(40.4688,20.4355)(42.5,20.0912)
		\drawline(42.5,31.7469)(44.5312,31.169)(46.5625,30.5877)(48.5938,29.9635)(50.625,29.2975)(52.6562,28.6045)(54.6875,27.9061)(56.7188,27.2272)(58.75,26.5929)(60.7812,26.0256)(62.8125,25.5454)(64.8438,25.1715)(66.875,24.9251)(68.9062,24.7515)(70.9375,24.561)(72.9688,24.3207)(75,24.0279)
		\drawline(75,30.3888)(77.0312,29.8277)(79.0625,29.4299)(81.0938,29.3191)(83.125,29.1786)(85.1562,28.6267)(87.1875,27.5698)(89.2188,26.1781)(91.25,24.7195)(93.2812,23.4146)(95.3125,22.3918)(97.3438,21.7476)(99.375,21.4134)(101.406,21.2139)(103.438,20.914)(105.469,20.4379)(107.5,19.805)
		\drawline(107.5,34.9113)(109.531,33.7357)(111.562,32.6718)(113.594,31.6041)(115.625,30.5071)(117.656,29.4152)(119.688,28.3978)(121.719,27.5523)(123.75,26.9698)(125.781,26.5269)(127.812,26.1972)(129.844,25.9305)(131.875,25.6986)(133.906,25.503)(135.938,25.3454)(137.969,25.2123)(140,25.0746)
		\drawline(140,28.37)(142.031,27.6811)(144.062,27.0853)(146.094,26.5715)(148.125,26.0924)(150.156,25.629)(152.188,25.1745)(154.219,24.7262)(156.25,24.2804)(158.281,23.8297)(160.312,23.3659)(162.344,22.8828)(164.375,22.3769)(166.406,21.847)(168.438,21.2949)(170.469,20.7248)(172.5,20.1436)
		\drawline(172.5,30.071)(174.531,29.3046)(176.562,28.5176)(178.594,27.6917)(180.625,26.8453)(182.656,26.0156)(184.688,25.2456)(186.719,24.5735)(188.75,24.0263)(190.781,23.6184)(192.812,23.3665)(194.844,23.2438)(196.875,23.1285)(198.906,22.9668)(200.938,22.7503)(202.969,22.4896)(205,22.2035)
		\drawline(205,27.5371)(207.031,26.7767)(209.062,26.3741)(211.094,26.2174)(213.125,26.0954)(215.156,25.8564)(217.188,25.4646)(219.219,24.9665)(221.25,24.4249)(223.281,23.8917)(225.312,23.4218)(227.344,23.0912)(229.375,22.9479)(231.406,22.8685)(233.438,22.7414)(235.469,22.5352)(237.5,22.2519)
		\drawline(237.5,26.0004)(239.531,25.2108)(241.562,24.5477)(243.594,23.8597)(245.625,23.1043)(247.656,22.2844)(249.688,21.4171)(251.719,20.5221)(253.75,19.6284)(255.781,18.7715)(257.812,17.9761)(259.844,17.251)(261.875,16.598)(263.906,16.0198)(265.938,15.5224)(267.969,15.1168)(270,14.8265)
		\drawline(10,30.9771)(12.0312,29.6375)(14.0625,28.7976)(16.0938,28.2936)(18.125,27.983)(20.1562,27.8089)(22.1875,27.6881)(24.2188,27.5606)(26.25,27.3909)(28.2812,27.1567)(30.3125,26.8462)(32.3438,26.4558)(34.375,25.9886)(36.4062,25.454)(38.4375,24.8667)(40.4688,24.2455)(42.5,23.6124)
		\drawline(42.5,33.6904)(44.5312,32.7388)(46.5625,32.1767)(48.5938,31.7187)(50.625,31.1448)(52.6562,30.4053)(54.6875,29.5379)(56.7188,28.6281)(58.75,27.8027)(60.7812,27.1106)(62.8125,26.5655)(64.8438,26.1568)(66.875,25.8792)(68.9062,25.7312)(70.9375,25.6037)(72.9688,25.4323)(75,25.2007)
		\drawline(75,31.7824)(77.0312,31.4509)(79.0625,31.1857)(81.0938,30.3596)(83.125,29.083)(85.1562,27.6377)(87.1875,26.3289)(89.2188,25.4077)(91.25,24.8885)(93.2812,24.7095)(95.3125,24.7896)(97.3438,24.8473)(99.375,24.7207)(101.406,24.3726)(103.438,23.8566)(105.469,23.2673)(107.5,22.7017)
		\drawline(107.5,25.4559)(109.531,24.285)(111.562,23.5815)(113.594,23.2719)(115.625,23.1367)(117.656,23.07)(119.688,22.9902)(121.719,22.8221)(123.75,22.5119)(125.781,22.0411)(127.812,21.4258)(129.844,20.7096)(131.875,19.9547)(133.906,19.2429)(135.938,18.6965)(137.969,18.4265)(140,18.333)
		\drawline(140,31.9211)(142.031,30.7654)(144.062,29.8552)(146.094,29.1578)(148.125,28.6411)(150.156,28.2627)(152.188,27.9873)(154.219,27.7983)(156.25,27.6706)(158.281,27.5742)(160.312,27.4805)(162.344,27.364)(164.375,27.205)(166.406,26.991)(168.438,26.7167)(170.469,26.3823)(172.5,25.9935)
		\drawline(172.5,32.167)(174.531,31.4063)(176.562,30.6588)(178.594,29.8985)(180.625,29.139)(182.656,28.4026)(184.688,27.7083)(186.719,27.0657)(188.75,26.4702)(190.781,25.902)(192.812,25.3342)(194.844,24.7485)(196.875,24.1415)(198.906,23.5242)(200.938,22.9134)(202.969,22.3187)(205,21.7505)
		\drawline(205,32.0637)(207.031,30.9004)(209.062,29.7288)(211.094,28.5797)(213.125,27.5013)(215.156,26.5274)(217.188,25.6748)(219.219,24.9329)(221.25,24.2626)(223.281,23.6305)(225.312,23.0248)(227.344,22.4475)(229.375,21.9064)(231.406,21.41)(233.438,20.9688)(235.469,20.608)(237.5,20.2993)
		\drawline(237.5,23.3648)(239.531,22.376)(241.562,21.6177)(243.594,20.934)(245.625,20.2175)(247.656,19.444)(249.688,18.6191)(251.719,17.7634)(253.75,16.9039)(255.781,16.0696)(257.812,15.2872)(259.844,14.5795)(261.875,13.9633)(263.906,13.4491)(265.938,13.0421)(267.969,12.7447)(270,12.5657)
		\drawline(10,35.4746)(12.0312,34.3261)(14.0625,33.8839)(16.0938,33.6438)(18.125,33.4265)(20.1562,33.1585)(22.1875,32.8023)(24.2188,32.3384)(26.25,31.7584)(28.2812,31.0614)(30.3125,30.2525)(32.3438,29.3433)(34.375,28.34)(36.4062,27.2539)(38.4375,26.1031)(40.4688,24.9124)(42.5,23.7126)
		\drawline(42.5,31.569)(44.5312,30.4304)(46.5625,29.5455)(48.5938,28.8972)(50.625,28.4435)(52.6562,28.0874)(54.6875,27.7211)(56.7188,27.2903)(58.75,26.7849)(60.7812,26.2222)(62.8125,25.6332)(64.8438,25.0526)(66.875,24.5126)(68.9062,24.0405)(70.9375,23.6852)(72.9688,23.4358)(75,23.1658)
		\drawline(75,27.6949)(77.0312,28.0135)(79.0625,28.4299)(81.0938,28.6431)(83.125,28.4588)(85.1562,27.8338)(87.1875,26.8627)(89.2188,25.7177)(91.25,24.5776)(93.2812,23.5798)(95.3125,22.8111)(97.3438,22.333)(99.375,22.158)(101.406,22.0956)(103.438,21.9488)(105.469,21.6728)(107.5,21.2863)
		\drawline(107.5,31.5439)(109.531,30.1572)(111.562,28.965)(113.594,27.8445)(115.625,26.8417)(117.656,26.0624)(119.688,25.6238)(121.719,25.3481)(123.75,25.1078)(125.781,24.8332)(127.812,24.4884)(129.844,24.0646)(131.875,23.5823)(133.906,23.0547)(135.938,22.4901)(137.969,21.903)(140,21.3121)
		\drawline(140,32.596)(142.031,32.0669)(144.062,32.0506)(146.094,32.3055)(148.125,32.661)(150.156,33.0226)(152.188,33.3316)(154.219,33.5513)(156.25,33.6598)(158.281,33.6479)(160.312,33.5186)(162.344,33.2943)(164.375,33.0354)(166.406,32.8142)(168.438,32.5488)(170.469,32.1943)(172.5,31.7545)
		\drawline(172.5,34.9698)(174.531,33.3858)(176.562,31.9924)(178.594,30.7106)(180.625,29.4921)(182.656,28.3015)(184.688,27.1251)(186.719,25.9764)(188.75,24.8869)(190.781,23.8907)(192.812,23.0135)(194.844,22.2695)(196.875,21.6639)(198.906,21.2157)(200.938,20.8671)(202.969,20.5311)(205,20.1822)
		\drawline(205,30.7773)(207.031,30.82)(209.062,31.0839)(211.094,31.3435)(213.125,31.2356)(215.156,30.5143)(217.188,29.1911)(219.219,27.4877)(221.25,25.7229)(223.281,24.1655)(225.312,22.9251)(227.344,22.0662)(229.375,21.5871)(231.406,21.3356)(233.438,21.1075)(235.469,20.752)(237.5,20.2465)
		\drawline(237.5,31.7189)(239.531,30.9593)(241.562,30.3778)(243.594,29.702)(245.625,28.7501)(247.656,27.5006)(249.688,26.0242)(251.719,24.4481)(253.75,22.9216)(255.781,21.5668)(257.812,20.4628)(259.844,19.6696)(261.875,19.2579)(263.906,19.1027)(265.938,19.0611)(267.969,19.0644)(270,19.0687)
		\drawline(10,34.9208)(12.0312,33.5966)(14.0625,32.7793)(16.0938,32.1726)(18.125,31.6665)(20.1562,31.2124)(22.1875,30.7875)(24.2188,30.3916)(26.25,30.0877)(28.2812,29.8727)(30.3125,29.5562)(32.3438,29.0985)(34.375,28.5041)(36.4062,27.7929)(38.4375,26.9942)(40.4688,26.144)(42.5,25.2809)
		\drawline(42.5,36.5881)(44.5312,35.1652)(46.5625,33.8713)(48.5938,32.7373)(50.625,31.8366)(52.6562,31.3647)(54.6875,31.054)(56.7188,30.6573)(58.75,30.1578)(60.7812,29.5592)(62.8125,28.883)(64.8438,28.1622)(66.875,27.4332)(68.9062,26.7294)(70.9375,26.077)(72.9688,25.4927)(75,24.9785)
		\drawline(75,32.2116)(77.0312,31.2772)(79.0625,30.4629)(81.0938,29.7206)(83.125,29.1098)(85.1562,28.7129)(87.1875,28.5276)(89.2188,28.5692)(91.25,28.7058)(93.2812,28.7938)(95.3125,28.7397)(97.3438,28.5211)(99.375,28.1696)(101.406,27.7435)(103.438,27.3044)(105.469,26.9077)(107.5,26.6299)
		\drawline(107.5,34.1917)(109.531,33.3294)(111.562,32.5262)(113.594,31.7194)(115.625,30.9318)(117.656,30.2226)(119.688,29.6691)(121.719,29.392)(123.75,29.3214)(125.781,29.3012)(127.812,29.253)(129.844,29.1314)(131.875,28.9114)(133.906,28.5842)(135.938,28.1544)(137.969,27.6372)(140,27.0535)
		\drawline(140,38.4276)(142.031,38.3043)(144.062,38.4092)(146.094,38.6122)(148.125,38.664)(150.156,38.5303)(152.188,38.2124)(154.219,37.7189)(156.25,37.06)(158.281,36.2462)(160.312,35.2892)(162.344,34.2034)(164.375,33.0109)(166.406,31.7436)(168.438,30.4527)(170.469,29.1966)(172.5,28.0516)
		\drawline(172.5,36.7248)(174.531,34.764)(176.562,32.9658)(178.594,31.3656)(180.625,29.9856)(182.656,28.8129)(184.688,27.8123)(186.719,26.946)(188.75,26.1808)(190.781,25.4909)(192.812,24.8585)(194.844,24.2732)(196.875,23.7314)(198.906,23.2357)(200.938,22.7946)(202.969,22.4281)(205,22.1268)
		\drawline(205,35.0803)(207.031,33.0348)(209.062,31.239)(211.094,29.7501)(213.125,28.5326)(215.156,27.5621)(217.188,26.8515)(219.219,26.3718)(221.25,25.9753)(223.281,25.5943)(225.312,25.2041)(227.344,24.8032)(229.375,24.4032)(231.406,24.0204)(233.438,23.67)(235.469,23.3668)(237.5,23.1519)
		\drawline(237.5,28.6818)(239.531,29.5407)(241.562,30.1312)(243.594,30.4987)(245.625,30.7442)(247.656,30.901)(249.688,30.9467)(251.719,30.836)(253.75,30.5905)(255.781,30.2777)(257.812,29.9711)(259.844,29.7111)(261.875,29.5218)(263.906,29.4463)(265.938,29.4382)(267.969,29.4532)(270,29.4764)
		\put(5,7){\tiny 0}
		\put(0,40){$\zeta$}
		\put(5,70){\tiny 3}
		\put(175,75){\makebox(96,4)[r]{$R=2500,\,\alpha=15,\,n=0$}}
	\end{picture}
	%\input{/home/elmar/src/floquet/integrator/dat/resrand/2000-75-0.epic}
	\begin{picture}(275,78)
		\thinlines
		\drawline(10,8)(10,73)
		\drawline(42.5,8)(42.5,73)
		\drawline(75,8)(75,73)
		\drawline(107.5,8)(107.5,73)
		\drawline(140,8)(140,73)
		\drawline(172.5,8)(172.5,73)
		\drawline(205,8)(205,73)
		\drawline(237.5,8)(237.5,73)
		\drawline(270,8)(270,73)
		\drawline(10,8)(270,8)
		\drawline(10,73)(270,73)
		\thicklines
		\drawline(10,34.5794)(12.0312,33.171)(14.0625,32.2062)(16.0938,31.5235)(18.125,31.0113)(20.1562,30.5964)(22.1875,30.2265)(24.2188,29.8636)(26.25,29.4807)(28.2812,29.0591)(30.3125,28.5871)(32.3438,28.0521)(34.375,27.4518)(36.4062,26.7903)(38.4375,26.0767)(40.4688,25.3218)(42.5,24.5381)
		\drawline(42.5,29.3259)(44.5312,28.3426)(46.5625,27.5148)(48.5938,26.8394)(50.625,26.3162)(52.6562,25.9458)(54.6875,25.7508)(56.7188,25.6073)(58.75,25.4231)(60.7812,25.1845)(62.8125,24.8946)(64.8438,24.5643)(66.875,24.2089)(68.9062,23.8452)(70.9375,23.4888)(72.9688,23.1526)(75,22.8458)
		\drawline(75,37.5852)(77.0312,36.7157)(79.0625,35.6371)(81.0938,34.3664)(83.125,32.9942)(85.1562,31.6329)(87.1875,30.3849)(89.2188,29.3317)(91.25,28.5483)(93.2812,28.034)(95.3125,27.7485)(97.3438,27.5641)(99.375,27.3529)(101.406,27.0579)(103.438,26.6622)(105.469,26.176)(107.5,25.6248)
		\drawline(107.5,26.2509)(109.531,24.9307)(111.562,24.1521)(113.594,23.5396)(115.625,22.9401)(117.656,22.2906)(119.688,21.5727)(121.719,20.7949)(123.75,19.9866)(125.781,19.172)(127.812,18.3722)(129.844,17.6171)(131.875,16.9395)(133.906,16.3774)(135.938,15.9087)(137.969,15.5134)(140,15.1802)
		\drawline(140,31.0496)(142.031,30.3355)(144.062,29.8398)(146.094,29.4443)(148.125,29.0961)(150.156,28.7766)(152.188,28.4965)(154.219,28.2636)(156.25,28.0738)(158.281,27.9261)(160.312,27.7825)(162.344,27.5961)(164.375,27.3422)(166.406,27.0123)(168.438,26.6061)(170.469,26.1289)(172.5,25.5896)
		\drawline(172.5,30.7615)(174.531,30.0096)(176.562,29.3765)(178.594,28.8326)(180.625,28.3697)(182.656,27.9849)(184.688,27.675)(186.719,27.4331)(188.75,27.2452)(190.781,27.0866)(192.812,26.9285)(194.844,26.7493)(196.875,26.5388)(198.906,26.2952)(200.938,26.0227)(202.969,25.729)(205,25.4235)
		\drawline(205,32.9175)(207.031,32.3286)(209.062,31.8041)(211.094,31.1142)(213.125,30.2155)(215.156,29.16)(217.188,28.0389)(219.219,26.9445)(221.25,25.9454)(223.281,25.075)(225.312,24.3728)(227.344,23.8677)(229.375,23.5268)(231.406,23.2763)(233.438,23.0156)(235.469,22.7018)(237.5,22.3216)
		\drawline(237.5,24.3283)(239.531,23.2298)(241.562,22.7932)(243.594,22.7544)(245.625,22.9016)(247.656,23.0962)(249.688,23.2261)(251.719,23.2622)(253.75,23.2153)(255.781,23.1599)(257.812,23.2059)(259.844,23.2316)(261.875,23.1506)(263.906,22.9384)(265.938,22.6032)(267.969,22.1661)(270,21.6527)
		\drawline(10,23.0706)(12.0312,21.8384)(14.0625,20.8897)(16.0938,20.1625)(18.125,19.6104)(20.1562,19.1746)(22.1875,18.8152)(24.2188,18.4993)(26.25,18.2059)(28.2812,17.9226)(30.3125,17.6424)(32.3438,17.3614)(34.375,17.0774)(36.4062,16.7895)(38.4375,16.4979)(40.4688,16.2036)(42.5,15.9085)
		\drawline(42.5,32.3932)(44.5312,31.7062)(46.5625,31.129)(48.5938,30.4512)(50.625,29.6177)(52.6562,28.6257)(54.6875,27.5025)(56.7188,26.2958)(58.75,25.0648)(60.7812,23.8704)(62.8125,22.7652)(64.8438,21.7871)(66.875,20.9572)(68.9062,20.2827)(70.9375,19.7658)(72.9688,19.412)(75,19.1873)
		\drawline(75,30.7749)(77.0312,29.3021)(79.0625,28.2727)(81.0938,27.6733)(83.125,27.3375)(85.1562,27.0749)(87.1875,26.7919)(89.2188,26.4559)(91.25,26.0706)(93.2812,25.6614)(95.3125,25.2629)(97.3438,24.913)(99.375,24.658)(101.406,24.5144)(103.438,24.4568)(105.469,24.4078)(107.5,24.3283)
		\drawline(107.5,35.8274)(109.531,35.7039)(111.562,35.3416)(113.594,34.7287)(115.625,33.9523)(117.656,33.2655)(119.688,33.046)(121.719,33.0763)(123.75,33.0057)(125.781,32.7204)(127.812,32.2266)(129.844,31.5781)(131.875,30.8709)(133.906,30.1719)(135.938,29.5066)(137.969,28.9446)(140,28.5949)
		\drawline(140,30.2205)(142.031,29.0621)(144.062,28.0595)(146.094,27.1872)(148.125,26.4326)(150.156,25.7743)(152.188,25.193)(154.219,24.6817)(156.25,24.2446)(158.281,23.9018)(160.312,23.7084)(162.344,23.6398)(164.375,23.6156)(166.406,23.6008)(168.438,23.5791)(170.469,23.5407)(172.5,23.4792)
		\drawline(172.5,34.031)(174.531,32.8473)(176.562,31.7387)(178.594,30.685)(180.625,29.6922)(182.656,28.7681)(184.688,27.9159)(186.719,27.134)(188.75,26.419)(190.781,25.7674)(192.812,25.1758)(194.844,24.6411)(196.875,24.1593)(198.906,23.7256)(200.938,23.335)(202.969,22.9828)(205,22.6672)
		\drawline(205,28.9207)(207.031,27.9832)(209.062,27.3325)(211.094,26.7733)(213.125,26.2576)(215.156,25.7661)(217.188,25.3029)(219.219,24.8841)(221.25,24.5197)(223.281,24.2003)(225.312,23.8995)(227.344,23.5886)(229.375,23.2479)(231.406,22.8684)(233.438,22.4517)(235.469,22.0075)(237.5,21.5499)
		\drawline(237.5,25.636)(239.531,24.5065)(241.562,23.4471)(243.594,22.4556)(245.625,21.564)(247.656,20.8245)(249.688,20.3079)(251.719,19.9844)(253.75,19.7896)(255.781,19.6822)(257.812,19.6285)(259.844,19.6018)(261.875,19.5822)(263.906,19.5566)(265.938,19.5172)(267.969,19.4607)(270,19.3876)
		\drawline(10,38.4913)(12.0312,38.3738)(14.0625,38.2937)(16.0938,38.2278)(18.125,38.3117)(20.1562,38.6063)(22.1875,38.8349)(24.2188,38.9406)(26.25,38.9019)(28.2812,38.7073)(30.3125,38.3524)(32.3438,37.8387)(34.375,37.1731)(36.4062,36.3662)(38.4375,35.4346)(40.4688,34.3993)(42.5,33.2865)
		\drawline(42.5,30.8371)(44.5312,29.6049)(46.5625,28.5137)(48.5938,27.5634)(50.625,26.7789)(52.6562,26.1856)(54.6875,25.776)(56.7188,25.4876)(58.75,25.2464)(60.7812,25.0035)(62.8125,24.7345)(64.8438,24.4323)(66.875,24.1012)(68.9062,23.7512)(70.9375,23.3948)(72.9688,23.0439)(75,22.7083)
		\drawline(75,32.5843)(77.0312,30.896)(79.0625,29.1105)(81.0938,27.2863)(83.125,25.5622)(85.1562,24.1039)(87.1875,23.0503)(89.2188,22.273)(91.25,21.6545)(93.2812,21.2251)(95.3125,20.9326)(97.3438,20.6941)(99.375,20.4408)(101.406,20.1092)(103.438,19.6552)(105.469,19.0937)(107.5,18.4683)
		\drawline(107.5,38.8477)(109.531,38.0473)(111.562,37.2502)(113.594,36.41)(115.625,35.6742)(117.656,35.1334)(119.688,34.6396)(121.719,34.0745)(123.75,33.3979)(125.781,32.5951)(127.812,31.6713)(129.844,30.6501)(131.875,29.5692)(133.906,28.4741)(135.938,27.416)(137.969,26.4515)(140,25.6505)
		\drawline(140,33.1576)(142.031,33.1812)(144.062,33.207)(146.094,33.1489)(148.125,32.9799)(150.156,32.6931)(152.188,32.2924)(154.219,31.788)(156.25,31.1911)(158.281,30.5119)(160.312,29.7605)(162.344,28.949)(164.375,28.097)(166.406,27.2288)(168.438,26.3328)(170.469,25.4186)(172.5,24.5082)
		\drawline(172.5,24.1612)(174.531,23.7711)(176.562,23.4075)(178.594,22.9929)(180.625,22.5035)(182.656,21.9464)(184.688,21.3513)(186.719,20.7387)(188.75,20.1282)(190.781,19.5422)(192.812,18.9998)(194.844,18.515)(196.875,18.0968)(198.906,17.7493)(200.938,17.4742)(202.969,17.2977)(205,17.1737)
		\drawline(205,30.9675)(207.031,30.2343)(209.062,29.4585)(211.094,28.587)(213.125,27.6305)(215.156,26.6345)(217.188,25.6611)(219.219,24.7691)(221.25,23.9956)(223.281,23.3486)(225.312,22.7979)(227.344,22.3371)(229.375,21.9788)(231.406,21.6937)(233.438,21.4367)(235.469,21.182)(237.5,20.9176)
		\drawline(237.5,31.3478)(239.531,30.4235)(241.562,29.8877)(243.594,29.6725)(245.625,29.6392)(247.656,29.7024)(249.688,29.8021)(251.719,29.8929)(253.75,29.934)(255.781,29.8903)(257.812,29.7371)(259.844,29.4636)(261.875,29.0731)(263.906,28.5794)(265.938,28.0025)(267.969,27.3657)(270,26.6924)
		\drawline(10,35.2008)(12.0312,34.6281)(14.0625,34.2583)(16.0938,33.8969)(18.125,33.4787)(20.1562,32.9882)(22.1875,32.4259)(24.2188,31.798)(26.25,31.1123)(28.2812,30.3771)(30.3125,29.5994)(32.3438,28.7863)(34.375,27.9438)(36.4062,27.0779)(38.4375,26.1948)(40.4688,25.3018)(42.5,24.407)
		\drawline(42.5,30.7112)(44.5312,29.7442)(46.5625,29.0164)(48.5938,28.5016)(50.625,28.1867)(52.6562,28.0524)(54.6875,28.0339)(56.7188,28.0149)(58.75,27.9348)(60.7812,27.7747)(62.8125,27.5351)(64.8438,27.2284)(66.875,26.8743)(68.9062,26.4952)(70.9375,26.1134)(72.9688,25.7484)(75,25.4159)
		\drawline(75,24.4462)(77.0312,23.4601)(79.0625,22.8548)(81.0938,22.4456)(83.125,22.1232)(85.1562,21.8107)(87.1875,21.4607)(89.2188,21.0645)(91.25,20.6519)(93.2812,20.2852)(95.3125,20.0278)(97.3438,19.8735)(99.375,19.744)(101.406,19.5714)(103.438,19.3313)(105.469,19.0293)(107.5,18.6849)
		\drawline(107.5,27.648)(109.531,26.3774)(111.562,25.3968)(113.594,24.5043)(115.625,23.6602)(117.656,22.8676)(119.688,22.1227)(121.719,21.4306)(123.75,20.8039)(125.781,20.2537)(127.812,19.7844)(129.844,19.392)(131.875,19.0679)(133.906,18.8041)(135.938,18.5973)(137.969,18.4228)(140,18.2465)
		\drawline(140,33.9766)(142.031,33.9764)(144.062,34.0596)(146.094,34.2698)(148.125,34.4485)(150.156,34.5332)(152.188,34.504)(154.219,34.3538)(156.25,34.0817)(158.281,33.6893)(160.312,33.1799)(162.344,32.559)(164.375,31.8331)(166.406,31.0124)(168.438,30.1104)(170.469,29.1445)(172.5,28.1327)
		\drawline(172.5,31.7111)(174.531,30.9101)(176.562,30.2562)(178.594,29.7139)(180.625,29.2701)(182.656,28.9212)(184.688,28.6661)(186.719,28.5009)(188.75,28.4085)(190.781,28.3526)(192.812,28.2968)(194.844,28.2203)(196.875,28.1146)(198.906,27.9791)(200.938,27.8181)(202.969,27.6385)(205,27.4477)
		\drawline(205,24.1793)(207.031,23.3626)(209.062,22.8535)(211.094,22.5093)(213.125,22.2084)(215.156,21.9094)(217.188,21.6018)(219.219,21.2871)(221.25,20.9738)(223.281,20.672)(225.312,20.3805)(227.344,20.0836)(229.375,19.766)(231.406,19.4221)(233.438,19.0563)(235.469,18.6787)(237.5,18.3023)
		\drawline(237.5,32.5678)(239.531,31.5213)(241.562,30.4868)(243.594,29.4824)(245.625,28.5676)(247.656,27.8661)(249.688,27.581)(251.719,27.3479)(253.75,27.0445)(255.781,26.6462)(257.812,26.1594)(259.844,25.6064)(261.875,25.0186)(263.906,24.432)(265.938,23.8866)(267.969,23.437)(270,23.139)
		\drawline(10,28.9951)(12.0312,27.91)(14.0625,27.0035)(16.0938,26.2213)(18.125,25.5462)(20.1562,24.9698)(22.1875,24.486)(24.2188,24.0924)(26.25,23.7919)(28.2812,23.5613)(30.3125,23.3795)(32.3438,23.2295)(34.375,23.0978)(36.4062,22.9748)(38.4375,22.8533)(40.4688,22.7285)(42.5,22.5974)
		\drawline(42.5,36.2492)(44.5312,34.8548)(46.5625,33.6431)(48.5938,32.5598)(50.625,31.5645)(52.6562,30.6265)(54.6875,29.726)(56.7188,28.85)(58.75,27.9916)(60.7812,27.1485)(62.8125,26.3228)(64.8438,25.5202)(66.875,24.7484)(68.9062,24.0151)(70.9375,23.3273)(72.9688,22.6898)(75,22.1056)
		\drawline(75,35.2487)(77.0312,33.4774)(79.0625,31.8134)(81.0938,30.1815)(83.125,28.6274)(85.1562,27.3141)(87.1875,26.1204)(89.2188,24.8097)(91.25,23.4217)(93.2812,22.0464)(95.3125,20.7747)(97.3438,19.6751)(99.375,18.7842)(101.406,18.1143)(103.438,17.6786)(105.469,17.4275)(107.5,17.2472)
		\drawline(107.5,28.8739)(109.531,28.3897)(111.562,28.0313)(113.594,27.7876)(115.625,27.5871)(117.656,27.2926)(119.688,26.8361)(121.719,26.2103)(123.75,25.4492)(125.781,24.6137)(127.812,23.7892)(129.844,23.0706)(131.875,22.5689)(133.906,22.2205)(135.938,21.9397)(137.969,21.6902)(140,21.4563)
		\drawline(140,38.4696)(142.031,37.7022)(144.062,37.1713)(146.094,36.8366)(148.125,36.6628)(150.156,36.6136)(152.188,36.6522)(154.219,36.7436)(156.25,36.8576)(158.281,36.9675)(160.312,37.0498)(162.344,37.086)(164.375,37.0589)(166.406,36.9566)(168.438,36.7703)(170.469,36.4964)(172.5,36.1346)
		\drawline(172.5,36.5229)(174.531,36.345)(176.562,36.1435)(178.594,35.8666)(180.625,35.5108)(182.656,35.0953)(184.688,34.6491)(186.719,34.2047)(188.75,33.7924)(190.781,33.438)(192.812,33.1611)(194.844,32.9802)(196.875,32.8989)(198.906,32.8616)(200.938,32.8153)(202.969,32.7375)(205,32.6248)
		\drawline(205,30.2989)(207.031,29.967)(209.062,29.6037)(211.094,29.1075)(213.125,28.4443)(215.156,27.6274)(217.188,26.7014)(219.219,25.7258)(221.25,24.7624)(223.281,23.8642)(225.312,23.0692)(227.344,22.4018)(229.375,21.8794)(231.406,21.5198)(233.438,21.2879)(235.469,21.0786)(237.5,20.8401)
		\drawline(237.5,38.1366)(239.531,38.4926)(241.562,38.7106)(243.594,38.574)(245.625,37.9717)(247.656,36.93)(249.688,35.565)(251.719,34.0364)(253.75,32.5147)(255.781,31.1909)(257.812,30.0814)(259.844,29.1152)(261.875,28.2806)(263.906,27.5806)(265.938,27.0947)(267.969,26.8643)(270,26.8423)
		\drawline(10,28.5504)(12.0312,27.3821)(14.0625,26.6438)(16.0938,26.0847)(18.125,25.6093)(20.1562,25.183)(22.1875,24.7852)(24.2188,24.4007)(26.25,24.0172)(28.2812,23.6247)(30.3125,23.2146)(32.3438,22.7801)(34.375,22.3166)(36.4062,21.822)(38.4375,21.2975)(40.4688,20.7473)(42.5,20.1789)
		\drawline(42.5,28.4232)(44.5312,27.3562)(46.5625,26.2978)(48.5938,25.2448)(50.625,24.2083)(52.6562,23.2046)(54.6875,22.2527)(56.7188,21.3723)(58.75,20.5806)(60.7812,19.8898)(62.8125,19.3055)(64.8438,18.8263)(66.875,18.4469)(68.9062,18.1663)(70.9375,17.9707)(72.9688,17.7965)(75,17.6109)
		\drawline(75,29.79)(77.0312,28.5213)(79.0625,27.4124)(81.0938,26.554)(83.125,25.9652)(85.1562,25.5077)(87.1875,25.0128)(89.2188,24.4146)(91.25,23.7201)(93.2812,22.9761)(95.3125,22.2334)(97.3438,21.5368)(99.375,20.9209)(101.406,20.4134)(103.438,20.0493)(105.469,19.8172)(107.5,19.6402)
		\drawline(107.5,38.0263)(109.531,37.2721)(111.562,36.4999)(113.594,35.6436)(115.625,34.6814)(117.656,33.6349)(119.688,32.5611)(121.719,31.5324)(123.75,30.6315)(125.781,29.9865)(127.812,29.5894)(129.844,29.345)(131.875,29.1694)(133.906,29.0151)(135.938,28.8557)(137.969,28.6789)(140,28.4841)
		\drawline(140,26.8484)(142.031,26.0327)(144.062,25.7373)(146.094,25.7188)(148.125,25.8018)(150.156,25.9078)(152.188,25.9981)(154.219,26.0534)(156.25,26.0651)(158.281,26.0346)(160.312,25.9662)(162.344,25.8428)(164.375,25.6577)(166.406,25.4092)(168.438,25.0998)(170.469,24.7355)(172.5,24.3255)
		\drawline(172.5,27.3419)(174.531,26.3254)(176.562,25.3778)(178.594,24.4617)(180.625,23.5692)(182.656,22.7143)(184.688,21.9233)(186.719,21.2342)(188.75,20.6872)(190.781,20.2873)(192.812,19.9928)(194.844,19.7485)(196.875,19.5131)(198.906,19.2646)(200.938,18.9968)(202.969,18.7143)(205,18.4262)
		\drawline(205,29.8894)(207.031,29.3624)(209.062,28.9692)(211.094,28.6879)(213.125,28.4965)(215.156,28.3136)(217.188,28.0693)(219.219,27.719)(221.25,27.2463)(223.281,26.659)(225.312,25.9833)(227.344,25.2554)(229.375,24.5144)(231.406,23.7959)(233.438,23.1243)(235.469,22.5188)(237.5,21.997)
		\drawline(237.5,34.9748)(239.531,33.9272)(241.562,33.2503)(243.594,32.6166)(245.625,31.9059)(247.656,31.0763)(249.688,30.1258)(251.719,29.075)(253.75,27.9569)(255.781,26.8084)(257.812,25.6654)(259.844,24.559)(261.875,23.5137)(263.906,22.5475)(265.938,21.6725)(267.969,20.8966)(270,20.226)
		\drawline(10,25.8652)(12.0312,24.8368)(14.0625,24.0869)(16.0938,23.5313)(18.125,23.1101)(20.1562,22.7742)(22.1875,22.4907)(24.2188,22.2395)(26.25,22.0113)(28.2812,21.8067)(30.3125,21.604)(32.3438,21.391)(34.375,21.1612)(36.4062,20.9115)(38.4375,20.6418)(40.4688,20.3544)(42.5,20.0536)
		\drawline(42.5,29.0545)(44.5312,27.8088)(46.5625,26.7509)(48.5938,25.8501)(50.625,25.106)(52.6562,24.5457)(54.6875,24.1945)(56.7188,23.9128)(58.75,23.6163)(60.7812,23.2861)(62.8125,22.9183)(64.8438,22.5138)(66.875,22.0814)(68.9062,21.6333)(70.9375,21.1826)(72.9688,20.7414)(75,20.3199)
		\drawline(75,34.5607)(77.0312,32.962)(79.0625,31.4451)(81.0938,30.0701)(83.125,28.8645)(85.1562,27.8187)(87.1875,26.9071)(89.2188,26.1052)(91.25,25.3963)(93.2812,24.7706)(95.3125,24.2221)(97.3438,23.7492)(99.375,23.3607)(101.406,23.0874)(103.438,22.9)(105.469,22.712)(107.5,22.5006)
		\drawline(107.5,34.8879)(109.531,33.9252)(111.562,32.812)(113.594,31.6065)(115.625,30.4222)(117.656,29.3845)(119.688,28.6408)(121.719,28.1425)(123.75,27.8388)(125.781,27.7238)(127.812,27.7262)(129.844,27.8004)(131.875,27.906)(133.906,28.0029)(135.938,28.0546)(137.969,28.0331)(140,27.922)
		\drawline(140,32.8346)(142.031,31.6221)(144.062,30.878)(146.094,30.428)(148.125,30.1594)(150.156,30.0066)(152.188,29.9319)(154.219,29.9078)(156.25,29.9156)(158.281,29.9457)(160.312,29.9947)(162.344,30.0413)(164.375,30.0716)(166.406,30.0746)(168.438,30.0417)(170.469,29.9672)(172.5,29.8471)
		\drawline(172.5,30.2207)(174.531,29.6114)(176.562,29.0345)(178.594,28.455)(180.625,27.8657)(182.656,27.2684)(184.688,26.6717)(186.719,26.0884)(188.75,25.5345)(190.781,25.0268)(192.812,24.5808)(194.844,24.2098)(196.875,23.9238)(198.906,23.7306)(200.938,23.6659)(202.969,23.6501)(205,23.6198)
		\drawline(205,33.3504)(207.031,32.7189)(209.062,32.0545)(211.094,31.3751)(213.125,30.7636)(215.156,30.2239)(217.188,29.6844)(219.219,29.0832)(221.25,28.3845)(223.281,27.5792)(225.312,26.7043)(227.344,25.816)(229.375,24.9696)(231.406,24.2086)(233.438,23.5638)(235.469,23.0609)(237.5,22.7347)
		\drawline(237.5,26.2605)(239.531,25.8052)(241.562,25.5515)(243.594,25.4088)(245.625,25.3215)(247.656,25.2442)(249.688,25.1345)(251.719,24.9544)(253.75,24.6746)(255.781,24.2824)(257.812,23.7819)(259.844,23.1912)(261.875,22.5372)(263.906,21.85)(265.938,21.1587)(267.969,20.4894)(270,19.8648)
		\drawline(10,33.2116)(12.0312,32.6729)(14.0625,32.2545)(16.0938,31.9317)(18.125,31.6864)(20.1562,31.4958)(22.1875,31.3333)(24.2188,31.1708)(26.25,30.9814)(28.2812,30.7418)(30.3125,30.4347)(32.3438,30.0497)(34.375,29.5825)(36.4062,29.0346)(38.4375,28.4128)(40.4688,27.7285)(42.5,26.9971)
		\drawline(42.5,30.5026)(44.5312,29.9735)(46.5625,29.7288)(48.5938,29.6446)(50.625,29.5878)(52.6562,29.4626)(54.6875,29.2243)(56.7188,28.8667)(58.75,28.4072)(60.7812,27.8755)(62.8125,27.3056)(64.8438,26.7298)(66.875,26.1758)(68.9062,25.6645)(70.9375,25.2099)(72.9688,24.8219)(75,24.5288)
		\drawline(75,26.5461)(77.0312,26.1047)(79.0625,25.631)(81.0938,25.1126)(83.125,24.5093)(85.1562,23.8347)(87.1875,23.139)(89.2188,22.4877)(91.25,21.9398)(93.2812,21.532)(95.3125,21.2936)(97.3438,21.1801)(99.375,20.9938)(101.406,20.7184)(103.438,20.3764)(105.469,20.0038)(107.5,19.6397)
		\drawline(107.5,36.4245)(109.531,35.5834)(111.562,34.816)(113.594,33.883)(115.625,32.7431)(117.656,31.4204)(119.688,29.9735)(121.719,28.4743)(123.75,26.9962)(125.781,25.6021)(127.812,24.341)(129.844,23.2479)(131.875,22.3494)(133.906,21.6801)(135.938,21.2672)(137.969,21.0359)(140,20.8956)
		\drawline(140,29.4345)(142.031,28.3555)(144.062,27.7983)(146.094,27.4456)(148.125,27.1618)(150.156,26.8901)(152.188,26.6039)(154.219,26.2897)(156.25,25.9404)(158.281,25.5521)(160.312,25.1228)(162.344,24.6523)(164.375,24.1419)(166.406,23.5942)(168.438,23.0138)(170.469,22.4074)(172.5,21.7836)
		\drawline(172.5,23.6713)(174.531,22.9441)(176.562,22.1979)(178.594,21.3997)(180.625,20.5586)(182.656,19.704)(184.688,18.8742)(186.719,18.1072)(188.75,17.4283)(190.781,16.8437)(192.812,16.3516)(194.844,15.9483)(196.875,15.6301)(198.906,15.3962)(200.938,15.2554)(202.969,15.1448)(205,15.021)
		\drawline(205,36.6938)(207.031,35.9953)(209.062,35.4759)(211.094,35.0749)(213.125,34.793)(215.156,34.6134)(217.188,34.4808)(219.219,34.3298)(221.25,34.1068)(223.281,33.7812)(225.312,33.3468)(227.344,32.8181)(229.375,32.2222)(231.406,31.5905)(233.438,30.9524)(235.469,30.3333)(237.5,29.7533)
		\drawline(237.5,28.4475)(239.531,28.0308)(241.562,27.7514)(243.594,27.3992)(245.625,26.903)(247.656,26.2489)(249.688,25.447)(251.719,24.5224)(253.75,23.5106)(255.781,22.4555)(257.812,21.4071)(259.844,20.4212)(261.875,19.5585)(263.906,18.8821)(265.938,18.4555)(267.969,18.2175)(270,18.1063)
		\drawline(10,26.8904)(12.0312,25.4237)(14.0625,24.2964)(16.0938,23.437)(18.125,22.7858)(20.1562,22.2656)(22.1875,21.7995)(24.2188,21.3538)(26.25,20.9149)(28.2812,20.4766)(30.3125,20.0351)(32.3438,19.5881)(34.375,19.1346)(36.4062,18.6741)(38.4375,18.2076)(40.4688,17.7374)(42.5,17.2673)
		\drawline(42.5,31.4479)(44.5312,30.3968)(46.5625,29.6384)(48.5938,29.087)(50.625,28.6885)(52.6562,28.3923)(54.6875,28.1533)(56.7188,27.9362)(58.75,27.7173)(60.7812,27.4835)(62.8125,27.2294)(64.8438,26.9547)(66.875,26.6619)(68.9062,26.3545)(70.9375,26.0359)(72.9688,25.7088)(75,25.3747)
		\drawline(75,32.3382)(77.0312,31.0453)(79.0625,29.6026)(81.0938,28.0919)(83.125,26.6149)(85.1562,25.2374)(87.1875,24.0354)(89.2188,23.0898)(91.25,22.4229)(93.2812,21.8944)(95.3125,21.4818)(97.3438,21.1178)(99.375,20.7725)(101.406,20.434)(103.438,20.094)(105.469,19.7503)(107.5,19.4043)
		\drawline(107.5,32.6991)(109.531,31.2724)(111.562,30.1256)(113.594,29.2062)(115.625,28.4952)(117.656,28.01)(119.688,27.7272)(121.719,27.5591)(123.75,27.4139)(125.781,27.2265)(127.812,26.9612)(129.844,26.6115)(131.875,26.1911)(133.906,25.7271)(135.938,25.2544)(137.969,24.8185)(140,24.477)
		\drawline(140,34.1636)(142.031,32.9693)(144.062,31.9352)(146.094,31.0301)(148.125,30.2532)(150.156,29.6091)(152.188,29.0816)(154.219,28.6325)(156.25,28.2255)(158.281,27.8351)(160.312,27.4444)(162.344,27.0415)(164.375,26.6182)(166.406,26.1687)(168.438,25.6894)(170.469,25.1785)(172.5,24.6364)
		\drawline(172.5,36.2941)(174.531,36.3147)(176.562,36.1667)(178.594,35.8046)(180.625,35.2376)(182.656,34.5076)(184.688,33.6722)(186.719,32.794)(188.75,31.9317)(190.781,31.1326)(192.812,30.4317)(194.844,29.85)(196.875,29.3972)(198.906,29.0851)(200.938,28.9015)(202.969,28.7431)(205,28.5436)
		\drawline(205,25.5131)(207.031,25.12)(209.062,24.7164)(211.094,24.2053)(213.125,23.5944)(215.156,22.9325)(217.188,22.281)(219.219,21.6989)(221.25,21.2432)(223.281,20.9454)(225.312,20.7859)(227.344,20.6898)(229.375,20.5778)(231.406,20.4123)(233.438,20.1805)(235.469,19.8869)(237.5,19.5464)
		\drawline(237.5,35.1979)(239.531,32.8406)(241.562,30.9968)(243.594,29.4112)(245.625,28.0151)(247.656,26.7937)(249.688,25.7544)(251.719,24.9067)(253.75,24.2481)(255.781,23.7714)(257.812,23.3986)(259.844,23.0714)(261.875,22.7611)(263.906,22.4544)(265.938,22.1484)(267.969,21.847)(270,21.5583)
		\put(5,7){\tiny 0}
		\put(0,40){$\zeta$}
		\put(5,70){\tiny 3}
		\put(8,2){\tiny $0$}
		\put(37,2){\tiny $\pi/4$}
		\put(70,2){\tiny $\pi/2$}
		\put(100,2){\tiny $3\pi/4$}
		\put(138,2){\tiny $\pi$}
		\put(165,2){\tiny $5\pi/4$}
		\put(198,2){\tiny $3\pi/2$}
		\put(230,2){\tiny $7\pi/4$}
		\put(265,2){\tiny $2\pi$}
		\put(152,-7){$\tau$}
		\put(175,75){\makebox(96,4)[r]{$R=2000,\,\alpha=15,\,n=0$}}
	\end{picture}
	\caption{\label{fig:awp:sym}Die Entwicklung symmetrischer St\"orungen als L\"osung des Anfangswertproblems.}
\end{figure}

\begin{figure} % Stoerungsenergie n=1
	%\input{/home/elmar/src/floquet/integrator/dat/resrand/2500-25-1.epic}
	\begin{picture}(275,78)
		\thinlines
		\drawline(10,8)(10,73)
		\drawline(42.5,8)(42.5,73)
		\drawline(75,8)(75,73)
		\drawline(107.5,8)(107.5,73)
		\drawline(140,8)(140,73)
		\drawline(172.5,8)(172.5,73)
		\drawline(205,8)(205,73)
		\drawline(237.5,8)(237.5,73)
		\drawline(270,8)(270,73)
		\drawline(10,8)(270,8)
		\drawline(10,73)(270,73)
		\thicklines
		\drawline(10,16.7514)(12.0312,18.2969)(14.0625,21.1776)(16.0938,24.0966)(18.125,27.2876)(20.1562,30.1864)(22.1875,32.1355)(24.2188,33.2456)(26.25,33.7395)(28.2812,33.6844)(30.3125,33.0349)(32.3438,31.7981)(34.375,30.1696)(36.4062,28.4662)(38.4375,26.9979)(40.4688,25.9923)(42.5,25.5625)
		\drawline(42.5,16.261)(44.5312,15.5692)(46.5625,15.958)(48.5938,15.6283)(50.625,15.4101)(52.6562,15.5162)(54.6875,16.0553)(56.7188,16.661)(58.75,17.2334)(60.7812,17.7971)(62.8125,18.4922)(64.8438,19.1012)(66.875,19.6295)(68.9062,20.1983)(70.9375,20.6356)(72.9688,21.0172)(75,21.2881)
		\drawline(75,14.7902)(77.0312,16.9407)(79.0625,18.3073)(81.0938,17.2847)(83.125,15.1724)(85.1562,14.5346)(87.1875,13.4433)(89.2188,12.5432)(91.25,11.8974)(93.2812,11.191)(95.3125,10.6929)(97.3438,10.1808)(99.375,9.78261)(101.406,9.42004)(103.438,9.12409)(105.469,8.88581)(107.5,8.69356)
		\drawline(107.5,17.2833)(109.531,17.2385)(111.562,20.4348)(113.594,24.095)(115.625,23.5102)(117.656,21.6183)(119.688,20.5674)(121.719,19.3685)(123.75,18.4729)(125.781,18.4032)(127.812,17.8496)(129.844,17.5729)(131.875,18.1083)(133.906,18.5139)(135.938,18.7792)(137.969,19.5939)(140,20.8168)
		\drawline(140,15.5849)(142.031,17.0443)(144.062,21.6299)(146.094,26.2887)(148.125,30.3199)(150.156,33.6379)(152.188,36.2989)(154.219,38.3308)(156.25,39.6813)(158.281,40.2517)(160.312,39.9699)(162.344,38.8722)(164.375,37.1644)(166.406,35.1983)(168.438,33.3541)(170.469,31.9619)(172.5,31.237)
		\drawline(172.5,15.9467)(174.531,15.6495)(176.562,15.6647)(178.594,15.7314)(180.625,15.4887)(182.656,15.0967)(184.688,14.9201)(186.719,14.9293)(188.75,14.9005)(190.781,14.9519)(192.812,15.1138)(194.844,15.2873)(196.875,15.4795)(198.906,15.7061)(200.938,15.9015)(202.969,16.0753)(205,16.1962)
		\drawline(205,18.1416)(207.031,17.3678)(209.062,19.1496)(211.094,17.624)(213.125,15.8125)(215.156,15.0517)(217.188,15.208)(219.219,14.9653)(221.25,14.3621)(223.281,14.086)(225.312,13.4184)(227.344,13.0327)(229.375,12.4614)(231.406,12.0186)(233.438,11.5534)(235.469,11.1135)(237.5,10.7484)
		\drawline(237.5,14.0288)(239.531,15.784)(241.562,19.3751)(243.594,20.0548)(245.625,18.2682)(247.656,16.1949)(249.688,14.8419)(251.719,14.1485)(253.75,13.3453)(255.781,12.9892)(257.812,12.7134)(259.844,12.5873)(261.875,12.7801)(263.906,12.9262)(265.938,13.1037)(267.969,13.543)(270,14.1041)
		\drawline(10,17.0354)(12.0312,19.1039)(14.0625,23.8835)(16.0938,27.7714)(18.125,30.189)(20.1562,30.9583)(22.1875,30.5695)(24.2188,29.4584)(26.25,28.1144)(28.2812,26.9448)(30.3125,26.0223)(32.3438,25.0695)(34.375,24.0786)(36.4062,23.1514)(38.4375,22.3946)(40.4688,21.902)(42.5,21.7416)
		\drawline(42.5,15.4673)(44.5312,15.6455)(46.5625,16.3832)(48.5938,17.3513)(50.625,18.4644)(52.6562,19.304)(54.6875,19.9316)(56.7188,20.6311)(58.75,21.3334)(60.7812,21.9508)(62.8125,22.577)(64.8438,23.212)(66.875,23.7978)(68.9062,24.3587)(70.9375,24.8707)(72.9688,25.2812)(75,25.577)
		\drawline(75,17.4709)(77.0312,16.5556)(79.0625,19.0716)(81.0938,18.5043)(83.125,16.5258)(85.1562,14.4235)(87.1875,13.7256)(89.2188,12.5298)(91.25,12.0559)(93.2812,11.3792)(95.3125,11.0129)(97.3438,10.5331)(99.375,10.2035)(101.406,9.8413)(103.438,9.55365)(105.469,9.28685)(107.5,9.06372)
		\drawline(107.5,15.7082)(109.531,16.0539)(111.562,17.8567)(113.594,21.9516)(115.625,23.8137)(117.656,22.2275)(119.688,19.4498)(121.719,17.4191)(123.75,16.3911)(125.781,15.4302)(127.812,15.1266)(129.844,14.6477)(131.875,14.5171)(133.906,14.975)(135.938,15.1703)(137.969,15.3178)(140,15.9829)
		\drawline(140,15.0337)(142.031,12.8414)(144.062,12.3299)(146.094,12.4169)(148.125,12.5305)(150.156,12.7955)(152.188,13.1449)(154.219,13.3226)(156.25,13.5145)(158.281,13.7026)(160.312,13.6852)(162.344,13.4435)(164.375,13.0349)(166.406,12.5587)(168.438,12.1215)(170.469,11.8043)(172.5,11.6386)
		\drawline(172.5,15.5113)(174.531,16.9885)(176.562,18.8141)(178.594,20.1792)(180.625,21.3594)(182.656,22.3302)(184.688,23.5557)(186.719,24.8378)(188.75,25.909)(190.781,26.9549)(192.812,28.0469)(194.844,29.0237)(196.875,29.9378)(198.906,30.8314)(200.938,31.5661)(202.969,32.1919)(205,32.6144)
		\drawline(205,17.9487)(207.031,19.327)(209.062,24.5288)(211.094,24.521)(213.125,21.2506)(215.156,16.8685)(217.188,15.4157)(219.219,13.4425)(221.25,12.4489)(223.281,11.4133)(225.312,10.846)(227.344,10.2578)(229.375,9.96141)(231.406,9.64345)(233.438,9.47827)(235.469,9.28475)(237.5,9.16596)
		\drawline(237.5,17.2241)(239.531,19.5291)(241.562,30.1623)(243.594,33.4728)(245.625,29.8494)(247.656,25.3159)(249.688,21.7375)(251.719,20.2147)(253.75,19.0689)(255.781,18.6963)(257.812,18.0969)(259.844,18.0504)(261.875,18.5279)(263.906,18.7802)(265.938,19.1967)(267.969,20.2112)(270,21.4452)
		\drawline(10,17.2315)(12.0312,19.6636)(14.0625,21.6923)(16.0938,23.1206)(18.125,24.2618)(20.1562,25.156)(22.1875,25.8459)(24.2188,26.2906)(26.25,26.5838)(28.2812,26.7106)(30.3125,26.5838)(32.3438,26.127)(34.375,25.4152)(36.4062,24.5485)(38.4375,23.6553)(40.4688,22.9121)(42.5,22.4747)
		\drawline(42.5,15.5189)(44.5312,14.5759)(46.5625,13.8447)(48.5938,13.6113)(50.625,13.3574)(52.6562,12.9019)(54.6875,12.8209)(56.7188,12.7598)(58.75,12.6624)(60.7812,12.8075)(62.8125,12.9034)(64.8438,12.9376)(66.875,13.0685)(68.9062,13.1791)(70.9375,13.2485)(72.9688,13.3412)(75,13.3722)
		\drawline(75,15.0395)(77.0312,15.6196)(79.0625,17.113)(81.0938,16.1662)(83.125,13.6632)(85.1562,12.3309)(87.1875,11.4508)(89.2188,10.5907)(91.25,10.1443)(93.2812,9.78638)(95.3125,9.44942)(97.3438,9.34495)(99.375,9.168)(101.406,9.11032)(103.438,8.99804)(105.469,8.906)(107.5,8.81927)
		\drawline(107.5,17.5887)(109.531,17.165)(111.562,17.756)(113.594,19.0335)(115.625,18.1496)(117.656,15.5926)(119.688,13.4906)(121.719,13.0986)(123.75,12.6171)(125.781,12.3481)(127.812,12.4533)(129.844,12.553)(131.875,12.5523)(133.906,12.7433)(135.938,13.1344)(137.969,13.569)(140,14.0148)
		\drawline(140,16.61)(142.031,22.5176)(144.062,31.6652)(146.094,41.0688)(148.125,49.8595)(150.156,57.4577)(152.188,63.5333)(154.219,67.9563)(156.25,70.6629)(158.281,71.5809)(160.312,70.6925)(162.344,68.1827)(164.375,64.5529)(166.406,60.5425)(168.438,56.8937)(170.469,54.2125)(172.5,52.8755)
		\drawline(172.5,15.8707)(174.531,14.5114)(176.562,14.3426)(178.594,15.1429)(180.625,15.583)(182.656,15.5304)(184.688,15.6058)(186.719,16.1031)(188.75,16.5256)(190.781,16.9713)(192.812,17.5918)(194.844,18.2701)(196.875,18.8588)(198.906,19.4104)(200.938,19.9439)(202.969,20.3417)(205,20.6734)
		\drawline(205,15.9345)(207.031,18.0053)(209.062,19.3028)(211.094,18.1)(213.125,14.8195)(215.156,12.6115)(217.188,11.7045)(219.219,11.2106)(221.25,10.7457)(223.281,10.5536)(225.312,10.2803)(227.344,10.0464)(229.375,9.85458)(231.406,9.61985)(233.438,9.47126)(235.469,9.27325)(237.5,9.13333)
		\drawline(237.5,18.2389)(239.531,28.6585)(241.562,40.6026)(243.594,47.5704)(245.625,44.5206)(247.656,39.605)(249.688,37.069)(251.719,34.3056)(253.75,32.5175)(255.781,32.4672)(257.812,31.5863)(259.844,31.2718)(261.875,32.4928)(263.906,33.5703)(265.938,34.5596)(267.969,36.6856)(270,39.5745)
		\drawline(10,16.7286)(12.0312,21.5879)(14.0625,27.9816)(16.0938,34.2736)(18.125,39.6226)(20.1562,43.7668)(22.1875,46.7777)(24.2188,48.8309)(26.25,50.0559)(28.2812,50.4229)(30.3125,49.8142)(32.3438,48.2251)(34.375,45.9069)(36.4062,43.3143)(38.4375,40.9369)(40.4688,39.1904)(42.5,38.333)
		\drawline(42.5,17.3103)(44.5312,17.082)(46.5625,18.0118)(48.5938,18.2814)(50.625,17.7794)(52.6562,16.848)(54.6875,16.7862)(56.7188,17.0302)(58.75,16.9209)(60.7812,16.9816)(62.8125,17.275)(64.8438,17.3974)(66.875,17.5975)(68.9062,17.8561)(70.9375,17.9982)(72.9688,18.1764)(75,18.25)
		\drawline(75,17.9899)(77.0312,18.2811)(79.0625,19.3797)(81.0938,17.9514)(83.125,15.4894)(85.1562,13.7894)(87.1875,12.9325)(89.2188,12.1151)(91.25,11.5796)(93.2812,11.1723)(95.3125,10.77)(97.3438,10.4674)(99.375,10.2242)(101.406,10.0008)(103.438,9.87943)(105.469,9.69798)(107.5,9.61206)
		\drawline(107.5,14.3275)(109.531,20.5751)(111.562,27.605)(113.594,28.7343)(115.625,24.7755)(117.656,21.2824)(119.688,19.2726)(121.719,18.2788)(123.75,17.526)(125.781,17.4645)(127.812,17.1866)(129.844,17.0433)(131.875,17.5037)(133.906,17.9585)(135.938,18.3718)(137.969,19.1872)(140,20.3)
		\drawline(140,14.8147)(142.031,16.6601)(144.062,21.6032)(146.094,26.4795)(148.125,30.605)(150.156,33.7789)(152.188,35.97)(154.219,37.3161)(156.25,38.0212)(158.281,38.1744)(160.312,37.6912)(162.344,36.5247)(164.375,34.8436)(166.406,32.9769)(168.438,31.2805)(170.469,30.0501)(172.5,29.4635)
		\drawline(172.5,19.6836)(174.531,20.3114)(176.562,20.8259)(178.594,20.9126)(180.625,21.1098)(182.656,21.8289)(184.688,22.5582)(186.719,23.1826)(188.75,23.827)(190.781,24.47)(192.812,25.0854)(194.844,25.6968)(196.875,26.2983)(198.906,26.8562)(200.938,27.3651)(202.969,27.7778)(205,28.0539)
		\drawline(205,17.7833)(207.031,21.6404)(209.062,22.4942)(211.094,19.2494)(213.125,16.0637)(215.156,14.7333)(217.188,14.6296)(219.219,14.074)(221.25,13.6249)(223.281,13.2901)(225.312,12.824)(227.344,12.5394)(229.375,12.0531)(231.406,11.7689)(233.438,11.3121)(235.469,11.0175)(237.5,10.6347)
		\drawline(237.5,16.0915)(239.531,16.0383)(241.562,23.2888)(243.594,28.2522)(245.625,26.4433)(247.656,23.1056)(249.688,19.4166)(251.719,17.8848)(253.75,16.8206)(255.781,16.321)(257.812,15.877)(259.844,15.7096)(261.875,16.0879)(263.906,16.3628)(265.938,16.6135)(267.969,17.3241)(270,18.2998)
		\drawline(10,14.6057)(12.0312,15.8178)(14.0625,19.3378)(16.0938,23.6333)(18.125,27.038)(20.1562,29.2115)(22.1875,30.4251)(24.2188,31.0796)(26.25,31.4423)(28.2812,31.5569)(30.3125,31.3096)(32.3438,30.6273)(34.375,29.539)(36.4062,28.235)(38.4375,26.9742)(40.4688,26.0086)(42.5,25.5097)
		\drawline(42.5,18.8016)(44.5312,17.2435)(46.5625,16.6399)(48.5938,16.0582)(50.625,15.4533)(52.6562,15.0686)(54.6875,15.1056)(56.7188,15.4298)(58.75,15.8134)(60.7812,16.2734)(62.8125,16.8612)(64.8438,17.4559)(66.875,18.0041)(68.9062,18.5026)(70.9375,18.9396)(72.9688,19.2859)(75,19.5376)
		\drawline(75,15.7687)(77.0312,19.7744)(79.0625,21.6962)(81.0938,20.3278)(83.125,16.6961)(85.1562,14.8582)(87.1875,14.1971)(89.2188,13.9984)(91.25,13.3211)(93.2812,12.9226)(95.3125,12.4357)(97.3438,11.9028)(99.375,11.5232)(101.406,11.0293)(103.438,10.7008)(105.469,10.2963)(107.5,9.99976)
		\drawline(107.5,17.5023)(109.531,19.2431)(111.562,21.5576)(113.594,21.4346)(115.625,20.1771)(117.656,19.046)(119.688,18.006)(121.719,17.4888)(123.75,16.6142)(125.781,15.7414)(127.812,15.5196)(129.844,15.1717)(131.875,14.692)(133.906,14.8515)(135.938,15.4145)(137.969,15.6182)(140,15.8226)
		\drawline(140,13.8603)(142.031,14.2052)(144.062,15.8999)(146.094,18.5531)(148.125,21.3351)(150.156,23.604)(152.188,25.2181)(154.219,26.1815)(156.25,26.5428)(158.281,26.3879)(160.312,25.7852)(162.344,24.8215)(164.375,23.6518)(166.406,22.481)(168.438,21.49)(170.469,20.8086)(172.5,20.5118)
		\drawline(172.5,14.8894)(174.531,13.7695)(176.562,13.4411)(178.594,13.4136)(180.625,13.4342)(182.656,13.4945)(184.688,13.5538)(186.719,13.8087)(188.75,14.1234)(190.781,14.3373)(192.812,14.5578)(194.844,14.8185)(196.875,15.0149)(198.906,15.2089)(200.938,15.3814)(202.969,15.5028)(205,15.5876)
		\drawline(205,16.5995)(207.031,13.9218)(209.062,12.8544)(211.094,12.8823)(213.125,11.7609)(215.156,10.8905)(217.188,10.3299)(219.219,9.85508)(221.25,9.51585)(223.281,9.26918)(225.312,9.04548)(227.344,8.88758)(229.375,8.73584)(231.406,8.63683)(233.438,8.53335)(235.469,8.46542)(237.5,8.39841)
		\drawline(237.5,16.5418)(239.531,16.6136)(241.562,16.5251)(243.594,15.3674)(245.625,13.1892)(247.656,11.75)(249.688,11.2371)(251.719,10.894)(253.75,10.6308)(255.781,10.5224)(257.812,10.3515)(259.844,10.1001)(261.875,9.99001)(263.906,10.0368)(265.938,9.92722)(267.969,9.85414)(270,10.0514)
		\drawline(10,15.6406)(12.0312,15.7863)(14.0625,17.8716)(16.0938,20.3609)(18.125,22.8938)(20.1562,25.2171)(22.1875,27.2476)(24.2188,28.9129)(26.25,30.0932)(28.2812,30.6705)(30.3125,30.5807)(32.3438,29.8581)(34.375,28.6731)(36.4062,27.2937)(38.4375,25.999)(40.4688,25.0239)(42.5,24.5164)
		\drawline(42.5,15.5792)(44.5312,15.2679)(46.5625,15.5936)(48.5938,16.2608)(50.625,16.7844)(52.6562,16.8615)(54.6875,17.0978)(56.7188,17.7998)(58.75,18.2638)(60.7812,18.6208)(62.8125,19.19)(64.8438,19.6262)(66.875,19.9772)(68.9062,20.4027)(70.9375,20.6807)(72.9688,20.9273)(75,21.0755)
		\drawline(75,14.733)(77.0312,16.3392)(79.0625,16.8217)(81.0938,15.3243)(83.125,13.8594)(85.1562,12.9199)(87.1875,12.741)(89.2188,12.3408)(91.25,12.2003)(93.2812,11.8124)(95.3125,11.6107)(97.3438,11.2769)(99.375,10.9987)(101.406,10.7193)(103.438,10.4151)(105.469,10.1837)(107.5,9.89558)
		\drawline(107.5,18.8663)(109.531,25.3809)(111.562,33.7618)(113.594,33.8744)(115.625,29.9844)(117.656,25.1313)(119.688,22.7618)(121.719,21.3892)(123.75,21.1585)(125.781,20.3424)(127.812,19.9187)(129.844,20.3757)(131.875,20.5052)(133.906,20.6721)(135.938,21.6598)(137.969,22.9029)(140,24.1098)
		\drawline(140,14.6751)(142.031,13.7025)(144.062,15.9529)(146.094,19.779)(148.125,23.234)(150.156,25.6994)(152.188,27.3043)(154.219,28.3059)(156.25,28.8835)(158.281,28.9731)(160.312,28.527)(162.344,27.6074)(164.375,26.3919)(166.406,25.1263)(168.438,24.0314)(170.469,23.2673)(172.5,22.9254)
		\drawline(172.5,18.2566)(174.531,15.6041)(176.562,14.2824)(178.594,14.203)(180.625,14.5125)(182.656,14.5137)(184.688,14.9076)(186.719,15.4604)(188.75,15.7942)(190.781,16.2472)(192.812,16.7878)(194.844,17.1645)(196.875,17.5897)(198.906,17.996)(200.938,18.2909)(202.969,18.5874)(205,18.7594)
		\drawline(205,14.9967)(207.031,17.788)(209.062,18.256)(211.094,17.6262)(213.125,16.9828)(215.156,16.3809)(217.188,15.2)(219.219,14.3388)(221.25,13.5406)(223.281,12.7586)(225.312,12.2144)(227.344,11.6133)(229.375,11.2369)(231.406,10.7909)(233.438,10.5242)(235.469,10.194)(237.5,9.98672)
		\drawline(237.5,18.6086)(239.531,28.3085)(241.562,32.5343)(243.594,31.8878)(245.625,29.8374)(247.656,27.9542)(249.688,25.7077)(251.719,24.3294)(253.75,23.0015)(255.781,21.7915)(257.812,21.852)(259.844,21.5641)(261.875,21.1865)(263.906,21.9647)(265.938,23.0729)(267.969,23.8593)(270,24.8579)
		\drawline(10,17.8286)(12.0312,14.8596)(14.0625,18.0644)(16.0938,23.2976)(18.125,28.4365)(20.1562,32.5559)(22.1875,35.4483)(24.2188,37.2581)(26.25,38.1983)(28.2812,38.396)(30.3125,37.8912)(32.3438,36.7535)(34.375,35.1502)(36.4062,33.3538)(38.4375,31.6786)(40.4688,30.4156)(42.5,29.767)
		\drawline(42.5,17.5747)(44.5312,18.9854)(46.5625,20.5429)(48.5938,20.3224)(50.625,19.6004)(52.6562,19.1278)(54.6875,19.4897)(56.7188,20.5539)(58.75,21.396)(60.7812,22.0566)(62.8125,23.0261)(64.8438,23.8802)(66.875,24.5194)(68.9062,25.2817)(70.9375,25.8412)(72.9688,26.3011)(75,26.6429)
		\drawline(75,15.8533)(77.0312,19.1266)(79.0625,21.2055)(81.0938,19.8762)(83.125,16.0666)(85.1562,13.9268)(87.1875,14.2637)(89.2188,13.9908)(91.25,13.5202)(93.2812,13.2137)(95.3125,12.7142)(97.3438,12.265)(99.375,11.8598)(101.406,11.4051)(103.438,11.0894)(105.469,10.6874)(107.5,10.4363)
		\drawline(107.5,16.1804)(109.531,22.5917)(111.562,27.2714)(113.594,27.6035)(115.625,25.5764)(117.656,22.6464)(119.688,21.8165)(121.719,21.1784)(123.75,20.1183)(125.781,19.9856)(127.812,20.029)(129.844,19.8361)(131.875,20.2098)(133.906,20.9741)(135.938,21.7309)(137.969,22.6557)(140,23.9999)
		\drawline(140,18.1377)(142.031,25.2154)(144.062,31.9576)(146.094,36.5753)(148.125,39.0233)(150.156,40.1475)(152.188,41.1102)(154.219,42.6498)(156.25,44.169)(158.281,45.1263)(160.312,45.1877)(162.344,44.2688)(164.375,42.5525)(166.406,40.4343)(168.438,38.3668)(170.469,36.7579)(172.5,35.8831)
		\drawline(172.5,16.8525)(174.531,17.4024)(176.562,17.647)(178.594,17.374)(180.625,16.7881)(182.656,16.9214)(184.688,17.5328)(186.719,17.881)(188.75,18.2573)(190.781,18.8077)(192.812,19.2294)(194.844,19.6211)(196.875,20.0644)(198.906,20.3995)(200.938,20.7189)(202.969,20.9641)(205,21.1138)
		\drawline(205,15.4076)(207.031,17.2912)(209.062,18.5131)(211.094,16.8204)(213.125,15.0637)(215.156,14.1278)(217.188,13.6935)(219.219,12.9104)(221.25,12.6612)(223.281,12.0934)(225.312,11.8515)(227.344,11.4005)(229.375,11.1353)(231.406,10.7771)(233.438,10.4986)(235.469,10.218)(237.5,9.94271)
		\drawline(237.5,15.5872)(239.531,16.6696)(241.562,16.4473)(243.594,17.9128)(245.625,18.7954)(247.656,19.1693)(249.688,17.9352)(251.719,16.8572)(253.75,16.8938)(255.781,16.5515)(257.812,16.0893)(259.844,16.3317)(261.875,16.585)(263.906,16.6858)(265.938,17.2223)(267.969,18.0916)(270,19.007)
		\drawline(10,14.3786)(12.0312,17.2818)(14.0625,21.6375)(16.0938,26.2209)(18.125,31.1044)(20.1562,35.9389)(22.1875,40.158)(24.2188,43.4271)(26.25,45.5369)(28.2812,46.3892)(30.3125,46.0005)(32.3438,44.5364)(34.375,42.3433)(36.4062,39.8934)(38.4375,37.6456)(40.4688,35.9705)(42.5,35.1027)
		\drawline(42.5,18.1223)(44.5312,16.793)(46.5625,16.3257)(48.5938,16.6076)(50.625,16.3231)(52.6562,15.8453)(54.6875,15.8024)(56.7188,16.1999)(58.75,16.6479)(60.7812,17.071)(62.8125,17.5935)(64.8438,18.1948)(66.875,18.8122)(68.9062,19.3404)(70.9375,19.8567)(72.9688,20.266)(75,20.5773)
		\drawline(75,19.1196)(77.0312,20.1212)(79.0625,18.3329)(81.0938,15.5185)(83.125,13.8112)(85.1562,13.4744)(87.1875,13.292)(89.2188,12.8649)(91.25,12.3129)(93.2812,12.016)(95.3125,11.5286)(97.3438,11.3026)(99.375,10.9116)(101.406,10.7144)(103.438,10.3967)(105.469,10.2013)(107.5,9.94877)
		\drawline(107.5,14.7174)(109.531,16.0293)(111.562,21.2656)(113.594,22.2737)(115.625,20.6642)(117.656,18.57)(119.688,17.1892)(121.719,16.6454)(123.75,16.1856)(125.781,15.6822)(127.812,15.8283)(129.844,15.9484)(131.875,15.911)(133.906,16.2804)(135.938,16.9546)(137.969,17.6522)(140,18.394)
		\drawline(140,15.6891)(142.031,19.7279)(144.062,25.3885)(146.094,30.3416)(148.125,34.2022)(150.156,37.1211)(152.188,39.3703)(154.219,41.0979)(156.25,42.2516)(158.281,42.6767)(160.312,42.2483)(162.344,40.9862)(164.375,39.1163)(166.406,37.0181)(168.438,35.0886)(170.469,33.6606)(172.5,32.9456)
		\drawline(172.5,16.6137)(174.531,14.8812)(176.562,14.1301)(178.594,13.7946)(180.625,13.5154)(182.656,13.3972)(184.688,13.5204)(186.719,13.7339)(188.75,14.0035)(190.781,14.2982)(192.812,14.6154)(194.844,15.0142)(196.875,15.4755)(198.906,15.9206)(200.938,16.3203)(202.969,16.6545)(205,16.9144)
		\drawline(205,16.4378)(207.031,19.2879)(209.062,19.7131)(211.094,18.37)(213.125,16.3004)(215.156,15.3644)(217.188,14.5517)(219.219,13.7996)(221.25,13.318)(223.281,12.6921)(225.312,12.3163)(227.344,11.7666)(229.375,11.4462)(231.406,10.9938)(233.438,10.7148)(235.469,10.3608)(237.5,10.1095)
		\drawline(237.5,17.6095)(239.531,27.19)(241.562,35.3538)(243.594,33.9928)(245.625,29.8984)(247.656,26.2033)(249.688,24.1058)(251.719,22.4429)(253.75,21.8829)(255.781,21.3047)(257.812,20.7886)(259.844,21.142)(261.875,21.5221)(263.906,21.8399)(265.938,22.7759)(267.969,24.1041)(270,25.5616)
		\drawline(10,16.4523)(12.0312,17.6665)(14.0625,22.1357)(16.0938,28.5015)(18.125,34.4523)(20.1562,39.0857)(22.1875,42.2619)(24.2188,44.2031)(26.25,45.183)(28.2812,45.3461)(30.3125,44.7043)(32.3438,43.304)(34.375,41.3319)(36.4062,39.1347)(38.4375,37.1038)(40.4688,35.5935)(42.5,34.8409)
		\drawline(42.5,16.7756)(44.5312,17.434)(46.5625,18.2468)(48.5938,17.9797)(50.625,17.2749)(52.6562,16.8366)(54.6875,17.0485)(56.7188,17.4423)(58.75,17.7797)(60.7812,18.1761)(62.8125,18.7033)(64.8438,19.1859)(66.875,19.6096)(68.9062,20.067)(70.9375,20.4359)(72.9688,20.7342)(75,20.9422)
		\drawline(75,17.7059)(77.0312,15.4008)(79.0625,15.3724)(81.0938,15.4478)(83.125,15.4182)(85.1562,14.5915)(87.1875,13.3447)(89.2188,12.635)(91.25,11.9605)(93.2812,11.4648)(95.3125,11.0628)(97.3438,10.7022)(99.375,10.4119)(101.406,10.1146)(103.438,9.88837)(105.469,9.63952)(107.5,9.4586)
		\drawline(107.5,16.9222)(109.531,20.6102)(111.562,22.8047)(113.594,22.6319)(115.625,21.608)(117.656,19.6646)(119.688,19.0581)(121.719,18.8847)(123.75,17.8823)(125.781,17.3042)(127.812,17.397)(129.844,17.0533)(131.875,17.0139)(133.906,17.669)(135.938,18.2472)(137.969,18.7135)(140,19.6132)
		\drawline(140,17.6044)(142.031,22.4539)(144.062,29.1294)(146.094,36.4644)(148.125,43.6675)(150.156,49.832)(152.188,54.4538)(154.219,57.5489)(156.25,59.2662)(158.281,59.6549)(160.312,58.6814)(162.344,56.4587)(164.375,53.3829)(166.406,50.0622)(168.438,47.105)(170.469,44.9901)(172.5,43.9952)
		\drawline(172.5,15.387)(174.531,14.0295)(176.562,14.1096)(178.594,13.7337)(180.625,12.7277)(182.656,11.6323)(184.688,11.2308)(186.719,10.9626)(188.75,10.5243)(190.781,10.3083)(192.812,10.2187)(194.844,10.0245)(196.875,9.92564)(198.906,9.88848)(200.938,9.81284)(202.969,9.79863)(205,9.77852)
		\drawline(205,17.5489)(207.031,16.9893)(209.062,18.8104)(211.094,19.5981)(213.125,17.7881)(215.156,15.154)(217.188,13.923)(219.219,12.9311)(221.25,12.3471)(223.281,11.8217)(225.312,11.45)(227.344,11.0707)(229.375,10.7183)(231.406,10.4277)(233.438,10.1105)(235.469,9.88399)(237.5,9.61792)
		\drawline(237.5,15.0921)(239.531,21.0973)(241.562,27.0977)(243.594,26.7532)(245.625,24.3532)(247.656,21.5366)(249.688,19.9065)(251.719,18.7713)(253.75,18.4893)(255.781,17.6602)(257.812,17.1562)(259.844,17.4801)(261.875,17.3625)(263.906,17.3163)(265.938,18.0867)(267.969,19.0189)(270,19.7818)
		\put(5,7){\tiny 0}
		\put(0,40){$\zeta$}
		\put(2,70){\tiny 10}
		\put(175,75){\makebox(96,4)[r]{$R=2500,\,\alpha=5,\,n=1$}}
	\end{picture}
	%\input{/home/elmar/src/floquet/integrator/dat/resrand/1500-25-1.epic}
	\begin{picture}(275,78)
		\thinlines
		\drawline(10,8)(10,73)
		\drawline(42.5,8)(42.5,73)
		\drawline(75,8)(75,73)
		\drawline(107.5,8)(107.5,73)
		\drawline(140,8)(140,73)
		\drawline(172.5,8)(172.5,73)
		\drawline(205,8)(205,73)
		\drawline(237.5,8)(237.5,73)
		\drawline(270,8)(270,73)
		\drawline(10,8)(270,8)
		\drawline(10,73)(270,73)
		\thicklines
		\drawline(10,18.481)(12.0312,20.4885)(14.0625,24.0924)(16.0938,28.3253)(18.125,32.5184)(20.1562,36.2448)(22.1875,39.3218)(24.2188,41.7085)(26.25,43.4378)(28.2812,44.5833)(30.3125,45.1391)(32.3438,45.0706)(34.375,44.4055)(36.4062,43.2193)(38.4375,41.6356)(40.4688,39.8115)(42.5,37.9072)
		\drawline(42.5,20.939)(44.5312,19.8323)(46.5625,19.2968)(48.5938,19.6095)(50.625,19.8629)(52.6562,19.7633)(54.6875,19.2362)(56.7188,18.2139)(58.75,16.9293)(60.7812,15.8244)(62.8125,15.4834)(64.8438,15.4683)(66.875,15.1704)(68.9062,14.6426)(70.9375,14.1438)(72.9688,13.8819)(75,13.6276)
		\drawline(75,19.2908)(77.0312,21.0804)(79.0625,21.0326)(81.0938,19.7767)(83.125,18.4603)(85.1562,17.4668)(87.1875,16.5949)(89.2188,15.2199)(91.25,13.9903)(93.2812,12.8215)(95.3125,12.3182)(97.3438,11.7993)(99.375,11.1982)(101.406,10.6214)(103.438,10.3495)(105.469,10.0822)(107.5,9.75671)
		\drawline(107.5,18.5723)(109.531,20.0706)(111.562,22.9282)(113.594,25.1878)(115.625,25.5446)(117.656,23.7969)(119.688,20.8149)(121.719,17.8245)(123.75,15.6157)(125.781,13.9735)(127.812,13.4592)(129.844,13.38)(131.875,13.0991)(133.906,12.7269)(135.938,12.4481)(137.969,12.4382)(140,12.574)
		\drawline(140,18.1836)(142.031,16.3573)(144.062,16.8213)(146.094,20.0764)(148.125,23.424)(150.156,26.2429)(152.188,28.4272)(154.219,29.9835)(156.25,30.9676)(158.281,31.4056)(160.312,31.3609)(162.344,30.9042)(164.375,30.105)(166.406,29.0381)(168.438,27.792)(170.469,26.4661)(172.5,25.1554)
		\drawline(172.5,17.4858)(174.531,16.472)(176.562,15.5049)(178.594,14.5147)(180.625,13.5615)(182.656,12.675)(184.688,11.9218)(186.719,11.5165)(188.75,11.1947)(190.781,10.7794)(192.812,10.3152)(194.844,9.9037)(196.875,9.73584)(198.906,9.68606)(200.938,9.58308)(202.969,9.41486)(205,9.24468)
		\drawline(205,17.5217)(207.031,16.3051)(209.062,16.2534)(211.094,15.9482)(213.125,14.97)(215.156,13.7073)(217.188,12.6233)(219.219,11.7523)(221.25,11.2194)(223.281,10.8084)(225.312,10.3471)(227.344,9.87464)(229.375,9.5313)(231.406,9.31889)(233.438,9.09229)(235.469,8.86702)(237.5,8.70498)
		\drawline(237.5,18.6363)(239.531,19.3391)(241.562,21.8724)(243.594,28.5162)(245.625,33.4747)(247.656,34.5565)(249.688,32.1438)(251.719,28.4537)(253.75,24.7868)(255.781,22.0738)(257.812,21.7083)(259.844,21.3467)(261.875,20.4568)(263.906,19.474)(265.938,18.9107)(267.969,19.1459)(270,19.6443)
		\drawline(10,20.5202)(12.0312,19.8627)(14.0625,20.5018)(16.0938,22.3154)(18.125,24.5171)(20.1562,26.6507)(22.1875,28.4238)(24.2188,29.8435)(26.25,30.955)(28.2812,31.6422)(30.3125,31.8755)(32.3438,31.6706)(34.375,31.0696)(36.4062,30.141)(38.4375,28.9789)(40.4688,27.6942)(42.5,26.3952)
		\drawline(42.5,19.1875)(44.5312,18.5531)(46.5625,18.1789)(48.5938,17.9054)(50.625,17.6272)(52.6562,17.2417)(54.6875,16.7778)(56.7188,16.2118)(58.75,15.557)(60.7812,15.1227)(62.8125,15.1282)(64.8438,15.1231)(66.875,14.9182)(68.9062,14.5422)(70.9375,14.1925)(72.9688,13.9952)(75,13.7396)
		\drawline(75,15.623)(77.0312,15.9659)(79.0625,17.172)(81.0938,17.8464)(83.125,17.1661)(85.1562,15.7516)(87.1875,14.5414)(89.2188,13.6788)(91.25,13.3095)(93.2812,12.6897)(95.3125,12.0143)(97.3438,11.331)(99.375,10.9268)(101.406,10.5591)(103.438,10.1896)(105.469,9.81931)(107.5,9.59444)
		\drawline(107.5,15.62)(109.531,14.1167)(111.562,16.1437)(113.594,20.3576)(115.625,23.0225)(117.656,23.5336)(119.688,22.6603)(121.719,21.1672)(123.75,20.1979)(125.781,19.7919)(127.812,19.1475)(129.844,18.3077)(131.875,17.6192)(133.906,17.4885)(135.938,17.7391)(137.969,18.0387)(140,18.2514)
		\drawline(140,18.4557)(142.031,15.5939)(144.062,14.4667)(146.094,14.7398)(148.125,15.7976)(150.156,17.0847)(152.188,18.4571)(154.219,20.082)(156.25,21.5618)(158.281,22.6668)(160.312,23.3996)(162.344,23.8245)(164.375,24.0335)(166.406,23.9841)(168.438,23.6895)(170.469,23.2007)(172.5,22.5836)
		\drawline(172.5,18.9635)(174.531,18.4858)(176.562,18.3432)(178.594,18.24)(180.625,17.914)(182.656,17.309)(184.688,16.4897)(186.719,15.8858)(188.75,15.6973)(190.781,15.7272)(192.812,15.6283)(194.844,15.2936)(196.875,14.8789)(198.906,14.6089)(200.938,14.4529)(202.969,14.1684)(205,13.7737)
		\drawline(205,20.6966)(207.031,19.4501)(209.062,21.3374)(211.094,23.006)(213.125,22.5573)(215.156,19.6063)(217.188,16.273)(219.219,14.4936)(221.25,14.0724)(223.281,13.3011)(225.312,12.337)(227.344,11.612)(229.375,11.21)(231.406,10.8273)(233.438,10.3965)(235.469,9.99677)(237.5,9.74787)
		\drawline(237.5,19.3811)(239.531,18.8163)(241.562,20.9042)(243.594,21.9427)(245.625,20.9951)(247.656,18.7756)(249.688,16.4513)(251.719,14.3847)(253.75,13.0909)(255.781,12.8259)(257.812,12.4978)(259.844,12.0664)(261.875,11.6761)(263.906,11.6478)(265.938,11.7319)(267.969,11.7491)(270,11.6866)
		\drawline(10,21.7099)(12.0312,24.6964)(14.0625,28.0501)(16.0938,30.7785)(18.125,33.0065)(20.1562,34.7612)(22.1875,36.0476)(24.2188,36.978)(26.25,37.6917)(28.2812,38.2029)(30.3125,38.3704)(32.3438,38.1234)(34.375,37.4488)(36.4062,36.3894)(38.4375,35.0376)(40.4688,33.5174)(42.5,31.9548)
		\drawline(42.5,19.2047)(44.5312,18.8971)(46.5625,18.4272)(48.5938,18.0731)(50.625,18.0504)(52.6562,18.0575)(54.6875,17.7402)(56.7188,17.0225)(58.75,16.1706)(60.7812,15.6234)(62.8125,15.5173)(64.8438,15.3944)(66.875,15.0654)(68.9062,14.665)(70.9375,14.3626)(72.9688,14.15)(75,13.8361)
		\drawline(75,17.2505)(77.0312,16.9296)(79.0625,16.6839)(81.0938,15.2799)(83.125,13.5111)(85.1562,12.1888)(87.1875,11.4201)(89.2188,11.2334)(91.25,10.7898)(93.2812,10.3343)(95.3125,9.87783)(97.3438,9.6501)(99.375,9.47515)(101.406,9.29052)(103.438,9.05371)(105.469,8.92475)(107.5,8.83892)
		\drawline(107.5,17.7849)(109.531,20.605)(111.562,23.9293)(113.594,24.8464)(115.625,23.2788)(117.656,21.1857)(119.688,19.4329)(121.719,18.8986)(123.75,19.324)(125.781,18.7666)(127.812,17.5323)(129.844,16.2229)(131.875,15.1945)(133.906,15.1975)(135.938,15.4574)(137.969,15.495)(140,15.2964)
		\drawline(140,21.0379)(142.031,19.1395)(144.062,20.4967)(146.094,24.2299)(148.125,28.4943)(150.156,32.298)(152.188,35.3951)(154.219,37.8175)(156.25,39.712)(158.281,40.934)(160.312,41.4332)(162.344,41.2464)(164.375,40.4486)(166.406,39.149)(168.438,37.4912)(170.469,35.6395)(172.5,33.7542)
		\drawline(172.5,18.5389)(174.531,18.8012)(176.562,19.2315)(178.594,19.1473)(180.625,18.397)(182.656,17.2402)(184.688,16.0541)(186.719,14.9724)(188.75,14.0693)(190.781,13.5057)(192.812,13.4758)(194.844,13.4441)(196.875,13.2512)(198.906,12.9101)(200.938,12.5888)(202.969,12.427)(205,12.2595)
		\drawline(205,20.6239)(207.031,21.2334)(209.062,21.4597)(211.094,19.7809)(213.125,17.42)(215.156,15.7834)(217.188,14.2879)(219.219,13.309)(221.25,12.6043)(223.281,12.5657)(225.312,12.2857)(227.344,11.7787)(229.375,11.1099)(231.406,10.7163)(233.438,10.4454)(235.469,10.1111)(237.5,9.73002)
		\drawline(237.5,19.2354)(239.531,17.812)(241.562,18.3613)(243.594,19.2037)(245.625,19.7522)(247.656,19.1174)(249.688,17.6518)(251.719,15.9967)(253.75,14.6239)(255.781,13.8209)(257.812,13.4507)(259.844,13.3631)(261.875,13.3917)(263.906,13.407)(265.938,13.386)(267.969,13.3554)(270,13.3689)
		\drawline(10,18.3028)(12.0312,21.3854)(14.0625,26.6354)(16.0938,32.5229)(18.125,38.0913)(20.1562,42.9192)(22.1875,46.817)(24.2188,49.7393)(26.25,51.7265)(28.2812,52.8663)(30.3125,53.2173)(32.3438,52.832)(34.375,51.7841)(36.4062,50.179)(38.4375,48.1637)(40.4688,45.9168)(42.5,43.6172)
		\drawline(42.5,18.8798)(44.5312,17.4885)(46.5625,17.1522)(48.5938,17.1395)(50.625,16.8738)(52.6562,16.6332)(54.6875,16.6062)(56.7188,16.4351)(58.75,16.1388)(60.7812,16.0865)(62.8125,16.1242)(64.8438,16.0534)(66.875,15.8186)(68.9062,15.4969)(70.9375,15.2569)(72.9688,15.0802)(75,14.7753)
		\drawline(75,18.8643)(77.0312,17.5185)(79.0625,18.551)(81.0938,19.37)(83.125,18.3763)(85.1562,16.404)(87.1875,14.4232)(89.2188,13.0276)(91.25,12.2644)(93.2812,11.8646)(95.3125,11.7554)(97.3438,11.3343)(99.375,10.8709)(101.406,10.3943)(103.438,10.1536)(105.469,9.88062)(107.5,9.60263)
		\drawline(107.5,19.8097)(109.531,18.7328)(111.562,19.8826)(113.594,22.353)(115.625,23.9242)(117.656,24.5429)(119.688,24.0393)(121.719,23.3758)(123.75,22.3475)(125.781,20.8102)(127.812,19.3969)(129.844,18.4886)(131.875,18.5466)(133.906,18.952)(135.938,19.2163)(137.969,19.2095)(140,19.0238)
		\drawline(140,17.3169)(142.031,16.1364)(144.062,15.5396)(146.094,15.6512)(148.125,16.3696)(150.156,17.6329)(152.188,18.9691)(154.219,20.1842)(156.25,21.2303)(158.281,22.1588)(160.312,22.8771)(162.344,23.2787)(164.375,23.3512)(166.406,23.1213)(168.438,22.6443)(170.469,21.9944)(172.5,21.2515)
		\drawline(172.5,16.8438)(174.531,15.2994)(176.562,14.0243)(178.594,13.0535)(180.625,12.3821)(182.656,11.9206)(184.688,11.3881)(186.719,10.8757)(188.75,10.5664)(190.781,10.4018)(192.812,10.1862)(194.844,9.93265)(196.875,9.73262)(198.906,9.63944)(200.938,9.58243)(202.969,9.47277)(205,9.33844)
		\drawline(205,18.9048)(207.031,15.7951)(209.062,13.932)(211.094,12.5497)(213.125,12.1351)(215.156,12.2159)(217.188,12.0561)(219.219,11.6894)(221.25,11.2978)(223.281,10.9386)(225.312,10.65)(227.344,10.3393)(229.375,10.0218)(231.406,9.71994)(233.438,9.48725)(235.469,9.28112)(237.5,9.08795)
		\drawline(237.5,16.5982)(239.531,15.9913)(241.562,17.839)(243.594,24.0543)(245.625,29.5666)(247.656,31.7468)(249.688,30.4959)(251.719,27.5043)(253.75,24.5206)(255.781,22.169)(257.812,21.8616)(259.844,21.7624)(261.875,21.1593)(263.906,20.3227)(265.938,19.6915)(267.969,19.7163)(270,20.1336)
		\drawline(10,19.4952)(12.0312,22.723)(14.0625,28.7044)(16.0938,36.0166)(18.125,43.3884)(20.1562,49.9798)(22.1875,55.3671)(24.2188,59.4548)(26.25,62.3244)(28.2812,64.1436)(30.3125,64.9842)(32.3438,64.828)(34.375,63.7435)(36.4062,61.8592)(38.4375,59.3668)(40.4688,56.5073)(42.5,53.5276)
		\drawline(42.5,21.162)(44.5312,20.0967)(46.5625,19.281)(48.5938,18.9854)(50.625,19.1613)(52.6562,19.2788)(54.6875,19.0231)(56.7188,18.3638)(58.75,17.6652)(60.7812,17.3968)(62.8125,17.385)(64.8438,17.2341)(66.875,16.8693)(68.9062,16.4757)(70.9375,16.2154)(72.9688,15.9572)(75,15.5554)
		\drawline(75,20.3326)(77.0312,23.7441)(79.0625,25.8106)(81.0938,24.7261)(83.125,21.5609)(85.1562,18.7075)(87.1875,15.9467)(89.2188,14.3584)(91.25,13.3928)(93.2812,12.415)(95.3125,11.2562)(97.3438,10.5456)(99.375,10.1759)(101.406,9.79635)(103.438,9.45444)(105.469,9.2397)(107.5,9.08349)
		\drawline(107.5,18.3039)(109.531,20.1539)(111.562,25.3897)(113.594,28.7396)(115.625,28.825)(117.656,26.489)(119.688,23.8443)(121.719,21.3065)(123.75,20.3216)(125.781,20.1654)(127.812,19.2397)(129.844,17.9977)(131.875,16.8637)(133.906,16.4827)(135.938,16.83)(137.969,17.1505)(140,17.2356)
		\drawline(140,20.3133)(142.031,20.4159)(144.062,20.9742)(146.094,21.6739)(148.125,22.5024)(150.156,23.6032)(152.188,25.0179)(154.219,26.4004)(156.25,27.4866)(158.281,28.1711)(160.312,28.4212)(162.344,28.2497)(164.375,27.7017)(166.406,26.8504)(168.438,25.7919)(170.469,24.6338)(172.5,23.4788)
		\drawline(172.5,19.3137)(174.531,21.0494)(176.562,22.7401)(178.594,23.7167)(180.625,24.1103)(182.656,24.1735)(184.688,23.9714)(186.719,23.5137)(188.75,22.8924)(190.781,22.5361)(192.812,22.532)(194.844,22.4347)(196.875,22.074)(198.906,21.5141)(200.938,20.978)(202.969,20.5604)(205,20.015)
		\drawline(205,16.8803)(207.031,16.189)(209.062,14.5954)(211.094,12.902)(213.125,12.2297)(215.156,11.8992)(217.188,11.5198)(219.219,10.9302)(221.25,10.2625)(223.281,9.64475)(225.312,9.2715)(227.344,9.14618)(229.375,9.00658)(231.406,8.7893)(233.438,8.61739)(235.469,8.5702)(237.5,8.49062)
		\drawline(237.5,19.8106)(239.531,16.5873)(241.562,17.3929)(243.594,18.8383)(245.625,19.1122)(247.656,18.005)(249.688,16.1353)(251.719,14.5154)(253.75,13.1904)(255.781,12.3546)(257.812,12.2784)(259.844,12.1716)(261.875,11.9204)(263.906,11.6462)(265.938,11.5072)(267.969,11.5593)(270,11.6783)
		\drawline(10,17.352)(12.0312,17.3468)(14.0625,17.8103)(16.0938,18.6661)(18.125,19.614)(20.1562,20.5582)(22.1875,21.2132)(24.2188,21.5496)(26.25,21.601)(28.2812,21.4122)(30.3125,21.023)(32.3438,20.4683)(34.375,19.7845)(36.4062,19.0158)(38.4375,18.2158)(40.4688,17.4407)(42.5,16.7363)
		\drawline(42.5,19.7749)(44.5312,19.5124)(46.5625,19.3041)(48.5938,18.6387)(50.625,17.5506)(52.6562,16.0471)(54.6875,14.3089)(56.7188,12.8299)(58.75,11.954)(60.7812,11.7602)(62.8125,11.5323)(64.8438,11.1262)(66.875,10.74)(68.9062,10.5558)(70.9375,10.4699)(72.9688,10.302)(75,10.1092)
		\drawline(75,19.9526)(77.0312,18.2937)(79.0625,16.5507)(81.0938,15.6003)(83.125,15.4361)(85.1562,14.4856)(87.1875,13.0099)(89.2188,11.8145)(91.25,10.9889)(93.2812,10.3664)(95.3125,9.93983)(97.3438,9.6375)(99.375,9.39758)(101.406,9.09048)(103.438,8.85295)(105.469,8.74554)(107.5,8.62218)
		\drawline(107.5,15.7341)(109.531,18.287)(111.562,21.0447)(113.594,22.3793)(115.625,22.03)(117.656,21.2274)(119.688,20.7444)(121.719,20.0147)(123.75,18.892)(125.781,17.7479)(127.812,17.1169)(129.844,17.0078)(131.875,17.008)(133.906,16.9493)(135.938,16.8381)(137.969,16.7931)(140,16.8976)
		\drawline(140,18.1583)(142.031,17.2816)(144.062,20.2162)(146.094,24.2411)(148.125,28.1392)(150.156,31.5369)(152.188,34.3096)(154.219,36.4456)(156.25,37.9894)(158.281,38.945)(160.312,39.3041)(162.344,39.0997)(164.375,38.3875)(166.406,37.2502)(168.438,35.8029)(170.469,34.1829)(172.5,32.5265)
		\drawline(172.5,17.3746)(174.531,17.9226)(176.562,18.8644)(178.594,19.8018)(180.625,20.0977)(182.656,19.8939)(184.688,19.5471)(186.719,19.2968)(188.75,19.0047)(190.781,18.7693)(192.812,18.8072)(194.844,18.819)(196.875,18.6516)(198.906,18.2962)(200.938,17.8989)(202.969,17.5906)(205,17.2311)
		\drawline(205,19.7341)(207.031,19.6052)(209.062,20.191)(211.094,19.845)(213.125,18.5571)(215.156,17.3638)(217.188,17.0983)(219.219,16.4114)(221.25,15.464)(223.281,14.3756)(225.312,13.9219)(227.344,13.3818)(229.375,12.6921)(231.406,11.9603)(233.438,11.5689)(235.469,11.166)(237.5,10.7161)
		\drawline(237.5,19.5655)(239.531,20.9508)(241.562,24.3223)(243.594,27.8261)(245.625,28.867)(247.656,27.3228)(249.688,24.9159)(251.719,22.528)(253.75,21.0029)(255.781,20.4322)(257.812,19.7708)(259.844,18.9791)(261.875,18.3534)(263.906,18.249)(265.938,18.4771)(267.969,18.7617)(270,18.9808)
		\drawline(10,16.2845)(12.0312,17.1045)(14.0625,19.3041)(16.0938,21.4424)(18.125,23.3951)(20.1562,25.0032)(22.1875,26.0252)(24.2188,26.5066)(26.25,26.5613)(28.2812,26.3064)(30.3125,25.831)(32.3438,25.1918)(34.375,24.4281)(36.4062,23.5765)(38.4375,22.6776)(40.4688,21.775)(42.5,20.9092)
		\drawline(42.5,19.484)(44.5312,20.2096)(46.5625,20.9724)(48.5938,21.373)(50.625,21.6692)(52.6562,22.065)(54.6875,22.3037)(56.7188,22.2993)(58.75,22.0096)(60.7812,21.5493)(62.8125,21.2068)(64.8438,21.0332)(66.875,20.8035)(68.9062,20.376)(70.9375,19.8513)(72.9688,19.354)(75,18.8557)
		\drawline(75,17.0569)(77.0312,17.4438)(79.0625,17.2133)(81.0938,16.2947)(83.125,14.9224)(85.1562,13.1544)(87.1875,11.526)(89.2188,10.5662)(91.25,10.3016)(93.2812,10.0545)(95.3125,9.75494)(97.3438,9.45095)(99.375,9.26917)(101.406,9.14381)(103.438,9.01887)(105.469,8.85916)(107.5,8.77234)
		\drawline(107.5,22.1114)(109.531,25.1153)(111.562,28.6551)(113.594,32.7474)(115.625,34.2785)(117.656,32.5657)(119.688,28.8826)(121.719,24.9915)(123.75,21.4365)(125.781,19.5914)(127.812,19.2477)(129.844,18.749)(131.875,18.0634)(133.906,17.5459)(135.938,17.5755)(137.969,17.8694)(140,18.1707)
		\drawline(140,18.8574)(142.031,19.6589)(144.062,24.7174)(146.094,30.4536)(148.125,35.6859)(150.156,40.1515)(152.188,43.7844)(154.219,46.6073)(156.25,48.6826)(158.281,49.9782)(160.312,50.4799)(162.344,50.2276)(164.375,49.2929)(166.406,47.7876)(168.438,45.868)(170.469,43.7198)(172.5,41.5255)
		\drawline(172.5,18.6147)(174.531,18.0319)(176.562,17.6668)(178.594,17.0089)(180.625,16.3276)(182.656,15.7502)(184.688,15.125)(186.719,14.3616)(188.75,13.5561)(190.781,13.0867)(192.812,13.0896)(194.844,13.102)(196.875,12.8639)(198.906,12.4858)(200.938,12.1611)(202.969,11.9999)(205,11.7937)
		\drawline(205,18.1671)(207.031,16.7707)(209.062,15.6074)(211.094,14.8962)(213.125,14.7165)(215.156,14.1041)(217.188,13.1153)(219.219,12.0949)(221.25,11.232)(223.281,10.7373)(225.312,10.3271)(227.344,9.87997)(229.375,9.4578)(231.406,9.24248)(233.438,9.1022)(235.469,8.91832)(237.5,8.74417)
		\drawline(237.5,18.1558)(239.531,18.4699)(241.562,18.3933)(243.594,18.3602)(245.625,18.0034)(247.656,17.2645)(249.688,16.3468)(251.719,15.5875)(253.75,14.7018)(255.781,13.5855)(257.812,12.4008)(259.844,11.3757)(261.875,10.6303)(263.906,10.4867)(265.938,10.4943)(267.969,10.3862)(270,10.1953)
		\drawline(10,17.7918)(12.0312,17.8075)(14.0625,18.7792)(16.0938,19.3615)(18.125,19.5074)(20.1562,19.3373)(22.1875,19.038)(24.2188,18.8037)(26.25,18.6294)(28.2812,18.4029)(30.3125,18.133)(32.3438,17.8393)(34.375,17.4903)(36.4062,17.0679)(38.4375,16.5759)(40.4688,16.0374)(42.5,15.4849)
		\drawline(42.5,17.2055)(44.5312,16.1712)(46.5625,15.0218)(48.5938,14.0489)(50.625,13.59)(52.6562,13.7197)(54.6875,13.9739)(56.7188,14.1835)(58.75,14.1552)(60.7812,13.9898)(62.8125,13.8056)(64.8438,13.6685)(66.875,13.5799)(68.9062,13.5246)(70.9375,13.447)(72.9688,13.2842)(75,13.0514)
		\drawline(75,18.4047)(77.0312,19.3169)(79.0625,18.9904)(81.0938,17.0541)(83.125,15.1377)(85.1562,13.8926)(87.1875,13.6395)(89.2188,13.1937)(91.25,12.6668)(93.2812,12.1133)(95.3125,11.9134)(97.3438,11.6221)(99.375,11.2271)(101.406,10.7388)(103.438,10.4345)(105.469,10.1783)(107.5,9.89617)
		\drawline(107.5,18.1036)(109.531,17.9244)(111.562,17.6349)(113.594,18.1031)(115.625,20.3749)(117.656,21.4934)(119.688,20.8238)(121.719,18.9518)(123.75,17.1082)(125.781,15.6325)(127.812,15.3208)(129.844,15.3259)(131.875,15.0078)(133.906,14.4959)(135.938,14.0635)(137.969,13.9775)(140,14.1908)
		\drawline(140,17.4396)(142.031,15.6757)(144.062,16.0378)(146.094,17.8308)(148.125,20.3992)(150.156,22.9308)(152.188,25.1207)(154.219,26.8684)(156.25,28.1473)(158.281,28.9692)(160.312,29.3362)(162.344,29.2833)(164.375,28.8611)(166.406,28.1334)(168.438,27.1814)(170.469,26.0996)(172.5,24.9818)
		\drawline(172.5,18.5395)(174.531,17.6033)(176.562,17.6798)(178.594,17.7702)(180.625,17.9535)(182.656,18.1213)(184.688,18.2031)(186.719,18.1432)(188.75,17.9701)(190.781,17.7572)(192.812,17.5475)(194.844,17.3194)(196.875,17.0426)(198.906,16.7175)(200.938,16.3649)(202.969,16.0025)(205,15.6156)
		\drawline(205,20.634)(207.031,18.7576)(209.062,21.1532)(211.094,22.6807)(213.125,21.9084)(215.156,19.2572)(217.188,16.2498)(219.219,13.9151)(221.25,12.9292)(223.281,12.3731)(225.312,11.4456)(227.344,10.7042)(229.375,10.2052)(231.406,9.8752)(233.438,9.52117)(235.469,9.24967)(237.5,9.05897)
		\drawline(237.5,17.2102)(239.531,18.6105)(241.562,21.8946)(243.594,26.2271)(245.625,29.156)(247.656,29.4194)(249.688,27.1852)(251.719,23.8028)(253.75,20.6927)(255.781,18.0631)(257.812,17.0992)(259.844,16.9576)(261.875,16.4442)(263.906,15.7541)(265.938,15.1822)(267.969,15.1123)(270,15.4082)
		\drawline(10,18.1701)(12.0312,15.7095)(14.0625,15.8298)(16.0938,15.9195)(18.125,15.7665)(20.1562,15.1934)(22.1875,14.3848)(24.2188,13.6209)(26.25,12.9574)(28.2812,12.4002)(30.3125,12.0948)(32.3438,12.0389)(34.375,12.1095)(36.4062,12.2035)(38.4375,12.2983)(40.4688,12.3494)(42.5,12.3439)
		\drawline(42.5,17.6785)(44.5312,18.7207)(46.5625,20.0561)(48.5938,21.0007)(50.625,21.4001)(52.6562,21.3997)(54.6875,21.2476)(56.7188,20.9714)(58.75,20.6557)(60.7812,20.4343)(62.8125,20.3964)(64.8438,20.3124)(66.875,20.0668)(68.9062,19.6707)(70.9375,19.2231)(72.9688,18.8176)(75,18.3691)
		\drawline(75,19.4407)(77.0312,18.847)(79.0625,17.6239)(81.0938,15.9575)(83.125,14.7284)(85.1562,14.4893)(87.1875,13.8537)(89.2188,12.9263)(91.25,12.0336)(93.2812,11.3461)(95.3125,10.7812)(97.3438,10.1987)(99.375,9.68102)(101.406,9.27771)(103.438,9.01641)(105.469,8.79115)(107.5,8.60218)
		\drawline(107.5,18.6728)(109.531,18.1819)(111.562,17.9655)(113.594,16.8155)(115.625,14.9768)(117.656,13.446)(119.688,12.6029)(121.719,11.804)(123.75,11.255)(125.781,11.2565)(127.812,11.1737)(129.844,10.9162)(131.875,10.6274)(133.906,10.475)(135.938,10.4925)(137.969,10.5557)(140,10.6128)
		\drawline(140,17.3047)(142.031,16.9178)(144.062,18.7257)(146.094,22.2339)(148.125,26.0225)(150.156,29.4026)(152.188,32.2293)(154.219,34.4296)(156.25,35.9265)(158.281,36.7341)(160.312,36.9209)(162.344,36.5668)(164.375,35.756)(166.406,34.5845)(168.438,33.1666)(170.469,31.6295)(172.5,30.094)
		\drawline(172.5,18.7847)(174.531,17.9057)(176.562,18.0367)(178.594,18.1233)(180.625,17.8106)(182.656,17.2974)(184.688,16.7948)(186.719,16.394)(188.75,16.2372)(190.781,16.357)(192.812,16.4034)(194.844,16.296)(196.875,16.0424)(198.906,15.7462)(200.938,15.5443)(202.969,15.3584)(205,15.0348)
		\drawline(205,21.5609)(207.031,23.0559)(209.062,23.2359)(211.094,21.3957)(213.125,18.8619)(215.156,16.1552)(217.188,13.8367)(219.219,13.272)(221.25,12.7717)(223.281,11.8572)(225.312,10.8673)(227.344,10.4585)(229.375,10.1978)(231.406,9.73516)(233.438,9.32023)(235.469,9.09419)(237.5,8.92492)
		\drawline(237.5,18.2005)(239.531,17.2802)(241.562,18.258)(243.594,21.1835)(245.625,24.4886)(247.656,25.8563)(249.688,24.9734)(251.719,22.8489)(253.75,20.4234)(255.781,18.2078)(257.812,17.4027)(259.844,17.3179)(261.875,16.9284)(263.906,16.3207)(265.938,15.8264)(267.969,15.7938)(270,16.0833)
		\put(5,7){\tiny 0}
		\put(0,40){$\zeta$}
		\put(5,70){\tiny 8}
		\put(175,75){\makebox(96,4)[r]{$R=1500,\,\alpha=5,\,n=1$}}
	\end{picture}
	%\input{/home/elmar/src/floquet/integrator/dat/resrand/2500-50-1.epic}
	\begin{picture}(275,78)
		\thinlines
		\drawline(10,8)(10,73)
		\drawline(42.5,8)(42.5,73)
		\drawline(75,8)(75,73)
		\drawline(107.5,8)(107.5,73)
		\drawline(140,8)(140,73)
		\drawline(172.5,8)(172.5,73)
		\drawline(205,8)(205,73)
		\drawline(237.5,8)(237.5,73)
		\drawline(270,8)(270,73)
		\drawline(10,8)(270,8)
		\drawline(10,73)(270,73)
		\thicklines
		\drawline(10,40.3197)(12.0312,39.2045)(14.0625,38.0872)(16.0938,36.896)(18.125,35.6263)(20.1562,34.3053)(22.1875,33.0083)(24.2188,31.8433)(26.25,30.8949)(28.2812,30.2781)(30.3125,29.954)(32.3438,29.7761)(34.375,29.6213)(36.4062,29.4285)(38.4375,29.1695)(40.4688,28.8385)(42.5,28.4451)
		\drawline(42.5,34.7538)(44.5312,34.6617)(46.5625,35.0844)(48.5938,35.518)(50.625,35.8501)(52.6562,36.0579)(54.6875,36.1134)(56.7188,36.1088)(58.75,36.16)(60.7812,36.3374)(62.8125,36.6266)(64.8438,36.9642)(66.875,37.2801)(68.9062,37.5249)(70.9375,37.6816)(72.9688,37.7576)(75,37.7661)
		\drawline(75,33.2202)(77.0312,32.1622)(79.0625,30.6458)(81.0938,28.7597)(83.125,26.699)(85.1562,24.7148)(87.1875,23.0486)(89.2188,21.8489)(91.25,21.1513)(93.2812,20.6363)(95.3125,20.0229)(97.3438,19.2803)(99.375,18.5009)(101.406,17.8619)(103.438,17.4965)(105.469,17.3071)(107.5,17.1173)
		\drawline(107.5,36.5487)(109.531,37.115)(111.562,37.6924)(113.594,37.8014)(115.625,37.3722)(117.656,36.8663)(119.688,36.5957)(121.719,36.2015)(123.75,35.9888)(125.781,35.7349)(127.812,35.0994)(129.844,34.25)(131.875,33.7051)(133.906,33.8485)(135.938,34.4992)(137.969,35.4954)(140,36.4078)
		\drawline(140,29.4565)(142.031,28.3502)(144.062,27.7726)(146.094,27.4252)(148.125,27.1912)(150.156,27.0268)(152.188,26.8929)(154.219,26.7509)(156.25,26.5715)(158.281,26.3312)(160.312,26.0133)(162.344,25.6075)(164.375,25.1095)(166.406,24.5188)(168.438,23.8383)(170.469,23.0777)(172.5,22.2556)
		\drawline(172.5,42.8151)(174.531,45.0658)(176.562,46.8776)(178.594,47.9475)(180.625,48.3405)(182.656,48.3225)(184.688,48.2712)(186.719,48.4179)(188.75,48.8551)(190.781,49.5066)(192.812,50.1872)(194.844,50.7654)(196.875,51.1505)(198.906,51.3303)(200.938,51.3468)(202.969,51.243)(205,51.0426)
		\drawline(205,36.1758)(207.031,35.8148)(209.062,35.5702)(211.094,34.8574)(213.125,33.3572)(215.156,31.5709)(217.188,30.2285)(219.219,29.8734)(221.25,30.2031)(223.281,30.199)(225.312,29.5027)(227.344,28.3757)(229.375,27.3338)(231.406,26.947)(233.438,26.9791)(235.469,26.9407)(237.5,26.4539)
		\drawline(237.5,34.3135)(239.531,32.971)(241.562,32.6445)(243.594,33.038)(245.625,33.2991)(247.656,33.3933)(249.688,32.5702)(251.719,31.2895)(253.75,30.3204)(255.781,30.035)(257.812,30.116)(259.844,30.4367)(261.875,30.9162)(263.906,31.5625)(265.938,31.9239)(267.969,32.008)(270,31.92)
		\drawline(10,40.3982)(12.0312,40.3496)(14.0625,41.6837)(16.0938,43.5869)(18.125,45.8103)(20.1562,48.2274)(22.1875,50.3694)(24.2188,52.0182)(26.25,53.1193)(28.2812,53.6551)(30.3125,53.6289)(32.3438,53.087)(34.375,52.0897)(36.4062,50.6989)(38.4375,49.0126)(40.4688,47.1946)(42.5,45.5518)
		\drawline(42.5,34.3148)(44.5312,34.7003)(46.5625,34.569)(48.5938,34.0316)(50.625,33.4594)(52.6562,32.7455)(54.6875,31.9527)(56.7188,31.2498)(58.75,30.7283)(60.7812,30.1602)(62.8125,29.533)(64.8438,28.8583)(66.875,28.1798)(68.9062,27.5612)(70.9375,27.0425)(72.9688,26.6661)(75,26.5032)
		\drawline(75,38.7788)(77.0312,39.7889)(79.0625,41.1836)(81.0938,42.0715)(83.125,42.3571)(85.1562,42.3744)(87.1875,42.294)(89.2188,42.7986)(91.25,43.5756)(93.2812,44.1597)(95.3125,44.5714)(97.3438,44.6214)(99.375,44.4275)(101.406,44.5397)(103.438,44.8481)(105.469,44.9913)(107.5,44.9865)
		\drawline(107.5,36.9055)(109.531,38.1301)(111.562,40.1072)(113.594,42.3638)(115.625,43.8859)(117.656,45.9583)(119.688,46.8704)(121.719,45.8066)(123.75,43.4811)(125.781,41.5194)(127.812,40.6902)(129.844,40.6099)(131.875,41.1498)(133.906,42.2359)(135.938,43.8334)(137.969,45.2994)(140,46.1702)
		\drawline(140,36.0783)(142.031,37.351)(144.062,38.5918)(146.094,39.5703)(148.125,40.2378)(150.156,40.602)(152.188,40.744)(154.219,40.8397)(156.25,40.7349)(158.281,40.3758)(160.312,39.8034)(162.344,38.9526)(164.375,37.8064)(166.406,36.4167)(168.438,34.863)(170.469,33.2339)(172.5,31.6214)
		\drawline(172.5,36.3081)(174.531,35.1518)(176.562,33.997)(178.594,32.8185)(180.625,31.637)(182.656,30.4369)(184.688,29.2504)(186.719,28.1237)(188.75,27.1258)(190.781,26.2793)(192.812,25.5278)(194.844,24.848)(196.875,24.2108)(198.906,23.5973)(200.938,23.0045)(202.969,22.4423)(205,21.9194)
		\drawline(205,33.8306)(207.031,33.5892)(209.062,33.3669)(211.094,33.215)(213.125,33.1093)(215.156,32.9093)(217.188,32.599)(219.219,32.1304)(221.25,31.3876)(223.281,30.4417)(225.312,29.613)(227.344,29.1291)(229.375,28.7946)(231.406,28.4043)(233.438,27.8168)(235.469,27.0674)(237.5,26.3864)
		\drawline(237.5,35.1709)(239.531,37.0414)(241.562,38.8538)(243.594,40.0734)(245.625,40.2313)(247.656,39.0485)(249.688,36.8765)(251.719,34.4002)(253.75,32.4259)(255.781,31.1467)(257.812,30.5468)(259.844,30.4911)(261.875,31.0581)(263.906,31.6986)(265.938,32)(267.969,32.0069)(270,32.0652)
		\drawline(10,37.1531)(12.0312,37.8198)(14.0625,38.9283)(16.0938,39.8827)(18.125,40.5431)(20.1562,40.8922)(22.1875,40.9409)(24.2188,40.7336)(26.25,40.3726)(28.2812,40.1254)(30.3125,39.9105)(32.3438,39.4277)(34.375,38.6544)(36.4062,37.6569)(38.4375,36.5491)(40.4688,35.4697)(42.5,34.5562)
		\drawline(42.5,32.9175)(44.5312,31.7582)(46.5625,30.7949)(48.5938,29.5539)(50.625,27.9562)(52.6562,26.1953)(54.6875,24.6229)(56.7188,23.5321)(58.75,22.6791)(60.7812,22.0434)(62.8125,21.5387)(64.8438,21.0941)(66.875,20.616)(68.9062,20.1825)(70.9375,19.8288)(72.9688,19.4843)(75,19.1328)
		\drawline(75,36.6301)(77.0312,36.3495)(79.0625,35.2967)(81.0938,33.7961)(83.125,32.4684)(85.1562,31.8362)(87.1875,31.8349)(89.2188,32.1644)(91.25,32.2294)(93.2812,31.904)(95.3125,31.3684)(97.3438,31.0371)(99.375,31.221)(101.406,31.3799)(103.438,31.4849)(105.469,31.3205)(107.5,30.8939)
		\drawline(107.5,39.0782)(109.531,37.9121)(111.562,37.115)(113.594,36.2542)(115.625,35.5015)(117.656,34.3839)(119.688,32.9886)(121.719,31.7148)(123.75,30.9348)(125.781,30.4267)(127.812,29.9934)(129.844,29.5686)(131.875,29.24)(133.906,29.2363)(135.938,29.1527)(137.969,29.018)(140,29.0233)
		\drawline(140,41.545)(142.031,40.7496)(144.062,40.0357)(146.094,39.1413)(148.125,37.9815)(150.156,36.5959)(152.188,35.1063)(154.219,33.662)(156.25,32.3098)(158.281,31.1389)(160.312,30.178)(162.344,29.555)(164.375,29.1005)(166.406,28.6734)(168.438,28.2177)(170.469,27.7217)(172.5,27.1973)
		\drawline(172.5,31.6349)(174.531,31.1038)(176.562,31.0505)(178.594,31.1591)(180.625,31.1716)(182.656,31.0726)(184.688,30.9266)(186.719,30.8094)(188.75,30.747)(190.781,30.7084)(192.812,30.6525)(194.844,30.5578)(196.875,30.4313)(198.906,30.2969)(200.938,30.1763)(202.969,30.076)(205,29.9895)
		\drawline(205,38.6107)(207.031,38.4761)(209.062,38.2794)(211.094,37.474)(213.125,36.3331)(215.156,35.2092)(217.188,34.3471)(219.219,34.016)(221.25,33.7144)(223.281,33.0807)(225.312,32.3057)(227.344,31.6119)(229.375,30.9727)(231.406,30.6363)(233.438,30.2939)(235.469,29.6444)(237.5,28.7261)
		\drawline(237.5,30.6486)(239.531,29.387)(241.562,29.047)(243.594,28.82)(245.625,28.3084)(247.656,27.5444)(249.688,26.4781)(251.719,25.1204)(253.75,23.7081)(255.781,22.5717)(257.812,21.9661)(259.844,21.5882)(261.875,21.3252)(263.906,21.0344)(265.938,20.7627)(267.969,20.8016)(270,21.0395)
		\drawline(10,39.5142)(12.0312,38.8748)(14.0625,38.4165)(16.0938,38.103)(18.125,37.8185)(20.1562,37.5581)(22.1875,37.3772)(24.2188,37.4103)(26.25,37.4394)(28.2812,37.2886)(30.3125,36.8858)(32.3438,36.2107)(34.375,35.2731)(36.4062,34.1133)(38.4375,32.8044)(40.4688,31.3777)(42.5,29.8989)
		\drawline(42.5,27.4522)(44.5312,26.9965)(46.5625,26.4557)(48.5938,25.8596)(50.625,25.3431)(52.6562,25.0078)(54.6875,24.7861)(56.7188,24.6214)(58.75,24.5569)(60.7812,24.4876)(62.8125,24.3522)(64.8438,24.1786)(66.875,23.9645)(68.9062,23.7122)(70.9375,23.437)(72.9688,23.1507)(75,22.8559)
		\drawline(75,36.9018)(77.0312,36.9226)(79.0625,35.7349)(81.0938,33.7214)(83.125,31.749)(85.1562,30.4126)(87.1875,30.2783)(89.2188,30.3011)(91.25,29.6944)(93.2812,28.5816)(95.3125,27.5296)(97.3438,26.9943)(99.375,27.0559)(101.406,26.9813)(103.438,26.4054)(105.469,25.4612)(107.5,24.5331)
		\drawline(107.5,33.5019)(109.531,32.4727)(111.562,31.6377)(113.594,30.6072)(115.625,29.3116)(117.656,27.8843)(119.688,26.7049)(121.719,26.1762)(123.75,25.8057)(125.781,25.3259)(127.812,24.744)(129.844,24.1392)(131.875,23.4207)(133.906,22.7036)(135.938,22.1972)(137.969,22.0361)(140,22.2814)
		\drawline(140,34.3057)(142.031,34.2004)(144.062,34.453)(146.094,35.0023)(148.125,36.0468)(150.156,37.5581)(152.188,39.2249)(154.219,40.8068)(156.25,42.148)(158.281,43.1422)(160.312,43.7197)(162.344,43.8425)(164.375,43.5065)(166.406,42.7447)(168.438,41.6336)(170.469,40.2981)(172.5,38.9298)
		\drawline(172.5,45.9277)(174.531,45.0247)(176.562,44.5586)(178.594,43.758)(180.625,42.7228)(182.656,41.6098)(184.688,40.5067)(186.719,39.4758)(188.75,38.6287)(190.781,38.1152)(192.812,37.703)(194.844,37.2686)(196.875,36.8435)(198.906,36.4698)(200.938,36.3058)(202.969,36.2707)(205,36.3452)
		\drawline(205,41.6951)(207.031,41.3413)(209.062,41.0194)(211.094,40.5576)(213.125,39.8396)(215.156,38.7963)(217.188,37.4489)(219.219,36.0128)(221.25,34.7817)(223.281,33.9129)(225.312,33.4254)(227.344,32.9964)(229.375,32.3345)(231.406,31.5099)(233.438,30.7151)(235.469,30.0445)(237.5,29.5634)
		\drawline(237.5,44.4236)(239.531,42.1465)(241.562,40.058)(243.594,37.7366)(245.625,35.4445)(247.656,33.7578)(249.688,32.9847)(251.719,32.7509)(253.75,32.6705)(255.781,32.3878)(257.812,31.7584)(259.844,30.8102)(261.875,29.7074)(263.906,28.7591)(265.938,28.1307)(267.969,27.9482)(270,28.0527)
		\drawline(10,37.4123)(12.0312,34.5852)(14.0625,32.5559)(16.0938,31.09)(18.125,30.004)(20.1562,29.1154)(22.1875,28.2657)(24.2188,27.4304)(26.25,26.6265)(28.2812,25.884)(30.3125,25.2331)(32.3438,24.6937)(34.375,24.2709)(36.4062,23.8381)(38.4375,23.3274)(40.4688,22.7223)(42.5,22.0318)
		\drawline(42.5,38.8039)(44.5312,37.6309)(46.5625,36.5846)(48.5938,35.456)(50.625,34.2034)(52.6562,32.8781)(54.6875,31.6104)(56.7188,30.539)(58.75,29.8584)(60.7812,29.6098)(62.8125,29.5203)(64.8438,29.6697)(66.875,30.0922)(68.9062,30.5327)(70.9375,30.9379)(72.9688,31.3144)(75,31.6637)
		\drawline(75,38.899)(77.0312,37.98)(79.0625,36.9068)(81.0938,35.9394)(83.125,35.2922)(85.1562,34.9091)(87.1875,34.4771)(89.2188,33.8546)(91.25,33.0618)(93.2812,32.3122)(95.3125,31.7566)(97.3438,31.3532)(99.375,30.9004)(101.406,30.3801)(103.438,29.8131)(105.469,29.1985)(107.5,28.6746)
		\drawline(107.5,36.449)(109.531,37.0873)(111.562,37.4203)(113.594,37.2335)(115.625,36.6819)(117.656,35.9569)(119.688,35.1447)(121.719,34.4825)(123.75,34.2572)(125.781,34.1057)(127.812,33.9281)(129.844,33.5838)(131.875,32.9193)(133.906,32.1761)(135.938,31.959)(137.969,32.3037)(140,33.1697)
		\drawline(140,35.5979)(142.031,37.1956)(144.062,39.6834)(146.094,42.2485)(148.125,44.5419)(150.156,46.3832)(152.188,47.6825)(154.219,48.4111)(156.25,48.5489)(158.281,48.1184)(160.312,47.1798)(162.344,45.8634)(164.375,44.5946)(166.406,43.6189)(168.438,42.7182)(170.469,41.7957)(172.5,40.8989)
		\drawline(172.5,36.6698)(174.531,35.3412)(176.562,34.1835)(178.594,33.1738)(180.625,32.2805)(182.656,31.4563)(184.688,30.6761)(186.719,29.9091)(188.75,29.1543)(190.781,28.442)(192.812,27.7944)(194.844,27.2044)(196.875,26.6579)(198.906,26.1197)(200.938,25.5745)(202.969,25.0245)(205,24.4774)
		\drawline(205,35.3631)(207.031,34.2128)(209.062,32.7615)(211.094,31.3198)(213.125,29.9483)(215.156,28.8301)(217.188,28.3738)(219.219,28.4708)(221.25,28.2473)(223.281,27.7111)(225.312,27.1859)(227.344,26.8768)(229.375,26.7023)(231.406,26.6773)(233.438,26.5323)(235.469,26.1281)(237.5,25.6274)
		\drawline(237.5,44.8383)(239.531,45.0938)(241.562,44.8853)(243.594,43.6974)(245.625,41.5487)(247.656,38.8583)(249.688,36.1987)(251.719,34.1059)(253.75,32.8608)(255.781,32.13)(257.812,31.5794)(259.844,31.3465)(261.875,31.2052)(263.906,30.7927)(265.938,30.1782)(267.969,29.5777)(270,29.2914)
		\drawline(10,35.3839)(12.0312,37.7401)(14.0625,40.8917)(16.0938,44.2934)(18.125,47.6119)(20.1562,50.6712)(22.1875,53.3713)(24.2188,55.6374)(26.25,57.39)(28.2812,58.5583)(30.3125,59.1008)(32.3438,59.0135)(34.375,58.3917)(36.4062,57.5476)(38.4375,56.4393)(40.4688,55.0795)(42.5,53.6127)
		\drawline(42.5,35.0781)(44.5312,35.3026)(46.5625,36.3944)(48.5938,37.5577)(50.625,38.3808)(52.6562,38.7502)(54.6875,38.7261)(56.7188,38.4614)(58.75,38.1498)(60.7812,38.0016)(62.8125,38.0571)(64.8438,38.2163)(66.875,38.3927)(68.9062,38.5257)(70.9375,38.5922)(72.9688,38.6206)(75,38.6449)
		\drawline(75,40.7708)(77.0312,41.581)(79.0625,42.0474)(81.0938,41.438)(83.125,39.3775)(85.1562,36.5775)(87.1875,34.0838)(89.2188,32.7932)(91.25,32.976)(93.2812,33.1086)(95.3125,32.5535)(97.3438,31.5588)(99.375,30.7327)(101.406,30.4014)(103.438,30.6232)(105.469,30.6863)(107.5,30.3747)
		\drawline(107.5,36.9776)(109.531,36.4642)(111.562,35.4296)(113.594,34.1408)(115.625,33.3985)(117.656,32.9674)(119.688,32.6317)(121.719,32.4428)(123.75,32.1068)(125.781,31.865)(127.812,31.3718)(129.844,30.7184)(131.875,30.2729)(133.906,30.3182)(135.938,30.9238)(137.969,31.8)(140,32.8395)
		\drawline(140,41.6945)(142.031,39.8619)(144.062,38.3771)(146.094,37.0951)(148.125,36.0765)(150.156,35.5724)(152.188,35.3763)(154.219,35.2822)(156.25,35.2231)(158.281,35.1236)(160.312,34.9696)(162.344,34.7713)(164.375,34.5688)(166.406,34.4327)(168.438,34.2017)(170.469,33.8043)(172.5,33.2289)
		\drawline(172.5,38.4482)(174.531,37.1917)(176.562,36.2466)(178.594,35.5303)(180.625,34.8597)(182.656,34.1575)(184.688,33.5082)(186.719,33.0447)(188.75,32.6023)(190.781,32.151)(192.812,31.7445)(194.844,31.3759)(196.875,31.023)(198.906,30.6748)(200.938,30.3199)(202.969,29.9503)(205,29.5629)
		\drawline(205,30.511)(207.031,29.3535)(209.062,28.3197)(211.094,27.1544)(213.125,26.0121)(215.156,25.0669)(217.188,24.3423)(219.219,24.0759)(221.25,23.8625)(223.281,23.5368)(225.312,23.2812)(227.344,23.1522)(229.375,23.0222)(231.406,22.9207)(233.438,22.718)(235.469,22.3609)(237.5,21.9784)
		\drawline(237.5,37.1408)(239.531,36.6672)(241.562,35.9851)(243.594,35.0164)(245.625,33.5981)(247.656,31.8069)(249.688,29.8699)(251.719,28.0506)(253.75,26.5622)(255.781,25.5213)(257.812,24.7775)(259.844,24.1955)(261.875,23.7168)(263.906,23.1862)(265.938,22.5346)(267.969,21.7786)(270,20.9719)
		\drawline(10,38.8973)(12.0312,39.0704)(14.0625,39.5824)(16.0938,39.845)(18.125,39.7098)(20.1562,39.1731)(22.1875,38.2985)(24.2188,37.1789)(26.25,35.9244)(28.2812,34.6767)(30.3125,33.446)(32.3438,32.2632)(34.375,31.1747)(36.4062,30.2053)(38.4375,29.2543)(40.4688,28.3416)(42.5,27.5231)
		\drawline(42.5,38.9493)(44.5312,40.1984)(46.5625,40.9944)(48.5938,41.2224)(50.625,41.064)(52.6562,40.773)(54.6875,40.5754)(56.7188,40.6642)(58.75,41.2995)(60.7812,42.2658)(62.8125,43.3032)(64.8438,44.2639)(66.875,45.0266)(68.9062,45.6205)(70.9375,46.105)(72.9688,46.5216)(75,46.8817)
		\drawline(75,32.1798)(77.0312,32.6025)(79.0625,32.955)(81.0938,32.5267)(83.125,31.293)(85.1562,29.7978)(87.1875,28.6241)(89.2188,28.1993)(91.25,28.4645)(93.2812,28.5343)(95.3125,28.2502)(97.3438,27.8677)(99.375,27.5764)(101.406,27.4912)(103.438,27.7335)(105.469,27.7909)(107.5,27.6176)
		\drawline(107.5,33.405)(109.531,33.2371)(111.562,33.1277)(113.594,32.8937)(115.625,32.8122)(117.656,33.3541)(119.688,33.872)(121.719,34.0009)(123.75,33.9474)(125.781,34.0895)(127.812,34.0914)(129.844,33.7946)(131.875,33.2408)(133.906,32.6413)(135.938,32.3371)(137.969,32.6248)(140,33.3307)
		\drawline(140,34.1512)(142.031,32.7128)(144.062,31.4193)(146.094,30.3875)(148.125,29.7119)(150.156,29.1673)(152.188,28.7795)(154.219,28.4544)(156.25,28.1704)(158.281,27.9543)(160.312,27.9326)(162.344,28.022)(164.375,28.0057)(166.406,27.8182)(168.438,27.4476)(170.469,26.9166)(172.5,26.2726)
		\drawline(172.5,32.6675)(174.531,31.2488)(176.562,29.8166)(178.594,28.3582)(180.625,26.9408)(182.656,25.6388)(184.688,24.5001)(186.719,23.5405)(188.75,22.7235)(190.781,21.9923)(192.812,21.3215)(194.844,20.7281)(196.875,20.2489)(198.906,19.8551)(200.938,19.4976)(202.969,19.1495)(205,18.8093)
		\drawline(205,27.5607)(207.031,27.2364)(209.062,26.7317)(211.094,25.9277)(213.125,24.9573)(215.156,24.1772)(217.188,23.891)(219.219,23.8561)(221.25,23.6569)(223.281,23.1915)(225.312,22.5889)(227.344,22.1036)(229.375,21.9475)(231.406,21.9286)(233.438,21.7659)(235.469,21.4175)(237.5,20.9768)
		\drawline(237.5,33.095)(239.531,32.8831)(241.562,32.7329)(243.594,32.0914)(245.625,30.8239)(247.656,29.11)(249.688,27.239)(251.719,25.4582)(253.75,24.0083)(255.781,22.9591)(257.812,22.0361)(259.844,21.1745)(261.875,20.4493)(263.906,19.7561)(265.938,19.0544)(267.969,18.3783)(270,17.7813)
		\drawline(10,38.8347)(12.0312,40.1044)(14.0625,41.9274)(16.0938,43.7806)(18.125,45.7791)(20.1562,47.6864)(22.1875,49.0282)(24.2188,49.8401)(26.25,50.2871)(28.2812,50.2918)(30.3125,49.7434)(32.3438,48.6727)(34.375,47.1842)(36.4062,45.4242)(38.4375,43.5006)(40.4688,41.6349)(42.5,40.0797)
		\drawline(42.5,41.9159)(44.5312,40.6242)(46.5625,39.2869)(48.5938,38.1171)(50.625,37.146)(52.6562,36.4752)(54.6875,35.7394)(56.7188,34.7904)(58.75,33.7471)(60.7812,32.7865)(62.8125,32.1189)(64.8438,31.6303)(66.875,31.1766)(68.9062,30.6167)(70.9375,29.9518)(72.9688,29.2492)(75,28.5553)
		\drawline(75,41.2911)(77.0312,40.3436)(79.0625,38.7641)(81.0938,36.4299)(83.125,33.7478)(85.1562,31.234)(87.1875,29.2826)(89.2188,28.2494)(91.25,27.9288)(93.2812,27.6315)(95.3125,26.8838)(97.3438,25.8613)(99.375,24.9123)(101.406,24.3872)(103.438,24.2147)(105.469,24.1713)(107.5,23.866)
		\drawline(107.5,42.8653)(109.531,41.166)(111.562,40.3574)(113.594,39.8439)(115.625,38.9432)(117.656,38.3764)(119.688,38.0655)(121.719,38.3091)(123.75,38.589)(125.781,38.929)(127.812,39.4318)(129.844,39.4851)(131.875,38.8447)(133.906,37.9078)(135.938,37.7353)(137.969,38.5153)(140,40.12)
		\drawline(140,41.0241)(142.031,38.6525)(144.062,36.6925)(146.094,34.8454)(148.125,33.124)(150.156,31.5772)(152.188,30.0476)(154.219,28.5868)(156.25,27.3162)(158.281,26.3937)(160.312,25.9295)(162.344,25.8395)(164.375,25.8315)(166.406,25.8182)(168.438,25.7814)(170.469,25.7344)(172.5,25.703)
		\drawline(172.5,40.7047)(174.531,39.2591)(176.562,38.0653)(178.594,37.248)(180.625,36.5757)(182.656,35.8872)(184.688,35.0617)(186.719,34.0795)(188.75,33.0224)(190.781,32.0129)(192.812,31.1593)(194.844,30.5153)(196.875,30.0242)(198.906,29.5763)(200.938,29.1165)(202.969,28.6368)(205,28.1448)
		\drawline(205,40.4716)(207.031,39.6931)(209.062,38.7946)(211.094,38.0473)(213.125,36.7991)(215.156,34.5698)(217.188,31.8381)(219.219,29.333)(221.25,27.4934)(223.281,26.615)(225.312,26.3498)(227.344,25.8059)(229.375,24.6661)(231.406,23.3845)(233.438,22.5119)(235.469,22.1486)(237.5,22.0925)
		\drawline(237.5,30.2419)(239.531,29.5928)(241.562,29.3473)(243.594,29.1803)(245.625,28.9715)(247.656,28.7169)(249.688,28.5289)(251.719,28.3245)(253.75,28.0075)(255.781,27.5953)(257.812,27.2421)(259.844,27.0617)(261.875,27.0225)(263.906,26.9587)(265.938,26.8177)(267.969,26.6132)(270,26.4338)
		\drawline(10,40.4859)(12.0312,41.758)(14.0625,43.5361)(16.0938,45.3072)(18.125,47.2052)(20.1562,49.1034)(22.1875,50.7202)(24.2188,51.8321)(26.25,52.3358)(28.2812,52.2238)(30.3125,51.5292)(32.3438,50.3117)(34.375,48.6623)(36.4062,46.7203)(38.4375,44.7358)(40.4688,43.142)(42.5,41.9603)
		\drawline(42.5,30.5689)(44.5312,29.1282)(46.5625,28.1513)(48.5938,27.4396)(50.625,26.7728)(52.6562,26.0333)(54.6875,25.2178)(56.7188,24.3773)(58.75,23.5633)(60.7812,22.8201)(62.8125,22.1702)(64.8438,21.57)(66.875,20.9707)(68.9062,20.3838)(70.9375,19.823)(72.9688,19.3009)(75,18.8141)
		\drawline(75,38.8306)(77.0312,37.9579)(79.0625,37.3718)(81.0938,36.1654)(83.125,34.8413)(85.1562,33.571)(87.1875,32.3219)(89.2188,31.4392)(91.25,31.259)(93.2812,31.0587)(95.3125,30.6542)(97.3438,30.3047)(99.375,29.7893)(101.406,29.3103)(103.438,29.1179)(105.469,28.8884)(107.5,28.4455)
		\drawline(107.5,38.4843)(109.531,36.9958)(111.562,35.7249)(113.594,35.5095)(115.625,36.6178)(117.656,37.4394)(119.688,38.345)(121.719,37.8606)(123.75,36.2618)(125.781,34.9189)(127.812,33.8353)(129.844,33.0729)(131.875,33.1193)(133.906,34.4138)(135.938,36.2568)(137.969,37.9797)(140,39.2403)
		\drawline(140,30.0313)(142.031,29.0545)(144.062,28.7277)(146.094,28.6565)(148.125,28.6715)(150.156,28.6581)(152.188,28.5717)(154.219,28.415)(156.25,28.2082)(158.281,27.9768)(160.312,27.7469)(162.344,27.596)(164.375,27.5269)(166.406,27.359)(168.438,27.0637)(170.469,26.6522)(172.5,26.1553)
		\drawline(172.5,33.8514)(174.531,31.8626)(176.562,30.2926)(178.594,28.9512)(180.625,27.6739)(182.656,26.4271)(184.688,25.2408)(186.719,24.1939)(188.75,23.3695)(190.781,22.7209)(192.812,22.2041)(194.844,21.7614)(196.875,21.403)(198.906,21.1489)(200.938,21.0056)(202.969,20.983)(205,21.0065)
		\drawline(205,42.27)(207.031,40.0207)(209.062,37.7375)(211.094,35.131)(213.125,32.7108)(215.156,31.2694)(217.188,30.8824)(219.219,30.7628)(221.25,30.2623)(223.281,29.166)(225.312,27.8717)(227.344,27.1194)(229.375,26.911)(231.406,26.9662)(233.438,26.765)(235.469,26.1295)(237.5,25.2599)
		\drawline(237.5,34.1237)(239.531,34.0537)(241.562,33.6958)(243.594,32.9327)(245.625,31.7404)(247.656,30.2259)(249.688,28.6585)(251.719,27.311)(253.75,26.4763)(255.781,26.1124)(257.812,25.9184)(259.844,25.7656)(261.875,25.4325)(263.906,24.9111)(265.938,24.266)(267.969,23.585)(270,22.965)
		\put(5,7){\tiny 0}
		\put(0,40){$\zeta$}
		\put(5,70){\tiny 3}
		\put(175,75){\makebox(96,4)[r]{$R=2500,\,\alpha=10,\,n=1$}}
	\end{picture}
	%\input{/home/elmar/src/floquet/integrator/dat/resrand/2000-75-1.epic}
		\begin{picture}(275,78)
		\thinlines
		\drawline(10,8)(10,73)
		\drawline(42.5,8)(42.5,73)
		\drawline(75,8)(75,73)
		\drawline(107.5,8)(107.5,73)
		\drawline(140,8)(140,73)
		\drawline(172.5,8)(172.5,73)
		\drawline(205,8)(205,73)
		\drawline(237.5,8)(237.5,73)
		\drawline(270,8)(270,73)
		\drawline(10,8)(270,8)
		\drawline(10,73)(270,73)
		\thicklines
		\drawline(10,39.0492)(12.0312,38.3424)(14.0625,37.7377)(16.0938,37.1757)(18.125,36.6275)(20.1562,36.0813)(22.1875,35.5308)(24.2188,34.972)(26.25,34.4017)(28.2812,33.816)(30.3125,33.2122)(32.3438,32.5871)(34.375,31.9386)(36.4062,31.2661)(38.4375,30.5708)(40.4688,29.856)(42.5,29.1285)
		\drawline(42.5,42.0199)(44.5312,41.2323)(46.5625,40.5726)(48.5938,39.9534)(50.625,39.3064)(52.6562,38.5914)(54.6875,37.7958)(56.7188,36.9261)(58.75,36.0044)(60.7812,35.0623)(62.8125,34.1363)(64.8438,33.2636)(66.875,32.4779)(68.9062,31.7948)(70.9375,31.2085)(72.9688,30.7123)(75,30.2915)
		\drawline(75,38.0785)(77.0312,37.4918)(79.0625,37.0894)(81.0938,36.7686)(83.125,36.4165)(85.1562,35.9719)(87.1875,35.425)(89.2188,34.7976)(91.25,34.1244)(93.2812,33.4417)(95.3125,32.7797)(97.3438,32.1598)(99.375,31.5954)(101.406,31.0965)(103.438,30.6763)(105.469,30.3529)(107.5,30.0996)
		\drawline(107.5,38.0372)(109.531,37.3624)(111.562,36.5528)(113.594,35.6309)(115.625,34.6927)(117.656,34.0832)(119.688,33.7534)(121.719,33.5628)(123.75,33.464)(125.781,33.3015)(127.812,33.0114)(129.844,32.6003)(131.875,32.0903)(133.906,31.5068)(135.938,30.8815)(137.969,30.2517)(140,29.6612)
		\drawline(140,43.5836)(142.031,42.4559)(144.062,41.4323)(146.094,40.4785)(148.125,39.5774)(150.156,38.7259)(152.188,37.9364)(154.219,37.2097)(156.25,36.5166)(158.281,35.8473)(160.312,35.1977)(162.344,34.5662)(164.375,33.9538)(166.406,33.3695)(168.438,32.8088)(170.469,32.2617)(172.5,31.7291)
		\drawline(172.5,36.0876)(174.531,35.5684)(176.562,35.1817)(178.594,34.8381)(180.625,34.4788)(182.656,34.1135)(184.688,33.7227)(186.719,33.2844)(188.75,32.8088)(190.781,32.3156)(192.812,31.8255)(194.844,31.3517)(196.875,30.8988)(198.906,30.4716)(200.938,30.0773)(202.969,29.7332)(205,29.4176)
		\drawline(205,38.0954)(207.031,37.5568)(209.062,36.9794)(211.094,36.2999)(213.125,35.5106)(215.156,34.6398)(217.188,33.7311)(219.219,32.83)(221.25,31.9711)(223.281,31.174)(225.312,30.4425)(227.344,29.7782)(229.375,29.1918)(231.406,28.6952)(233.438,28.3085)(235.469,27.96)(237.5,27.6073)
		\drawline(237.5,34.1053)(239.531,33.7162)(241.562,33.3565)(243.594,32.9975)(245.625,32.591)(247.656,32.0907)(249.688,31.4737)(251.719,30.7422)(253.75,29.918)(255.781,29.034)(257.812,28.1303)(259.844,27.2406)(261.875,26.3937)(263.906,25.6159)(265.938,24.9327)(267.969,24.3777)(270,23.9602)
		\drawline(10,43.9521)(12.0312,43.3925)(14.0625,42.9531)(16.0938,42.5765)(18.125,42.2422)(20.1562,41.9549)(22.1875,41.7296)(24.2188,41.5012)(26.25,41.2369)(28.2812,40.9199)(30.3125,40.541)(32.3438,40.0957)(34.375,39.5829)(36.4062,39.0041)(38.4375,38.3645)(40.4688,37.6712)(42.5,36.9348)
		\drawline(42.5,39.8192)(44.5312,39.5127)(46.5625,39.1701)(48.5938,38.7677)(50.625,38.2915)(52.6562,37.7373)(54.6875,37.1096)(56.7188,36.42)(58.75,35.6861)(60.7812,34.9291)(62.8125,34.1707)(64.8438,33.4317)(66.875,32.7293)(68.9062,32.0749)(70.9375,31.473)(72.9688,30.9179)(75,30.3945)
		\drawline(75,37.3635)(77.0312,36.8927)(79.0625,36.3396)(81.0938,35.7275)(83.125,35.0955)(85.1562,34.4923)(87.1875,33.9725)(89.2188,33.5808)(91.25,33.3545)(93.2812,33.2662)(95.3125,33.1801)(97.3438,33.0467)(99.375,32.8564)(101.406,32.6166)(103.438,32.3419)(105.469,32.0506)(107.5,31.7621)
		\drawline(107.5,41.1682)(109.531,40.8642)(111.562,40.5017)(113.594,40.0966)(115.625,39.6554)(117.656,39.1617)(119.688,38.5771)(121.719,37.8679)(123.75,37.0288)(125.781,36.0804)(127.812,35.0593)(129.844,34.0093)(131.875,32.9767)(133.906,32.0054)(135.938,31.1385)(137.969,30.4254)(140,29.9533)
		\drawline(140,31.1458)(142.031,30.9199)(144.062,30.6915)(146.094,30.4369)(148.125,30.1516)(150.156,29.8404)(152.188,29.5062)(154.219,29.1467)(156.25,28.7533)(158.281,28.318)(160.312,27.8413)(162.344,27.327)(164.375,26.7835)(166.406,26.2165)(168.438,25.6263)(170.469,25.0142)(172.5,24.3839)
		\drawline(172.5,38.3537)(174.531,38.0068)(176.562,37.6027)(178.594,37.1239)(180.625,36.5686)(182.656,35.9439)(184.688,35.2632)(186.719,34.5425)(188.75,33.8007)(190.781,33.0555)(192.812,32.3247)(194.844,31.6249)(196.875,30.9712)(198.906,30.3793)(200.938,29.8474)(202.969,29.3492)(205,28.8691)
		\drawline(205,32.1464)(207.031,31.7395)(209.062,31.252)(211.094,30.6815)(213.125,30.0415)(215.156,29.3547)(217.188,28.6518)(219.219,27.9658)(221.25,27.329)(223.281,26.7751)(225.312,26.3145)(227.344,25.9289)(229.375,25.5955)(231.406,25.2914)(233.438,24.9954)(235.469,24.6948)(237.5,24.3886)
		\drawline(237.5,29.6316)(239.531,29.0468)(241.562,28.5087)(243.594,27.9763)(245.625,27.4284)(247.656,26.8626)(249.688,26.2899)(251.719,25.7314)(253.75,25.2019)(255.781,24.7193)(257.812,24.2949)(259.844,23.9265)(261.875,23.6105)(263.906,23.3397)(265.938,23.1048)(267.969,22.8972)(270,22.7134)
		\drawline(10,35.0842)(12.0312,34.1393)(14.0625,33.3647)(16.0938,32.6972)(18.125,32.1005)(20.1562,31.5541)(22.1875,31.0486)(24.2188,30.5762)(26.25,30.1314)(28.2812,29.7061)(30.3125,29.2877)(32.3438,28.8729)(34.375,28.4748)(36.4062,28.1071)(38.4375,27.727)(40.4688,27.3208)(42.5,26.8869)
		\drawline(42.5,39.9973)(44.5312,39.7856)(46.5625,39.5543)(48.5938,39.2399)(50.625,38.8223)(52.6562,38.308)(54.6875,37.72)(56.7188,37.0905)(58.75,36.4572)(60.7812,35.8558)(62.8125,35.3197)(64.8438,34.8799)(66.875,34.5653)(68.9062,34.3605)(70.9375,34.2598)(72.9688,34.1794)(75,34.0739)
		\drawline(75,36.9761)(77.0312,36.7651)(79.0625,36.4468)(81.0938,35.9509)(83.125,35.2573)(85.1562,34.3883)(87.1875,33.3979)(89.2188,32.3583)(91.25,31.3495)(93.2812,30.4451)(95.3125,29.674)(97.3438,29.0404)(99.375,28.5498)(101.406,28.1593)(103.438,27.7883)(105.469,27.4012)(107.5,26.9805)
		\drawline(107.5,46.4891)(109.531,46.2094)(111.562,45.8694)(113.594,45.3429)(115.625,44.5811)(117.656,43.5845)(119.688,42.3876)(121.719,41.0519)(123.75,39.6455)(125.781,38.1973)(127.812,36.7538)(129.844,35.365)(131.875,34.0776)(133.906,32.9344)(135.938,31.993)(137.969,31.2657)(140,30.7097)
		\drawline(140,34.5302)(142.031,34.2446)(144.062,33.9911)(146.094,33.755)(148.125,33.5249)(150.156,33.2928)(152.188,33.0523)(154.219,32.7986)(156.25,32.5284)(158.281,32.2389)(160.312,31.9297)(162.344,31.6008)(164.375,31.2531)(166.406,30.8891)(168.438,30.5143)(170.469,30.1379)(172.5,29.7503)
		\drawline(172.5,33.2269)(174.531,32.8629)(176.562,32.4922)(178.594,32.0842)(180.625,31.6273)(182.656,31.1238)(184.688,30.5884)(186.719,30.0426)(188.75,29.5095)(190.781,28.9906)(192.812,28.4916)(194.844,28.0194)(196.875,27.5791)(198.906,27.173)(200.938,26.799)(202.969,26.4392)(205,26.0641)
		\drawline(205,36.7898)(207.031,36.298)(209.062,35.7745)(211.094,35.2207)(213.125,34.6496)(215.156,34.0813)(217.188,33.5402)(219.219,33.0532)(221.25,32.6558)(223.281,32.3499)(225.312,32.076)(227.344,31.8108)(229.375,31.546)(231.406,31.2709)(233.438,30.984)(235.469,30.6896)(237.5,30.3927)
		\drawline(237.5,39.0603)(239.531,38.7996)(241.562,38.4672)(243.594,38.0265)(245.625,37.4658)(247.656,36.7926)(249.688,36.0297)(251.719,35.2092)(253.75,34.3703)(255.781,33.555)(257.812,32.8133)(259.844,32.2084)(261.875,31.7092)(263.906,31.3047)(265.938,30.9818)(267.969,30.7316)(270,30.5182)
		\drawline(10,36.5926)(12.0312,35.851)(14.0625,35.2556)(16.0938,34.7577)(18.125,34.3096)(20.1562,33.8821)(22.1875,33.4594)(24.2188,33.0332)(26.25,32.6003)(28.2812,32.1592)(30.3125,31.7109)(32.3438,31.2542)(34.375,30.7886)(36.4062,30.3204)(38.4375,29.8563)(40.4688,29.3865)(42.5,28.9098)
		\drawline(42.5,28.9933)(44.5312,28.3423)(46.5625,27.773)(48.5938,27.257)(50.625,26.7678)(52.6562,26.2877)(54.6875,25.8095)(56.7188,25.3369)(58.75,24.8866)(60.7812,24.4874)(62.8125,24.1136)(64.8438,23.7517)(66.875,23.4028)(68.9062,23.0722)(70.9375,22.7654)(72.9688,22.4889)(75,22.2514)
		\drawline(75,44.2109)(77.0312,43.8722)(79.0625,43.6499)(81.0938,43.2248)(83.125,42.4552)(85.1562,41.3734)(87.1875,40.0805)(89.2188,38.6975)(91.25,37.3399)(93.2812,36.1077)(95.3125,35.0896)(97.3438,34.3712)(99.375,33.8377)(101.406,33.467)(103.438,33.2354)(105.469,33.0519)(107.5,32.838)
		\drawline(107.5,41.3684)(109.531,40.6855)(111.562,39.9689)(113.594,39.1653)(115.625,38.2987)(117.656,37.4517)(119.688,36.6574)(121.719,35.8432)(123.75,35.0185)(125.781,34.2268)(127.812,33.522)(129.844,32.9121)(131.875,32.4053)(133.906,32.0431)(135.938,31.8524)(137.969,31.7373)(140,31.6435)
		\drawline(140,36.5354)(142.031,36.4048)(144.062,36.2923)(146.094,36.1849)(148.125,36.0692)(150.156,35.9322)(152.188,35.7619)(154.219,35.5498)(156.25,35.2903)(158.281,34.9802)(160.312,34.6197)(162.344,34.2093)(164.375,33.7515)(166.406,33.2493)(168.438,32.7063)(170.469,32.1271)(172.5,31.5172)
		\drawline(172.5,40.4014)(174.531,39.465)(176.562,38.6085)(178.594,37.7997)(180.625,37.0192)(182.656,36.2531)(184.688,35.4948)(186.719,34.7458)(188.75,34.0132)(190.781,33.3073)(192.812,32.6374)(194.844,32.0082)(196.875,31.4191)(198.906,30.8692)(200.938,30.3598)(202.969,29.8926)(205,29.4675)
		\drawline(205,33.7742)(207.031,33.6585)(209.062,33.3953)(211.094,32.9284)(213.125,32.2651)(215.156,31.4472)(217.188,30.5318)(219.219,29.5794)(221.25,28.6446)(223.281,27.7698)(225.312,26.9827)(227.344,26.2975)(229.375,25.7242)(231.406,25.2918)(233.438,24.9823)(235.469,24.6992)(237.5,24.3942)
		\drawline(237.5,32.6998)(239.531,32.3046)(241.562,31.8435)(243.594,31.2889)(245.625,30.6339)(247.656,29.8862)(249.688,29.0667)(251.719,28.2088)(253.75,27.3555)(255.781,26.5579)(257.812,25.874)(259.844,25.3346)(261.875,24.96)(263.906,24.7024)(265.938,24.4563)(267.969,24.1826)(270,23.8732)
		\drawline(10,41.0666)(12.0312,40.3821)(14.0625,39.7191)(16.0938,39.0501)(18.125,38.3639)(20.1562,37.6599)(22.1875,36.9443)(24.2188,36.2263)(26.25,35.5186)(28.2812,34.8218)(30.3125,34.1328)(32.3438,33.4542)(34.375,32.7908)(36.4062,32.1533)(38.4375,31.535)(40.4688,30.9329)(42.5,30.3518)
		\drawline(42.5,39.8864)(44.5312,39.5473)(46.5625,39.1692)(48.5938,38.7118)(50.625,38.1628)(52.6562,37.5297)(54.6875,36.8344)(56.7188,36.1084)(58.75,35.3862)(60.7812,34.6983)(62.8125,34.0589)(64.8438,33.4648)(66.875,32.9113)(68.9062,32.3995)(70.9375,31.9317)(72.9688,31.5088)(75,31.0969)
		\drawline(75,41.3147)(77.0312,40.9231)(79.0625,40.398)(81.0938,39.7917)(83.125,39.1257)(85.1562,38.3422)(87.1875,37.3934)(89.2188,36.2934)(91.25,35.0944)(93.2812,33.8631)(95.3125,32.6686)(97.3438,31.5707)(99.375,30.6152)(101.406,29.8309)(103.438,29.245)(105.469,28.8369)(107.5,28.5359)
		\drawline(107.5,33.4124)(109.531,32.6114)(111.562,31.9356)(113.594,31.3257)(115.625,30.7558)(117.656,30.2332)(119.688,29.8168)(121.719,29.477)(123.75,29.1317)(125.781,28.7537)(127.812,28.3387)(129.844,27.8959)(131.875,27.4445)(133.906,27.0163)(135.938,26.6391)(137.969,26.3535)(140,26.1106)
		\drawline(140,34.642)(142.031,34.3616)(144.062,34.1924)(146.094,34.0797)(148.125,33.9898)(150.156,33.9044)(152.188,33.8139)(154.219,33.7123)(156.25,33.5977)(158.281,33.4679)(160.312,33.3218)(162.344,33.157)(164.375,32.968)(166.406,32.7481)(168.438,32.4903)(170.469,32.1884)(172.5,31.8407)
		\drawline(172.5,44.6901)(174.531,44.3192)(176.562,43.9437)(178.594,43.4722)(180.625,42.8688)(182.656,42.136)(184.688,41.3034)(186.719,40.4194)(188.75,39.5087)(190.781,38.5959)(192.812,37.7054)(194.844,36.857)(196.875,36.0659)(198.906,35.3446)(200.938,34.7111)(202.969,34.1859)(205,33.665)
		\drawline(205,37.2281)(207.031,36.683)(209.062,36.1339)(211.094,35.5899)(213.125,35.0909)(215.156,34.6719)(217.188,34.2576)(219.219,33.828)(221.25,33.3816)(223.281,32.9034)(225.312,32.3891)(227.344,31.8366)(229.375,31.2607)(231.406,30.6802)(233.438,30.1141)(235.469,29.5814)(237.5,29.1131)
		\drawline(237.5,30.3388)(239.531,30.027)(241.562,29.6922)(243.594,29.3528)(245.625,28.9621)(247.656,28.5027)(249.688,27.9871)(251.719,27.4208)(253.75,26.8178)(255.781,26.201)(257.812,25.5957)(259.844,25.0259)(261.875,24.5099)(263.906,24.0602)(265.938,23.6874)(267.969,23.3796)(270,23.127)
		\drawline(10,33.3801)(12.0312,32.9201)(14.0625,32.443)(16.0938,31.9493)(18.125,31.4414)(20.1562,30.9238)(22.1875,30.3999)(24.2188,29.8725)(26.25,29.3447)(28.2812,28.8184)(30.3125,28.2958)(32.3438,27.7793)(34.375,27.2714)(36.4062,26.7751)(38.4375,26.2942)(40.4688,25.8319)(42.5,25.3927)
		\drawline(42.5,39.6039)(44.5312,39.4071)(46.5625,39.1588)(48.5938,38.8334)(50.625,38.43)(52.6562,37.9628)(54.6875,37.4515)(56.7188,36.9152)(58.75,36.3745)(60.7812,35.8482)(62.8125,35.3548)(64.8438,34.9113)(66.875,34.5343)(68.9062,34.2437)(70.9375,34.0585)(72.9688,33.8906)(75,33.7014)
		\drawline(75,37.9347)(77.0312,37.4775)(79.0625,37.1044)(81.0938,36.7781)(83.125,36.4559)(85.1562,36.1038)(87.1875,35.7034)(89.2188,35.2526)(91.25,34.7612)(93.2812,34.2457)(95.3125,33.7261)(97.3438,33.2204)(99.375,32.7444)(101.406,32.3107)(103.438,31.93)(105.469,31.6158)(107.5,31.3816)
		\drawline(107.5,40.5007)(109.531,39.7605)(111.562,39.0041)(113.594,38.1795)(115.625,37.2649)(117.656,36.2548)(119.688,35.1537)(121.719,33.9833)(123.75,32.7791)(125.781,31.5816)(127.812,30.3945)(129.844,29.2375)(131.875,28.1346)(133.906,27.107)(135.938,26.1716)(137.969,25.342)(140,24.6306)
		\drawline(140,40.3618)(142.031,39.8331)(144.062,39.483)(146.094,39.2141)(148.125,38.9783)(150.156,38.7463)(152.188,38.5)(154.219,38.2282)(156.25,37.9243)(158.281,37.5845)(160.312,37.2069)(162.344,36.7896)(164.375,36.3322)(166.406,35.8341)(168.438,35.2952)(170.469,34.7167)(172.5,34.1003)
		\drawline(172.5,35.3739)(174.531,34.453)(176.562,33.6187)(178.594,32.8517)(180.625,32.146)(182.656,31.457)(184.688,30.7615)(186.719,30.0627)(188.75,29.3733)(190.781,28.7105)(192.812,28.0928)(194.844,27.5363)(196.875,27.0502)(198.906,26.6365)(200.938,26.2914)(202.969,26.0114)(205,25.7819)
		\drawline(205,35.0222)(207.031,34.5529)(209.062,34.1272)(211.094,33.7203)(213.125,33.3093)(215.156,32.867)(217.188,32.3921)(219.219,31.8968)(221.25,31.3974)(223.281,30.909)(225.312,30.4425)(227.344,30.0023)(229.375,29.5874)(231.406,29.1943)(233.438,28.8179)(235.469,28.4514)(237.5,28.0917)
		\drawline(237.5,38.7238)(239.531,38.1366)(241.562,37.4933)(243.594,36.7332)(245.625,35.8518)(247.656,34.8716)(249.688,33.836)(251.719,32.8118)(253.75,31.8786)(255.781,31.0553)(257.812,30.3572)(259.844,29.7822)(261.875,29.2826)(263.906,28.8301)(265.938,28.4062)(267.969,27.9993)(270,27.6037)
		\drawline(10,31.2925)(12.0312,31.0477)(14.0625,30.8495)(16.0938,30.6798)(18.125,30.5314)(20.1562,30.3996)(22.1875,30.2803)(24.2188,30.1672)(26.25,30.0536)(28.2812,29.9327)(30.3125,29.7993)(32.3438,29.6482)(34.375,29.4759)(36.4062,29.2794)(38.4375,29.0563)(40.4688,28.8052)(42.5,28.5261)
		\drawline(42.5,38.189)(44.5312,37.8352)(46.5625,37.4959)(48.5938,37.1033)(50.625,36.6358)(52.6562,36.0943)(54.6875,35.4926)(56.7188,34.8496)(58.75,34.1863)(60.7812,33.5246)(62.8125,32.8839)(64.8438,32.2803)(66.875,31.7254)(68.9062,31.2312)(70.9375,30.8276)(72.9688,30.466)(75,30.0892)
		\drawline(75,30.4837)(77.0312,29.97)(79.0625,29.4937)(81.0938,29.0491)(83.125,28.617)(85.1562,28.1966)(87.1875,27.7957)(89.2188,27.4203)(91.25,27.0762)(93.2812,26.7587)(95.3125,26.4531)(97.3438,26.1475)(99.375,25.8379)(101.406,25.5269)(103.438,25.2242)(105.469,24.929)(107.5,24.628)
		\drawline(107.5,32.4632)(109.531,32.0036)(111.562,31.6778)(113.594,31.384)(115.625,31.0583)(117.656,30.6791)(119.688,30.2436)(121.719,29.7609)(123.75,29.2516)(125.781,28.7559)(127.812,28.3032)(129.844,27.8889)(131.875,27.5292)(133.906,27.2708)(135.938,27.0873)(137.969,26.9342)(140,26.7936)
		\drawline(140,38.4713)(142.031,37.7921)(144.062,37.1404)(146.094,36.491)(148.125,35.8371)(150.156,35.1754)(152.188,34.5046)(154.219,33.8252)(156.25,33.1379)(158.281,32.4452)(160.312,31.7493)(162.344,31.0535)(164.375,30.3615)(166.406,29.6766)(168.438,29.0029)(170.469,28.3442)(172.5,27.7047)
		\drawline(172.5,37.1815)(174.531,36.8713)(176.562,36.5944)(178.594,36.3099)(180.625,35.9684)(182.656,35.5474)(184.688,35.0441)(186.719,34.4684)(188.75,33.8386)(190.781,33.1777)(192.812,32.5108)(194.844,31.8602)(196.875,31.244)(198.906,30.6744)(200.938,30.1559)(202.969,29.6812)(205,29.2287)
		\drawline(205,34.0475)(207.031,33.6722)(209.062,33.3125)(211.094,32.9869)(213.125,32.6948)(215.156,32.3761)(217.188,31.9709)(219.219,31.4609)(221.25,30.8646)(223.281,30.2205)(225.312,29.5763)(227.344,28.9856)(229.375,28.4787)(231.406,28.0408)(233.438,27.6683)(235.469,27.3699)(237.5,27.1293)
		\drawline(237.5,34.8535)(239.531,34.1536)(241.562,33.569)(243.594,33.1329)(245.625,32.864)(247.656,32.6619)(249.688,32.4233)(251.719,32.0866)(253.75,31.6351)(255.781,31.0834)(257.812,30.4599)(259.844,29.7928)(261.875,29.1087)(263.906,28.4334)(265.938,27.7928)(267.969,27.2217)(270,26.757)
		\drawline(10,31.6373)(12.0312,31.2446)(14.0625,30.9268)(16.0938,30.5886)(18.125,30.2075)(20.1562,29.7778)(22.1875,29.3007)(24.2188,28.7809)(26.25,28.2243)(28.2812,27.6372)(30.3125,27.0257)(32.3438,26.3953)(34.375,25.7509)(36.4062,25.0972)(38.4375,24.4386)(40.4688,23.7797)(42.5,23.1257)
		\drawline(42.5,34.8168)(44.5312,34.2715)(46.5625,33.8754)(48.5938,33.5528)(50.625,33.2304)(52.6562,32.859)(54.6875,32.4229)(56.7188,31.92)(58.75,31.3575)(60.7812,30.7502)(62.8125,30.1173)(64.8438,29.4805)(66.875,28.8614)(68.9062,28.2807)(70.9375,27.7567)(72.9688,27.3038)(75,26.9267)
		\drawline(75,32.4311)(77.0312,32.1453)(79.0625,31.87)(81.0938,31.5668)(83.125,31.2026)(85.1562,30.7652)(87.1875,30.2703)(89.2188,29.7581)(91.25,29.2262)(93.2812,28.6777)(95.3125,28.1239)(97.3438,27.579)(99.375,27.0567)(101.406,26.5671)(103.438,26.1197)(105.469,25.7277)(107.5,25.4148)
		\drawline(107.5,39.6604)(109.531,38.3847)(111.562,37.3267)(113.594,36.493)(115.625,35.8022)(117.656,35.1628)(119.688,34.5187)(121.719,33.8481)(123.75,33.1524)(125.781,32.445)(127.812,31.7428)(129.844,31.0603)(131.875,30.4135)(133.906,29.8233)(135.938,29.3243)(137.969,28.9359)(140,28.6441)
		\drawline(140,39.5183)(142.031,38.8388)(144.062,38.1819)(146.094,37.5431)(148.125,36.9298)(150.156,36.364)(152.188,35.864)(154.219,35.3956)(156.25,34.9405)(158.281,34.4773)(160.312,33.9946)(162.344,33.4876)(164.375,32.9548)(166.406,32.3988)(168.438,31.8257)(170.469,31.2438)(172.5,30.664)
		\drawline(172.5,32.856)(174.531,32.614)(176.562,32.3704)(178.594,32.0977)(180.625,31.7824)(182.656,31.4225)(184.688,31.025)(186.719,30.6022)(188.75,30.1717)(190.781,29.7518)(192.812,29.3631)(194.844,29.0263)(196.875,28.7362)(198.906,28.4881)(200.938,28.2756)(202.969,28.0792)(205,27.8727)
		\drawline(205,32.7152)(207.031,32.5343)(209.062,32.362)(211.094,32.167)(213.125,31.9319)(215.156,31.6479)(217.188,31.3129)(219.219,30.9309)(221.25,30.5095)(223.281,30.0584)(225.312,29.5902)(227.344,29.1179)(229.375,28.654)(231.406,28.21)(233.438,27.7964)(235.469,27.4273)(237.5,27.0965)
		\drawline(237.5,33.6815)(239.531,33.285)(241.562,32.8354)(243.594,32.346)(245.625,31.7363)(247.656,30.9803)(249.688,30.103)(251.719,29.1467)(253.75,28.1585)(255.781,27.1838)(257.812,26.2621)(259.844,25.4281)(261.875,24.7241)(263.906,24.1361)(265.938,23.6271)(267.969,23.189)(270,22.8121)
		\drawline(10,37.8811)(12.0312,37.4684)(14.0625,37.1512)(16.0938,36.8862)(18.125,36.6505)(20.1562,36.431)(22.1875,36.2193)(24.2188,36.0091)(26.25,35.7949)(28.2812,35.5719)(30.3125,35.3366)(32.3438,35.0859)(34.375,34.8179)(36.4062,34.5319)(38.4375,34.2264)(40.4688,33.9006)(42.5,33.5534)
		\drawline(42.5,32.3902)(44.5312,32.1085)(46.5625,31.7798)(48.5938,31.4041)(50.625,30.9862)(52.6562,30.5355)(54.6875,30.064)(56.7188,29.5852)(58.75,29.1123)(60.7812,28.6543)(62.8125,28.2127)(64.8438,27.7836)(66.875,27.3664)(68.9062,26.9613)(70.9375,26.5668)(72.9688,26.1785)(75,25.7921)
		\drawline(75,37.4472)(77.0312,37.2606)(79.0625,37.0416)(81.0938,36.7679)(83.125,36.4368)(85.1562,36.0583)(87.1875,35.6506)(89.2188,35.2361)(91.25,34.8381)(93.2812,34.455)(95.3125,34.0886)(97.3438,33.745)(99.375,33.433)(101.406,33.1667)(103.438,32.9555)(105.469,32.7966)(107.5,32.6276)
		\drawline(107.5,43.0454)(109.531,42.8235)(111.562,42.6223)(113.594,42.4307)(115.625,42.2385)(117.656,42.024)(119.688,41.7552)(121.719,41.4)(123.75,40.9392)(125.781,40.3691)(127.812,39.7014)(129.844,38.9575)(131.875,38.1689)(133.906,37.3523)(135.938,36.5138)(137.969,35.6705)(140,34.8417)
		\drawline(140,41.3931)(142.031,40.4835)(144.062,39.7022)(146.094,39.0137)(148.125,38.4194)(150.156,37.9039)(152.188,37.4041)(154.219,36.8758)(156.25,36.306)(158.281,35.6945)(160.312,35.0452)(162.344,34.3631)(164.375,33.6542)(166.406,32.9253)(168.438,32.1859)(170.469,31.4474)(172.5,30.7296)
		\drawline(172.5,35.5875)(174.531,35.2677)(176.562,34.9795)(178.594,34.6842)(180.625,34.3629)(182.656,34.008)(184.688,33.6172)(186.719,33.1927)(188.75,32.7412)(190.781,32.2736)(192.812,31.803)(194.844,31.3428)(196.875,30.9056)(198.906,30.5006)(200.938,30.1325)(202.969,29.7982)(205,29.4801)
		\drawline(205,33.9827)(207.031,33.6802)(209.062,33.3968)(211.094,33.0943)(213.125,32.7529)(215.156,32.3646)(217.188,31.9358)(219.219,31.4843)(221.25,31.0332)(223.281,30.6046)(225.312,30.2148)(227.344,29.8697)(229.375,29.5651)(231.406,29.2898)(233.438,29.03)(235.469,28.7699)(237.5,28.5035)
		\drawline(237.5,38.7849)(239.531,38.3039)(241.562,37.8489)(243.594,37.4487)(245.625,37.1198)(247.656,36.855)(249.688,36.657)(251.719,36.5014)(253.75,36.298)(255.781,36.0053)(257.812,35.6096)(259.844,35.116)(261.875,34.5419)(263.906,33.9112)(265.938,33.2512)(267.969,32.5904)(270,31.9568)
		\put(5,7){\tiny 0}
		\put(0,40){$\zeta$}
		\put(5,70){\tiny 3}
		\put(8,2){\tiny $0$}
		\put(37,2){\tiny $\pi/4$}
		\put(70,2){\tiny $\pi/2$}
		\put(100,2){\tiny $3\pi/4$}
		\put(138,2){\tiny $\pi$}
		\put(165,2){\tiny $5\pi/4$}
		\put(198,2){\tiny $3\pi/2$}
		\put(230,2){\tiny $7\pi/4$}
		\put(265,2){\tiny $2\pi$}
		\put(152,-7){$\tau$}
		\put(175,75){\makebox(96,4)[r]{$R=2000,\,\alpha=15,\,n=1$}}
	\end{picture}
	\caption{\label{fig:awp:asy}Antimetrische St\"orungen k\"onnen zu bestimmten Zeiten verst\"arkt werden.}
\end{figure}
\newpage

Abschlie\ss end l\"a\ss t sich zusammenfassen, da\ss\ sich die oszillierende Rohrstr\"omung im untersuchten Parameterbereich asymptotisch stabil gegen\"uber infinitessimalen St\"orungen verh\"alt.
Die Floquetsche Theorie ist hierbei die geeignete Methode um den Stabilit\"atsnachweis zu erbringen.\\
Trotz der asymptotischen Stabilit\"at gibt es Bereiche, in denen St\"orungen kurzzeitig angefacht werden und zu einem zeitlich begrenzten turbulenten Erscheinungsbild f\"uhren k\"onnen.
Um diese zu erkennen, hat sich die quasistatische Theorie bew\"ahrt.
Mit einem Verfahren basierend auf der Galerkin-Entwicklung wurden Parameterstudien durchgef\"uhrt.
Die Parameter umfassen die beiden Kenngr\"o\ss en der Grundstr\"omung, Womersley- und Reynoldszahl, den Zeitparameter und die angulare und axiale Wellenzahl einer St\"orungskomponente.
In umfangreichen numerischen Berechnungen wurden die Aufklingrate und die Kreisfrequenz der St\"orung als Eigenwerte und die radiale Verteilung als Eigenfunktion ermittelt.\\

Im klassischen Sinne folgt aus den Eigenwerten eines dynamischen Systems mit konstanten Koeffizienten ein asymptotischer Stabilit\"atsbegriff f\"ur lange Zeiten.
Das exponentielle Wachstum ist unter Umst\"anden erst nach gewisser Zeit erkennbar, schlie\ss lich setzt sich die L\"osung aus einer \"Uberlagerung von Schwingungen zusammen.
Daher ist die \"Ubertragung des Stabilit\"atsbegriffes auf instation\"are Systeme heikel.
Erstens sind die Koeffizienten des Systems durch die Instationarit\"at nicht mehr konstant, wodurch die L\"osung keine exponentielle Gestalt annimmt.
Zweitens mag sich die Grundstr\"omung---und damit die Eigenwerte---ge\"andert haben, bevor eine St\"orung mit positiver Aufklingrate wesentlich angewachsen ist.\\
Die Aussagekraft der Eigenwerte als Stabilit\"atsbegriff f\"ur instation\"are Systeme ist daher nur in abgeschw\"achter Form als quasistatische Stabilit\"at g\"ultig.
N\"utzlich zur Beurteilung dieser Asymptotik ist der Quasistatikparameter beziehungsweise die Aufklingrate in der Oszillationszeitskala.\\
Zur weiteren Interpretation der quasistatischen Eigenwerte, haben wir das System als Anfangswertproblem formuliert und numerisch integriert.\\

\begin{figure}[htbp] % instabilitaetsgebiete
	\begin{picture}(275,82)
		\drawline(87,12)(187,12)
		\drawline(87,12)(87,80)
		\put(190,0){$\alpha$}
		\put(73,78){$R$}
		\drawline(133,12)(133,9)\put(129,0){\small 10}
		\drawline(180,12)(180,9)\put(175,0){\small 20}
		\drawline(84,31)(87,31)\put(63,67){\small 3000}
		\drawline(84,50)(87,50)\put(63,47){\small 2000}
		\drawline(84,70)(87,70)\put(63,28){\small 1000}
		\thicklines %\drawline(180,50)(105.5,23.5)(96,70)
		\drawline(180,50)(116,28.5)(109.5,26.5)(105.5,26.5)(102.8,29.9)(99.5,43.7)(96,70)
		\thinlines
		\drawline(101.4,35.5)(104.2,42.5)(108.4,51.1)(116.5,60.3)(126.5,70)
		\drawline(125.3,31.6)(130.7,37.6)(135.4,44.2)(142,54.5)(151,69.3)
		\put(105,60){I}
		\put(117,44){III}
		\put(160,55){II}
		\put(140,20){stabil}
	\end{picture}
	\caption{\label{fig:instabilitaetsgebiete}Gebiete der quasistatischen Instabilit\"at (schematisch). \,I: Merkliche Verst\"arkung vornehmlich in der Rohrmitte. II: Schwache Instabilit\"at der wandnahen Grenzschicht. III: Mischform ohne erkennbare Verst\"arkung.}
\end{figure}
Aus beiden Berechnungen, sowohl der quasistatischen Eigenwertbestimmung als auch des Anfangswertproblems, ziehen wir folgende Schl\"usse:
Das quasistatische Instabilit\"atsgebiet l\"a\ss t sich \"uber der $R$-$\alpha$-Ebene in drei Bereiche aufteilen.
Diese sind in Abbildung \ref{fig:instabilitaetsgebiete} skizziert.
In allen Bereichen gibt es einen Zeitpunkt mit positiver Aufklingrate im quasistatischen Sinne, dennoch unterscheiden sich die Gebiete bez\"uglich ihres Ph\"anotypes.\\
In Bereich I gibt es wachsende St\"orungen.
Die Aufklingraten in der Oszillationszeitskala sind gro\ss\ genug, da\ss\ sie trotz ver\"anderlicher Grundstr\"omung um die Gr\"o\ss enordnung $10^1$ verst\"arkt werden k\"onnen.
Aus den Eigenfunktionen geht hervor, da\ss\ haupt"-s\"ach"-lich St\"orungen in der Rohrmitte verst\"arkt werden.
Hingegen konzentriert sich die Wirksamkeit der St\"orungen f\"ur go\ss e Womersleysche Zahlen auf die wandnahen Grenzschicht.
Diesen Bereich bezeichnen wir als Gebiet II.
Die zugeh\"origen Aufklingraten sind allerdings nicht gro\ss\ genug, da\ss\ die Instabilit\"aten sichtbar werden.
Gleiches gilt f\"ur Bereich III.
In diesem Mischgebiet wandeln sich die Eigenfunktionen der am st\"arksten angefachten St\"orungen mit nur m\"a\ss igen Verst\"arkungen von den zentralen Formen zur Grenzschichtform.\\

Die Interaktion von kleinen St\"orungen mit der Grund"-str\"o"-mung, wie sie durch die lineare Theorie beschrieben wird, ist der zentrale Instabilit\"atsmechanismus.
Dennoch reicht er nicht aus, den Zusammenbruch der laminaren Str\"omung zu beschreiben.
Beispielsweise hat die station\"are Rohrstr\"omung nach dieser Theorie keinen Bereich positiver Aufklingraten, obwohl wir im realen System eine Stabilt\"atsgrenze beobachten.
Daher mu\ss\ es noch andere Mechanismen geben, welche die Grundstr\"omung beeinflussen k\"onnen.\\
Nichtlineare Effekte tragen zwar nichts zum Energiehaushalt der St\"orungen bei, k\"onnen jedoch die Energie zwischen den Freiheitsgraden umverteilen.
Au\ss erdem k\"onnen sie den Grundstr\"omungszustand soweit verzerren, da\ss\ er eine instabilere Form einnimmt, die wiederum durch den linearen Mechanismus ge"-st\"ort werden kann.
Bevor diese Effekte auftreten k\"onnen, m\"ussen die als klein angenommenen St\"orungen eine gewisse Gr\"o\ss e erreichen.\\
Eine M\"oglichkeit dazu bieten neben der behandelten linearen Interaktion transiente Verst\"arkungen durch nichtorthogonale Eigenfunktionen.
Um diesen Effekt zu erl\"autern, geben wir im Anschlu\ss\ ein Modellbeispiel.\\
%Der Effekt beruht darauf, da\ss\ eine Anfangsst\"orung kleiner Amplitude, aus einer \"Uberlagerung nichtorthogonaler Eigenfunktionen bestehen kann, welche ihrerseits eine wesentlich gr\"o\ss ere Amplitude besitzen.
%Im Laufe der zeitlichen Entwicklung k\"onnten sich die Anteile der Eigenfunktionen \"uberlagern, welche sich vorher nahezu aufhoben.
%Dadurch kann es zu einer zeitlich begrenzten Verst\"arkung der Amplitude der St\"orung kommen.\\
Die Eigenschaften der nichtlinearen und nichtorthogonalen Effekte sind wesentlich komplizierter zu analysieren, als jene der linearen Interaktion.
Au\ss erdem lassen sich nur spezielle Konfigurationen analysieren, welche nicht die Gesamtheit der St\"orungen widerspiegeln.\\
Wir k\"onnen davon ausgehen, da\ss\ auch im vorliegenden Falle der oszillierenden Rohrstr\"omung der lineare Instabilit\"atsmechanismus nicht alleine f\"ur das Einsetzen der turbulenten Str\"omung verantwortlich ist.
Im Gegensatz zur station\"aren Str\"omung tr\"agt er jedoch unmittelbar dazu bei.\\

\paragraph{Nichtorthogonalit\"at}
Mit einem einfachen Modell wollen wir den Effekt transienter Verst\"arkung infolge nichtorthogonaler Eigenfunktionen demonstrieren.\\
Dazu betrachten wir ein lineares dynamisches System zweiter Ordnung mit dem Zustandsvektor $x(t)=(u,v)$
\begin{equation}
	\frac{\mathrm{d}x}{\mathrm{d}t} = Ax.
\end{equation}
Die Systemmatrix $A$ habe Eigenwerte $\lambda_{1,2}$ mit negativem Realteil
\begin{equation}
	\Re\lambda_{1,2}<0
\end{equation}
und nichtorthogonale Eigenvektoren
\begin{equation}
	x_1 = (1,0) \qquad x_2 = (\cos{\varphi},\sin\varphi).
\end{equation}
Um den Effekt der nichtorthogonalen Eigenvektoren hervorzuheben soll der eingeschlossene Winkel $\varphi$ klein sein $0<\varphi\ll\pi/2$.\\
Die Eigenwerte und Eigenvektoren korrespondieren mit der Systemmatrix
\begin{equation}
	A = \begin{pmatrix}\lambda_1&(\lambda_2-\lambda_1)\cot\varphi\\0&\lambda_2\end{pmatrix}.
\end{equation}
Die allgemeine L\"osung des homogenen Systems setzt sich linear aus den Basisl\"osungen zusammen
\begin{equation}
	x(t) = \alpha x_1 e^{\lambda_1 t} + \beta x_2 e^{\lambda_2 t}.
\end{equation}
W\"ahlen wir als Anfangsbedingung einen Vektor, welcher orthogonal zur Vorzugsrichtung des Systems steht
\begin{equation}
	x(0) = (0,1),
\end{equation}
dann lautet die L\"osung
\begin{eqnarray}
	u(t) &=& (e^{\lambda_2 t} - e^{\lambda_1 t}) \cot\varphi \\
	v(t) &=& e^{\lambda_2 t}.
\end{eqnarray}
Da die Realteile der Eigenwerte $\lambda_{1,2}$ negativ sind, n\"ahern sich beide Komponenten f\"ur lange Zeiten exponentiell der Null\"osung.
Trotzdem w\"achst die Komponente in Vorzugsrichtung des Systems $u$ zwischenzeitlich stark, da der Faktor $\cot\varphi$ f\"ur kleine Winkel wie $1/\varphi$ singul\"ar wird.
Das h\"angt damit zusammen, da\ss\ die Faktoren $\alpha$ und $\beta$ sehr gro\ss\ sein m\"ussen, um eine endliche Anfangskonfiguration $x(0)$ darzustellen, welche orthogonal zur Vorzugsrichtung des Systems ist.
Zum Zeitpunkt $t=0$ kompensieren sich die Anteile nahezu gegenseitig.
Wenn die Eigenwerte ungleich sind $\lambda_1\neq\lambda_2$, dann klingen die Anteile im weiteren Verlauf unterschiedlich schnell ab, soda\ss\ vor\"ubergehend einer der beiden Anteile dominiert.\\
\begin{figure}[ htbp ]
	\begin{center}
	\begin{picture}(150,110)
	\thicklines\drawline(40,110)(60,110)\thinlines\put(65,108){$u$}
	\drawline(85,110)(105,110)\put(110,108){$v$}
	\drawline(15,10)(140,10)(140,100)(15,100)(15,10)
	\drawline(15,32.5)(140,32.5)
	\drawline(15,55)(140,55)
	\drawline(15,77.5)(140,77.5)
	\drawline(46.25,10)(46.25,100)
	\drawline(77.5,10)(77.5,100)
	\drawline(108.5,10)(108.5,100)
	\put(14,5){\tiny 0}
	\put(42.5,5){\tiny 1/2}
	\put(76,5){\tiny 1}
	\put(104.5,5){\tiny 3/2}
	\put(139,5){\tiny 2}
	\put(10,8){\tiny 0}
	\put(10,30.5){\tiny 1}
	\put(10,53.5){\tiny 2}
	\put(10,75.5){\tiny 3}
	\put(10,98){\tiny 4}
	\put(90,-4){$t$}
	\thicklines\drawline(15,10)(17.0833,32.7806)(19.1667,49.4826)(21.25,61.5522)(23.3333,70.0982)(25.4167,75.968)(27.5,79.8103)(29.5833,82.1208)(31.6667,83.2794)(33.75,83.575)(35.8333,83.2296)(37.9167,82.4117)(40,81.2503)(42.0833,79.8429)(44.1667,78.2645)(46.25,76.5712)(56.6667,67.5595)(67.0833,59.0417)(77.5,51.5969)(87.9167,45.233)(98.3333,39.83)(108.75,35.252)(119.167,31.3758)(129.583,28.0943)(140,25.3166)\thinlines
	\drawline(15,32.5)(17.0833,31.7624)(19.1667,31.0489)(21.25,30.3588)(23.3333,29.6914)(25.4167,29.0458)(27.5,28.4214)(29.5833,27.8175)(31.6667,27.2334)(33.75,26.6684)(35.8333,26.122)(37.9167,25.5934)(40,25.0822)(42.0833,24.5878)(44.1667,24.1095)(46.25,23.6469)(56.6667,21.5519)(67.0833,19.7785)(77.5,18.2773)(87.9167,17.0066)(98.3333,15.9309)(108.75,15.0204)(119.167,14.2497)(129.583,13.5973)(140,13.045)
	\end{picture}
	\end{center}
	\caption{Wird ein ged\"ampftes System orthogonal zu seiner Vorzugsrichtung angeregt, so weicht es parallel zur Vorzugsrichtung aus, bevor es asymptotisch abklingt.
	Die Intensit\"at der Verst\"arkung ist das Produkt aus dem Kotangens des eingeschlossenen Winkels der Eigenvektoren und der Differenz der Exponentiall\"osungen.
	In diesem Beispiel ist $\lambda_1=-8$, $\lambda_2=-1$ und $\varphi=\pi/16$.}
\end{figure}

\clearpage
%% Experimente
\chapter{Vergleich mit Experimenten}\label{sec:experimente}
	%Neben den Station\"aren Experimenten \textsl{reynolds} und 100 Jahre sp\"ater in Manchester\textsl{Darbyshire,Mullin}\footnote{\label{bib:darbyshire_mullin}\textsl{Darbyshire, Mullin}: \textit{J.\ Fluid.\ Mech.} \textbf{289} (1995)} auch instation\"are Str\"o"-mungen.\\
	%Transiente Experimente\footnote{\label{bib:he_jackson}\textsl{He, Jackson}: \textit{J.\ Fluid.\ Mech.} \textbf{408} (2000)}, schrittweise linear\footnote{\label{bib:das_arakeri}\textsl{Das, Arakeri}: \textit{Journal of Fluid Mechanics} \textbf{374} (1998)}\\
	%\footnote{\label{bib:yu_zhao}\textsl{Yu, Zhao}: \textit{Experiments in Fluids} \textbf{26} (1999)} mit Ausbeulung.

	%Ohne Hauptanteil: Allgemein unterscheiden die Jungs 3(4) Gebiete, im leicht turbulenten (2) geschiehts in der acc Phase, im schlimmeren Fall in der dec phase sehr sprunghaft, und wir in der acc phase wieder stabilisiert, auch f\"ur recht hohe R.
	%\textsl{Hino}\footnote{\label{bib:hino_sawamoto}\textsl{Hino, Sawamoto, Takasu}: \textit{J.\ Fluid.\ Mech.} \textbf{75} (1976)}: Hot wire, Machen drei Gebiete aus, zeichnen stabilit\"atsdiagramm \"uber Rd,lambda, oder Re,lambda. Zweite kritische Rey"-noldszahl liegt konstant bei Rd550, erste Kritische h\"angt von lambda ab, und f\"allt ab. In diesem Bereich stimmts mit Rechnungen ganz gut.
	%\textsl{mit anmerken}\footnote{\label{bib:ohmi_igushi}\textsl{Ohmi, Iguchi, Kakehashi, Masuda}: \textit{Bull.\ JSME} \textbf{25} (1982)}
	%\textsl{Akhavan}\footnote{\label{bib:akhavan1}\textsl{Akhavan, Kamm, Sharpiro}: \textit{J.\ Fluid.\ Mech.} \textbf{225} (1991)} turbulent in acc phase, zu hohe R, oberhalb der kritischen Werte.
	%\textsl{Eckmann}\footnote{\label{bib:eckmann_grotberg}\textsl{Eckmann, Grotberg}: \textit{J.\ Fluid.\ Mech.} \textbf{222} (1991)}: LDV f\"ur hochfrequente Anregungen, Instabilit\"at in der Grenzschicht f\"ur $R^\delta>500$, kein schwach turbulenter Bereich.
	%\textsl{Merkli,Thomann}\footnote{\label{bib:merkli_thomann}\textsl{Merkli, Thomann}: \textit{J.\ Fluid.\ Mech.} \textbf{68} (1975)}: hot wire, smoke visu, Luft, geschlossenes Rohr.

Es wurden in der Vergangenheit au\ss er Experimenten an der station\"aren Rohrstr\"o"-mung\footnote{\label{bib:darbyshire_mullin}\textsl{Darbyshire, Mullin}: \textit{J.\ Fluid.\ Mech.} \textbf{289} (1995)}
auch einige Experimente an instation\"aren Str\"o"-mungen durchgef\"uhrt.
Bei diesen unterscheiden wir transiente und oszillierende Str\"o"-mungsvorg\"ange.\\
\textsl{He und Jackson}\footnote{\label{bib:he_jackson}\textsl{He, Jackson}: \textit{J.\ Fluid.\ Mech.} \textbf{408} (2000)}
untersuchen eine Rohrstr\"omung, deren Flu\ss\ linear zwischen zwei konstanten Werten ver\"andert wird.
\textsl{Das und Arakeri}\footnote{\label{bib:das_arakeri}\textsl{Das, Arakeri}: \textit{J.\ Fluid.\ Mech.} \textbf{374} (1998)}
untersuchen die gleiche Konfiguration und betonen hierbei die instabilen Eigenschaften eines Geschwindigkeitsprofils mit R\"uckstr\"omung, welches durch eine schnelle Abbremsung des Flusses hervorgeht.\\
Die oszillierenden Str\"o"-mungen gliedern sich in solche mit einem \"uberlagerten station\"aren Anteil\footnote{\label{bib:lodahl_sumer}\textsl{Lodahl, Sumer, Fredsoe}: \textit{J.\ Fluid.\ Mech.} \textbf{373} (1998)\\\bibspace\label{bib:sarpkaya}\textsl{Sarpkaya}: \textit{J.\ Basic Eng.} \textbf{Sep} (1966)\\\bibspace\label{bib:shemer_wygnanski}\textsl{Shemer, Wygnanski, Kit}: \textit{J.\ Fluid.\ Mech.} \textbf{153} (1985)}
und jene, deren Flu\ss\ rein harmonisch oszilliert.
Letztere lassen sich mit den Rechnungen dieser Arbeit vergleichen.\\

\textsl{Hino, Sawamoto und Takasu}\footnote{\label{bib:hino_sawamoto}\textsl{Hino, Sawamoto, Takasu}: \textit{J.\ Fluid.\ Mech.} \textbf{75} (1976)}
sowie \textsl{Omhi, Igushi, Kakehashi und Masuda}\footnote{\label{bib:ohmi_igushi}\textsl{Ohmi, Iguchi, Kakehashi, Masuda}: \textit{Bull.\ JSME} \textbf{25} (1982)}
untersuchen die Str\"o"-mung mit der Hitzdrahtanemometrie.
Im wesentlichen unterscheiden sie drei charakteristische Bereiche:
\begin{itemize}
	\item Stabilit\"at: Die Str\"o"-mung ist w\"ahrend des gesamten Zyklus stabil.
	\item Transition: In der Beschleunigungsphase kommt es zu leichten Unregelm\"a\ss igkeiten im Str\"o"-mungsbild.
	\item Turbulenz: In der Verz\"ogerungsphase setzen sprunghaft turbulente St\"or"-ungen ein, welche jedoch auch f\"ur hohe Rey"-noldszahlen nie \"uber den gesamten Zyklus anhalten.
\end{itemize}
Der \"Ubergang vom schwach turbulenten zum stark turbulenten Bereich, findet laut den Untersuchungen bei $R/\alpha > 700$ statt.
Dieser steht auch im Blickpunkt neuerer Untersuchungen.\footnote{\label{bib:akhavan1}\textsl{Akhavan, Kamm, Sharpiro}: \textit{J.\ Fluid.\ Mech.} \textbf{225} (1991)\\\bibspace\label{bib:eckmann_grotberg}\textsl{Eckmann, Grotberg}: \textit{J.\ Fluid.\ Mech.} \textbf{222} (1991)}
In der vorliegenden Arbeit ist er jedoch nicht von Bedeutung.
Wir beschr\"anken uns auf den \"Ubergang vom stabilen zum schwach turbulenten Bereich, welcher bei kleineren Grenzschicht-Rey"-noldszahlen stattfindet.\\

\begin{figure}[htbp]	% Vergleich mit Hinos Experimenten
	\begin{center}
	%\input{fig/hino}
		\begin{picture}(188,132)(0,0)
		%\drawline(0,0)(0,130)(188,130)(188,0)(0,0)
		\put(161,-3){$\alpha$}
		\put(-10,96){$R$}
		\thinlines
		\drawline(16,10)(16,130)
		\drawline(58.5,10)(58.5,130)
		\drawline(101,10)(101,130)
		\drawline(143.5,10)(143.5,130)
		\drawline(186,10)(186,130)
		\drawline(16,10)(186,10)
		\drawline(16,30)(186,30)
		\drawline(16,50)(186,50)
		\drawline(16,70)(186,70)
		\drawline(16,90)(186,90)
		\drawline(16,110)(186,110)
		\drawline(16,130)(186,130)
		\put(14,4){\tiny 0}
		\put(56.5,4){\tiny 5}
		\put(97,4){\tiny 10}
		\put(139.5,4){\tiny 15}
		\put(182,4){\tiny 20}
		\put(0,8){\makebox(15,4)[r]{\tiny 0}}
		\put(0,28){\makebox(15,4)[r]{\tiny 500}}
		\put(0,48){\makebox(15,4)[r]{\tiny 1000}}
		\put(0,68){\makebox(15,4)[r]{\tiny 1500}}
		\put(0,88){\makebox(15,4)[r]{\tiny 2000}}
		\put(0,108){\makebox(15,4)[r]{\tiny 2500}}
		\put(0,128){\makebox(15,4)[r]{\tiny 3000}}
		\thicklines
		\drawline(16,10)(186,90)
		\drawline(16,10)(52.4284,130)
		\drawline(69.0888,130)(69.6681,109.636)(70.2475,97.018)(70.7659,87.9636)(71.4367,78.218)(72.687,69.0908)(74.2422,60.8)(76.3157,54.2544)(78.1758,50.7636)(80.7372,47.9273)(83.3901,46.1454)(86.9273,45.2)(90.6475,45.0546)(94.4592,45.6727)(97.6609,46.4727)(103.547,48.5091)(108.242,50.3636)(113.67,52.7636)(128.855,60.182)(146.969,69.6)(186,90.8)
		\drawline(28.7157,130)(29.8745,110.909)(32.1004,86.4364)(34.3875,69.2364)(35.6376,63.018)(36.9488,57.2364)(38.2296,52.582)(39.6323,48.5818)(41.3704,44.3273)(43.4134,40.9454)(46.4628,37.3091)(49.5426,35.2)(52.8053,34.1818)(56.1901,34.4)(60.0018,35.6)(64.0574,37.4546)(68.5399,38.6909)(73.3273,39.7818)(78.6941,40.7273)(86.0736,42.8364)(92.9955,45.3091)(102.54,49.3818)(114.31,54.982)(125.075,60.4)(136.143,65.9272)(160.141,78.4364)(172.522,85.0544)(186,92.1456)

		\put(49.15,14.2){\circle{3}}
		\put(49.15,18.4){\circle{3}}
		\put(49.15,22.6){\circle{3}}
		\put(49.15,38.4){\circle{3}}
		\put(49.15,66.8){\circle{3}}
		\put(38.95,80){\circle{3}}
		\put(32.235,80.8){\circle{3}}
		\put(69.55,28){\circle{3}}
		\put(69.55,34){\circle{3}}
		\put(53.485,44.8){\circle{3}}
		\put(53.485,62.8){\circle{3}}
		\put(49.15,95.2){\circle*{3}}
		\put(62.75,66.8){\circle*{3}}
		\put(62.75,123.6){\circle*{3}}
		\put(90.375,44){\circle*{3}}
		\put(69.55,42){\circle*{3}}
		\put(53.485,80){\circle*{3}}
		\put(53.485,114.8){\circle*{3}}
		\put(49.15,80){\circle*{3}}
		\put(38.95,92.8){\circle*{3}}
		\put(32.235,124.8){\circle*{3}}
		\put(120,62){\begin{rotate}{29}{\whiten\vrule width 40pt height 6pt}\end{rotate}}
		\put(120,62){\begin{rotate}{29}{\tiny $R/\alpha = 100$}\end{rotate}}
		\put(38,95){\begin{rotate}{73}{\whiten\vrule width 30pt height 6pt}\end{rotate}}
		\put(38,95){\begin{rotate}{73}{\tiny $R/\alpha = 700$}\end{rotate}}
		\put(37.5,60){\begin{rotate}{-70}{\whiten\vrule width 16pt height 4pt}\end{rotate}}
		\put(37.5,60){\begin{rotate}{-70}{\tiny $n = 1$}\end{rotate}}
		\put(74,70){\begin{rotate}{-73}{\whiten\vrule width 20pt height 4pt}\end{rotate}}
		\put(74,70){\begin{rotate}{-73}{\tiny $n = 0$}\end{rotate}}
		\end{picture}
	\caption{\label{fig:hino}Die Experimente von \textsl{Hino, Sawamoto und Takasu} zeigen Stabilit\"at (helle Punkte) und Instabilit\"at (dunkle Punkte). Au{\ss}erdem sind die neutralen Kurven der quasista"-tisch"-en Rechnung f\"ur $n=0$ und $n=1$, sowie die Konstanten $R/\alpha = 100$ und $R/\alpha = 700$ eingezeichnet.}
	\end{center}
\end{figure}

Abbildung \ref{fig:hino} zeigt die Ergebnisse der Untersuchungen von \textsl{Hino Sawamoto und Takasu} vergleichend mit den vorliegenden Rechnungen der quasista"-tisch"-en Theorie.
Die instabilen Befunde liegen im Bereich, f\"ur den die Rechnung Instabilit\"at be"-z\"ug"-lich antimetrischer St\"or"-ungen voraussagt.
F\"ur Parameterkombinationen nahe der neutralen Kurve k\"onnen keine me\ss baren Anzeichen f\"ur In"-sta"-bi"-li"-t\"at"-en erwartet werden, da zum einen der Aufklingfaktor zu klein ist und zum anderen die Zeitdauer des instabilen Bereichs zu kurz daf\"ur ist, da\ss\ kleine St\"or"-ungen auf ein me\ss bares Niveau ansteigen k\"onnen.\\

In keiner der Arbeiten wird die Grenze f\"ur schwache In"-sta"-bi"-li"-t\"at"-en bei gro\ss en Womersleyschen Zahlen $\alpha$ ermittelt, die nach der quasistatischen Theorie einen Wert von $R/\alpha \simeq 100$ einnimmt.
Das bekr\"aftigt die Vermutung, welche bei der Diskussion der Berechnungsergebnisse am Ende des letzten Kapitels angestellt wurde, da\ss\ die Wirkdauer des quasistatischen Verst\"arkungsfaktors in diesem Parameterbereich zu kurz ist, um die laminare Bewegung wesentlich zu st\"oren.

%% Zusammenfassung
\chapter*{Zusammenfassung}\label{sec:zusammenfassung}
\pagestyle{epilogue}

Bei Str\"omungsvorg\"angen unterscheiden wir prinzipiell zwischen laminarer und turbulenter Bewegung.
Welche der beiden Formen vorkommt, ist von grundlegendem Interesse.
Die L\"osung der Bewegungsgleichungen f\"uhrt auch dann auf die laminare Str\"omung, wenn diese praktisch nicht vorkommt.
Ursache dieses Ph\"anomens ist die Instabilit\"at der Str\"omung gegen\"uber kleinen St\"orungen oder Imperfektionen im System, die in der Realit\"at stets vorhanden sind.
Die Stabilit\"atsfrage kann mathematisch formuliert und gel\"ost werden.
Ziel der Analyse ist die Entscheidung, unter welchen Parameterrestriktionen eine Grundstr\"omung entweder stabil oder instabil ist.
Falls sie stabil ist, kann die laminare Str\"omung aufrecht erhalten werden.
Andernfalls zerf\"allt sie in eine m\"oglicherweise turbulente Str\"omung.\\

In dieser Arbeit untersuchen wir die Str\"omung einer reibungsbehafteten Fl\"ussigkeit in einem Rohr, welche sich unter dem Einflu\ss\ eines harmonisch oszillierenden axialen Druckgradienten einstellt.
Die Grundstr\"omung l\"a\ss t sich durch zwei unabh\"angige Kenngr\"o\ss en (Womersley- und Reynoldszahl) parametrisieren.
Sie ist ein Prototyp f\"ur instation\"are Str\"omungen, wie sie beispielsweise in biologischen Gef\"a\ss systemen oder in verfahrenstechnischen Anlagen vorkommen.\\

Bisher wurde die oszillierende Rohrstr\"omung vorwiegend experimentell untersucht.
Aus den Befunden geht hervor, da\ss\ die Str\"o"-mung f\"ur ausgew\"ahlte Paramterkombinationen zu bestimmten Zeiten ein instabiles Verhalten aufweist.

Rechnerisch konnte das bisher nicht best\"atigt werden.
Die durchge"-f\"uhrten Stabilit\"atsanalysen, basierend auf der Floquetschen Theorie, deuten auf asymptotisches Stabilit\"atsverhalten der Str\"omung gegen\"uber axialsymmetrischen St\"orungen hin.\\

Es ist Ziel dieser Arbeit, die in den Experimenten gefundenen Instabilit\"atsph\"anomene theoretisch zu erkl\"aren.
Dazu werden die vorhandenen Ans\"atze zur hydrodynamischen Stabilit\"at verglichen und an die Problemstellung angepa\ss t.
Mit einer geeigneten Methode werden anschlie\ss end Parameterstudien \"uber alle sinnvollen Kombinationen von Grundstr\"omungen und St\"orungen durchgef\"uhrt.\\

Ausgehend von den Bewegungsgleichungen einer reibungbehafteten inkompressiblen Fl\"ussigkeit (Navier-Stokes-Gleichung"-en), leiten wir die Evolutionsgleichungen der St\"orungsgeschwindigkeiten her.
Es handelt sich dabei um ein System partieller Differentialgleichungen mit nichtkonstanten Koeffizienten.
Zur Diskussion des Systems vergleichen wir verschiedene Ans\"atze.

Mit der Floquetschen Theorie k\"onnen lineare Systeme ge"-w\"ohnlicher Differentialgleichungen mit periodischen Koeffizienten auf ihre asymptotische Stabilit\"at hin untersucht werden.
Es gelingt uns, die Evolutionsgleichungen mit einer Diskretisierung in diese Form zu \"uberf\"uhren.
Die Stabilit\"atsaussage gilt allerdings nur \"uber einen Zeitraum, welcher mehrere Perioden der oszillierenden Str\"omung umfa\ss t und f\"uhrt ausschlie\ss lich auf Stabilit\"at.\\
Um die kurzzeitigen Effekte (w\"ahrend einer Schwingungsperiode) zu untersuchen, f\"uhren wir den quasistatischen Stabilit\"atsbegriff ein.
Durch eine zeitliche Skalenseparation betrachten wir die Dynamik von schnellen St\"orungen \"uber einer langsam oszillierenden (quasistatischen) Grundstr\"omung.
Das mathematische System vereinfacht sich dadurch zu einem System mit konstanten Koeffizienten in der Zeit und f\"uhrt auf ein Rand-Eigenwertproblem.\\
Zur L\"osung des Rand-Eigenwertproblems vergleichen wir ein Schie\ss verfahren mit einer Galerkin-Entwicklung.
Eine besondere Schwierigkeit stellt hierbei die Vermeidung von Singularit\"aten auf der Rohrmitte dar, die durch die Darstellung der Gleichungen im zylindrischen Koordinatensystem erscheinen.
Die Hauptrechnungen dieser Arbeit, welche die Parameterstudien beinhalten, basieren auf der Galerkin-Entwicklung.\\
Um die Ergebnisse der quasistatischen Studien zu interpretieren, vergleichen wir sie mit direkten numerischen L\"osungen des Anfangswertproblems.
Hierbei w\"ahlen wir spezielle Parameterkombinationen und zuf\"allige Anfangsbedingungen aus.\\

Es ist gelungen, mit der quasistatischen Methode kurzzeitige instabile St\"orungen nachzuweisen.
Sie treten jeweils zu den Zeitpunkten der Flu\ss umkehr auf und werden anschlie\ss end wieder stark ged\"ampft.\\
Sowohl symmetrische als auch antimetrische St\"orungen bez\"ug"-lich der Rohr"-achse dominieren das Instabilit\"atsgebiet.
Auff\"allig ist, da\ss\ antimetrische St\"orungen am meisten verst\"arkt werden.
St\"orungen mit h\"oheren Winkelzahlen sind hingegen allgemein st\"arker ged\"ampft.\\
Anf\"allig f\"ur die kurzzeitigen Verst\"arkungen sind St\"orungen im Bereich mittlerer axialer Wellenl\"angen.
Sowohl langwellige als auch kurzwellige St\"orungen beliebiger Form werden zu jeder Zeit ged\"ampft.\\
Durch die vergleichenden Studien k\"onnen wir das quasistatische Instabilit\"atsgebiet in drei Bereiche in Abh\"angigkeit der Reynolds- und Womersleyzahl aufteilen.
Am auff\"alligsten sind die Verst\"arkungen bei Womersleyschen Zahlen von 8 f\"ur symmetrische (4 f\"ur antimetrische) St\"orungen in der Rohrmitte bereits ab Reynoldsschen Zahlen von 1000.\\

Die verwendete lineare Stabilit\"atstheorie beschreibt den pri"-m\"aren Instabilit\"atsmechanismus, welcher durch den Energietransfer von der Grundstr\"omung in die St\"orungen verursacht wird.
Wir m\"ussen davon ausgehen, da\ss\ auch andere Mechanismen eine wichtige Rolle spielen (Nichtlinearit\"at und Nichtorthogonalit\"at), wie aus dem Grenzfall der station\"aren Str\"omung ersichtlich wird.\\


In dieser Arbeit haben wir erstmals die quasistatische Theorie erfolgreich auf die Stabilit\"atsfrage der oszillierenden Rohrstr\"omung angewendet.
Mithilfe umfangreicher Parameterstudien haben wir die aus Experimenten bekannten Instabilit\"atsph\"anomene erkl\"art und das Instabilit\"atsgebiet eingehend analysiert.

\chapter*{Literatur}\label{sec:literatur}
\pagestyle{literatur}
\newcommand{\bibfill}{\,\hrulefill\,} %\dotfill
\begin{enumerate}
	\item \textsl{R. Akhavan, R.~D.~Kamm, A.~H.~Sharpiro:} An investigation of transition to turbulence in bounded oscillatory {S}tokes flows Part 1. Experiments. \textit{Journal of Fluid Mechanics} \textbf{225}, 395--422 (1991)\bibfill\pageref{bib:akhavan1}
	\item \textsl{E. Anderson, Z. Bai, C. Bischof, S. Blackford, J. Demmel, J. Dongarra, J. Du Croz, A. Greenbaum, S. Hammarling, A. McKenney, D. Sorensen:} {LAPACK} Users' Guide. \textit{Society for Industrial and Applied Mathematics} (1999)\bibfill\pageref{bib:lapack}
	\item \textsl{D.~E.~Amos:} A portable package for {B}essel functions of a complex argument and nonnegative order. \textit{ACM Transactions on Mathematical Software} \textbf{12}, 265--273 (1986)\bibfill\pageref{bib:amos}
	\item \textsl{Peter J. Blennerhassett, Andrew P. Bassom:} The linear stability of flat {S}tokes layers. \textit{Journal of Fluid Mechanics} \textbf{464}, 393--410 (2002)\bibfill\pageref{bib:blennerhassett_bassom}
	\item \textsl{K.~M.~Case:} Hydrodynamic stability and the inviscid limit. \textit{Journal of Fluid Mechanics} \textbf{10}, 420 (1961)\bibfill\pageref{bib:case2}
	\item \textsl{A.~G.~Darbyshire, T. Mullin:} Transition to turbulence in constant-mass-flux pipe flow. \textit{Journal of Fluid Mechanics} \textbf{289}, 83--114 (1995)\bibfill\pageref{bib:darbyshire_mullin}
	\item \textsl{Debopam Das, Jaywant H. Arakeri:} Transition of unsteady velocity profiles with reverse flow. \textit{Journal of Fluid Mechanics} \textbf{374}, 215--183 (1998)\bibfill\pageref{bib:das_arakeri}
	\item \textsl{Stephen H. Davis:} The stability of time-periodic flows. \textit{Annual Review of Fluid Mechanics} \textbf{8}, 57--74 (1976)\bibfill\pageref{bib:davis_review}
	\item \textsl{David M. Eckmann, James B. Grotberg:} Experiments on transition to turbulence in oscillatory pipe flow. \textit{Journal of Fluid Mechanics} \textbf{222}, 329--350 (1991)\bibfill\pageref{bib:eckmann_grotberg}
	\item \textsl{F. Fedele, D.~L.~Hitt, R.~D.~Prabhu:} Revisiting the stability of pulsatile pipe flow. \textit{European Journal of Mechanics B/Fluids} \textbf{24}, 237--254 (2005)\bibfill\pageref{bib:fedele_hitt}
	\item \textsl{Gaston Floquet:} Sur les {\'e}quations diff{\'e}rentielles lin{\'e}aires {\`a} coefficients p{\'e}riodiques. \textit{Annales Scientifiques de l'{\'E}cole Normale Sup{\'e}rieure} \textbf{12}, 47--88 (1883)\bibfill\pageref{bib:floquet}
	\item \textsl{Mohamed S. Ghidaoui, A.~A.~Kolyshkin:} A quasi-steady approach to the instability of time-dependent flows in pipes. \textit{Journal of Fluid Mechanics} \textbf{465}, 301--330 (2002)\bibfill\pageref{bib:ghidaoui_kolyshkin}
	\item \textsl{A.~E.~Gill:} The least-damped disturbance to {P}oiseuille flow in a circular pipe. \textit{Journal of Fluid Mechanics} \textbf{61}, 97--107 (1973)\bibfill\pageref{bib:gill2}
	\item \textsl{Chester E. Grosch, Harold Salwen:} The stability of steady time-depen"-dent plane {P}oiseuille flow. \textit{Journal of Fluid Mechanics} \textbf{34}, 177--205 (1968)\bibfill\pageref{bib:grosch_salwen}
	\item \textsl{Siegfried Grossmann:} The onset of shear flow turbulence. \textit{Reviews of Modern Physics} \textbf{72}, 603--618 (2000)\bibfill\pageref{bib:grossmann}
	\item \textsl{Philip Hall:} The linear stability of flat {S}tokes layers. \textit{Proceedings of the Royal Society of London A} \textbf{359}, 151--166 (1978)\bibfill\pageref{bib:hall}
	\item \textsl{Philip Hall:} On the instability of {S}tokes layers at high {R}eynolds numbers. \textit{Journal of Fluid Mechanics} \textbf{482}, 1--15 (2003)\bibfill\pageref{bib:hall3}
	\item \textsl{Georg Hamel:} {Z}um {T}urbulenzproblem. \textit{Nachrichten von der Gesellschaft der Wissenschaften zu G{\"o}ttingen}, 261--270 (1911)\bibfill\pageref{bib:hamel}
	\item \textsl{S. He, J.~D.~Jackson:} A study of turbulence under conditions of transient flow in a pipe. \textit{Journal of Fluid Mechanics} \textbf{408}, 1--38 (2000)\bibfill\pageref{bib:he_jackson}
	\item \textsl{Werner Heisenberg:} {\"U}ber {S}tabilit{\"a}t und {T}urbulenz von {F}l{\"u}ssigkeitsstr{\"o}men. \textit{Annalen der Physik} \textbf{74}, 577 (1924)\bibfill\pageref{bib:heisenberg}
	\item \textsl{Hermann Helmholtz:} {\"U}ber discontinuierliche {F}l{\"u}ssigkeits-{B}ewegungen. \textit{Monatsberichte der Preu{\ss}ischen Akademie der Wissenschaften zu Berlin} \textbf{23}, 215--218 (1868)\bibfill\pageref{bib:helmholtz_fluessigkeit}
	\item \textsl{Mikio Hino, Masaki Sawamoto, Shuji Takasu:} Experiments on transition to turbulence in an oscillatory pipe flow. \textit{Journal of Fluid Mechanics} \textbf{75}, 193--207 (1976)\bibfill\pageref{bib:hino_sawamoto}
	\item \textsl{R. Hooke, T.~A.~Jeeves:} Direct search solution of numerical and statistical problems. \textit{Journal of the ACM} \textbf{8}, 212--229 (1961)\bibfill\pageref{bib:hooke_jeeves}
	\item \textsl{Ludwig Hopf:} {D}er {V}erlauf kleiner {S}chwingungen auf einer {S}tr{\"o}mung reibender {F}l{\"u}ssigkeit. \textit{Annalen der Physik} \textbf{44}, 1--60 (1914)\bibfill\pageref{bib:hopf}
	\item \textsl{Patrick Huerre, Peter A. Monkewitz:} Local and global instabilities in spatially developing flows. \textit{Annual Review of Fluid Mechanics} \textbf{22}, 473--537 (1990)\bibfill\pageref{bib:huerre_monkewitz}
	\item \textsl{Kelvin:} Stability of fluid motion -- rectilinear motion of viscous fluid between two parallel planes. \textit{Philosophical Magazine} \textbf{24}, 188--196 (1887)\bibfill\pageref{bib:kelvin}
	%\item \textsl{Martin Wilhem Kutta:} Beitrag zur n{\"a}herungsweisen Integration totaler Differentialgleichungen. \textit{Zeitschrift f{\"ur} Mathematik und Physik} \textbf{46}, 435 (1901)\bibfill\pageref{bib:kutta}
	\item \textsl{C.~R.~Lodahl, B.~M.~Sumer, J. Fredsoe:} Turbulent combined oscillatory flow and current in a pipe. \textit{Journal of Fluid Mechanics} \textbf{373}, 331--348 (1998)\bibfill\pageref{bib:lodahl_sumer}
	\item \textsl{L.~M.~Mack:} A numerical study of the temporal eigenvalue spectrum of the {B}lasius boundary layer. \textit{Journal of Fluid Mechanics} \textbf{73}, 497--520 (1976)\bibfill\pageref{bib:mack}
	%\item \textsl{P. Merkli, H. Thomann:} Transition to turbulence in oscillating pipe flow. \textit{Journal of Fluid Mechanics} \textbf{68}, 567--575 (1975)\bibfill\pageref{bib:merkli_thomann}
	\item \textsl{Steven L.~B.~Moshier:} Methods and programmes for mathematical functions. Wiley (1989)\bibfill\pageref{bib:moshier}
	\item \textsl{Ali Hasan Nayfeh, Dean T. Mook:} Nonlinear oscillations. Wiley (1979)\bibfill\pageref{bib:nayfeh_mook}
	\item \textsl{Munekazu Ohmi, Manabu Iguchi, Koichiro Kakehashi, Tetsuya Masuda:} Transition to turbulence and velocity distribution in an oscillating pipe flow. \textit{Bulletin of the JSME} \textbf{25}, 365 (1982)\bibfill\pageref{bib:ohmi_igushi}
	\item \textsl{William Orr:} The stability or instability of the steady motions of a perfect liquid and of a viscous liquid. \textit{Proceedings of the Royal Irish Academy} \textbf{XXVII}, 9 (1907)\bibfill\pageref{bib:orr},\,\pageref{bib:orr:2}
	\item \textsl{Lord Rayleigh:} On the question of the stability of the flow of fluids. \textit{The London, Edinburgh, and Dublin philosophical magazine and journal of science} \textbf{34}, 59--70 (1892)\bibfill\pageref{bib:rayleigh}
	\item \textsl{Satish C. Reddy, Peter J. Schmidt, Dan S. Henningson:} Pseudospectra of the {O}rr-{S}ommerfeld operator. \textit{SIAM Journal of Applied Mathematics} \textbf{53}, 15--47 (1993)\bibfill\pageref{bib:reddy}
	\item \textsl{Osborne Reynolds:} On the dynamical theory of incompressible viscous fluids and the determination of the criterion. \textit{Philosophical Transactions of the Royal Society of London} \textbf{186}, 123--164 (1895)\bibfill\pageref{bib:reynolds_criterion},\,\pageref{bib:reynolds_criterion:2}
	\item \textsl{Osborne Reynolds:} An experimental investigation of the circumstances which determine wether the motion of water shall be direct or sinuous, and of the law of resistance in parallel channels. \textit{Philosophical Transactions of the Royal Society of London} \textbf{174}, 935 (1883)\bibfill\pageref{bib:reynolds_experiment}
	\item \textsl{Bernd Rummler:} Eigenfunctions of the {S}tokes operator in special domains {I}. \textit{Zeitschrift f{\"u}r angewandte Mathematik und Mechanik} \textbf{77(8)}, 619--627 (1997)\bibfill\pageref{bib:rummler}
	\item \textsl{Harold Salwen, Chester E. Grosch:} The stability of {P}oiseuille flow in a pipe of circular cross-section. \textit{Journal of Fluid Mechanics} \textbf{54}, 93--112 (1972)\bibfill\pageref{bib:salwen_grosch},\,\pageref{bib:salwen_grosch:2}
	\item \textsl{Turgut Sarpkaya:} Experimental determination of the critical {R}eynolds number for pulsating {P}oiseuille flow. \textit{Journal of Basic Engineering} \textbf{Sep.}, 589--598 (1966)\bibfill\pageref{bib:sarpkaya}
	\item \textsl{James Serrin:} On the stability of viscous fluid motions. \textit{Archive for Rational Mechanics and Analysis} \textbf{3}, 1--13 (1959)\bibfill\pageref{bib:serrin}
	\item \textsl{Theodor Sexl:} {Z}ur {S}tabilit{\"a}tsfrage der {P}oiseuilleschen und {C}ouetteschen {S}tr{\"o}mung. \textit{Annalen der Physik} \textbf{83}, 835--848 (1927)\bibfill\pageref{bib:sexl}
	\item \textsl{Theodor Sexl:} {\"U}ber den von {E}. {G}. {R}ichardson entdeckten ``{A}nnulareffekt''. \textit{Zeitschrift f{\"u}r Physik} \textbf{61}, 349--362 (1930)\bibfill\pageref{bib:sexl2}
	\item \textsl{L. Shemer, I. Wygnanski, E. Kit:} Pulsating flow in a pipe. \textit{Journal of Fluid Mechanics} \textbf{153}, 313--337 (1985)\bibfill\pageref{bib:shemer_wygnanski}
	\item \textsl{Arnold Sommerfeld:} Ein {B}eitrag zur hydrodynamischen {E}rkl{\"a}rung der turbulenten {F}l{\"u}ssigkeitsbewegung. \textit{Atti del 4 Congresso Internazionale dei Matematici, Roma} \textbf{III}, 116--126 (1908)\bibfill\pageref{bib:sommerfeld_congresso}
	\item \textsl{George Gabriel Stokes:} On the effect of internal friction of fluids on the motion of pendulums. \textit{Transactions of the Cambridge Philosophical Society} \textbf{ix}, 8--16 (1851)\bibfill\pageref{bib:stokes}
	\item \textsl{J.~T.~Tozzi, Christian H. von Kerczek:} The stability of oscillatory {H}agen-{P}oiseuille flow. \textit{Journal of Applied Mechanics} \textbf{53}, 187 (1986)\bibfill\pageref{bib:tozzi_kerczek}
	\item \textsl{Lloyd N. Trefethen, Anne E. Trefethen, Satish C. Reddy, Tobin A. Driscoll:} Hydrodynamic stability without eigenvalues. \textit{Science} \textbf{261}, 578 (1993)\bibfill\pageref{bib:trefethen}
	\item \textsl{Shigeo Uchida:} The pulsating viscous flow superposed on the steady laminar motion of incompressible fluid in a circular pipe. \textit{Zeitschrift f{\"u}r angewandte Mathematik und Mechanik} \textbf{7}, 403 (1956)\bibfill\pageref{bib:uchida}
	\item \textsl{J. Watson:} On spatially-growing finite disturbances in plane {P}oiseuille flow. \textit{Journal of Fluid Mechanics} \textbf{14}, 211--221 (1962)\bibfill\pageref{bib:watson_j}
	\item \textsl{John R.\ Womersley:} An elastic tube theory of puls transmission and oscillatory flow in mammalian arteries. \textit{WADC Technical Report} \textbf{TR 56-614}, (1959)\bibfill\pageref{bib:womersley_rep}
	\item \textsl{W.~H.~Yang, Chia-Shun Yih:} Stability of time-periodic flows in a circular pipe. \textit{Journal of Fluid Mechanics} \textbf{82}, 497--505 (1977)\bibfill\pageref{bib:yang_yih},\,\pageref{bib:yang_yih:2}
	\item \textsl{M. Zhao, M.~S.~Ghidaoui, A.~A.~Kolyshkin, R. Vaillancourt:} On the stability of oscillatory pipe flow. \textit{Technische Mechanik} \textbf{24}, 289--296 (2004)\bibfill\pageref{bib:zhao_ghidaoui}
	%\item \textsl{S.C.M. Yu, J.~B.~Zhao:} Steady and pulsating flow characteristics in straight tubes with and without a lateral circular protrusion. \textit{Experiments in Fluids} \textbf{26}, 505--512 (1999)\bibfill\pageref{bib:yu_zhao}
\end{enumerate}
\end{document}
