\input{header}
\begin{document}
\thispagestyle{empty}
\section*{Inhaltsverzeichnis}
\renewcommand\baselinestretch{1.2}\normalfont

\begin{tabular}{lll}
	\textbf{\large{1}}&\multicolumn{2}{l}{\textbf{\large{{Einleitung}}}}\\
	&1.1&{Definition und Einteilung der Mechanik}\\
	&1.2&{Physikalische Gr��en und Einheiten}\\
	&1.3&{Mathematische Grundlagen}\\
	&1.4&{Lehrb�cher zur Technischen Mechanik}\\
	&1.5&{Aufgaben}\\
	\\\textbf{\large{2}}&\multicolumn{2}{l}{\textbf{\large{{Statik starrer K�rper}}}}\\
	&2.1&{Grundoperationen mit Kr�ften}\\
	&2.2&{Kr�ftepaar und Moment}\\
	&2.3&{Gleichgewicht starrer K�rper}\\
	&2.4&{Verteilte Kr�fte und Schwerpunkt}\\
	&2.5&{Lagerreaktionen}\\
	&2.6&{Haften und Gleiten}\\
	&2.7&{Aufgaben}\\
	\\\textbf{\large{3}}&\multicolumn{2}{l}{\textbf{\large{{Einblick in die Elastostatik}}}}\\
	&3.1&{Einachsiger Spannungszustand}\\
	&3.2&{Statisch unbestimmte Stabsysteme}\\
	&3.3&{Biegung gerader Balken}\\
	&3.4&{Form�nderungsenergien}\\
	&3.5&{Aufgaben}\\
	\\\textbf{\large{4}}&\multicolumn{2}{l}{\textbf{\large{{Kinematik}}}}\\
	&4.1&{Punkt-Bewegungen}\\
	&4.2&{Ebene Bewegungen starrer K�rper}\\
	&4.3&{Aufgaben}\\
	\\\textbf{\large{5}}&\multicolumn{2}{l}{\textbf{\large{{Kinetik}}}}\\
	&{5.1}&Kinetische Grundbegriffe\\
	&{5.2}&Kinetik des Massenpunktes\\
	&{5.3}&Rotation um eine feste Achse\\
	&{5.4}&Kinetik ebener Starrk�rperbewegungen\\
	&{5.5}&Lineare Schwingungen\\
	&{5.6}&Aufgaben\\
\end{tabular}
\end{document}
